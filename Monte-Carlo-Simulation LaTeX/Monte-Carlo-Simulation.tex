
% Default to the notebook output style

    


% Inherit from the specified cell style.




    
%\documentclass[11pt]{article}
\documentclass[paper=landscape]{scrartcl}
	\usepackage[paperwidth=40cm,paperheight=235cm,margin=1in]{geometry}

    
    
    \usepackage[T1]{fontenc}
    % Nicer default font (+ math font) than Computer Modern for most use cases
    \usepackage{mathpazo}

    % Basic figure setup, for now with no caption control since it's done
    % automatically by Pandoc (which extracts ![](path) syntax from Markdown).
    \usepackage{graphicx}
    % We will generate all images so they have a width \maxwidth. This means
    % that they will get their normal width if they fit onto the page, but
    % are scaled down if they would overflow the margins.
    \makeatletter
    \def\maxwidth{\ifdim\Gin@nat@width>\linewidth\linewidth
    \else\Gin@nat@width\fi}
    \makeatother
    \let\Oldincludegraphics\includegraphics
    % Set max figure width to be 80% of text width, for now hardcoded.
    \renewcommand{\includegraphics}[1]{\Oldincludegraphics[width=.8\maxwidth]{#1}}
    % Ensure that by default, figures have no caption (until we provide a
    % proper Figure object with a Caption API and a way to capture that
    % in the conversion process - todo).
    \usepackage{caption}
    \DeclareCaptionLabelFormat{nolabel}{}
    \captionsetup{labelformat=nolabel}

    \usepackage{adjustbox} % Used to constrain images to a maximum size 
    \usepackage{xcolor} % Allow colors to be defined
    \usepackage{enumerate} % Needed for markdown enumerations to work
    \usepackage{geometry} % Used to adjust the document margins
    \usepackage{amsmath} % Equations
    \usepackage{amssymb} % Equations
    \usepackage{textcomp} % defines textquotesingle
    % Hack from http://tex.stackexchange.com/a/47451/13684:
    \AtBeginDocument{%
        \def\PYZsq{\textquotesingle}% Upright quotes in Pygmentized code
    }
    \usepackage{upquote} % Upright quotes for verbatim code
    \usepackage{eurosym} % defines \euro
    \usepackage[mathletters]{ucs} % Extended unicode (utf-8) support
    \usepackage[utf8x]{inputenc} % Allow utf-8 characters in the tex document
    \usepackage{fancyvrb} % verbatim replacement that allows latex
    \usepackage{grffile} % extends the file name processing of package graphics 
                         % to support a larger range 
    % The hyperref package gives us a pdf with properly built
    % internal navigation ('pdf bookmarks' for the table of contents,
    % internal cross-reference links, web links for URLs, etc.)
    \usepackage{hyperref}
    \usepackage{longtable} % longtable support required by pandoc >1.10
    \usepackage{booktabs}  % table support for pandoc > 1.12.2
    \usepackage[inline]{enumitem} % IRkernel/repr support (it uses the enumerate* environment)
    \usepackage[normalem]{ulem} % ulem is needed to support strikethroughs (\sout)
                                % normalem makes italics be italics, not underlines
    \usepackage{mathrsfs}
    

    
    
    % Colors for the hyperref package
    \definecolor{urlcolor}{rgb}{0,.145,.698}
    \definecolor{linkcolor}{rgb}{.71,0.21,0.01}
    \definecolor{citecolor}{rgb}{.12,.54,.11}

    % ANSI colors
    \definecolor{ansi-black}{HTML}{3E424D}
    \definecolor{ansi-black-intense}{HTML}{282C36}
    \definecolor{ansi-red}{HTML}{E75C58}
    \definecolor{ansi-red-intense}{HTML}{B22B31}
    \definecolor{ansi-green}{HTML}{00A250}
    \definecolor{ansi-green-intense}{HTML}{007427}
    \definecolor{ansi-yellow}{HTML}{DDB62B}
    \definecolor{ansi-yellow-intense}{HTML}{B27D12}
    \definecolor{ansi-blue}{HTML}{208FFB}
    \definecolor{ansi-blue-intense}{HTML}{0065CA}
    \definecolor{ansi-magenta}{HTML}{D160C4}
    \definecolor{ansi-magenta-intense}{HTML}{A03196}
    \definecolor{ansi-cyan}{HTML}{60C6C8}
    \definecolor{ansi-cyan-intense}{HTML}{258F8F}
    \definecolor{ansi-white}{HTML}{C5C1B4}
    \definecolor{ansi-white-intense}{HTML}{A1A6B2}
    \definecolor{ansi-default-inverse-fg}{HTML}{FFFFFF}
    \definecolor{ansi-default-inverse-bg}{HTML}{000000}

    % commands and environments needed by pandoc snippets
    % extracted from the output of `pandoc -s`
    \providecommand{\tightlist}{%
      \setlength{\itemsep}{0pt}\setlength{\parskip}{0pt}}
    \DefineVerbatimEnvironment{Highlighting}{Verbatim}{commandchars=\\\{\}}
    % Add ',fontsize=\small' for more characters per line
    \newenvironment{Shaded}{}{}
    \newcommand{\KeywordTok}[1]{\textcolor[rgb]{0.00,0.44,0.13}{\textbf{{#1}}}}
    \newcommand{\DataTypeTok}[1]{\textcolor[rgb]{0.56,0.13,0.00}{{#1}}}
    \newcommand{\DecValTok}[1]{\textcolor[rgb]{0.25,0.63,0.44}{{#1}}}
    \newcommand{\BaseNTok}[1]{\textcolor[rgb]{0.25,0.63,0.44}{{#1}}}
    \newcommand{\FloatTok}[1]{\textcolor[rgb]{0.25,0.63,0.44}{{#1}}}
    \newcommand{\CharTok}[1]{\textcolor[rgb]{0.25,0.44,0.63}{{#1}}}
    \newcommand{\StringTok}[1]{\textcolor[rgb]{0.25,0.44,0.63}{{#1}}}
    \newcommand{\CommentTok}[1]{\textcolor[rgb]{0.38,0.63,0.69}{\textit{{#1}}}}
    \newcommand{\OtherTok}[1]{\textcolor[rgb]{0.00,0.44,0.13}{{#1}}}
    \newcommand{\AlertTok}[1]{\textcolor[rgb]{1.00,0.00,0.00}{\textbf{{#1}}}}
    \newcommand{\FunctionTok}[1]{\textcolor[rgb]{0.02,0.16,0.49}{{#1}}}
    \newcommand{\RegionMarkerTok}[1]{{#1}}
    \newcommand{\ErrorTok}[1]{\textcolor[rgb]{1.00,0.00,0.00}{\textbf{{#1}}}}
    \newcommand{\NormalTok}[1]{{#1}}
    
    % Additional commands for more recent versions of Pandoc
    \newcommand{\ConstantTok}[1]{\textcolor[rgb]{0.53,0.00,0.00}{{#1}}}
    \newcommand{\SpecialCharTok}[1]{\textcolor[rgb]{0.25,0.44,0.63}{{#1}}}
    \newcommand{\VerbatimStringTok}[1]{\textcolor[rgb]{0.25,0.44,0.63}{{#1}}}
    \newcommand{\SpecialStringTok}[1]{\textcolor[rgb]{0.73,0.40,0.53}{{#1}}}
    \newcommand{\ImportTok}[1]{{#1}}
    \newcommand{\DocumentationTok}[1]{\textcolor[rgb]{0.73,0.13,0.13}{\textit{{#1}}}}
    \newcommand{\AnnotationTok}[1]{\textcolor[rgb]{0.38,0.63,0.69}{\textbf{\textit{{#1}}}}}
    \newcommand{\CommentVarTok}[1]{\textcolor[rgb]{0.38,0.63,0.69}{\textbf{\textit{{#1}}}}}
    \newcommand{\VariableTok}[1]{\textcolor[rgb]{0.10,0.09,0.49}{{#1}}}
    \newcommand{\ControlFlowTok}[1]{\textcolor[rgb]{0.00,0.44,0.13}{\textbf{{#1}}}}
    \newcommand{\OperatorTok}[1]{\textcolor[rgb]{0.40,0.40,0.40}{{#1}}}
    \newcommand{\BuiltInTok}[1]{{#1}}
    \newcommand{\ExtensionTok}[1]{{#1}}
    \newcommand{\PreprocessorTok}[1]{\textcolor[rgb]{0.74,0.48,0.00}{{#1}}}
    \newcommand{\AttributeTok}[1]{\textcolor[rgb]{0.49,0.56,0.16}{{#1}}}
    \newcommand{\InformationTok}[1]{\textcolor[rgb]{0.38,0.63,0.69}{\textbf{\textit{{#1}}}}}
    \newcommand{\WarningTok}[1]{\textcolor[rgb]{0.38,0.63,0.69}{\textbf{\textit{{#1}}}}}
    
    
    % Define a nice break command that doesn't care if a line doesn't already
    % exist.
    \def\br{\hspace*{\fill} \\* }
    % Math Jax compatibility definitions
    \def\gt{>}
    \def\lt{<}
    \let\Oldtex\TeX
    \let\Oldlatex\LaTeX
    \renewcommand{\TeX}{\textrm{\Oldtex}}
    \renewcommand{\LaTeX}{\textrm{\Oldlatex}}
    % Document parameters
    % Document title
    \title{Monte-Carlo-Simulation}
    
    
    
    
    

    % Pygments definitions
    
\makeatletter
\def\PY@reset{\let\PY@it=\relax \let\PY@bf=\relax%
    \let\PY@ul=\relax \let\PY@tc=\relax%
    \let\PY@bc=\relax \let\PY@ff=\relax}
\def\PY@tok#1{\csname PY@tok@#1\endcsname}
\def\PY@toks#1+{\ifx\relax#1\empty\else%
    \PY@tok{#1}\expandafter\PY@toks\fi}
\def\PY@do#1{\PY@bc{\PY@tc{\PY@ul{%
    \PY@it{\PY@bf{\PY@ff{#1}}}}}}}
\def\PY#1#2{\PY@reset\PY@toks#1+\relax+\PY@do{#2}}

\expandafter\def\csname PY@tok@w\endcsname{\def\PY@tc##1{\textcolor[rgb]{0.73,0.73,0.73}{##1}}}
\expandafter\def\csname PY@tok@c\endcsname{\let\PY@it=\textit\def\PY@tc##1{\textcolor[rgb]{0.25,0.50,0.50}{##1}}}
\expandafter\def\csname PY@tok@cp\endcsname{\def\PY@tc##1{\textcolor[rgb]{0.74,0.48,0.00}{##1}}}
\expandafter\def\csname PY@tok@k\endcsname{\let\PY@bf=\textbf\def\PY@tc##1{\textcolor[rgb]{0.00,0.50,0.00}{##1}}}
\expandafter\def\csname PY@tok@kp\endcsname{\def\PY@tc##1{\textcolor[rgb]{0.00,0.50,0.00}{##1}}}
\expandafter\def\csname PY@tok@kt\endcsname{\def\PY@tc##1{\textcolor[rgb]{0.69,0.00,0.25}{##1}}}
\expandafter\def\csname PY@tok@o\endcsname{\def\PY@tc##1{\textcolor[rgb]{0.40,0.40,0.40}{##1}}}
\expandafter\def\csname PY@tok@ow\endcsname{\let\PY@bf=\textbf\def\PY@tc##1{\textcolor[rgb]{0.67,0.13,1.00}{##1}}}
\expandafter\def\csname PY@tok@nb\endcsname{\def\PY@tc##1{\textcolor[rgb]{0.00,0.50,0.00}{##1}}}
\expandafter\def\csname PY@tok@nf\endcsname{\def\PY@tc##1{\textcolor[rgb]{0.00,0.00,1.00}{##1}}}
\expandafter\def\csname PY@tok@nc\endcsname{\let\PY@bf=\textbf\def\PY@tc##1{\textcolor[rgb]{0.00,0.00,1.00}{##1}}}
\expandafter\def\csname PY@tok@nn\endcsname{\let\PY@bf=\textbf\def\PY@tc##1{\textcolor[rgb]{0.00,0.00,1.00}{##1}}}
\expandafter\def\csname PY@tok@ne\endcsname{\let\PY@bf=\textbf\def\PY@tc##1{\textcolor[rgb]{0.82,0.25,0.23}{##1}}}
\expandafter\def\csname PY@tok@nv\endcsname{\def\PY@tc##1{\textcolor[rgb]{0.10,0.09,0.49}{##1}}}
\expandafter\def\csname PY@tok@no\endcsname{\def\PY@tc##1{\textcolor[rgb]{0.53,0.00,0.00}{##1}}}
\expandafter\def\csname PY@tok@nl\endcsname{\def\PY@tc##1{\textcolor[rgb]{0.63,0.63,0.00}{##1}}}
\expandafter\def\csname PY@tok@ni\endcsname{\let\PY@bf=\textbf\def\PY@tc##1{\textcolor[rgb]{0.60,0.60,0.60}{##1}}}
\expandafter\def\csname PY@tok@na\endcsname{\def\PY@tc##1{\textcolor[rgb]{0.49,0.56,0.16}{##1}}}
\expandafter\def\csname PY@tok@nt\endcsname{\let\PY@bf=\textbf\def\PY@tc##1{\textcolor[rgb]{0.00,0.50,0.00}{##1}}}
\expandafter\def\csname PY@tok@nd\endcsname{\def\PY@tc##1{\textcolor[rgb]{0.67,0.13,1.00}{##1}}}
\expandafter\def\csname PY@tok@s\endcsname{\def\PY@tc##1{\textcolor[rgb]{0.73,0.13,0.13}{##1}}}
\expandafter\def\csname PY@tok@sd\endcsname{\let\PY@it=\textit\def\PY@tc##1{\textcolor[rgb]{0.73,0.13,0.13}{##1}}}
\expandafter\def\csname PY@tok@si\endcsname{\let\PY@bf=\textbf\def\PY@tc##1{\textcolor[rgb]{0.73,0.40,0.53}{##1}}}
\expandafter\def\csname PY@tok@se\endcsname{\let\PY@bf=\textbf\def\PY@tc##1{\textcolor[rgb]{0.73,0.40,0.13}{##1}}}
\expandafter\def\csname PY@tok@sr\endcsname{\def\PY@tc##1{\textcolor[rgb]{0.73,0.40,0.53}{##1}}}
\expandafter\def\csname PY@tok@ss\endcsname{\def\PY@tc##1{\textcolor[rgb]{0.10,0.09,0.49}{##1}}}
\expandafter\def\csname PY@tok@sx\endcsname{\def\PY@tc##1{\textcolor[rgb]{0.00,0.50,0.00}{##1}}}
\expandafter\def\csname PY@tok@m\endcsname{\def\PY@tc##1{\textcolor[rgb]{0.40,0.40,0.40}{##1}}}
\expandafter\def\csname PY@tok@gh\endcsname{\let\PY@bf=\textbf\def\PY@tc##1{\textcolor[rgb]{0.00,0.00,0.50}{##1}}}
\expandafter\def\csname PY@tok@gu\endcsname{\let\PY@bf=\textbf\def\PY@tc##1{\textcolor[rgb]{0.50,0.00,0.50}{##1}}}
\expandafter\def\csname PY@tok@gd\endcsname{\def\PY@tc##1{\textcolor[rgb]{0.63,0.00,0.00}{##1}}}
\expandafter\def\csname PY@tok@gi\endcsname{\def\PY@tc##1{\textcolor[rgb]{0.00,0.63,0.00}{##1}}}
\expandafter\def\csname PY@tok@gr\endcsname{\def\PY@tc##1{\textcolor[rgb]{1.00,0.00,0.00}{##1}}}
\expandafter\def\csname PY@tok@ge\endcsname{\let\PY@it=\textit}
\expandafter\def\csname PY@tok@gs\endcsname{\let\PY@bf=\textbf}
\expandafter\def\csname PY@tok@gp\endcsname{\let\PY@bf=\textbf\def\PY@tc##1{\textcolor[rgb]{0.00,0.00,0.50}{##1}}}
\expandafter\def\csname PY@tok@go\endcsname{\def\PY@tc##1{\textcolor[rgb]{0.53,0.53,0.53}{##1}}}
\expandafter\def\csname PY@tok@gt\endcsname{\def\PY@tc##1{\textcolor[rgb]{0.00,0.27,0.87}{##1}}}
\expandafter\def\csname PY@tok@err\endcsname{\def\PY@bc##1{\setlength{\fboxsep}{0pt}\fcolorbox[rgb]{1.00,0.00,0.00}{1,1,1}{\strut ##1}}}
\expandafter\def\csname PY@tok@kc\endcsname{\let\PY@bf=\textbf\def\PY@tc##1{\textcolor[rgb]{0.00,0.50,0.00}{##1}}}
\expandafter\def\csname PY@tok@kd\endcsname{\let\PY@bf=\textbf\def\PY@tc##1{\textcolor[rgb]{0.00,0.50,0.00}{##1}}}
\expandafter\def\csname PY@tok@kn\endcsname{\let\PY@bf=\textbf\def\PY@tc##1{\textcolor[rgb]{0.00,0.50,0.00}{##1}}}
\expandafter\def\csname PY@tok@kr\endcsname{\let\PY@bf=\textbf\def\PY@tc##1{\textcolor[rgb]{0.00,0.50,0.00}{##1}}}
\expandafter\def\csname PY@tok@bp\endcsname{\def\PY@tc##1{\textcolor[rgb]{0.00,0.50,0.00}{##1}}}
\expandafter\def\csname PY@tok@fm\endcsname{\def\PY@tc##1{\textcolor[rgb]{0.00,0.00,1.00}{##1}}}
\expandafter\def\csname PY@tok@vc\endcsname{\def\PY@tc##1{\textcolor[rgb]{0.10,0.09,0.49}{##1}}}
\expandafter\def\csname PY@tok@vg\endcsname{\def\PY@tc##1{\textcolor[rgb]{0.10,0.09,0.49}{##1}}}
\expandafter\def\csname PY@tok@vi\endcsname{\def\PY@tc##1{\textcolor[rgb]{0.10,0.09,0.49}{##1}}}
\expandafter\def\csname PY@tok@vm\endcsname{\def\PY@tc##1{\textcolor[rgb]{0.10,0.09,0.49}{##1}}}
\expandafter\def\csname PY@tok@sa\endcsname{\def\PY@tc##1{\textcolor[rgb]{0.73,0.13,0.13}{##1}}}
\expandafter\def\csname PY@tok@sb\endcsname{\def\PY@tc##1{\textcolor[rgb]{0.73,0.13,0.13}{##1}}}
\expandafter\def\csname PY@tok@sc\endcsname{\def\PY@tc##1{\textcolor[rgb]{0.73,0.13,0.13}{##1}}}
\expandafter\def\csname PY@tok@dl\endcsname{\def\PY@tc##1{\textcolor[rgb]{0.73,0.13,0.13}{##1}}}
\expandafter\def\csname PY@tok@s2\endcsname{\def\PY@tc##1{\textcolor[rgb]{0.73,0.13,0.13}{##1}}}
\expandafter\def\csname PY@tok@sh\endcsname{\def\PY@tc##1{\textcolor[rgb]{0.73,0.13,0.13}{##1}}}
\expandafter\def\csname PY@tok@s1\endcsname{\def\PY@tc##1{\textcolor[rgb]{0.73,0.13,0.13}{##1}}}
\expandafter\def\csname PY@tok@mb\endcsname{\def\PY@tc##1{\textcolor[rgb]{0.40,0.40,0.40}{##1}}}
\expandafter\def\csname PY@tok@mf\endcsname{\def\PY@tc##1{\textcolor[rgb]{0.40,0.40,0.40}{##1}}}
\expandafter\def\csname PY@tok@mh\endcsname{\def\PY@tc##1{\textcolor[rgb]{0.40,0.40,0.40}{##1}}}
\expandafter\def\csname PY@tok@mi\endcsname{\def\PY@tc##1{\textcolor[rgb]{0.40,0.40,0.40}{##1}}}
\expandafter\def\csname PY@tok@il\endcsname{\def\PY@tc##1{\textcolor[rgb]{0.40,0.40,0.40}{##1}}}
\expandafter\def\csname PY@tok@mo\endcsname{\def\PY@tc##1{\textcolor[rgb]{0.40,0.40,0.40}{##1}}}
\expandafter\def\csname PY@tok@ch\endcsname{\let\PY@it=\textit\def\PY@tc##1{\textcolor[rgb]{0.25,0.50,0.50}{##1}}}
\expandafter\def\csname PY@tok@cm\endcsname{\let\PY@it=\textit\def\PY@tc##1{\textcolor[rgb]{0.25,0.50,0.50}{##1}}}
\expandafter\def\csname PY@tok@cpf\endcsname{\let\PY@it=\textit\def\PY@tc##1{\textcolor[rgb]{0.25,0.50,0.50}{##1}}}
\expandafter\def\csname PY@tok@c1\endcsname{\let\PY@it=\textit\def\PY@tc##1{\textcolor[rgb]{0.25,0.50,0.50}{##1}}}
\expandafter\def\csname PY@tok@cs\endcsname{\let\PY@it=\textit\def\PY@tc##1{\textcolor[rgb]{0.25,0.50,0.50}{##1}}}

\def\PYZbs{\char`\\}
\def\PYZus{\char`\_}
\def\PYZob{\char`\{}
\def\PYZcb{\char`\}}
\def\PYZca{\char`\^}
\def\PYZam{\char`\&}
\def\PYZlt{\char`\<}
\def\PYZgt{\char`\>}
\def\PYZsh{\char`\#}
\def\PYZpc{\char`\%}
\def\PYZdl{\char`\$}
\def\PYZhy{\char`\-}
\def\PYZsq{\char`\'}
\def\PYZdq{\char`\"}
\def\PYZti{\char`\~}
% for compatibility with earlier versions
\def\PYZat{@}
\def\PYZlb{[}
\def\PYZrb{]}
\makeatother


    % Exact colors from NB
    \definecolor{incolor}{rgb}{0.0, 0.0, 0.5}
    \definecolor{outcolor}{rgb}{0.545, 0.0, 0.0}



    
    % Prevent overflowing lines due to hard-to-break entities
    \sloppy 
    % Setup hyperref package
    \hypersetup{
      breaklinks=true,  % so long urls are correctly broken across lines
      colorlinks=true,
      urlcolor=urlcolor,
      linkcolor=linkcolor,
      citecolor=citecolor,
      }
    % Slightly bigger margins than the latex defaults
    
    \geometry{verbose,tmargin=1in,bmargin=1in,lmargin=1in,rmargin=1in}
    
    

    \begin{document}
    
    
    \maketitle
    
    

    
    \hypertarget{monte-carlo-simulation}{%
\section{Monte-Carlo-Simulation}\label{monte-carlo-simulation}}

© Thomas Robert Holy 2019 Version 1.1.2 Visit me on GitHub:
https://github.com/trh0ly \#\# Grundlegende Einstellungen: Zunächst
müssen die notwendigen Pakete (auch Module) importiert werden, damit auf
diese zugegriffen werden kann.

    \begin{Verbatim}[commandchars=\\\{\}]
{\color{incolor}In [{\color{incolor}1}]:} \PY{k+kn}{import} \PY{n+nn}{pandas} \PY{k}{as} \PY{n+nn}{pd} \PY{c+c1}{\PYZsh{} Programmbibliothek die Hilfsmittel für die Verwaltung von Daten und deren Analyse anbietet}
        \PY{k+kn}{import} \PY{n+nn}{scipy}\PY{n+nn}{.}\PY{n+nn}{stats} \PY{k}{as} \PY{n+nn}{st} \PY{c+c1}{\PYZsh{} SciPy ist ein Python\PYZhy{}basiertes Ökosystem für Open\PYZhy{}Source\PYZhy{}Software für Mathematik, Naturwissenschaften und Ingenieurwissenschaften}
        \PY{k+kn}{from} \PY{n+nn}{scipy}\PY{n+nn}{.}\PY{n+nn}{stats} \PY{k}{import} \PY{n}{rankdata}\PY{p}{,} \PY{n}{norm}  
        \PY{k+kn}{from} \PY{n+nn}{scipy} \PY{k}{import} \PY{n}{array}\PY{p}{,} \PY{n}{linalg}\PY{p}{,} \PY{n}{dot}
        \PY{k+kn}{import} \PY{n+nn}{random} \PY{c+c1}{\PYZsh{} Dieses Modul wird verwendet um Zufallszahlen zu ziehen}
        \PY{k+kn}{import} \PY{n+nn}{numpy} \PY{k}{as} \PY{n+nn}{np} \PY{c+c1}{\PYZsh{} Programmbibliothek die eine einfache Handhabung von Vektoren, Matrizen oder generell großen mehrdimensionalen Arrays ermöglicht}
        \PY{k+kn}{import} \PY{n+nn}{math} \PY{c+c1}{\PYZsh{} Dieses Modul wird verwendet um Skalardaten zu berechnen, z. B. trigonometrische Berechnungen.}
        \PY{k+kn}{import} \PY{n+nn}{operator} \PY{c+c1}{\PYZsh{} Programmbibliothek, welche die Ausgaben übersichtlicher gestaltet}
        \PY{k+kn}{import} \PY{n+nn}{matplotlib}\PY{n+nn}{.}\PY{n+nn}{pyplot} \PY{k}{as} \PY{n+nn}{plt} \PY{c+c1}{\PYZsh{} Programmbibliothek die es erlaubt mathematische Darstellungen aller Art anzufertigen}
        \PY{k+kn}{import} \PY{n+nn}{matplotlib}\PY{n+nn}{.}\PY{n+nn}{patches} \PY{k}{as} \PY{n+nn}{mpatches}
        \PY{k+kn}{from} \PY{n+nn}{riskmeasure\PYZus{}module} \PY{k}{import} \PY{n}{risk\PYZus{}measure} \PY{k}{as} \PY{n}{rm}
        \PY{k+kn}{from} \PY{n+nn}{IPython}\PY{n+nn}{.}\PY{n+nn}{core}\PY{n+nn}{.}\PY{n+nn}{display} \PY{k}{import} \PY{n}{display}\PY{p}{,} \PY{n}{HTML}
\end{Verbatim}

    Anschließend werden Einstellungen definiert, die die Formatierung der
Ausgaben betreffen. Hierfür wird das Modul \texttt{operator} genutzt.
Außerdem wird die Größe der Grafiken modifiziert, welche später
angezeigt werden sollen.

    \begin{Verbatim}[commandchars=\\\{\}]
{\color{incolor}In [{\color{incolor}2}]:} \PY{o}{\PYZpc{}}\PY{o}{\PYZpc{}}\PY{n+nx}{javascript}
        \PY{n+nx}{IPython}\PY{p}{.}\PY{n+nx}{OutputArea}\PY{p}{.}\PY{n+nx}{auto\PYZus{}scroll\PYZus{}threshold} \PY{o}{=} \PY{l+m+mi}{9999}\PY{p}{;}
\end{Verbatim}

    
    \begin{verbatim}
<IPython.core.display.Javascript object>
    \end{verbatim}

    
    \begin{Verbatim}[commandchars=\\\{\}]
{\color{incolor}In [{\color{incolor}3}]:} \PY{n}{display}\PY{p}{(}\PY{n}{HTML}\PY{p}{(}\PY{l+s+s2}{\PYZdq{}}\PY{l+s+s2}{\PYZlt{}style\PYZgt{}.container }\PY{l+s+s2}{\PYZob{}}\PY{l+s+s2}{ width:100}\PY{l+s+s2}{\PYZpc{}}\PY{l+s+s2}{ !important; \PYZcb{}\PYZlt{}/style\PYZgt{}}\PY{l+s+s2}{\PYZdq{}}\PY{p}{)}\PY{p}{)}
        \PY{n}{SCREEN\PYZus{}WIDTH} \PY{o}{=} \PY{l+m+mi}{115}
        \PY{n}{centered} \PY{o}{=} \PY{n}{operator}\PY{o}{.}\PY{n}{methodcaller}\PY{p}{(}\PY{l+s+s1}{\PYZsq{}}\PY{l+s+s1}{center}\PY{l+s+s1}{\PYZsq{}}\PY{p}{,} \PY{n}{SCREEN\PYZus{}WIDTH}\PY{p}{)}
        \PY{n}{plt}\PY{o}{.}\PY{n}{rcParams}\PY{p}{[}\PY{l+s+s2}{\PYZdq{}}\PY{l+s+s2}{figure.figsize}\PY{l+s+s2}{\PYZdq{}}\PY{p}{]} \PY{o}{=} \PY{l+m+mi}{15}\PY{p}{,}\PY{l+m+mi}{15}
\end{Verbatim}

    
    \begin{verbatim}
<IPython.core.display.HTML object>
    \end{verbatim}

    
    \hypertarget{monte-carlo-simulation}{%
\subsection{Monte-Carlo Simulation:}\label{monte-carlo-simulation}}

\hypertarget{variablen-spezifizieren}{%
\subsubsection{Variablen spezifizieren}\label{variablen-spezifizieren}}

Im ersten Schritt werden die für die Simulation notwenigen Variablen
definiert.

    \begin{Verbatim}[commandchars=\\\{\}]
{\color{incolor}In [{\color{incolor}4}]:} \PY{c+c1}{\PYZsh{}\PYZsh{}\PYZsh{}\PYZsh{}\PYZsh{}\PYZsh{}\PYZsh{}\PYZsh{}\PYZsh{}\PYZsh{}\PYZsh{}\PYZsh{}\PYZsh{}\PYZsh{}\PYZsh{}\PYZsh{}\PYZsh{}\PYZsh{}\PYZsh{}\PYZsh{}\PYZsh{}\PYZsh{}\PYZsh{}\PYZsh{}\PYZsh{}\PYZsh{}\PYZsh{}\PYZsh{}\PYZsh{}\PYZsh{}\PYZsh{}\PYZsh{}\PYZsh{}\PYZsh{}\PYZsh{}\PYZsh{}\PYZsh{}\PYZsh{}\PYZsh{}\PYZsh{}\PYZsh{}\PYZsh{}\PYZsh{}\PYZsh{}\PYZsh{}\PYZsh{}\PYZsh{}\PYZsh{}\PYZsh{}\PYZsh{}\PYZsh{}\PYZsh{}\PYZsh{}\PYZsh{}\PYZsh{}\PYZsh{}\PYZsh{}\PYZsh{}\PYZsh{}\PYZsh{}\PYZsh{}\PYZsh{}\PYZsh{}\PYZsh{}\PYZsh{}\PYZsh{}\PYZsh{}\PYZsh{}\PYZsh{}\PYZsh{}\PYZsh{}\PYZsh{}\PYZsh{}\PYZsh{}}
        \PY{c+c1}{\PYZsh{}\PYZhy{}\PYZhy{}\PYZhy{}\PYZhy{}\PYZhy{}\PYZhy{}\PYZhy{}\PYZhy{}\PYZhy{}\PYZhy{}\PYZhy{}\PYZhy{}\PYZhy{}\PYZhy{}\PYZhy{}\PYZhy{}\PYZhy{}\PYZhy{}\PYZhy{}\PYZhy{}\PYZhy{}\PYZhy{}\PYZhy{}\PYZhy{}\PYZhy{}\PYZhy{}\PYZhy{}\PYZhy{}\PYZhy{}\PYZhy{}\PYZhy{}\PYZhy{}\PYZhy{}\PYZhy{}\PYZhy{}\PYZhy{}\PYZhy{}\PYZhy{}\PYZhy{}\PYZhy{}\PYZhy{}\PYZhy{}\PYZhy{}\PYZhy{}\PYZhy{}\PYZhy{}\PYZhy{}\PYZhy{}\PYZhy{}\PYZhy{}\PYZhy{}\PYZhy{}\PYZhy{}\PYZhy{}\PYZhy{}\PYZhy{}\PYZhy{}\PYZhy{}\PYZhy{}\PYZhy{}\PYZhy{}\PYZhy{}\PYZhy{}\PYZhy{}\PYZhy{}\PYZhy{}\PYZhy{}\PYZhy{}\PYZhy{}\PYZhy{}\PYZhy{}\PYZhy{}\PYZhy{}}
        \PY{c+c1}{\PYZsh{} Anzahl Simulationsdurchläufe}
        \PY{n}{n} \PY{o}{=} \PY{l+m+mi}{10000}
        
        \PY{c+c1}{\PYZsh{}\PYZhy{}\PYZhy{}\PYZhy{}\PYZhy{}\PYZhy{}\PYZhy{}\PYZhy{}\PYZhy{}\PYZhy{}\PYZhy{}\PYZhy{}\PYZhy{}\PYZhy{}\PYZhy{}\PYZhy{}\PYZhy{}\PYZhy{}\PYZhy{}\PYZhy{}\PYZhy{}\PYZhy{}\PYZhy{}\PYZhy{}\PYZhy{}\PYZhy{}\PYZhy{}\PYZhy{}\PYZhy{}\PYZhy{}\PYZhy{}\PYZhy{}\PYZhy{}\PYZhy{}\PYZhy{}\PYZhy{}\PYZhy{}\PYZhy{}\PYZhy{}\PYZhy{}\PYZhy{}\PYZhy{}\PYZhy{}\PYZhy{}\PYZhy{}\PYZhy{}\PYZhy{}\PYZhy{}\PYZhy{}\PYZhy{}\PYZhy{}\PYZhy{}\PYZhy{}}
        \PY{c+c1}{\PYZsh{} Neue Randverteilungen (Gleichverteilung)}
        \PY{n}{rand\PYZus{}x} \PY{o}{=} \PY{p}{[}\PY{l+m+mi}{10}\PY{p}{,}\PY{l+m+mi}{20}\PY{p}{]}
        \PY{n}{rand\PYZus{}y} \PY{o}{=} \PY{p}{[}\PY{l+m+mi}{8}\PY{p}{,}\PY{l+m+mi}{22}\PY{p}{]}
        
        \PY{c+c1}{\PYZsh{}\PYZhy{}\PYZhy{}\PYZhy{}\PYZhy{}\PYZhy{}\PYZhy{}\PYZhy{}\PYZhy{}\PYZhy{}\PYZhy{}\PYZhy{}\PYZhy{}\PYZhy{}\PYZhy{}\PYZhy{}\PYZhy{}\PYZhy{}\PYZhy{}\PYZhy{}\PYZhy{}\PYZhy{}\PYZhy{}\PYZhy{}\PYZhy{}\PYZhy{}\PYZhy{}\PYZhy{}\PYZhy{}\PYZhy{}\PYZhy{}\PYZhy{}\PYZhy{}\PYZhy{}\PYZhy{}\PYZhy{}\PYZhy{}\PYZhy{}\PYZhy{}\PYZhy{}\PYZhy{}\PYZhy{}\PYZhy{}\PYZhy{}\PYZhy{}\PYZhy{}\PYZhy{}\PYZhy{}\PYZhy{}\PYZhy{}\PYZhy{}\PYZhy{}\PYZhy{}}
        \PY{c+c1}{\PYZsh{} Varianzen und Korrelation(en)}
        \PY{n}{var\PYZus{}x} \PY{o}{=} \PY{l+m+mi}{4} 
        \PY{n}{var\PYZus{}y} \PY{o}{=} \PY{l+m+mi}{9}
        
        \PY{n}{corr\PYZus{}list} \PY{o}{=} \PY{p}{[}\PY{l+m+mi}{0}\PY{p}{]}
        \PY{n}{std\PYZus{}list} \PY{o}{=} \PY{p}{[}\PY{n}{math}\PY{o}{.}\PY{n}{sqrt}\PY{p}{(}\PY{n}{var\PYZus{}x}\PY{p}{)}\PY{p}{,} \PY{n}{math}\PY{o}{.}\PY{n}{sqrt}\PY{p}{(}\PY{n}{var\PYZus{}y}\PY{p}{)}\PY{p}{]}
        
        \PY{c+c1}{\PYZsh{}\PYZhy{}\PYZhy{}\PYZhy{}\PYZhy{}\PYZhy{}\PYZhy{}\PYZhy{}\PYZhy{}\PYZhy{}\PYZhy{}\PYZhy{}\PYZhy{}\PYZhy{}\PYZhy{}\PYZhy{}\PYZhy{}\PYZhy{}\PYZhy{}\PYZhy{}\PYZhy{}\PYZhy{}\PYZhy{}\PYZhy{}\PYZhy{}\PYZhy{}\PYZhy{}\PYZhy{}\PYZhy{}\PYZhy{}\PYZhy{}\PYZhy{}\PYZhy{}\PYZhy{}\PYZhy{}\PYZhy{}\PYZhy{}\PYZhy{}\PYZhy{}\PYZhy{}\PYZhy{}\PYZhy{}\PYZhy{}\PYZhy{}\PYZhy{}\PYZhy{}\PYZhy{}\PYZhy{}\PYZhy{}\PYZhy{}\PYZhy{}\PYZhy{}\PYZhy{}}
        \PY{c+c1}{\PYZsh{} Erwartungswerte}
        \PY{n}{mu} \PY{o}{=} \PY{p}{[}\PY{l+m+mi}{2}\PY{p}{,} \PY{l+m+mi}{3}\PY{p}{]}
        \PY{c+c1}{\PYZsh{}\PYZhy{}\PYZhy{}\PYZhy{}\PYZhy{}\PYZhy{}\PYZhy{}\PYZhy{}\PYZhy{}\PYZhy{}\PYZhy{}\PYZhy{}\PYZhy{}\PYZhy{}\PYZhy{}\PYZhy{}\PYZhy{}\PYZhy{}\PYZhy{}\PYZhy{}\PYZhy{}\PYZhy{}\PYZhy{}\PYZhy{}\PYZhy{}\PYZhy{}\PYZhy{}\PYZhy{}\PYZhy{}\PYZhy{}\PYZhy{}\PYZhy{}\PYZhy{}\PYZhy{}\PYZhy{}\PYZhy{}\PYZhy{}\PYZhy{}\PYZhy{}\PYZhy{}\PYZhy{}\PYZhy{}\PYZhy{}\PYZhy{}\PYZhy{}\PYZhy{}\PYZhy{}\PYZhy{}\PYZhy{}\PYZhy{}\PYZhy{}\PYZhy{}\PYZhy{}\PYZhy{}\PYZhy{}\PYZhy{}\PYZhy{}\PYZhy{}\PYZhy{}\PYZhy{}\PYZhy{}\PYZhy{}\PYZhy{}\PYZhy{}\PYZhy{}\PYZhy{}\PYZhy{}\PYZhy{}\PYZhy{}\PYZhy{}\PYZhy{}\PYZhy{}\PYZhy{}\PYZhy{}}
        \PY{c+c1}{\PYZsh{}\PYZsh{}\PYZsh{}\PYZsh{}\PYZsh{}\PYZsh{}\PYZsh{}\PYZsh{}\PYZsh{}\PYZsh{}\PYZsh{}\PYZsh{}\PYZsh{}\PYZsh{}\PYZsh{}\PYZsh{}\PYZsh{}\PYZsh{}\PYZsh{}\PYZsh{}\PYZsh{}\PYZsh{}\PYZsh{}\PYZsh{}\PYZsh{}\PYZsh{}\PYZsh{}\PYZsh{}\PYZsh{}\PYZsh{}\PYZsh{}\PYZsh{}\PYZsh{}\PYZsh{}\PYZsh{}\PYZsh{}\PYZsh{}\PYZsh{}\PYZsh{}\PYZsh{}\PYZsh{}\PYZsh{}\PYZsh{}\PYZsh{}\PYZsh{}\PYZsh{}\PYZsh{}\PYZsh{}\PYZsh{}\PYZsh{}\PYZsh{}\PYZsh{}\PYZsh{}\PYZsh{}\PYZsh{}\PYZsh{}\PYZsh{}\PYZsh{}\PYZsh{}\PYZsh{}\PYZsh{}\PYZsh{}\PYZsh{}\PYZsh{}\PYZsh{}\PYZsh{}\PYZsh{}\PYZsh{}\PYZsh{}\PYZsh{}\PYZsh{}\PYZsh{}\PYZsh{}\PYZsh{}}
\end{Verbatim}

    \hypertarget{funktionen-definieren}{%
\subsubsection{Funktionen definieren}\label{funktionen-definieren}}

Als nächstes werden Funktionen definiert. Die erste Funktion berechnet
aus den gegeben Varianzen und dazugehörigen Korrelationen die
Varianz-Kovarianz-Matrix, die Zweite legt ein Array mit den gegebenen
Varianzen an und die Dritte führt die Cholesky-Zerlegung auf der
Grundlage Varianz-Kovarianz-Matrix durch. Die weiteren Funktionen
vereinfachen das Plotten.

    \begin{Verbatim}[commandchars=\\\{\}]
{\color{incolor}In [{\color{incolor}5}]:} \PY{c+c1}{\PYZsh{}\PYZhy{}\PYZhy{}\PYZhy{}\PYZhy{}\PYZhy{}\PYZhy{}\PYZhy{}\PYZhy{}\PYZhy{}\PYZhy{}\PYZhy{}\PYZhy{}\PYZhy{}\PYZhy{}\PYZhy{}\PYZhy{}\PYZhy{}\PYZhy{}\PYZhy{}\PYZhy{}\PYZhy{}\PYZhy{}\PYZhy{}\PYZhy{}\PYZhy{}\PYZhy{}\PYZhy{}\PYZhy{}\PYZhy{}\PYZhy{}\PYZhy{}\PYZhy{}\PYZhy{}\PYZhy{}\PYZhy{}\PYZhy{}\PYZhy{}\PYZhy{}\PYZhy{}\PYZhy{}\PYZhy{}\PYZhy{}\PYZhy{}\PYZhy{}\PYZhy{}\PYZhy{}\PYZhy{}\PYZhy{}\PYZhy{}\PYZhy{}\PYZhy{}\PYZhy{}\PYZhy{}\PYZhy{}\PYZhy{}\PYZhy{}\PYZhy{}\PYZhy{}\PYZhy{}\PYZhy{}\PYZhy{}\PYZhy{}\PYZhy{}\PYZhy{}\PYZhy{}\PYZhy{}\PYZhy{}\PYZhy{}\PYZhy{}\PYZhy{}\PYZhy{}\PYZhy{}\PYZhy{}\PYZhy{}\PYZhy{}\PYZhy{}\PYZhy{}\PYZhy{}\PYZhy{}\PYZhy{}\PYZhy{}\PYZhy{}\PYZhy{}\PYZhy{}\PYZhy{}\PYZhy{}\PYZhy{}\PYZhy{}\PYZhy{}\PYZhy{}\PYZhy{}\PYZhy{}\PYZhy{}\PYZhy{}\PYZhy{}\PYZhy{}\PYZhy{}\PYZhy{}\PYZhy{}\PYZhy{}\PYZhy{}\PYZhy{}\PYZhy{}\PYZhy{}\PYZhy{}\PYZhy{}\PYZhy{}\PYZhy{}\PYZhy{}\PYZhy{}\PYZhy{}}
        \PY{c+c1}{\PYZsh{} Definition einer Funktion, welche eine Varianz\PYZhy{}Kovarianz\PYZhy{}Matrix erstellt}
        \PY{c+c1}{\PYZsh{} Argumente:}
        \PY{c+c1}{\PYZsh{} \PYZhy{} std\PYZus{}list: Liste mit Standardabweichungen}
        \PY{c+c1}{\PYZsh{} \PYZhy{} corr\PYZus{}list: Liste mit Korrelationskoeffizienten}
        \PY{c+c1}{\PYZsh{}\PYZhy{}\PYZhy{}\PYZhy{}\PYZhy{}\PYZhy{}\PYZhy{}\PYZhy{}\PYZhy{}\PYZhy{}\PYZhy{}\PYZhy{}\PYZhy{}}
        \PY{k}{def} \PY{n+nf}{var\PYZus{}covar\PYZus{}matrix\PYZus{}func}\PY{p}{(}\PY{n}{std\PYZus{}list}\PY{p}{,} \PY{n}{corr\PYZus{}list}\PY{p}{)}\PY{p}{:}    
            \PY{n}{counter\PYZus{}0}\PY{p}{,} \PY{n}{counter\PYZus{}1} \PY{o}{=} \PY{l+m+mi}{0}\PY{p}{,} \PY{l+m+mi}{0}
            \PY{n}{len\PYZus{}std\PYZus{}list} \PY{o}{=} \PY{n+nb}{len}\PY{p}{(}\PY{n}{std\PYZus{}list}\PY{p}{)}
            \PY{n}{array} \PY{o}{=} \PY{p}{[}\PY{p}{[}\PY{l+m+mi}{0}\PY{p}{]} \PY{o}{*} \PY{n}{len\PYZus{}std\PYZus{}list}\PY{p}{]} \PY{o}{*} \PY{n}{len\PYZus{}std\PYZus{}list}    
            \PY{n}{val\PYZus{}list} \PY{o}{=} \PY{p}{[}\PY{p}{]}
            
            \PY{c+c1}{\PYZsh{} Für jedes i und j in len\PYZus{}std\PYZus{}list..}
            \PY{k}{for} \PY{n}{i} \PY{o+ow}{in} \PY{n+nb}{range}\PY{p}{(}\PY{l+m+mi}{0}\PY{p}{,}\PY{n}{len\PYZus{}std\PYZus{}list}\PY{p}{)}\PY{p}{:}
                \PY{k}{for} \PY{n}{j} \PY{o+ow}{in} \PY{n+nb}{range}\PY{p}{(}\PY{l+m+mi}{0}\PY{p}{,}\PY{n}{len\PYZus{}std\PYZus{}list}\PY{p}{)}\PY{p}{:}
                    \PY{c+c1}{\PYZsh{} Wenn i = j, dann multipliziere beide Werte}
                    \PY{k}{if} \PY{n}{i} \PY{o}{==} \PY{n}{j}\PY{p}{:}
                        \PY{n}{val} \PY{o}{=} \PY{n}{std\PYZus{}list}\PY{p}{[}\PY{n}{i}\PY{p}{]} \PY{o}{*} \PY{n}{std\PYZus{}list}\PY{p}{[}\PY{n}{i}\PY{p}{]}
                        \PY{n}{val\PYZus{}list}\PY{o}{.}\PY{n}{append}\PY{p}{(}\PY{n}{val}\PY{p}{)}
                    \PY{c+c1}{\PYZsh{} Wenn i kleiner j, dann multipliziere beide Standardabweichungen}
                    \PY{c+c1}{\PYZsh{} und die dazugehörige Korrelation}
                    \PY{k}{if} \PY{n}{i} \PY{o}{\PYZlt{}} \PY{n}{j}\PY{p}{:}
                        \PY{n}{val} \PY{o}{=} \PY{p}{(}\PY{n}{std\PYZus{}list}\PY{p}{[}\PY{n}{i}\PY{p}{]} \PY{o}{*} \PY{n}{std\PYZus{}list}\PY{p}{[}\PY{n}{j}\PY{p}{]} \PY{o}{*} \PY{n}{corr\PYZus{}list}\PY{p}{[}\PY{n}{counter\PYZus{}0}\PY{p}{]}\PY{p}{)}
                        \PY{n}{counter\PYZus{}0} \PY{o}{+}\PY{o}{=} \PY{l+m+mi}{1}
                        \PY{n}{val\PYZus{}list}\PY{o}{.}\PY{n}{append}\PY{p}{(}\PY{n}{val}\PY{p}{)}
                    \PY{c+c1}{\PYZsh{} Wenn i größer j, dann multipliziere beide Standardabweichungen}
                    \PY{c+c1}{\PYZsh{} und die dazugehörige Korrelation}
                    \PY{k}{if} \PY{n}{i} \PY{o}{\PYZgt{}} \PY{n}{j}\PY{p}{:}
                        \PY{n}{val} \PY{o}{=} \PY{p}{(}\PY{n}{std\PYZus{}list}\PY{p}{[}\PY{n}{i}\PY{p}{]} \PY{o}{*} \PY{n}{std\PYZus{}list}\PY{p}{[}\PY{n}{j}\PY{p}{]} \PY{o}{*} \PY{n}{corr\PYZus{}list}\PY{p}{[}\PY{n}{counter\PYZus{}1}\PY{p}{]}\PY{p}{)}
                        \PY{n}{counter\PYZus{}1} \PY{o}{+}\PY{o}{=} \PY{l+m+mi}{1}
                        \PY{n}{val\PYZus{}list}\PY{o}{.}\PY{n}{append}\PY{p}{(}\PY{n}{val}\PY{p}{)}
                        
            \PY{n}{var\PYZus{}covar} \PY{o}{=} \PY{n}{np}\PY{o}{.}\PY{n}{array}\PY{p}{(}\PY{n}{val\PYZus{}list}\PY{p}{)}\PY{o}{.}\PY{n}{reshape}\PY{p}{(}\PY{n}{len\PYZus{}std\PYZus{}list}\PY{p}{,} \PY{n}{len\PYZus{}std\PYZus{}list}\PY{p}{)}      
            \PY{k}{return} \PY{n}{var\PYZus{}covar}
            
        \PY{c+c1}{\PYZsh{}\PYZhy{}\PYZhy{}\PYZhy{}\PYZhy{}\PYZhy{}\PYZhy{}\PYZhy{}\PYZhy{}\PYZhy{}\PYZhy{}\PYZhy{}\PYZhy{}\PYZhy{}\PYZhy{}\PYZhy{}\PYZhy{}\PYZhy{}\PYZhy{}\PYZhy{}\PYZhy{}\PYZhy{}\PYZhy{}\PYZhy{}\PYZhy{}\PYZhy{}\PYZhy{}\PYZhy{}\PYZhy{}\PYZhy{}\PYZhy{}\PYZhy{}\PYZhy{}\PYZhy{}\PYZhy{}\PYZhy{}\PYZhy{}\PYZhy{}\PYZhy{}\PYZhy{}\PYZhy{}\PYZhy{}\PYZhy{}\PYZhy{}\PYZhy{}\PYZhy{}\PYZhy{}\PYZhy{}\PYZhy{}\PYZhy{}\PYZhy{}\PYZhy{}\PYZhy{}\PYZhy{}\PYZhy{}\PYZhy{}\PYZhy{}\PYZhy{}\PYZhy{}\PYZhy{}\PYZhy{}\PYZhy{}\PYZhy{}\PYZhy{}\PYZhy{}\PYZhy{}\PYZhy{}\PYZhy{}\PYZhy{}\PYZhy{}\PYZhy{}\PYZhy{}\PYZhy{}\PYZhy{}\PYZhy{}\PYZhy{}\PYZhy{}\PYZhy{}\PYZhy{}\PYZhy{}\PYZhy{}\PYZhy{}\PYZhy{}\PYZhy{}\PYZhy{}\PYZhy{}\PYZhy{}\PYZhy{}\PYZhy{}\PYZhy{}\PYZhy{}\PYZhy{}\PYZhy{}\PYZhy{}\PYZhy{}\PYZhy{}\PYZhy{}\PYZhy{}\PYZhy{}\PYZhy{}\PYZhy{}\PYZhy{}\PYZhy{}\PYZhy{}\PYZhy{}\PYZhy{}\PYZhy{}\PYZhy{}\PYZhy{}\PYZhy{}\PYZhy{}\PYZhy{}}
        \PY{c+c1}{\PYZsh{} Definition einer Funktion, welche ein Varianz\PYZhy{}Array erstellt}
        \PY{c+c1}{\PYZsh{} Argumente:}
        \PY{c+c1}{\PYZsh{} \PYZhy{} std\PYZus{}list: Liste mit Standardabweichungen}
        \PY{c+c1}{\PYZsh{}\PYZhy{}\PYZhy{}\PYZhy{}\PYZhy{}\PYZhy{}\PYZhy{}\PYZhy{}\PYZhy{}\PYZhy{}\PYZhy{}\PYZhy{}\PYZhy{}}
        \PY{k}{def} \PY{n+nf}{var\PYZus{}func}\PY{p}{(}\PY{n}{std\PYZus{}list}\PY{p}{)}\PY{p}{:}    
            \PY{n}{var\PYZus{}list} \PY{o}{=} \PY{p}{[}\PY{p}{]}    
            \PY{k}{for} \PY{n}{i} \PY{o+ow}{in} \PY{n+nb}{range}\PY{p}{(}\PY{l+m+mi}{0}\PY{p}{,} \PY{n+nb}{len}\PY{p}{(}\PY{n}{std\PYZus{}list}\PY{p}{)}\PY{p}{)}\PY{p}{:}
                \PY{n}{var} \PY{o}{=} \PY{n}{np}\PY{o}{.}\PY{n}{power}\PY{p}{(}\PY{n}{std\PYZus{}list}\PY{p}{[}\PY{n}{i}\PY{p}{]}\PY{p}{,}\PY{l+m+mi}{2}\PY{p}{)}
                \PY{n}{var\PYZus{}list}\PY{o}{.}\PY{n}{append}\PY{p}{(}\PY{n}{var}\PY{p}{)}        
            \PY{k}{return} \PY{n}{var\PYZus{}list}
        
        \PY{c+c1}{\PYZsh{}\PYZhy{}\PYZhy{}\PYZhy{}\PYZhy{}\PYZhy{}\PYZhy{}\PYZhy{}\PYZhy{}\PYZhy{}\PYZhy{}\PYZhy{}\PYZhy{}\PYZhy{}\PYZhy{}\PYZhy{}\PYZhy{}\PYZhy{}\PYZhy{}\PYZhy{}\PYZhy{}\PYZhy{}\PYZhy{}\PYZhy{}\PYZhy{}\PYZhy{}\PYZhy{}\PYZhy{}\PYZhy{}\PYZhy{}\PYZhy{}\PYZhy{}\PYZhy{}\PYZhy{}\PYZhy{}\PYZhy{}\PYZhy{}\PYZhy{}\PYZhy{}\PYZhy{}\PYZhy{}\PYZhy{}\PYZhy{}\PYZhy{}\PYZhy{}\PYZhy{}\PYZhy{}\PYZhy{}\PYZhy{}\PYZhy{}\PYZhy{}\PYZhy{}\PYZhy{}\PYZhy{}\PYZhy{}\PYZhy{}\PYZhy{}\PYZhy{}\PYZhy{}\PYZhy{}\PYZhy{}\PYZhy{}\PYZhy{}\PYZhy{}\PYZhy{}\PYZhy{}\PYZhy{}\PYZhy{}\PYZhy{}\PYZhy{}\PYZhy{}\PYZhy{}\PYZhy{}\PYZhy{}\PYZhy{}\PYZhy{}\PYZhy{}\PYZhy{}\PYZhy{}\PYZhy{}\PYZhy{}\PYZhy{}\PYZhy{}\PYZhy{}\PYZhy{}\PYZhy{}\PYZhy{}\PYZhy{}\PYZhy{}\PYZhy{}\PYZhy{}\PYZhy{}\PYZhy{}\PYZhy{}\PYZhy{}\PYZhy{}\PYZhy{}\PYZhy{}\PYZhy{}\PYZhy{}\PYZhy{}\PYZhy{}\PYZhy{}\PYZhy{}\PYZhy{}\PYZhy{}\PYZhy{}\PYZhy{}\PYZhy{}\PYZhy{}\PYZhy{}\PYZhy{}}
        \PY{c+c1}{\PYZsh{} Definition einer Funktion, welche die Cholesky\PYZhy{}Zerlegung durchführt}
        \PY{c+c1}{\PYZsh{} Argumente:}
        \PY{c+c1}{\PYZsh{} \PYZhy{} var\PYZus{}covar: Varianz\PYZhy{}Kovarianz\PYZhy{}Array}
        \PY{c+c1}{\PYZsh{}\PYZhy{}\PYZhy{}\PYZhy{}\PYZhy{}\PYZhy{}\PYZhy{}\PYZhy{}\PYZhy{}\PYZhy{}\PYZhy{}\PYZhy{}\PYZhy{}}
        \PY{k}{def} \PY{n+nf}{cholesky\PYZus{}func}\PY{p}{(}\PY{n}{var\PYZus{}covar}\PY{p}{)}\PY{p}{:}    
            \PY{n}{cholesky} \PY{o}{=} \PY{n}{linalg}\PY{o}{.}\PY{n}{cholesky}\PY{p}{(}\PY{n}{var\PYZus{}covar}\PY{p}{,} \PY{n}{lower}\PY{o}{=}\PY{k+kc}{True}\PY{p}{)}
            \PY{k}{return} \PY{n}{cholesky}
        
        \PY{c+c1}{\PYZsh{}\PYZhy{}\PYZhy{}\PYZhy{}\PYZhy{}\PYZhy{}\PYZhy{}\PYZhy{}\PYZhy{}\PYZhy{}\PYZhy{}\PYZhy{}\PYZhy{}\PYZhy{}\PYZhy{}\PYZhy{}\PYZhy{}\PYZhy{}\PYZhy{}\PYZhy{}\PYZhy{}\PYZhy{}\PYZhy{}\PYZhy{}\PYZhy{}\PYZhy{}\PYZhy{}\PYZhy{}\PYZhy{}\PYZhy{}\PYZhy{}\PYZhy{}\PYZhy{}\PYZhy{}\PYZhy{}\PYZhy{}\PYZhy{}\PYZhy{}\PYZhy{}\PYZhy{}\PYZhy{}\PYZhy{}\PYZhy{}\PYZhy{}\PYZhy{}\PYZhy{}\PYZhy{}\PYZhy{}\PYZhy{}\PYZhy{}\PYZhy{}\PYZhy{}\PYZhy{}\PYZhy{}\PYZhy{}\PYZhy{}\PYZhy{}\PYZhy{}\PYZhy{}\PYZhy{}\PYZhy{}\PYZhy{}\PYZhy{}\PYZhy{}\PYZhy{}\PYZhy{}\PYZhy{}\PYZhy{}\PYZhy{}\PYZhy{}\PYZhy{}\PYZhy{}\PYZhy{}\PYZhy{}\PYZhy{}\PYZhy{}\PYZhy{}\PYZhy{}\PYZhy{}\PYZhy{}\PYZhy{}\PYZhy{}\PYZhy{}\PYZhy{}\PYZhy{}\PYZhy{}\PYZhy{}\PYZhy{}\PYZhy{}\PYZhy{}\PYZhy{}\PYZhy{}\PYZhy{}\PYZhy{}\PYZhy{}\PYZhy{}\PYZhy{}\PYZhy{}\PYZhy{}\PYZhy{}\PYZhy{}\PYZhy{}\PYZhy{}\PYZhy{}\PYZhy{}\PYZhy{}\PYZhy{}\PYZhy{}\PYZhy{}\PYZhy{}\PYZhy{}\PYZhy{}}
        \PY{c+c1}{\PYZsh{} Hilfsfunktionen zum Plotten}
        
        \PY{c+c1}{\PYZsh{}\PYZhy{}\PYZhy{}\PYZhy{}\PYZhy{}\PYZhy{}\PYZhy{}\PYZhy{}\PYZhy{}\PYZhy{}\PYZhy{}\PYZhy{}\PYZhy{}\PYZhy{}\PYZhy{}\PYZhy{}\PYZhy{}\PYZhy{}\PYZhy{}\PYZhy{}\PYZhy{}\PYZhy{}\PYZhy{}\PYZhy{}\PYZhy{}\PYZhy{}\PYZhy{}\PYZhy{}}
        \PY{c+c1}{\PYZsh{} Definition einer Funktion, welche eine gegebene Liste mit 2er\PYZhy{}Tuples in X\PYZhy{} und Y\PYZhy{}Realisationen splittet}
        \PY{c+c1}{\PYZsh{} Argumente:}
        \PY{c+c1}{\PYZsh{} \PYZhy{} liste: Liste mit 2er\PYZhy{}Tuple}
        \PY{c+c1}{\PYZsh{}\PYZhy{}\PYZhy{}\PYZhy{}\PYZhy{}\PYZhy{}\PYZhy{}\PYZhy{}\PYZhy{}\PYZhy{}\PYZhy{}\PYZhy{}\PYZhy{}}
        \PY{k}{def} \PY{n+nf}{split\PYZus{}liste}\PY{p}{(}\PY{n}{liste}\PY{p}{)}\PY{p}{:}     
            \PY{n}{counter} \PY{o}{=} \PY{l+m+mi}{0}
            \PY{n}{x\PYZus{}liste}\PY{p}{,} \PY{n}{y\PYZus{}liste} \PY{o}{=} \PY{p}{[}\PY{p}{]}\PY{p}{,} \PY{p}{[}\PY{p}{]}
            
            \PY{k}{for} \PY{n}{i} \PY{o+ow}{in} \PY{n+nb}{range}\PY{p}{(}\PY{l+m+mi}{0}\PY{p}{,} \PY{n+nb}{len}\PY{p}{(}\PY{n}{liste}\PY{p}{)}\PY{p}{)}\PY{p}{:}
                \PY{n}{x} \PY{o}{=} \PY{n}{liste}\PY{p}{[}\PY{n}{counter}\PY{p}{]}\PY{p}{[}\PY{l+m+mi}{0}\PY{p}{]}
                \PY{n}{y} \PY{o}{=} \PY{n}{liste}\PY{p}{[}\PY{n}{counter}\PY{p}{]}\PY{p}{[}\PY{l+m+mi}{1}\PY{p}{]}
                \PY{n}{x\PYZus{}liste}\PY{o}{.}\PY{n}{append}\PY{p}{(}\PY{n}{x}\PY{p}{)}
                \PY{n}{y\PYZus{}liste}\PY{o}{.}\PY{n}{append}\PY{p}{(}\PY{n}{y}\PY{p}{)}
                \PY{n}{counter} \PY{o}{+}\PY{o}{=} \PY{l+m+mi}{1}        
            \PY{k}{return} \PY{n}{x\PYZus{}liste}\PY{p}{,} \PY{n}{y\PYZus{}liste}
        
        \PY{c+c1}{\PYZsh{}\PYZhy{}\PYZhy{}\PYZhy{}\PYZhy{}\PYZhy{}\PYZhy{}\PYZhy{}\PYZhy{}\PYZhy{}\PYZhy{}\PYZhy{}\PYZhy{}\PYZhy{}\PYZhy{}\PYZhy{}\PYZhy{}\PYZhy{}\PYZhy{}\PYZhy{}\PYZhy{}\PYZhy{}\PYZhy{}\PYZhy{}\PYZhy{}\PYZhy{}\PYZhy{}\PYZhy{}}
        \PY{c+c1}{\PYZsh{} Definition einer Funktion, welche die Grafiken plottet}
        \PY{c+c1}{\PYZsh{} Argumente:}
        \PY{c+c1}{\PYZsh{} \PYZhy{} x\PYZus{}liste: x\PYZhy{}Realisationen}
        \PY{c+c1}{\PYZsh{} \PYZhy{} y\PYZus{}liste:y\PYZhy{}Realisationen}
        \PY{c+c1}{\PYZsh{} \PYZhy{} show:}
        \PY{c+c1}{\PYZsh{} \PYZhy{}\PYZhy{}\PYZgt{} Wenn True: Grafik wird unmittelbar darstellt}
        \PY{c+c1}{\PYZsh{} \PYZhy{}\PYZhy{}\PYZgt{} Wenn False: Grafik wird nicht unmittelbar darstellt}
        \PY{c+c1}{\PYZsh{} \PYZhy{} get\PYZus{}xy\PYZus{}lim:}
        \PY{c+c1}{\PYZsh{} \PYZhy{}\PYZhy{}\PYZgt{} Wenn True: X\PYZhy{} und Y\PYZhy{} Kooridnaten werden ermittelt und die Grafik wird darauf begrenzt}
        \PY{c+c1}{\PYZsh{} \PYZhy{}\PYZhy{}\PYZgt{} Wenn False: Keine Anpassung der Koordinatenachsen}
        \PY{c+c1}{\PYZsh{}\PYZhy{}\PYZhy{}\PYZhy{}\PYZhy{}\PYZhy{}\PYZhy{}\PYZhy{}\PYZhy{}\PYZhy{}\PYZhy{}\PYZhy{}\PYZhy{}}
        \PY{k}{def} \PY{n+nf}{plot\PYZus{}func}\PY{p}{(}\PY{n}{x\PYZus{}liste}\PY{p}{,} \PY{n}{y\PYZus{}liste}\PY{p}{,} \PY{n}{show}\PY{o}{=}\PY{k+kc}{True}\PY{p}{,} \PY{n}{get\PYZus{}xy\PYZus{}lim}\PY{o}{=}\PY{k+kc}{True}\PY{p}{)}\PY{p}{:}
            \PY{n}{plt}\PY{o}{.}\PY{n}{scatter}\PY{p}{(}\PY{n}{x\PYZus{}liste}\PY{p}{,} \PY{n}{y\PYZus{}liste}\PY{p}{)}
            
            \PY{k}{if} \PY{n}{get\PYZus{}xy\PYZus{}lim} \PY{o}{==} \PY{k+kc}{True}\PY{p}{:}    
                \PY{n}{left}\PY{p}{,} \PY{n}{right} \PY{o}{=} \PY{n}{plt}\PY{o}{.}\PY{n}{xlim}\PY{p}{(}\PY{p}{)}
                \PY{n}{plt}\PY{o}{.}\PY{n}{xlim}\PY{p}{(}\PY{p}{(}\PY{n}{left}\PY{p}{,} \PY{n}{right}\PY{p}{)}\PY{p}{)}   
                \PY{n}{plt}\PY{o}{.}\PY{n}{xlim}\PY{p}{(}\PY{n}{left}\PY{p}{,} \PY{n}{right}\PY{p}{)}     
                \PY{n}{left}\PY{p}{,} \PY{n}{right} \PY{o}{=} \PY{n}{plt}\PY{o}{.}\PY{n}{ylim}\PY{p}{(}\PY{p}{)}
                \PY{n}{plt}\PY{o}{.}\PY{n}{ylim}\PY{p}{(}\PY{p}{(}\PY{n}{left}\PY{p}{,} \PY{n}{right}\PY{p}{)}\PY{p}{)} 
                \PY{n}{plt}\PY{o}{.}\PY{n}{ylim}\PY{p}{(}\PY{n}{left}\PY{p}{,} \PY{n}{right}\PY{p}{)}   
                
            \PY{n}{plt}\PY{o}{.}\PY{n}{grid}\PY{p}{(}\PY{p}{)}
            \PY{n}{plt}\PY{o}{.}\PY{n}{xlabel}\PY{p}{(}\PY{l+s+s1}{\PYZsq{}}\PY{l+s+s1}{Realisation X}\PY{l+s+s1}{\PYZsq{}}\PY{p}{)}
            \PY{n}{plt}\PY{o}{.}\PY{n}{ylabel}\PY{p}{(}\PY{l+s+s1}{\PYZsq{}}\PY{l+s+s1}{Realisation Y}\PY{l+s+s1}{\PYZsq{}}\PY{p}{)}
            \PY{n}{plt}\PY{o}{.}\PY{n}{axhline}\PY{p}{(}\PY{l+m+mi}{0}\PY{p}{,} \PY{n}{color}\PY{o}{=}\PY{l+s+s1}{\PYZsq{}}\PY{l+s+s1}{black}\PY{l+s+s1}{\PYZsq{}}\PY{p}{)}
            \PY{n}{plt}\PY{o}{.}\PY{n}{axvline}\PY{p}{(}\PY{l+m+mi}{0}\PY{p}{,} \PY{n}{color}\PY{o}{=}\PY{l+s+s1}{\PYZsq{}}\PY{l+s+s1}{black}\PY{l+s+s1}{\PYZsq{}}\PY{p}{)}
                
            \PY{k}{if} \PY{n}{show} \PY{o}{==} \PY{k+kc}{True}\PY{p}{:}
                \PY{n}{plt}\PY{o}{.}\PY{n}{show}\PY{p}{(}\PY{p}{)}
        
        \PY{c+c1}{\PYZsh{}\PYZhy{}\PYZhy{}\PYZhy{}\PYZhy{}\PYZhy{}\PYZhy{}\PYZhy{}\PYZhy{}\PYZhy{}\PYZhy{}\PYZhy{}\PYZhy{}\PYZhy{}\PYZhy{}\PYZhy{}\PYZhy{}\PYZhy{}\PYZhy{}\PYZhy{}\PYZhy{}\PYZhy{}\PYZhy{}\PYZhy{}\PYZhy{}\PYZhy{}\PYZhy{}\PYZhy{}       }
        \PY{c+c1}{\PYZsh{} Definition einer Funktion, ein Histogramm plottet}
        \PY{c+c1}{\PYZsh{}\PYZhy{}\PYZhy{}\PYZhy{}\PYZhy{}\PYZhy{}\PYZhy{}\PYZhy{}\PYZhy{}\PYZhy{}\PYZhy{}\PYZhy{}\PYZhy{}        }
        \PY{k}{def} \PY{n+nf}{hist\PYZus{}func}\PY{p}{(}\PY{n}{H}\PY{p}{,} \PY{n}{X1}\PY{p}{)}\PY{p}{:}
            \PY{n}{dx} \PY{o}{=} \PY{n}{X1}\PY{p}{[}\PY{l+m+mi}{1}\PY{p}{]} \PY{o}{\PYZhy{}} \PY{n}{X1}\PY{p}{[}\PY{l+m+mi}{0}\PY{p}{]}
            \PY{n}{F1} \PY{o}{=} \PY{n}{np}\PY{o}{.}\PY{n}{cumsum}\PY{p}{(}\PY{n}{H}\PY{p}{)} \PY{o}{*} \PY{n}{dx}
            \PY{n}{plt}\PY{o}{.}\PY{n}{plot}\PY{p}{(}\PY{n}{X1}\PY{p}{[}\PY{l+m+mi}{1}\PY{p}{:}\PY{p}{]}\PY{p}{,} \PY{n}{F1}\PY{p}{)}
        
        \PY{c+c1}{\PYZsh{}\PYZhy{}\PYZhy{}\PYZhy{}\PYZhy{}\PYZhy{}\PYZhy{}\PYZhy{}\PYZhy{}\PYZhy{}\PYZhy{}\PYZhy{}\PYZhy{}\PYZhy{}\PYZhy{}\PYZhy{}\PYZhy{}\PYZhy{}\PYZhy{}\PYZhy{}\PYZhy{}\PYZhy{}\PYZhy{}\PYZhy{}\PYZhy{}\PYZhy{}\PYZhy{}\PYZhy{}}
        \PY{c+c1}{\PYZsh{} Definition einer Funktion, welche eine Verteilungsfunktion plottet}
        \PY{c+c1}{\PYZsh{}\PYZhy{}\PYZhy{}\PYZhy{}\PYZhy{}\PYZhy{}\PYZhy{}\PYZhy{}\PYZhy{}\PYZhy{}\PYZhy{}\PYZhy{}\PYZhy{}    }
        \PY{k}{def} \PY{n+nf}{verteilung\PYZus{}func}\PY{p}{(}\PY{p}{)}\PY{p}{:}
            \PY{n}{plt}\PY{o}{.}\PY{n}{title}\PY{p}{(}\PY{l+s+s1}{\PYZsq{}}\PY{l+s+s1}{Verteilung X+Y, Realisationen gleichverteilte X,Y mit Gauss\PYZhy{}Copula}\PY{l+s+s1}{\PYZsq{}}\PY{p}{)}
            \PY{n}{blue\PYZus{}patch} \PY{o}{=} \PY{n}{mpatches}\PY{o}{.}\PY{n}{Patch}\PY{p}{(}\PY{n}{color}\PY{o}{=}\PY{l+s+s1}{\PYZsq{}}\PY{l+s+s1}{blue}\PY{l+s+s1}{\PYZsq{}}\PY{p}{,} \PY{n}{label}\PY{o}{=}\PY{l+s+s1}{\PYZsq{}}\PY{l+s+s1}{Monte\PYZhy{}Carlo\PYZhy{}Simulation(en)}\PY{l+s+s1}{\PYZsq{}}\PY{p}{)}
            \PY{n}{plt}\PY{o}{.}\PY{n}{legend}\PY{p}{(}\PY{n}{handles}\PY{o}{=}\PY{p}{[}\PY{n}{blue\PYZus{}patch}\PY{p}{]}\PY{p}{,} \PY{n}{loc}\PY{o}{=}\PY{l+s+s1}{\PYZsq{}}\PY{l+s+s1}{upper left}\PY{l+s+s1}{\PYZsq{}}\PY{p}{)}
            \PY{n}{plt}\PY{o}{.}\PY{n}{grid}\PY{p}{(}\PY{p}{)}
            \PY{n}{plt}\PY{o}{.}\PY{n}{xlabel}\PY{p}{(}\PY{l+s+s1}{\PYZsq{}}\PY{l+s+s1}{PF\PYZhy{}Realisation}\PY{l+s+s1}{\PYZsq{}}\PY{p}{)}
            \PY{n}{plt}\PY{o}{.}\PY{n}{ylabel}\PY{p}{(}\PY{l+s+s1}{\PYZsq{}}\PY{l+s+s1}{Wahrscheinlichkeit}\PY{l+s+s1}{\PYZsq{}}\PY{p}{)}
            \PY{n}{left}\PY{p}{,} \PY{n}{right} \PY{o}{=} \PY{n}{plt}\PY{o}{.}\PY{n}{xlim}\PY{p}{(}\PY{p}{)}
            \PY{n}{plt}\PY{o}{.}\PY{n}{xlim}\PY{p}{(}\PY{p}{(}\PY{n}{left}\PY{p}{,} \PY{n}{right}\PY{p}{)}\PY{p}{)}
            \PY{n}{plt}\PY{o}{.}\PY{n}{xlim}\PY{p}{(}\PY{n}{left}\PY{p}{,} \PY{n}{right}\PY{p}{)}
            \PY{n}{plt}\PY{o}{.}\PY{n}{axhline}\PY{p}{(}\PY{l+m+mi}{0}\PY{p}{,} \PY{n}{color}\PY{o}{=}\PY{l+s+s1}{\PYZsq{}}\PY{l+s+s1}{black}\PY{l+s+s1}{\PYZsq{}}\PY{p}{)}
            \PY{n}{plt}\PY{o}{.}\PY{n}{axvline}\PY{p}{(}\PY{l+m+mi}{0}\PY{p}{,} \PY{n}{color}\PY{o}{=}\PY{l+s+s1}{\PYZsq{}}\PY{l+s+s1}{black}\PY{l+s+s1}{\PYZsq{}}\PY{p}{)}
            \PY{n}{plt}\PY{o}{.}\PY{n}{show}\PY{p}{(}\PY{p}{)}
\end{Verbatim}

    \hypertarget{funktion-fuxfcr-die-monte-carlo-simulation-definieren}{%
\subsubsection{Funktion für die Monte-Carlo-Simulation
definieren}\label{funktion-fuxfcr-die-monte-carlo-simulation-definieren}}

Nun wird eine Funktion definiert, in welcher die Simulation durchgeführt
wird. Dabei werden zunächst zwei gleichverteilte Pseudo-Zufallszahlen
generiert, welche anschließend in unabhängige standardnormalverteilte
Zufallszahlen transformiert werden. Letztere werden unter Anwendung der
Cholesky-Zerlegung in abhängige standardnormalverteilte Zufallszahlen
überführt. Danach werden diese unter Anwendung der oben gegebenen
Standardabweichungen und der gegebenen Erwartungswerte in Realisationen
der Gauss-Copula überführt, bevor auf diese anschließend die neuen
Randverteilungen angewendet werden. Zum Schluss wird die Summe aus
beiden Realisationen gebildet, welche den Portfolio-Wert ergibt.

    \begin{Verbatim}[commandchars=\\\{\}]
{\color{incolor}In [{\color{incolor}6}]:} \PY{c+c1}{\PYZsh{} Definition einer Funktion, welche die Monte\PYZhy{}Carlo\PYZhy{}Sumulation durchführt}
        \PY{c+c1}{\PYZsh{} Argumente:}
        \PY{c+c1}{\PYZsh{} \PYZhy{} simulation\PYZus{}runs: Anzahl an Durchläufen der Simulation}
        \PY{c+c1}{\PYZsh{} \PYZhy{} randverteilung\PYZus{}x: Neue Randverteilung der Variablen X}
        \PY{c+c1}{\PYZsh{} \PYZhy{} randverteilung\PYZus{}y: Neue Randverteilung der Variablen Y}
        \PY{c+c1}{\PYZsh{} \PYZhy{} mu\PYZus{}list: Liste mit Erwartungswerten}
        \PY{c+c1}{\PYZsh{} \PYZhy{} std\PYZus{}list: Liste mit Standardabweichungen}
        \PY{c+c1}{\PYZsh{} \PYZhy{} corr\PYZus{}list: Liste mit Korrelationen}
        \PY{c+c1}{\PYZsh{} \PYZhy{} m: Anzahl der Variablen die pro Lauf gezogen werden (hier gerade nur zwei möglich)}
        \PY{c+c1}{\PYZsh{} \PYZhy{} full\PYZus{}log:}
        \PY{c+c1}{\PYZsh{} \PYZhy{}\PYZhy{}\PYZgt{} Wenn True: Alle berechneten Größen werden den Listen angefügt}
        \PY{c+c1}{\PYZsh{} \PYZhy{}\PYZhy{}\PYZgt{} Wenn False: Nur die Summe (das letztendliche Ergebnis der Simulation) der Realisationen wird einer Liste angefügt}
        \PY{c+c1}{\PYZsh{}\PYZhy{}\PYZhy{}\PYZhy{}\PYZhy{}\PYZhy{}\PYZhy{}\PYZhy{}\PYZhy{}\PYZhy{}\PYZhy{}\PYZhy{}\PYZhy{}}
        \PY{k}{def} \PY{n+nf}{copula\PYZus{}sim}\PY{p}{(}\PY{n}{simulation\PYZus{}runs}\PY{p}{,} \PY{n}{randverteilung\PYZus{}x}\PY{p}{,} \PY{n}{randverteilung\PYZus{}y}\PY{p}{,} \PY{n}{mu\PYZus{}list}\PY{p}{,} \PY{n}{std\PYZus{}list}\PY{p}{,} \PY{n}{corr\PYZus{}list}\PY{p}{,} \PY{n}{m}\PY{o}{=}\PY{l+m+mi}{2}\PY{p}{,} \PY{n}{full\PYZus{}log}\PY{o}{=}\PY{k+kc}{False}\PY{p}{)}\PY{p}{:}
            
            \PY{c+c1}{\PYZsh{} Listen für die zu berechnenden Größen}
            \PY{n}{total\PYZus{}standard\PYZus{}norm\PYZus{}ab\PYZus{}list}\PY{p}{,} \PY{n}{total\PYZus{}realisation\PYZus{}cop\PYZus{}list} \PY{o}{=} \PY{p}{[}\PY{p}{]}\PY{p}{,} \PY{p}{[}\PY{p}{]}
            \PY{n}{total\PYZus{}xy\PYZus{}list}\PY{p}{,} \PY{n}{total\PYZus{}summe\PYZus{}liste} \PY{o}{=} \PY{p}{[}\PY{p}{]}\PY{p}{,} \PY{p}{[}\PY{p}{]}
            
            \PY{c+c1}{\PYZsh{} Funktionen aufrufen um die für die folgenden Berechnungen nötige Werte zu erhalten}
            \PY{n}{var\PYZus{}covar} \PY{o}{=} \PY{n}{var\PYZus{}covar\PYZus{}matrix\PYZus{}func}\PY{p}{(}\PY{n}{std\PYZus{}list}\PY{p}{,} \PY{n}{corr\PYZus{}list}\PY{p}{)}
            \PY{n}{var\PYZus{}list} \PY{o}{=} \PY{n}{var\PYZus{}func}\PY{p}{(}\PY{n}{std\PYZus{}list}\PY{p}{)}
            \PY{n}{cholesky} \PY{o}{=} \PY{n}{cholesky\PYZus{}func}\PY{p}{(}\PY{n}{var\PYZus{}covar}\PY{p}{)}
            
            \PY{c+c1}{\PYZsh{} Durchführung der n Durchläufe}
            \PY{k}{for} \PY{n}{i} \PY{o+ow}{in} \PY{n+nb}{range}\PY{p}{(}\PY{l+m+mi}{0}\PY{p}{,} \PY{n}{simulation\PYZus{}runs}\PY{p}{)}\PY{p}{:}        
                \PY{c+c1}{\PYZsh{}\PYZhy{}\PYZhy{}\PYZhy{}\PYZhy{}\PYZhy{}\PYZhy{}\PYZhy{}\PYZhy{}\PYZhy{}\PYZhy{}\PYZhy{}\PYZhy{}\PYZhy{}\PYZhy{}\PYZhy{}\PYZhy{}\PYZhy{}\PYZhy{}\PYZhy{}\PYZhy{}\PYZhy{}\PYZhy{}\PYZhy{}\PYZhy{}\PYZhy{}\PYZhy{}\PYZhy{}\PYZhy{}\PYZhy{}\PYZhy{}\PYZhy{}\PYZhy{}\PYZhy{}\PYZhy{}\PYZhy{}\PYZhy{}\PYZhy{}\PYZhy{}\PYZhy{}\PYZhy{}\PYZhy{}\PYZhy{}\PYZhy{}\PYZhy{}\PYZhy{}\PYZhy{}\PYZhy{}\PYZhy{}\PYZhy{}\PYZhy{}\PYZhy{}\PYZhy{}\PYZhy{}\PYZhy{}\PYZhy{}\PYZhy{}\PYZhy{}\PYZhy{}\PYZhy{}\PYZhy{}\PYZhy{}\PYZhy{}\PYZhy{}\PYZhy{}\PYZhy{}\PYZhy{}\PYZhy{}\PYZhy{}\PYZhy{}\PYZhy{}\PYZhy{}\PYZhy{}\PYZhy{}\PYZhy{}\PYZhy{}\PYZhy{}\PYZhy{}\PYZhy{}\PYZhy{}\PYZhy{}\PYZhy{}\PYZhy{}\PYZhy{}\PYZhy{}\PYZhy{}\PYZhy{}\PYZhy{}\PYZhy{}\PYZhy{}\PYZhy{}\PYZhy{}\PYZhy{}\PYZhy{}\PYZhy{}\PYZhy{}\PYZhy{}\PYZhy{}\PYZhy{}\PYZhy{}\PYZhy{}\PYZhy{}\PYZhy{}\PYZhy{}\PYZhy{}\PYZhy{}\PYZhy{}\PYZhy{}\PYZhy{}\PYZhy{}\PYZhy{}\PYZhy{}}
                \PY{c+c1}{\PYZsh{} Gleichverteilte Zufallszahlen ziehen        }
                \PY{n}{random\PYZus{}ZV\PYZus{}list} \PY{o}{=} \PY{p}{[}\PY{p}{]}  
                \PY{k}{for} \PY{n}{i} \PY{o+ow}{in} \PY{n+nb}{range}\PY{p}{(}\PY{l+m+mi}{0}\PY{p}{,} \PY{n}{m}\PY{p}{)}\PY{p}{:}
                    \PY{n}{x} \PY{o}{=} \PY{n}{random}\PY{o}{.}\PY{n}{random}\PY{p}{(}\PY{p}{)}
                    \PY{n}{random\PYZus{}ZV\PYZus{}list}\PY{o}{.}\PY{n}{append}\PY{p}{(}\PY{n}{x}\PY{p}{)}
                \PY{k}{if} \PY{n}{simulation\PYZus{}runs} \PY{o}{==} \PY{l+m+mi}{1}\PY{p}{:}
                    \PY{n+nb}{print}\PY{p}{(}\PY{l+s+s1}{\PYZsq{}}\PY{l+s+s1}{1) Gleichverteilte Zufallszahlen: }\PY{l+s+si}{\PYZob{}\PYZcb{}}\PY{l+s+se}{\PYZbs{}n}\PY{l+s+s1}{\PYZsq{}}\PY{o}{.}\PY{n}{format}\PY{p}{(}\PY{n}{random\PYZus{}ZV\PYZus{}list}\PY{p}{)}\PY{p}{)}
                
                \PY{c+c1}{\PYZsh{}\PYZhy{}\PYZhy{}\PYZhy{}\PYZhy{}\PYZhy{}\PYZhy{}\PYZhy{}\PYZhy{}\PYZhy{}\PYZhy{}\PYZhy{}\PYZhy{}\PYZhy{}\PYZhy{}\PYZhy{}\PYZhy{}\PYZhy{}\PYZhy{}\PYZhy{}\PYZhy{}\PYZhy{}\PYZhy{}\PYZhy{}\PYZhy{}\PYZhy{}\PYZhy{}\PYZhy{}\PYZhy{}\PYZhy{}\PYZhy{}\PYZhy{}\PYZhy{}\PYZhy{}\PYZhy{}\PYZhy{}\PYZhy{}\PYZhy{}\PYZhy{}\PYZhy{}\PYZhy{}\PYZhy{}\PYZhy{}\PYZhy{}\PYZhy{}\PYZhy{}\PYZhy{}\PYZhy{}\PYZhy{}\PYZhy{}\PYZhy{}\PYZhy{}\PYZhy{}\PYZhy{}\PYZhy{}\PYZhy{}\PYZhy{}\PYZhy{}\PYZhy{}\PYZhy{}\PYZhy{}\PYZhy{}\PYZhy{}\PYZhy{}\PYZhy{}\PYZhy{}\PYZhy{}\PYZhy{}\PYZhy{}\PYZhy{}\PYZhy{}\PYZhy{}\PYZhy{}\PYZhy{}\PYZhy{}\PYZhy{}\PYZhy{}\PYZhy{}\PYZhy{}\PYZhy{}\PYZhy{}\PYZhy{}\PYZhy{}\PYZhy{}\PYZhy{}\PYZhy{}\PYZhy{}\PYZhy{}\PYZhy{}\PYZhy{}\PYZhy{}\PYZhy{}\PYZhy{}\PYZhy{}\PYZhy{}\PYZhy{}\PYZhy{}\PYZhy{}\PYZhy{}\PYZhy{}\PYZhy{}\PYZhy{}\PYZhy{}\PYZhy{}\PYZhy{}\PYZhy{}\PYZhy{}\PYZhy{}\PYZhy{}\PYZhy{}\PYZhy{}\PYZhy{}}
                \PY{c+c1}{\PYZsh{} Transformation der gleichverteilten Zufallszahlen in unabhängige standardnormalverteilte Zufallszahlen}
                \PY{n}{standard\PYZus{}norm\PYZus{}list} \PY{o}{=} \PY{n}{norm}\PY{o}{.}\PY{n}{ppf}\PY{p}{(}\PY{n}{random\PYZus{}ZV\PYZus{}list}\PY{p}{)}   
                \PY{k}{if} \PY{n}{simulation\PYZus{}runs} \PY{o}{==} \PY{l+m+mi}{1}\PY{p}{:}
                    \PY{n+nb}{print}\PY{p}{(}\PY{l+s+s1}{\PYZsq{}}\PY{l+s+s1}{2) Standardnormalverteilte Zufallszahlen: }\PY{l+s+si}{\PYZob{}\PYZcb{}}\PY{l+s+se}{\PYZbs{}n}\PY{l+s+s1}{\PYZsq{}}\PY{o}{.}\PY{n}{format}\PY{p}{(}\PY{n}{standard\PYZus{}norm\PYZus{}list}\PY{p}{)}\PY{p}{)}
                
                \PY{c+c1}{\PYZsh{}\PYZhy{}\PYZhy{}\PYZhy{}\PYZhy{}\PYZhy{}\PYZhy{}\PYZhy{}\PYZhy{}\PYZhy{}\PYZhy{}\PYZhy{}\PYZhy{}\PYZhy{}\PYZhy{}\PYZhy{}\PYZhy{}\PYZhy{}\PYZhy{}\PYZhy{}\PYZhy{}\PYZhy{}\PYZhy{}\PYZhy{}\PYZhy{}\PYZhy{}\PYZhy{}\PYZhy{}\PYZhy{}\PYZhy{}\PYZhy{}\PYZhy{}\PYZhy{}\PYZhy{}\PYZhy{}\PYZhy{}\PYZhy{}\PYZhy{}\PYZhy{}\PYZhy{}\PYZhy{}\PYZhy{}\PYZhy{}\PYZhy{}\PYZhy{}\PYZhy{}\PYZhy{}\PYZhy{}\PYZhy{}\PYZhy{}\PYZhy{}\PYZhy{}\PYZhy{}\PYZhy{}\PYZhy{}\PYZhy{}\PYZhy{}\PYZhy{}\PYZhy{}\PYZhy{}\PYZhy{}\PYZhy{}\PYZhy{}\PYZhy{}\PYZhy{}\PYZhy{}\PYZhy{}\PYZhy{}\PYZhy{}\PYZhy{}\PYZhy{}\PYZhy{}\PYZhy{}\PYZhy{}\PYZhy{}\PYZhy{}\PYZhy{}\PYZhy{}\PYZhy{}\PYZhy{}\PYZhy{}\PYZhy{}\PYZhy{}\PYZhy{}\PYZhy{}\PYZhy{}\PYZhy{}\PYZhy{}\PYZhy{}\PYZhy{}\PYZhy{}\PYZhy{}\PYZhy{}\PYZhy{}\PYZhy{}\PYZhy{}\PYZhy{}\PYZhy{}\PYZhy{}\PYZhy{}\PYZhy{}\PYZhy{}\PYZhy{}\PYZhy{}\PYZhy{}\PYZhy{}\PYZhy{}\PYZhy{}\PYZhy{}\PYZhy{}\PYZhy{}\PYZhy{}}
                \PY{c+c1}{\PYZsh{} Transformation in standardnormalverteilte abhängige Zufallszahlen        }
                \PY{n}{standard\PYZus{}norm\PYZus{}ab\PYZus{}list} \PY{o}{=} \PY{p}{[}\PY{p}{]}
                \PY{n}{counter\PYZus{}cholesky\PYZus{}0}\PY{p}{,} \PY{n}{counter\PYZus{}cholesky\PYZus{}1} \PY{o}{=} \PY{l+m+mi}{0}\PY{p}{,} \PY{l+m+mi}{0}
                \PY{n}{counter\PYZus{}mu}\PY{p}{,} \PY{n}{counter\PYZus{}standard\PYZus{}norm\PYZus{}list} \PY{o}{=} \PY{l+m+mi}{0}\PY{p}{,} \PY{l+m+mi}{0}
        
                \PY{c+c1}{\PYZsh{} Für jede zuvor gezogene unabhängige, standardnormalverteilte Zufallszahlen }
                \PY{c+c1}{\PYZsh{} wird die Abhänigkeitsstruktur (aus der Cholesky\PYZhy{}Matrix) auf diese Zufallszahlen übertragen}
                \PY{k}{for} \PY{n}{i} \PY{o+ow}{in} \PY{n+nb}{range}\PY{p}{(}\PY{l+m+mi}{0}\PY{p}{,} \PY{n}{m}\PY{p}{)}\PY{p}{:}
                    \PY{c+c1}{\PYZsh{} 1. Schritt: Berechnung a, mit a = 1. unabhängige standardnormalverteilte Zufallsvariable * Eintrag in Cholesky\PYZhy{}Matrix}
                    \PY{n}{a} \PY{o}{=} \PY{n}{cholesky}\PY{p}{[}\PY{n}{counter\PYZus{}cholesky\PYZus{}0}\PY{p}{]}\PY{p}{[}\PY{n}{counter\PYZus{}cholesky\PYZus{}1}\PY{p}{]} \PY{o}{*} \PY{n}{standard\PYZus{}norm\PYZus{}list}\PY{p}{[}\PY{n}{counter\PYZus{}standard\PYZus{}norm\PYZus{}list}\PY{p}{]}
                    \PY{n}{counter\PYZus{}cholesky\PYZus{}1} \PY{o}{+}\PY{o}{=} \PY{l+m+mi}{1}
                    \PY{n}{counter\PYZus{}standard\PYZus{}norm\PYZus{}list} \PY{o}{+}\PY{o}{=} \PY{l+m+mi}{1}
        
                    \PY{c+c1}{\PYZsh{} 2. Schritt: Berechnung b, mit b = 2. unabhängige standardnormalverteilte Zufallsvariable * Eintrag in Cholesky\PYZhy{}Matrix}
                    \PY{n}{b} \PY{o}{=} \PY{n}{cholesky}\PY{p}{[}\PY{n}{counter\PYZus{}cholesky\PYZus{}0}\PY{p}{]}\PY{p}{[}\PY{n}{counter\PYZus{}cholesky\PYZus{}1}\PY{p}{]} \PY{o}{*} \PY{n}{standard\PYZus{}norm\PYZus{}list}\PY{p}{[}\PY{n}{counter\PYZus{}standard\PYZus{}norm\PYZus{}list}\PY{p}{]}
                    \PY{n}{counter\PYZus{}cholesky\PYZus{}0} \PY{o}{+}\PY{o}{=} \PY{l+m+mi}{1}  
                    \PY{n}{counter\PYZus{}cholesky\PYZus{}1} \PY{o}{=} \PY{l+m+mi}{0}
                    \PY{n}{counter\PYZus{}standard\PYZus{}norm\PYZus{}list} \PY{o}{=} \PY{l+m+mi}{0}  
        
                    \PY{c+c1}{\PYZsh{} 3. Schritt: Endergebnis r ist der Erwartungswert + a + b und stellt die abhängige standardnormalverteilte Zufallsvariable dar}
                    \PY{n}{r} \PY{o}{=} \PY{n}{a} \PY{o}{+} \PY{n}{b} \PY{o}{+} \PY{n}{mu\PYZus{}list}\PY{p}{[}\PY{n}{counter\PYZus{}mu}\PY{p}{]}
                    \PY{n}{standard\PYZus{}norm\PYZus{}ab\PYZus{}list}\PY{o}{.}\PY{n}{append}\PY{p}{(}\PY{n}{r}\PY{p}{)}
                    \PY{n}{counter\PYZus{}mu} \PY{o}{+}\PY{o}{=} \PY{l+m+mi}{1}
                \PY{k}{if} \PY{n}{simulation\PYZus{}runs} \PY{o}{==} \PY{l+m+mi}{1}\PY{p}{:}
                    \PY{n+nb}{print}\PY{p}{(}\PY{l+s+s1}{\PYZsq{}}\PY{l+s+s1}{3) Standardnormalverteilte abhängige Zufallszahlen: }\PY{l+s+si}{\PYZob{}\PYZcb{}}\PY{l+s+se}{\PYZbs{}n}\PY{l+s+s1}{\PYZsq{}}\PY{o}{.}\PY{n}{format}\PY{p}{(}\PY{n}{standard\PYZus{}norm\PYZus{}ab\PYZus{}list}\PY{p}{)}\PY{p}{)}
        
                \PY{c+c1}{\PYZsh{}\PYZhy{}\PYZhy{}\PYZhy{}\PYZhy{}\PYZhy{}\PYZhy{}\PYZhy{}\PYZhy{}\PYZhy{}\PYZhy{}\PYZhy{}\PYZhy{}\PYZhy{}\PYZhy{}\PYZhy{}\PYZhy{}\PYZhy{}\PYZhy{}\PYZhy{}\PYZhy{}\PYZhy{}\PYZhy{}\PYZhy{}\PYZhy{}\PYZhy{}\PYZhy{}\PYZhy{}\PYZhy{}\PYZhy{}\PYZhy{}\PYZhy{}\PYZhy{}\PYZhy{}\PYZhy{}\PYZhy{}\PYZhy{}\PYZhy{}\PYZhy{}\PYZhy{}\PYZhy{}\PYZhy{}\PYZhy{}\PYZhy{}\PYZhy{}\PYZhy{}\PYZhy{}\PYZhy{}\PYZhy{}\PYZhy{}\PYZhy{}\PYZhy{}\PYZhy{}\PYZhy{}\PYZhy{}\PYZhy{}\PYZhy{}\PYZhy{}\PYZhy{}\PYZhy{}\PYZhy{}\PYZhy{}\PYZhy{}\PYZhy{}\PYZhy{}\PYZhy{}\PYZhy{}\PYZhy{}\PYZhy{}\PYZhy{}\PYZhy{}\PYZhy{}\PYZhy{}\PYZhy{}\PYZhy{}\PYZhy{}\PYZhy{}\PYZhy{}\PYZhy{}\PYZhy{}\PYZhy{}\PYZhy{}\PYZhy{}\PYZhy{}\PYZhy{}\PYZhy{}\PYZhy{}\PYZhy{}\PYZhy{}\PYZhy{}\PYZhy{}\PYZhy{}\PYZhy{}\PYZhy{}\PYZhy{}\PYZhy{}\PYZhy{}\PYZhy{}\PYZhy{}\PYZhy{}\PYZhy{}\PYZhy{}\PYZhy{}\PYZhy{}\PYZhy{}\PYZhy{}\PYZhy{}\PYZhy{}\PYZhy{}\PYZhy{}\PYZhy{}\PYZhy{}}
                \PY{c+c1}{\PYZsh{} Transformation Realisationen der Gauss\PYZhy{}Copula}
                \PY{n}{counter} \PY{o}{=} \PY{l+m+mi}{0}
                \PY{n}{realisation\PYZus{}cop\PYZus{}list} \PY{o}{=} \PY{p}{[}\PY{p}{]}     
                \PY{c+c1}{\PYZsh{} Transformation indem von jeweiliger abhängiger standardnormalverteilter Zufallszahlen}
                \PY{c+c1}{\PYZsh{} der dazugehörige Erwartungswert subtrahiert wird, und dieses Ergebnis dann durch die dazugehörige Standardabweichung geteilt wird}
                \PY{c+c1}{\PYZsh{} Die Realisation der Gauss\PYZhy{}Copula ergibt sich dann als der Wert (Wahrscheinlichkeit) der kumulierten Normalverteilungsfunktion}
                \PY{k}{for} \PY{n}{i} \PY{o+ow}{in} \PY{n+nb}{range}\PY{p}{(}\PY{l+m+mi}{0}\PY{p}{,} \PY{n}{m}\PY{p}{)}\PY{p}{:}
                    \PY{n}{r\PYZus{}cop} \PY{o}{=} \PY{p}{(}\PY{n}{standard\PYZus{}norm\PYZus{}ab\PYZus{}list}\PY{p}{[}\PY{n}{counter}\PY{p}{]} \PY{o}{\PYZhy{}} \PY{n}{mu\PYZus{}list}\PY{p}{[}\PY{n}{counter}\PY{p}{]}\PY{p}{)} \PY{o}{/} \PY{n}{math}\PY{o}{.}\PY{n}{sqrt}\PY{p}{(}\PY{n}{var\PYZus{}list}\PY{p}{[}\PY{n}{counter}\PY{p}{]}\PY{p}{)}   
                    \PY{n}{p\PYZus{}value} \PY{o}{=} \PY{n}{st}\PY{o}{.}\PY{n}{norm}\PY{o}{.}\PY{n}{cdf}\PY{p}{(}\PY{n}{r\PYZus{}cop}\PY{p}{)}
                    \PY{n}{realisation\PYZus{}cop\PYZus{}list}\PY{o}{.}\PY{n}{append}\PY{p}{(}\PY{n}{p\PYZus{}value}\PY{p}{)}
                    \PY{n}{counter} \PY{o}{+}\PY{o}{=} \PY{l+m+mi}{1}
                \PY{k}{if} \PY{n}{simulation\PYZus{}runs} \PY{o}{==} \PY{l+m+mi}{1}\PY{p}{:}
                    \PY{n+nb}{print}\PY{p}{(}\PY{l+s+s1}{\PYZsq{}}\PY{l+s+s1}{4) Realisationen der Gauss\PYZhy{}Copula: }\PY{l+s+si}{\PYZob{}\PYZcb{}}\PY{l+s+se}{\PYZbs{}n}\PY{l+s+s1}{\PYZsq{}}\PY{o}{.}\PY{n}{format}\PY{p}{(}\PY{n}{realisation\PYZus{}cop\PYZus{}list}\PY{p}{)}\PY{p}{)}
                    
                \PY{c+c1}{\PYZsh{}\PYZhy{}\PYZhy{}\PYZhy{}\PYZhy{}\PYZhy{}\PYZhy{}\PYZhy{}\PYZhy{}\PYZhy{}\PYZhy{}\PYZhy{}\PYZhy{}\PYZhy{}\PYZhy{}\PYZhy{}\PYZhy{}\PYZhy{}\PYZhy{}\PYZhy{}\PYZhy{}\PYZhy{}\PYZhy{}\PYZhy{}\PYZhy{}\PYZhy{}\PYZhy{}\PYZhy{}\PYZhy{}\PYZhy{}\PYZhy{}\PYZhy{}\PYZhy{}\PYZhy{}\PYZhy{}\PYZhy{}\PYZhy{}\PYZhy{}\PYZhy{}\PYZhy{}\PYZhy{}\PYZhy{}\PYZhy{}\PYZhy{}\PYZhy{}\PYZhy{}\PYZhy{}\PYZhy{}\PYZhy{}\PYZhy{}\PYZhy{}\PYZhy{}\PYZhy{}\PYZhy{}\PYZhy{}\PYZhy{}\PYZhy{}\PYZhy{}\PYZhy{}\PYZhy{}\PYZhy{}\PYZhy{}\PYZhy{}\PYZhy{}\PYZhy{}\PYZhy{}\PYZhy{}\PYZhy{}\PYZhy{}\PYZhy{}\PYZhy{}\PYZhy{}\PYZhy{}\PYZhy{}\PYZhy{}\PYZhy{}\PYZhy{}\PYZhy{}\PYZhy{}\PYZhy{}\PYZhy{}\PYZhy{}\PYZhy{}\PYZhy{}\PYZhy{}\PYZhy{}\PYZhy{}\PYZhy{}\PYZhy{}\PYZhy{}\PYZhy{}\PYZhy{}\PYZhy{}\PYZhy{}\PYZhy{}\PYZhy{}\PYZhy{}\PYZhy{}\PYZhy{}\PYZhy{}\PYZhy{}\PYZhy{}\PYZhy{}\PYZhy{}\PYZhy{}\PYZhy{}\PYZhy{}\PYZhy{}\PYZhy{}\PYZhy{}\PYZhy{}\PYZhy{}}
                \PY{c+c1}{\PYZsh{} Gemeinsame Verteilung: Übertragung der neuen Ränder auf die ermittelte Abhängigkeitsstruktur  }
                \PY{c+c1}{\PYZsh{} Für x und y wird jeweils (Obergrenze \PYZhy{} Untergrenze) * Realisation der Gauss\PYZhy{}Copula + Untergrenze gerechnet}
                \PY{c+c1}{\PYZsh{} Dabei sind Obergrenze bzw. Untergrenze der linke bzw. rechte Rand des Intervalls der neuen Randverteilungen}
                \PY{n}{x} \PY{o}{=} \PY{p}{(}\PY{n}{randverteilung\PYZus{}x}\PY{p}{[}\PY{l+m+mi}{0}\PY{p}{]} \PY{o}{\PYZhy{}} \PY{n}{randverteilung\PYZus{}x}\PY{p}{[}\PY{l+m+mi}{1}\PY{p}{]}\PY{p}{)} \PY{o}{*} \PY{n}{realisation\PYZus{}cop\PYZus{}list}\PY{p}{[}\PY{l+m+mi}{0}\PY{p}{]} \PY{o}{+} \PY{n}{randverteilung\PYZus{}x}\PY{p}{[}\PY{l+m+mi}{1}\PY{p}{]}
                \PY{n}{y} \PY{o}{=} \PY{p}{(}\PY{n}{randverteilung\PYZus{}y}\PY{p}{[}\PY{l+m+mi}{0}\PY{p}{]} \PY{o}{\PYZhy{}} \PY{n}{randverteilung\PYZus{}y}\PY{p}{[}\PY{l+m+mi}{1}\PY{p}{]}\PY{p}{)} \PY{o}{*} \PY{n}{realisation\PYZus{}cop\PYZus{}list}\PY{p}{[}\PY{l+m+mi}{1}\PY{p}{]} \PY{o}{+} \PY{n}{randverteilung\PYZus{}y}\PY{p}{[}\PY{l+m+mi}{1}\PY{p}{]}
                
                \PY{c+c1}{\PYZsh{} Die Summe bzw. das Endergebnis der Simulation ist dann X + Y, also die gemeinsame Realisation}
                \PY{n}{summe} \PY{o}{=} \PY{n}{x} \PY{o}{+} \PY{n}{y} 
                \PY{k}{if} \PY{n}{simulation\PYZus{}runs} \PY{o}{==} \PY{l+m+mi}{1}\PY{p}{:}
                    \PY{n+nb}{print}\PY{p}{(}\PY{l+s+s1}{\PYZsq{}}\PY{l+s+s1}{5) Sumulationsergebnisse: x=}\PY{l+s+si}{\PYZob{}\PYZcb{}}\PY{l+s+s1}{, y=}\PY{l+s+si}{\PYZob{}\PYZcb{}}\PY{l+s+s1}{, Summe=}\PY{l+s+si}{\PYZob{}\PYZcb{}}\PY{l+s+s1}{\PYZsq{}}\PY{o}{.}\PY{n}{format}\PY{p}{(}\PY{n}{x}\PY{p}{,} \PY{n}{y}\PY{p}{,} \PY{n}{summe}\PY{p}{)}\PY{p}{)}
                
                \PY{c+c1}{\PYZsh{} Berechnete Werte werden den jeweiligen Listen angefügt}
                \PY{k}{if} \PY{n}{full\PYZus{}log} \PY{o}{==} \PY{k+kc}{True}\PY{p}{:}
                    \PY{n}{total\PYZus{}standard\PYZus{}norm\PYZus{}ab\PYZus{}list}\PY{o}{.}\PY{n}{append}\PY{p}{(}\PY{n}{standard\PYZus{}norm\PYZus{}ab\PYZus{}list}\PY{p}{)}
                    \PY{n}{total\PYZus{}realisation\PYZus{}cop\PYZus{}list}\PY{o}{.}\PY{n}{append}\PY{p}{(}\PY{n}{realisation\PYZus{}cop\PYZus{}list}\PY{p}{)}
                    \PY{n}{total\PYZus{}xy\PYZus{}list}\PY{o}{.}\PY{n}{append}\PY{p}{(}\PY{p}{(}\PY{n}{x}\PY{p}{,} \PY{n}{y}\PY{p}{)}\PY{p}{)}
                \PY{n}{total\PYZus{}summe\PYZus{}liste}\PY{o}{.}\PY{n}{append}\PY{p}{(}\PY{n}{summe}\PY{p}{)}        
          
            \PY{k}{return} \PY{n}{total\PYZus{}standard\PYZus{}norm\PYZus{}ab\PYZus{}list}\PY{p}{,} \PY{n}{total\PYZus{}realisation\PYZus{}cop\PYZus{}list}\PY{p}{,} \PY{n}{total\PYZus{}xy\PYZus{}list}\PY{p}{,} \PY{n}{total\PYZus{}summe\PYZus{}liste}
\end{Verbatim}

    \begin{Verbatim}[commandchars=\\\{\}]
{\color{incolor}In [{\color{incolor}7}]:} \PY{n}{\PYZus{}}\PY{p}{,} \PY{n}{\PYZus{}}\PY{p}{,} \PY{n}{\PYZus{}}\PY{p}{,} \PY{n}{\PYZus{}} \PY{o}{=} \PY{n}{copula\PYZus{}sim}\PY{p}{(}\PY{l+m+mi}{1}\PY{p}{,} \PY{n}{rand\PYZus{}x}\PY{p}{,} \PY{n}{rand\PYZus{}y}\PY{p}{,} \PY{n}{mu}\PY{p}{,} \PY{n}{std\PYZus{}list}\PY{p}{,} \PY{n}{corr\PYZus{}list}\PY{p}{,} \PY{n}{full\PYZus{}log}\PY{o}{=}\PY{k+kc}{False}\PY{p}{)}
\end{Verbatim}

    \begin{Verbatim}[commandchars=\\\{\}]
1) Gleichverteilte Zufallszahlen: [0.5474698703901343, 0.20312175454000392]

2) Standardnormalverteilte Zufallszahlen: [ 0.1192715  -0.83052236]

3) Standardnormalverteilte abhängige Zufallszahlen: [2.23854300702303, 0.5084329092563715]

4) Realisationen der Gauss-Copula: [0.5474698703901343, 0.20312175454000397]

5) Sumulationsergebnisse: x=14.525301296098657, y=19.156295436439944, Summe=33.6815967325386

    \end{Verbatim}

    \begin{Verbatim}[commandchars=\\\{\}]
{\color{incolor}In [{\color{incolor}8}]:} \PY{n}{total\PYZus{}standard\PYZus{}norm\PYZus{}ab\PYZus{}list}\PY{p}{,} \PY{n}{total\PYZus{}realisation\PYZus{}cop\PYZus{}list}\PY{p}{,} \PY{n}{total\PYZus{}xy\PYZus{}list}\PY{p}{,} \PY{n}{total\PYZus{}summe\PYZus{}liste} \PY{o}{=} \PY{n}{copula\PYZus{}sim}\PY{p}{(}\PY{n}{n}\PY{p}{,} \PY{n}{rand\PYZus{}x}\PY{p}{,} \PY{n}{rand\PYZus{}y}\PY{p}{,} \PY{n}{mu}\PY{p}{,} \PY{n}{std\PYZus{}list}\PY{p}{,} \PY{n}{corr\PYZus{}list}\PY{p}{,} \PY{n}{full\PYZus{}log}\PY{o}{=}\PY{k+kc}{True}\PY{p}{)}
\end{Verbatim}

    \hypertarget{graphische-ausgaben}{%
\subsection{Graphische Ausgaben:}\label{graphische-ausgaben}}

\hypertarget{realisationen-abhuxe4ngiger-bivariat-normalverteilter-zufallszahlen}{%
\subsubsection{Realisationen abhängiger bivariat normalverteilter
Zufallszahlen}\label{realisationen-abhuxe4ngiger-bivariat-normalverteilter-zufallszahlen}}

Die Zufallszahlen \(x\) und \(y\) streuen mit der Varianz \(\sigma_x\)
bzw. \(\sigma_y\) um ihren jeweiligen Erwartungswert \(\mu_x\) bzw.
\(\mu_y\).

    \begin{Verbatim}[commandchars=\\\{\}]
{\color{incolor}In [{\color{incolor}9}]:} \PY{n}{x\PYZus{}liste}\PY{p}{,} \PY{n}{y\PYZus{}liste} \PY{o}{=} \PY{n}{split\PYZus{}liste}\PY{p}{(}\PY{n}{total\PYZus{}standard\PYZus{}norm\PYZus{}ab\PYZus{}list}\PY{p}{)} \PY{c+c1}{\PYZsh{} X\PYZhy{} und Y\PYZhy{}Realisationen aus gemeinsamer Liste extrahieren}
        \PY{n}{plt}\PY{o}{.}\PY{n}{title}\PY{p}{(}\PY{l+s+s1}{\PYZsq{}}\PY{l+s+s1}{Realisationen abhängiger bivariat normalverteilter Zufallszahlen}\PY{l+s+s1}{\PYZsq{}}\PY{p}{)} \PY{c+c1}{\PYZsh{} Spezifischer Titel}
        \PY{n}{plot\PYZus{}func}\PY{p}{(}\PY{n}{x\PYZus{}liste}\PY{p}{,} \PY{n}{y\PYZus{}liste}\PY{p}{,} \PY{n}{show}\PY{o}{=}\PY{k+kc}{True}\PY{p}{,} \PY{n}{get\PYZus{}xy\PYZus{}lim}\PY{o}{=}\PY{k+kc}{True}\PY{p}{)} \PY{c+c1}{\PYZsh{} Plot erzeugen und anzeigen}
\end{Verbatim}

    \begin{center}
    \adjustimage{max size={0.9\linewidth}{0.9\paperheight}}{output_14_0.png}
    \end{center}
    { \hspace*{\fill} \\}
    
    \hypertarget{realisationen-gauss-copula}{%
\subsection{Realisationen
Gauss-Copula}\label{realisationen-gauss-copula}}

Darstellung der Abhänigkeitsstruktur ohne die (alten) Randverteilungen.
Diese Realisationen lassen sich als Wahrscheinlichkeiten interpretieren.

    \begin{Verbatim}[commandchars=\\\{\}]
{\color{incolor}In [{\color{incolor}10}]:} \PY{n}{x\PYZus{}liste}\PY{p}{,} \PY{n}{y\PYZus{}liste} \PY{o}{=} \PY{n}{split\PYZus{}liste}\PY{p}{(}\PY{n}{total\PYZus{}realisation\PYZus{}cop\PYZus{}list}\PY{p}{)} \PY{c+c1}{\PYZsh{} X\PYZhy{} und Y\PYZhy{}Realisationen aus gemeinsamer Liste extrahieren}
         \PY{n}{plt}\PY{o}{.}\PY{n}{title}\PY{p}{(}\PY{l+s+s1}{\PYZsq{}}\PY{l+s+s1}{Realisationen Gauss\PYZhy{}Copula}\PY{l+s+s1}{\PYZsq{}}\PY{p}{)} \PY{c+c1}{\PYZsh{} Spezifischer Titel}
         \PY{n}{plot\PYZus{}func}\PY{p}{(}\PY{n}{x\PYZus{}liste}\PY{p}{,} \PY{n}{y\PYZus{}liste}\PY{p}{,} \PY{n}{show}\PY{o}{=}\PY{k+kc}{True}\PY{p}{,} \PY{n}{get\PYZus{}xy\PYZus{}lim}\PY{o}{=}\PY{k+kc}{True}\PY{p}{)} \PY{c+c1}{\PYZsh{} Plot erzeugen und anzeigen}
\end{Verbatim}

    \begin{center}
    \adjustimage{max size={0.9\linewidth}{0.9\paperheight}}{output_16_0.png}
    \end{center}
    { \hspace*{\fill} \\}
    
    \hypertarget{realisationen-gleichverteilte-x-y-mit-gauss-copula}{%
\subsection{Realisationen gleichverteilte X, Y mit
Gauss-Copula}\label{realisationen-gleichverteilte-x-y-mit-gauss-copula}}

Die Abhängikeitsstruktur wurde auf die neuen Randverteilungen
übertragen.\\
Die Realisationen sind auf das Intervall \([18,42]\) beschränkt (siehe
rand\_y, rand\_y), da dies die gemeinsame Ober- bzw. Untergrenze der
vorgegebenen neuen Randverteilungen ist.

    \begin{Verbatim}[commandchars=\\\{\}]
{\color{incolor}In [{\color{incolor}11}]:} \PY{n}{x\PYZus{}liste}\PY{p}{,} \PY{n}{y\PYZus{}liste} \PY{o}{=} \PY{n}{split\PYZus{}liste}\PY{p}{(}\PY{n}{total\PYZus{}xy\PYZus{}list}\PY{p}{)} \PY{c+c1}{\PYZsh{} X\PYZhy{} und Y\PYZhy{}Realisationen aus gemeinsamer Liste extrahieren}
         \PY{n}{plt}\PY{o}{.}\PY{n}{title}\PY{p}{(}\PY{l+s+s1}{\PYZsq{}}\PY{l+s+s1}{Realisationen gleichverteilte X, Y mit Gauss\PYZhy{}Copula}\PY{l+s+s1}{\PYZsq{}}\PY{p}{)} \PY{c+c1}{\PYZsh{} Spezifischer Titel}
         \PY{n}{plot\PYZus{}func}\PY{p}{(}\PY{n}{x\PYZus{}liste}\PY{p}{,} \PY{n}{y\PYZus{}liste}\PY{p}{,} \PY{n}{show}\PY{o}{=}\PY{k+kc}{True}\PY{p}{,} \PY{n}{get\PYZus{}xy\PYZus{}lim}\PY{o}{=}\PY{k+kc}{True}\PY{p}{)} \PY{c+c1}{\PYZsh{} Plot erzeugen und anzeigen}
\end{Verbatim}

    \begin{center}
    \adjustimage{max size={0.9\linewidth}{0.9\paperheight}}{output_18_0.png}
    \end{center}
    { \hspace*{\fill} \\}
    
    \hypertarget{histogramm-xy-realisationen-gleichverteilte-x-y-mit-gauss-copula}{%
\subsection{Histogramm X+Y, Realisationen gleichverteilte X, Y mit
Gauss-Copula}\label{histogramm-xy-realisationen-gleichverteilte-x-y-mit-gauss-copula}}

Darstellung der Simulationsergebnisse als Summe der \(x\) und \(y\)
Realisationen unter Berücksichtung Abhänigkeitsstruktur und neuen
Randverteilungen. Es zeigt sich ein auf das Intervall \([18,42]\)
beschränkter glockenförmiger Verlauf, wie er für eine Normalverteilung
üblich ist. Hinweis: Obige Aussage gilt nur bei einem
Korrelationskoeffizienten von \(0\).

    \begin{Verbatim}[commandchars=\\\{\}]
{\color{incolor}In [{\color{incolor}12}]:} \PY{n}{bins} \PY{o}{=} \PY{l+m+mi}{71}
         \PY{n}{data} \PY{o}{=} \PY{n}{total\PYZus{}summe\PYZus{}liste}
         \PY{n}{plt}\PY{o}{.}\PY{n}{hist}\PY{p}{(}\PY{n}{data}\PY{p}{,} \PY{n}{bins}\PY{o}{=}\PY{n}{bins}\PY{p}{)}
         \PY{n}{plt}\PY{o}{.}\PY{n}{grid}\PY{p}{(}\PY{p}{)}
         \PY{n}{plt}\PY{o}{.}\PY{n}{title}\PY{p}{(}\PY{l+s+s1}{\PYZsq{}}\PY{l+s+s1}{Histogram X+Y, Realisationen gleichverteilte X,Y mit Gauss\PYZhy{}Copula}\PY{l+s+s1}{\PYZsq{}}\PY{p}{)}
         \PY{n}{plt}\PY{o}{.}\PY{n}{xlabel}\PY{p}{(}\PY{l+s+s1}{\PYZsq{}}\PY{l+s+s1}{PF\PYZhy{}Realisation}\PY{l+s+s1}{\PYZsq{}}\PY{p}{)}
         \PY{n}{plt}\PY{o}{.}\PY{n}{ylabel}\PY{p}{(}\PY{l+s+s1}{\PYZsq{}}\PY{l+s+s1}{Anzahl}\PY{l+s+s1}{\PYZsq{}}\PY{p}{)}
         \PY{n}{plt}\PY{o}{.}\PY{n}{show}\PY{p}{(}\PY{p}{)}
\end{Verbatim}

    \begin{center}
    \adjustimage{max size={0.9\linewidth}{0.9\paperheight}}{output_20_0.png}
    \end{center}
    { \hspace*{\fill} \\}
    
    \hypertarget{verteilung-xy-realisationen-gleichverteilte-x-y-mit-gauss-copula}{%
\subsection{Verteilung X+Y, Realisationen gleichverteilte X, Y mit
Gauss-Copula}\label{verteilung-xy-realisationen-gleichverteilte-x-y-mit-gauss-copula}}

Darstellung der resultierenden kumulativen Verteilungsfunktion. Diese
nimmt, wie für eine Normalverteilung üblich, einen S-förmigen Verlauf.

    \begin{Verbatim}[commandchars=\\\{\}]
{\color{incolor}In [{\color{incolor}13}]:} \PY{n}{H}\PY{p}{,} \PY{n}{X1} \PY{o}{=} \PY{n}{np}\PY{o}{.}\PY{n}{histogram}\PY{p}{(}\PY{n}{total\PYZus{}summe\PYZus{}liste}\PY{p}{,} \PY{n}{bins}\PY{o}{=}\PY{n}{n}\PY{p}{,} \PY{n}{density}\PY{o}{=}\PY{k+kc}{True}\PY{p}{)}
         \PY{n}{hist\PYZus{}func}\PY{p}{(}\PY{n}{H}\PY{p}{,} \PY{n}{X1}\PY{p}{)}
         \PY{n}{verteilung\PYZus{}func}\PY{p}{(}\PY{p}{)}
\end{Verbatim}

    \begin{center}
    \adjustimage{max size={0.9\linewidth}{0.9\paperheight}}{output_22_0.png}
    \end{center}
    { \hspace*{\fill} \\}
    
    \hypertarget{risikomauxdfe-schuxe4tzen}{%
\subsection{Risikomaße schätzen:}\label{risikomauxdfe-schuxe4tzen}}

\hypertarget{parameterfestlegung-und-aufruf-der-funktionen}{%
\subsubsection{Parameterfestlegung und Aufruf der
Funktionen}\label{parameterfestlegung-und-aufruf-der-funktionen}}

    \begin{Verbatim}[commandchars=\\\{\}]
{\color{incolor}In [{\color{incolor}14}]:} \PY{c+c1}{\PYZsh{}\PYZhy{}\PYZhy{}\PYZhy{}\PYZhy{}\PYZhy{}\PYZhy{}\PYZhy{}\PYZhy{}\PYZhy{}\PYZhy{}\PYZhy{}\PYZhy{}\PYZhy{}\PYZhy{}\PYZhy{}\PYZhy{}\PYZhy{}\PYZhy{}\PYZhy{}\PYZhy{}\PYZhy{}\PYZhy{}\PYZhy{}\PYZhy{}\PYZhy{}\PYZhy{}\PYZhy{}}
         \PY{c+c1}{\PYZsh{} Nutzung der objektorientierten Programmierung }
         \PY{n}{x} \PY{o}{=} \PY{n}{rm}\PY{p}{(}\PY{n}{total\PYZus{}summe\PYZus{}liste}\PY{p}{,} \PY{n}{alpha}\PY{o}{=}\PY{l+m+mf}{0.1}\PY{p}{,} \PY{n}{gamma}\PY{o}{=}\PY{l+m+mf}{0.5}\PY{p}{)}
         
         \PY{c+c1}{\PYZsh{}\PYZhy{}\PYZhy{}\PYZhy{}\PYZhy{}\PYZhy{}\PYZhy{}\PYZhy{}\PYZhy{}\PYZhy{}\PYZhy{}\PYZhy{}\PYZhy{}\PYZhy{}\PYZhy{}\PYZhy{}\PYZhy{}\PYZhy{}\PYZhy{}\PYZhy{}\PYZhy{}\PYZhy{}\PYZhy{}\PYZhy{}\PYZhy{}\PYZhy{}\PYZhy{}\PYZhy{}}
         \PY{c+c1}{\PYZsh{} Value at Risk}
         \PY{n+nb}{print}\PY{p}{(}\PY{l+s+s1}{\PYZsq{}}\PY{l+s+s1}{\PYZsh{}}\PY{l+s+s1}{\PYZsq{}} \PY{o}{+} \PY{n}{SCREEN\PYZus{}WIDTH} \PY{o}{*} \PY{l+s+s1}{\PYZsq{}}\PY{l+s+s1}{\PYZhy{}}\PY{l+s+s1}{\PYZsq{}} \PY{o}{+} \PY{l+s+s1}{\PYZsq{}}\PY{l+s+s1}{\PYZsh{}}\PY{l+s+s1}{\PYZsq{}}\PY{p}{)}
         \PY{n+nb}{print}\PY{p}{(}\PY{l+s+s1}{\PYZsq{}}\PY{l+s+s1}{|}\PY{l+s+s1}{\PYZsq{}} \PY{o}{+} \PY{n}{centered}\PY{p}{(}\PY{l+s+s1}{\PYZsq{}}\PY{l+s+s1}{Der VaR beträgt: }\PY{l+s+s1}{\PYZsq{}} \PY{o}{+} \PY{n+nb}{str}\PY{p}{(}\PY{n}{x}\PY{o}{.}\PY{n}{VaR}\PY{p}{(}\PY{p}{)}\PY{p}{)} \PY{o}{+} \PY{l+s+s1}{\PYZsq{}}\PY{l+s+s1}{.}\PY{l+s+s1}{\PYZsq{}}\PY{p}{)} \PY{o}{+} \PY{l+s+s1}{\PYZsq{}}\PY{l+s+s1}{| }\PY{l+s+s1}{\PYZsq{}}\PY{p}{)}
         
         \PY{c+c1}{\PYZsh{}\PYZhy{}\PYZhy{}\PYZhy{}\PYZhy{}\PYZhy{}\PYZhy{}\PYZhy{}\PYZhy{}\PYZhy{}\PYZhy{}\PYZhy{}\PYZhy{}\PYZhy{}\PYZhy{}\PYZhy{}\PYZhy{}\PYZhy{}\PYZhy{}\PYZhy{}\PYZhy{}\PYZhy{}\PYZhy{}\PYZhy{}\PYZhy{}\PYZhy{}\PYZhy{}\PYZhy{}}
         \PY{c+c1}{\PYZsh{} Conditional Value at Risk}
         \PY{n+nb}{print}\PY{p}{(}\PY{l+s+s1}{\PYZsq{}}\PY{l+s+s1}{\PYZsh{}}\PY{l+s+s1}{\PYZsq{}} \PY{o}{+} \PY{n}{SCREEN\PYZus{}WIDTH} \PY{o}{*} \PY{l+s+s1}{\PYZsq{}}\PY{l+s+s1}{\PYZhy{}}\PY{l+s+s1}{\PYZsq{}} \PY{o}{+} \PY{l+s+s1}{\PYZsq{}}\PY{l+s+s1}{\PYZsh{}}\PY{l+s+s1}{\PYZsq{}}\PY{p}{)}
         \PY{n+nb}{print}\PY{p}{(}\PY{l+s+s1}{\PYZsq{}}\PY{l+s+s1}{|}\PY{l+s+s1}{\PYZsq{}} \PY{o}{+} \PY{n}{centered}\PY{p}{(}\PY{l+s+s1}{\PYZsq{}}\PY{l+s+s1}{Der CVaR beträgt: }\PY{l+s+s1}{\PYZsq{}} \PY{o}{+} \PY{n+nb}{str}\PY{p}{(}\PY{n}{x}\PY{o}{.}\PY{n}{CVaR}\PY{p}{(}\PY{p}{)}\PY{p}{)} \PY{o}{+} \PY{l+s+s1}{\PYZsq{}}\PY{l+s+s1}{.}\PY{l+s+s1}{\PYZsq{}}\PY{p}{)} \PY{o}{+} \PY{l+s+s1}{\PYZsq{}}\PY{l+s+s1}{| }\PY{l+s+s1}{\PYZsq{}}\PY{p}{)}
         
         \PY{c+c1}{\PYZsh{}\PYZhy{}\PYZhy{}\PYZhy{}\PYZhy{}\PYZhy{}\PYZhy{}\PYZhy{}\PYZhy{}\PYZhy{}\PYZhy{}\PYZhy{}\PYZhy{}\PYZhy{}\PYZhy{}\PYZhy{}\PYZhy{}\PYZhy{}\PYZhy{}\PYZhy{}\PYZhy{}\PYZhy{}\PYZhy{}\PYZhy{}\PYZhy{}\PYZhy{}\PYZhy{}\PYZhy{}}
         \PY{c+c1}{\PYZsh{} Power\PYZhy{}Spektrales Risikomaß}
         \PY{n+nb}{print}\PY{p}{(}\PY{l+s+s1}{\PYZsq{}}\PY{l+s+s1}{\PYZsh{}}\PY{l+s+s1}{\PYZsq{}} \PY{o}{+} \PY{n}{SCREEN\PYZus{}WIDTH} \PY{o}{*} \PY{l+s+s1}{\PYZsq{}}\PY{l+s+s1}{\PYZhy{}}\PY{l+s+s1}{\PYZsq{}} \PY{o}{+} \PY{l+s+s1}{\PYZsq{}}\PY{l+s+s1}{\PYZsh{}}\PY{l+s+s1}{\PYZsq{}}\PY{p}{)}
         \PY{n+nb}{print}\PY{p}{(}\PY{l+s+s1}{\PYZsq{}}\PY{l+s+s1}{|}\PY{l+s+s1}{\PYZsq{}} \PY{o}{+} \PY{n}{centered}\PY{p}{(}\PY{l+s+s1}{\PYZsq{}}\PY{l+s+s1}{Power\PYZhy{}Spektrales Risikomaß bei der Monte\PYZhy{}Carlo\PYZhy{}Simulation:}\PY{l+s+s1}{\PYZsq{}}\PY{p}{)} \PY{o}{+} \PY{l+s+s1}{\PYZsq{}}\PY{l+s+s1}{| }\PY{l+s+s1}{\PYZsq{}}\PY{p}{)}
         \PY{n+nb}{print}\PY{p}{(}\PY{l+s+s1}{\PYZsq{}}\PY{l+s+s1}{\PYZsh{}}\PY{l+s+s1}{\PYZsq{}} \PY{o}{+} \PY{n}{SCREEN\PYZus{}WIDTH} \PY{o}{*} \PY{l+s+s1}{\PYZsq{}}\PY{l+s+s1}{\PYZhy{}}\PY{l+s+s1}{\PYZsq{}} \PY{o}{+} \PY{l+s+s1}{\PYZsq{}}\PY{l+s+s1}{\PYZsh{}}\PY{l+s+s1}{\PYZsq{}}\PY{p}{)}
         \PY{n+nb}{print}\PY{p}{(}\PY{l+s+s1}{\PYZsq{}}\PY{l+s+s1}{|}\PY{l+s+s1}{\PYZsq{}} \PY{o}{+} \PY{n}{centered}\PY{p}{(}\PY{l+s+s1}{\PYZsq{}}\PY{l+s+s1}{Der Erwartungswert beträgt: }\PY{l+s+s1}{\PYZsq{}} \PY{o}{+} \PY{n+nb}{str}\PY{p}{(}\PY{n}{x}\PY{o}{.}\PY{n}{expected\PYZus{}value}\PY{p}{)} \PY{o}{+} \PY{l+s+s1}{\PYZsq{}}\PY{l+s+s1}{.}\PY{l+s+s1}{\PYZsq{}}\PY{p}{)} \PY{o}{+} \PY{l+s+s1}{\PYZsq{}}\PY{l+s+s1}{| }\PY{l+s+s1}{\PYZsq{}}\PY{p}{)}
         \PY{n+nb}{print}\PY{p}{(}\PY{l+s+s1}{\PYZsq{}}\PY{l+s+s1}{|}\PY{l+s+s1}{\PYZsq{}} \PY{o}{+} \PY{n}{centered}\PY{p}{(}\PY{l+s+s1}{\PYZsq{}}\PY{l+s+s1}{Das Risiko beträgt: }\PY{l+s+s1}{\PYZsq{}} \PY{o}{+} \PY{n+nb}{str}\PY{p}{(}\PY{n}{x}\PY{o}{.}\PY{n}{Power}\PY{p}{(}\PY{p}{)}\PY{p}{)} \PY{o}{+} \PY{l+s+s1}{\PYZsq{}}\PY{l+s+s1}{.}\PY{l+s+s1}{\PYZsq{}}\PY{p}{)} \PY{o}{+} \PY{l+s+s1}{\PYZsq{}}\PY{l+s+s1}{| }\PY{l+s+s1}{\PYZsq{}}\PY{p}{)}
         \PY{n+nb}{print}\PY{p}{(}\PY{l+s+s1}{\PYZsq{}}\PY{l+s+s1}{\PYZsh{}}\PY{l+s+s1}{\PYZsq{}} \PY{o}{+} \PY{n}{SCREEN\PYZus{}WIDTH} \PY{o}{*} \PY{l+s+s1}{\PYZsq{}}\PY{l+s+s1}{\PYZhy{}}\PY{l+s+s1}{\PYZsq{}} \PY{o}{+} \PY{l+s+s1}{\PYZsq{}}\PY{l+s+s1}{\PYZsh{}}\PY{l+s+s1}{\PYZsq{}}\PY{p}{)}
\end{Verbatim}

    \begin{Verbatim}[commandchars=\\\{\}]
\#-------------------------------------------------------------------------------------------------------------------\#
|                                       Der VaR beträgt: -23.292246488674394.                                       | 
\#-------------------------------------------------------------------------------------------------------------------\#
|                                       Der CVaR beträgt: -21.527951677274842.                                      | 
\#-------------------------------------------------------------------------------------------------------------------\#
|                             Power-Spektrales Risikomaß bei der Monte-Carlo-Simulation:                            | 
\#-------------------------------------------------------------------------------------------------------------------\#
|                                   Der Erwartungswert beträgt: 29.92361231216212.                                  | 
|                                      Das Risiko beträgt: 26.775350767136274.                                      | 
\#-------------------------------------------------------------------------------------------------------------------\#

    \end{Verbatim}

    \hypertarget{instabilituxe4t-monte-carlo-simulation}{%
\subsection{``Instabilität''
Monte-Carlo-Simulation}\label{instabilituxe4t-monte-carlo-simulation}}

Ein Kritikpunkt an der Monte-Carlo-Simulation ist, dass das Ergebnis des
Verfahrens großen Schwankungen unterliegen kann, sofern nur wenige
Realisationen in einem Simulationslauf simuliert werden. Dies soll die
folgende Grafik veranschaulichen. Dabei kann sowohl die Anzahl der
Simulationsläufe als auch die Anzahl der in jeder Simulation
durchgeführten Simulationen variiert werden.

    \begin{Verbatim}[commandchars=\\\{\}]
{\color{incolor}In [{\color{incolor}15}]:} \PY{c+c1}{\PYZsh{}\PYZsh{}\PYZsh{}\PYZsh{}\PYZsh{}\PYZsh{}\PYZsh{}\PYZsh{}\PYZsh{}\PYZsh{}\PYZsh{}\PYZsh{}\PYZsh{}\PYZsh{}\PYZsh{}\PYZsh{}\PYZsh{}\PYZsh{}\PYZsh{}\PYZsh{}\PYZsh{}\PYZsh{}\PYZsh{}\PYZsh{}\PYZsh{}\PYZsh{}\PYZsh{}\PYZsh{}\PYZsh{}\PYZsh{}\PYZsh{}\PYZsh{}\PYZsh{}\PYZsh{}\PYZsh{}\PYZsh{}\PYZsh{}\PYZsh{}\PYZsh{}\PYZsh{}\PYZsh{}\PYZsh{}\PYZsh{}\PYZsh{}\PYZsh{}\PYZsh{}\PYZsh{}\PYZsh{}\PYZsh{}\PYZsh{}\PYZsh{}\PYZsh{}\PYZsh{}\PYZsh{}\PYZsh{}\PYZsh{}\PYZsh{}\PYZsh{}\PYZsh{}\PYZsh{}\PYZsh{}\PYZsh{}\PYZsh{}\PYZsh{}\PYZsh{}\PYZsh{}\PYZsh{}\PYZsh{}\PYZsh{}\PYZsh{}\PYZsh{}\PYZsh{}\PYZsh{}\PYZsh{}}
         \PY{c+c1}{\PYZsh{}\PYZhy{}\PYZhy{}\PYZhy{}\PYZhy{}\PYZhy{}\PYZhy{}\PYZhy{}\PYZhy{}\PYZhy{}\PYZhy{}\PYZhy{}\PYZhy{}\PYZhy{}\PYZhy{}\PYZhy{}\PYZhy{}\PYZhy{}\PYZhy{}\PYZhy{}\PYZhy{}\PYZhy{}\PYZhy{}\PYZhy{}\PYZhy{}\PYZhy{}\PYZhy{}\PYZhy{}\PYZhy{}\PYZhy{}\PYZhy{}\PYZhy{}\PYZhy{}\PYZhy{}\PYZhy{}\PYZhy{}\PYZhy{}\PYZhy{}\PYZhy{}\PYZhy{}\PYZhy{}\PYZhy{}\PYZhy{}\PYZhy{}\PYZhy{}\PYZhy{}\PYZhy{}\PYZhy{}\PYZhy{}\PYZhy{}\PYZhy{}\PYZhy{}\PYZhy{}\PYZhy{}\PYZhy{}\PYZhy{}\PYZhy{}\PYZhy{}\PYZhy{}\PYZhy{}\PYZhy{}\PYZhy{}\PYZhy{}\PYZhy{}\PYZhy{}\PYZhy{}\PYZhy{}\PYZhy{}\PYZhy{}\PYZhy{}\PYZhy{}\PYZhy{}\PYZhy{}\PYZhy{}}
         \PY{c+c1}{\PYZsh{} Parameter Risikomaße}
         \PY{n}{alpha} \PY{o}{=} \PY{l+m+mf}{0.1}
         \PY{n}{gamma} \PY{o}{=} \PY{l+m+mf}{0.5}
         
         \PY{c+c1}{\PYZsh{}\PYZhy{}\PYZhy{}\PYZhy{}\PYZhy{}\PYZhy{}\PYZhy{}\PYZhy{}\PYZhy{}\PYZhy{}\PYZhy{}\PYZhy{}\PYZhy{}\PYZhy{}\PYZhy{}\PYZhy{}\PYZhy{}\PYZhy{}\PYZhy{}\PYZhy{}\PYZhy{}\PYZhy{}\PYZhy{}\PYZhy{}\PYZhy{}\PYZhy{}\PYZhy{}\PYZhy{}\PYZhy{}\PYZhy{}\PYZhy{}\PYZhy{}\PYZhy{}\PYZhy{}\PYZhy{}\PYZhy{}\PYZhy{}\PYZhy{}\PYZhy{}\PYZhy{}\PYZhy{}\PYZhy{}\PYZhy{}\PYZhy{}\PYZhy{}\PYZhy{}\PYZhy{}\PYZhy{}\PYZhy{}\PYZhy{}\PYZhy{}\PYZhy{}\PYZhy{}\PYZhy{}\PYZhy{}\PYZhy{}\PYZhy{}\PYZhy{}\PYZhy{}\PYZhy{}\PYZhy{}\PYZhy{}\PYZhy{}\PYZhy{}\PYZhy{}\PYZhy{}\PYZhy{}\PYZhy{}\PYZhy{}\PYZhy{}\PYZhy{}\PYZhy{}\PYZhy{}\PYZhy{}}
         \PY{c+c1}{\PYZsh{} Wiederholungen der Simulationen}
         \PY{n}{runs\PYZus{}sim} \PY{o}{=} \PY{l+m+mi}{100} \PY{c+c1}{\PYZsh{} Legt die Anzahl der Durchläufe einer Simulation fest}
         \PY{n}{runs\PYZus{}func} \PY{o}{=} \PY{l+m+mi}{10} \PY{c+c1}{\PYZsh{} Legt fest, wie viele Simulationen durchgeführt werden}
         
         \PY{c+c1}{\PYZsh{}\PYZhy{}\PYZhy{}\PYZhy{}\PYZhy{}\PYZhy{}\PYZhy{}\PYZhy{}\PYZhy{}\PYZhy{}\PYZhy{}\PYZhy{}\PYZhy{}\PYZhy{}\PYZhy{}\PYZhy{}\PYZhy{}\PYZhy{}\PYZhy{}\PYZhy{}\PYZhy{}\PYZhy{}\PYZhy{}\PYZhy{}\PYZhy{}\PYZhy{}\PYZhy{}\PYZhy{}\PYZhy{}\PYZhy{}\PYZhy{}\PYZhy{}\PYZhy{}\PYZhy{}\PYZhy{}\PYZhy{}\PYZhy{}\PYZhy{}\PYZhy{}\PYZhy{}\PYZhy{}\PYZhy{}\PYZhy{}\PYZhy{}\PYZhy{}\PYZhy{}\PYZhy{}\PYZhy{}\PYZhy{}\PYZhy{}\PYZhy{}\PYZhy{}\PYZhy{}\PYZhy{}\PYZhy{}\PYZhy{}\PYZhy{}\PYZhy{}\PYZhy{}\PYZhy{}\PYZhy{}\PYZhy{}\PYZhy{}\PYZhy{}\PYZhy{}\PYZhy{}\PYZhy{}\PYZhy{}\PYZhy{}\PYZhy{}\PYZhy{}\PYZhy{}\PYZhy{}\PYZhy{}}
         \PY{c+c1}{\PYZsh{}\PYZsh{}\PYZsh{}\PYZsh{}\PYZsh{}\PYZsh{}\PYZsh{}\PYZsh{}\PYZsh{}\PYZsh{}\PYZsh{}\PYZsh{}\PYZsh{}\PYZsh{}\PYZsh{}\PYZsh{}\PYZsh{}\PYZsh{}\PYZsh{}\PYZsh{}\PYZsh{}\PYZsh{}\PYZsh{}\PYZsh{}\PYZsh{}\PYZsh{}\PYZsh{}\PYZsh{}\PYZsh{}\PYZsh{}\PYZsh{}\PYZsh{}\PYZsh{}\PYZsh{}\PYZsh{}\PYZsh{}\PYZsh{}\PYZsh{}\PYZsh{}\PYZsh{}\PYZsh{}\PYZsh{}\PYZsh{}\PYZsh{}\PYZsh{}\PYZsh{}\PYZsh{}\PYZsh{}\PYZsh{}\PYZsh{}\PYZsh{}\PYZsh{}\PYZsh{}\PYZsh{}\PYZsh{}\PYZsh{}\PYZsh{}\PYZsh{}\PYZsh{}\PYZsh{}\PYZsh{}\PYZsh{}\PYZsh{}\PYZsh{}\PYZsh{}\PYZsh{}\PYZsh{}\PYZsh{}\PYZsh{}\PYZsh{}\PYZsh{}\PYZsh{}\PYZsh{}\PYZsh{}}
         
         \PY{n}{RM\PYZus{}VaR\PYZus{}list}\PY{p}{,} \PY{n}{RM\PYZus{}CVaR\PYZus{}list} \PY{o}{=} \PY{p}{[}\PY{p}{]}\PY{p}{,} \PY{p}{[}\PY{p}{]}
         \PY{n}{RM\PYZus{}PSRM\PYZus{}list}\PY{p}{,} \PY{n}{mega\PYZus{}summe\PYZus{}list} \PY{o}{=} \PY{p}{[}\PY{p}{]}\PY{p}{,} \PY{p}{[}\PY{p}{]}
         
         \PY{c+c1}{\PYZsh{} Führe die Simulation \PYZdq{}runs\PYZus{}func mal\PYZdq{} durch und speichere die Ergebnisse in der jeweiligen Liste}
         \PY{k}{for} \PY{n}{i} \PY{o+ow}{in} \PY{n+nb}{range}\PY{p}{(}\PY{l+m+mi}{0}\PY{p}{,} \PY{n}{runs\PYZus{}func}\PY{p}{)}\PY{p}{:}
             \PY{n}{\PYZus{}}\PY{p}{,} \PY{n}{\PYZus{}}\PY{p}{,} \PY{n}{\PYZus{}}\PY{p}{,} \PY{n}{total\PYZus{}summe\PYZus{}liste} \PY{o}{=} \PY{n}{copula\PYZus{}sim}\PY{p}{(}\PY{n}{runs\PYZus{}sim}\PY{p}{,} \PY{n}{rand\PYZus{}x}\PY{p}{,} \PY{n}{rand\PYZus{}y}\PY{p}{,} \PY{n}{mu}\PY{p}{,} \PY{n}{std\PYZus{}list}\PY{p}{,} \PY{n}{corr\PYZus{}list}\PY{p}{,} \PY{n}{full\PYZus{}log}\PY{o}{=}\PY{k+kc}{False}\PY{p}{)}
             \PY{n}{mega\PYZus{}summe\PYZus{}list} \PY{o}{+}\PY{o}{=} \PY{n}{total\PYZus{}summe\PYZus{}liste}
             \PY{n}{x} \PY{o}{=} \PY{n}{rm}\PY{p}{(}\PY{n}{total\PYZus{}summe\PYZus{}liste}\PY{p}{,} \PY{n}{alpha}\PY{p}{,} \PY{n}{gamma}\PY{p}{)}    
             \PY{n}{RM\PYZus{}VaR\PYZus{}list}\PY{o}{.}\PY{n}{append}\PY{p}{(}\PY{n}{x}\PY{o}{.}\PY{n}{VaR}\PY{p}{(}\PY{p}{)}\PY{p}{)}    
             \PY{n}{RM\PYZus{}CVaR\PYZus{}list}\PY{o}{.}\PY{n}{append}\PY{p}{(}\PY{n}{x}\PY{o}{.}\PY{n}{CVaR}\PY{p}{(}\PY{p}{)}\PY{p}{)}
             \PY{n}{RM\PYZus{}PSRM\PYZus{}list}\PY{o}{.}\PY{n}{append}\PY{p}{(}\PY{n}{x}\PY{o}{.}\PY{n}{Power}\PY{p}{(}\PY{p}{)}\PY{p}{)}
         
         \PY{c+c1}{\PYZsh{}\PYZhy{}\PYZhy{}\PYZhy{}\PYZhy{}\PYZhy{}\PYZhy{}\PYZhy{}\PYZhy{}\PYZhy{}\PYZhy{}\PYZhy{}\PYZhy{}\PYZhy{}\PYZhy{}\PYZhy{}\PYZhy{}\PYZhy{}\PYZhy{}\PYZhy{}\PYZhy{}\PYZhy{}\PYZhy{}\PYZhy{}\PYZhy{}\PYZhy{}\PYZhy{}\PYZhy{}}
         \PY{c+c1}{\PYZsh{} Erzeuge ein DataFrame mit den Simulationsvergebnissen}
         \PY{c+c1}{\PYZsh{} und deren prozentualen Änderung vom jeweils vorherigen Ergebnis}
         \PY{n}{RM\PYZus{}frame} \PY{o}{=} \PY{n}{pd}\PY{o}{.}\PY{n}{DataFrame}\PY{p}{(}\PY{p}{)}
         \PY{n}{RM\PYZus{}frame}\PY{p}{[}\PY{l+s+s1}{\PYZsq{}}\PY{l+s+s1}{VaR}\PY{l+s+s1}{\PYZsq{}}\PY{p}{]} \PY{o}{=} \PY{n}{RM\PYZus{}VaR\PYZus{}list}
         \PY{n}{RM\PYZus{}frame}\PY{p}{[}\PY{l+s+s1}{\PYZsq{}}\PY{l+s+s1}{VaR\PYZhy{}Change}\PY{l+s+s1}{\PYZsq{}}\PY{p}{]} \PY{o}{=} \PY{n}{RM\PYZus{}frame}\PY{p}{[}\PY{l+s+s1}{\PYZsq{}}\PY{l+s+s1}{VaR}\PY{l+s+s1}{\PYZsq{}}\PY{p}{]}\PY{o}{.}\PY{n}{pct\PYZus{}change}\PY{p}{(}\PY{p}{)}
         \PY{n}{RM\PYZus{}frame}\PY{p}{[}\PY{l+s+s1}{\PYZsq{}}\PY{l+s+s1}{CVaR}\PY{l+s+s1}{\PYZsq{}}\PY{p}{]} \PY{o}{=} \PY{n}{RM\PYZus{}CVaR\PYZus{}list}
         \PY{n}{RM\PYZus{}frame}\PY{p}{[}\PY{l+s+s1}{\PYZsq{}}\PY{l+s+s1}{CVaR\PYZhy{}Change}\PY{l+s+s1}{\PYZsq{}}\PY{p}{]} \PY{o}{=} \PY{n}{RM\PYZus{}frame}\PY{p}{[}\PY{l+s+s1}{\PYZsq{}}\PY{l+s+s1}{CVaR}\PY{l+s+s1}{\PYZsq{}}\PY{p}{]}\PY{o}{.}\PY{n}{pct\PYZus{}change}\PY{p}{(}\PY{p}{)}
         \PY{n}{RM\PYZus{}frame}\PY{p}{[}\PY{l+s+s1}{\PYZsq{}}\PY{l+s+s1}{Power}\PY{l+s+s1}{\PYZsq{}}\PY{p}{]} \PY{o}{=} \PY{n}{RM\PYZus{}PSRM\PYZus{}list}
         \PY{n}{RM\PYZus{}frame}\PY{p}{[}\PY{l+s+s1}{\PYZsq{}}\PY{l+s+s1}{Power\PYZhy{}Change}\PY{l+s+s1}{\PYZsq{}}\PY{p}{]} \PY{o}{=} \PY{n}{RM\PYZus{}frame}\PY{p}{[}\PY{l+s+s1}{\PYZsq{}}\PY{l+s+s1}{Power}\PY{l+s+s1}{\PYZsq{}}\PY{p}{]}\PY{o}{.}\PY{n}{pct\PYZus{}change}\PY{p}{(}\PY{p}{)}
         
         \PY{c+c1}{\PYZsh{}\PYZhy{}\PYZhy{}\PYZhy{}\PYZhy{}\PYZhy{}\PYZhy{}\PYZhy{}\PYZhy{}\PYZhy{}\PYZhy{}\PYZhy{}\PYZhy{}\PYZhy{}\PYZhy{}\PYZhy{}\PYZhy{}\PYZhy{}\PYZhy{}\PYZhy{}\PYZhy{}\PYZhy{}\PYZhy{}\PYZhy{}\PYZhy{}\PYZhy{}\PYZhy{}\PYZhy{}}
         \PY{c+c1}{\PYZsh{} Ermittle die kleinste und größte Relaisation des jweiligen Risikomaßes}
         \PY{n}{Min\PYZus{}Max\PYZus{}VaR} \PY{o}{=} \PY{p}{(}\PY{n+nb}{min}\PY{p}{(}\PY{n}{RM\PYZus{}VaR\PYZus{}list}\PY{p}{)}\PY{p}{,} \PY{n+nb}{max}\PY{p}{(}\PY{n}{RM\PYZus{}VaR\PYZus{}list}\PY{p}{)}\PY{p}{)}
         \PY{n}{Min\PYZus{}Max\PYZus{}CVaR} \PY{o}{=} \PY{p}{(}\PY{n+nb}{min}\PY{p}{(}\PY{n}{RM\PYZus{}CVaR\PYZus{}list}\PY{p}{)}\PY{p}{,} \PY{n+nb}{max}\PY{p}{(}\PY{n}{RM\PYZus{}CVaR\PYZus{}list}\PY{p}{)}\PY{p}{)}
         \PY{n}{Min\PYZus{}Max\PYZus{}PSRM} \PY{o}{=} \PY{p}{(}\PY{n+nb}{min}\PY{p}{(}\PY{n}{RM\PYZus{}PSRM\PYZus{}list}\PY{p}{)}\PY{p}{,} \PY{n+nb}{max}\PY{p}{(}\PY{n}{RM\PYZus{}PSRM\PYZus{}list}\PY{p}{)}\PY{p}{)}
         
         \PY{c+c1}{\PYZsh{}\PYZhy{}\PYZhy{}\PYZhy{}\PYZhy{}\PYZhy{}\PYZhy{}\PYZhy{}\PYZhy{}\PYZhy{}\PYZhy{}\PYZhy{}\PYZhy{}\PYZhy{}\PYZhy{}\PYZhy{}\PYZhy{}\PYZhy{}\PYZhy{}\PYZhy{}\PYZhy{}\PYZhy{}\PYZhy{}\PYZhy{}\PYZhy{}\PYZhy{}\PYZhy{}\PYZhy{}}
         \PY{c+c1}{\PYZsh{} Gib den DataFrame und die Infos zurück}
         \PY{n+nb}{print}\PY{p}{(}\PY{l+s+s1}{\PYZsq{}}\PY{l+s+s1}{\PYZsh{}}\PY{l+s+s1}{\PYZsq{}} \PY{o}{+} \PY{n}{SCREEN\PYZus{}WIDTH} \PY{o}{*} \PY{l+s+s1}{\PYZsq{}}\PY{l+s+s1}{\PYZhy{}}\PY{l+s+s1}{\PYZsq{}} \PY{o}{+} \PY{l+s+s1}{\PYZsq{}}\PY{l+s+s1}{\PYZsh{}}\PY{l+s+s1}{\PYZsq{}}\PY{p}{)}
         \PY{n+nb}{print}\PY{p}{(}\PY{l+s+s1}{\PYZsq{}}\PY{l+s+s1}{|}\PY{l+s+s1}{\PYZsq{}} \PY{o}{+} \PY{n}{centered}\PY{p}{(}\PY{l+s+s1}{\PYZsq{}}\PY{l+s+s1}{[INFO] Der DataFrame mit den auf den auf }\PY{l+s+s1}{\PYZsq{}} \PY{o}{+}\PY{n+nb}{str}\PY{p}{(}\PY{n}{runs\PYZus{}func}\PY{p}{)} \PY{o}{+} \PY{l+s+s1}{\PYZsq{}}\PY{l+s+s1}{ mal }\PY{l+s+s1}{\PYZsq{}} \PY{o}{+} \PY{n+nb}{str}\PY{p}{(}\PY{n}{runs\PYZus{}sim}\PY{p}{)} \PY{o}{+} \PY{l+s+s1}{\PYZsq{}}\PY{l+s+s1}{ Durchläufen beruhenden Risikomaßen ergibt sich wie folgt: }\PY{l+s+s1}{\PYZsq{}}\PY{p}{)} \PY{o}{+} \PY{l+s+s1}{\PYZsq{}}\PY{l+s+s1}{| }\PY{l+s+s1}{\PYZsq{}}\PY{p}{)}
         \PY{n+nb}{print}\PY{p}{(}\PY{l+s+s1}{\PYZsq{}}\PY{l+s+s1}{\PYZsh{}}\PY{l+s+s1}{\PYZsq{}} \PY{o}{+} \PY{n}{SCREEN\PYZus{}WIDTH} \PY{o}{*} \PY{l+s+s1}{\PYZsq{}}\PY{l+s+s1}{\PYZhy{}}\PY{l+s+s1}{\PYZsq{}} \PY{o}{+} \PY{l+s+s1}{\PYZsq{}}\PY{l+s+s1}{\PYZsh{}}\PY{l+s+s1}{\PYZsq{}}\PY{p}{)}
         \PY{n+nb}{print}\PY{p}{(}\PY{n}{RM\PYZus{}frame}\PY{p}{)}
         \PY{n+nb}{print}\PY{p}{(}\PY{l+s+s1}{\PYZsq{}}\PY{l+s+s1}{\PYZsh{}}\PY{l+s+s1}{\PYZsq{}} \PY{o}{+} \PY{n}{SCREEN\PYZus{}WIDTH} \PY{o}{*} \PY{l+s+s1}{\PYZsq{}}\PY{l+s+s1}{\PYZhy{}}\PY{l+s+s1}{\PYZsq{}} \PY{o}{+} \PY{l+s+s1}{\PYZsq{}}\PY{l+s+s1}{\PYZsh{}}\PY{l+s+s1}{\PYZsq{}}\PY{p}{)}
         \PY{n+nb}{print}\PY{p}{(}\PY{l+s+s1}{\PYZsq{}}\PY{l+s+s1}{|}\PY{l+s+s1}{\PYZsq{}} \PY{o}{+} \PY{n}{centered}\PY{p}{(}\PY{l+s+s1}{\PYZsq{}}\PY{l+s+s1}{Nach }\PY{l+s+s1}{\PYZsq{}} \PY{o}{+} \PY{n+nb}{str}\PY{p}{(}\PY{n}{runs\PYZus{}func}\PY{p}{)} \PY{o}{+} \PY{l+s+s1}{\PYZsq{}}\PY{l+s+s1}{ Simulationsläufen mit je }\PY{l+s+s1}{\PYZsq{}} \PY{o}{+} \PY{n+nb}{str}\PY{p}{(}\PY{n}{runs\PYZus{}sim}\PY{p}{)} \PY{o}{+} \PY{l+s+s1}{\PYZsq{}}\PY{l+s+s1}{ Durchläufen beträgt der kleinste VaR }\PY{l+s+s1}{\PYZsq{}} \PY{o}{+} \PY{n+nb}{str}\PY{p}{(}\PY{n+nb}{round}\PY{p}{(}\PY{n}{Min\PYZus{}Max\PYZus{}VaR}\PY{p}{[}\PY{l+m+mi}{0}\PY{p}{]}\PY{p}{,}\PY{l+m+mi}{2}\PY{p}{)}\PY{p}{)} \PY{o}{+}\PY{l+s+s1}{\PYZsq{}}\PY{l+s+s1}{, der größte }\PY{l+s+s1}{\PYZsq{}} \PY{o}{+} \PY{n+nb}{str}\PY{p}{(}\PY{n+nb}{round}\PY{p}{(}\PY{n}{Min\PYZus{}Max\PYZus{}VaR}\PY{p}{[}\PY{l+m+mi}{1}\PY{p}{]}\PY{p}{,}\PY{l+m+mi}{2}\PY{p}{)}\PY{p}{)} \PY{o}{+} \PY{l+s+s1}{\PYZsq{}}\PY{l+s+s1}{ (}\PY{l+s+se}{\PYZbs{}u0394}\PY{l+s+s1}{ = }\PY{l+s+s1}{\PYZsq{}} \PY{o}{+} \PY{n+nb}{str}\PY{p}{(}\PY{p}{(}\PY{n+nb}{round}\PY{p}{(}\PY{p}{(}\PY{n+nb}{float}\PY{p}{(}\PY{n}{Min\PYZus{}Max\PYZus{}VaR}\PY{p}{[}\PY{l+m+mi}{0}\PY{p}{]}\PY{o}{/}\PY{n}{Min\PYZus{}Max\PYZus{}VaR}\PY{p}{[}\PY{l+m+mi}{1}\PY{p}{]}\PY{p}{)}\PY{o}{\PYZhy{}}\PY{l+m+mi}{1}\PY{p}{)}\PY{o}{*}\PY{l+m+mi}{100}\PY{p}{,}\PY{l+m+mi}{2}\PY{p}{)}\PY{p}{)}\PY{p}{)} \PY{o}{+} \PY{l+s+s1}{\PYZsq{}}\PY{l+s+s1}{\PYZpc{}}\PY{l+s+s1}{).}\PY{l+s+s1}{\PYZsq{}}\PY{p}{)} \PY{o}{+} \PY{l+s+s1}{\PYZsq{}}\PY{l+s+s1}{| }\PY{l+s+s1}{\PYZsq{}}\PY{p}{)}
         \PY{n+nb}{print}\PY{p}{(}\PY{l+s+s1}{\PYZsq{}}\PY{l+s+s1}{|}\PY{l+s+s1}{\PYZsq{}} \PY{o}{+} \PY{n}{centered}\PY{p}{(}\PY{l+s+s1}{\PYZsq{}}\PY{l+s+s1}{Nach }\PY{l+s+s1}{\PYZsq{}} \PY{o}{+} \PY{n+nb}{str}\PY{p}{(}\PY{n}{runs\PYZus{}func}\PY{p}{)} \PY{o}{+} \PY{l+s+s1}{\PYZsq{}}\PY{l+s+s1}{ Simulationsläufen mit je }\PY{l+s+s1}{\PYZsq{}} \PY{o}{+} \PY{n+nb}{str}\PY{p}{(}\PY{n}{runs\PYZus{}sim}\PY{p}{)} \PY{o}{+} \PY{l+s+s1}{\PYZsq{}}\PY{l+s+s1}{ Durchläufen beträgt der kleinste CVaR }\PY{l+s+s1}{\PYZsq{}} \PY{o}{+} \PY{n+nb}{str}\PY{p}{(}\PY{n+nb}{round}\PY{p}{(}\PY{n}{Min\PYZus{}Max\PYZus{}CVaR}\PY{p}{[}\PY{l+m+mi}{0}\PY{p}{]}\PY{p}{,}\PY{l+m+mi}{2}\PY{p}{)}\PY{p}{)} \PY{o}{+}\PY{l+s+s1}{\PYZsq{}}\PY{l+s+s1}{, der größte }\PY{l+s+s1}{\PYZsq{}} \PY{o}{+} \PY{n+nb}{str}\PY{p}{(}\PY{n+nb}{round}\PY{p}{(}\PY{n}{Min\PYZus{}Max\PYZus{}CVaR}\PY{p}{[}\PY{l+m+mi}{1}\PY{p}{]}\PY{p}{,}\PY{l+m+mi}{2}\PY{p}{)}\PY{p}{)} \PY{o}{+} \PY{l+s+s1}{\PYZsq{}}\PY{l+s+s1}{ (}\PY{l+s+se}{\PYZbs{}u0394}\PY{l+s+s1}{ = }\PY{l+s+s1}{\PYZsq{}} \PY{o}{+} \PY{n+nb}{str}\PY{p}{(}\PY{p}{(}\PY{n+nb}{round}\PY{p}{(}\PY{p}{(}\PY{n+nb}{float}\PY{p}{(}\PY{n}{Min\PYZus{}Max\PYZus{}CVaR}\PY{p}{[}\PY{l+m+mi}{0}\PY{p}{]}\PY{o}{/}\PY{n}{Min\PYZus{}Max\PYZus{}CVaR}\PY{p}{[}\PY{l+m+mi}{1}\PY{p}{]}\PY{p}{)}\PY{o}{\PYZhy{}}\PY{l+m+mi}{1}\PY{p}{)}\PY{o}{*}\PY{l+m+mi}{100}\PY{p}{,}\PY{l+m+mi}{2}\PY{p}{)}\PY{p}{)}\PY{p}{)} \PY{o}{+} \PY{l+s+s1}{\PYZsq{}}\PY{l+s+s1}{\PYZpc{}}\PY{l+s+s1}{).}\PY{l+s+s1}{\PYZsq{}}\PY{p}{)} \PY{o}{+} \PY{l+s+s1}{\PYZsq{}}\PY{l+s+s1}{| }\PY{l+s+s1}{\PYZsq{}}\PY{p}{)}
         \PY{n+nb}{print}\PY{p}{(}\PY{l+s+s1}{\PYZsq{}}\PY{l+s+s1}{|}\PY{l+s+s1}{\PYZsq{}} \PY{o}{+} \PY{n}{centered}\PY{p}{(}\PY{l+s+s1}{\PYZsq{}}\PY{l+s+s1}{Nach }\PY{l+s+s1}{\PYZsq{}} \PY{o}{+} \PY{n+nb}{str}\PY{p}{(}\PY{n}{runs\PYZus{}func}\PY{p}{)} \PY{o}{+} \PY{l+s+s1}{\PYZsq{}}\PY{l+s+s1}{ Simulationsläufen mit je }\PY{l+s+s1}{\PYZsq{}} \PY{o}{+} \PY{n+nb}{str}\PY{p}{(}\PY{n}{runs\PYZus{}sim}\PY{p}{)} \PY{o}{+} \PY{l+s+s1}{\PYZsq{}}\PY{l+s+s1}{ Durchläufen beträgt das kleinste P\PYZhy{}SRM }\PY{l+s+s1}{\PYZsq{}} \PY{o}{+} \PY{n+nb}{str}\PY{p}{(}\PY{n+nb}{round}\PY{p}{(}\PY{n}{Min\PYZus{}Max\PYZus{}PSRM}\PY{p}{[}\PY{l+m+mi}{0}\PY{p}{]}\PY{p}{,}\PY{l+m+mi}{2}\PY{p}{)}\PY{p}{)} \PY{o}{+}\PY{l+s+s1}{\PYZsq{}}\PY{l+s+s1}{, das größte }\PY{l+s+s1}{\PYZsq{}} \PY{o}{+} \PY{n+nb}{str}\PY{p}{(}\PY{n+nb}{round}\PY{p}{(}\PY{n}{Min\PYZus{}Max\PYZus{}PSRM}\PY{p}{[}\PY{l+m+mi}{1}\PY{p}{]}\PY{p}{,}\PY{l+m+mi}{2}\PY{p}{)}\PY{p}{)} \PY{o}{+} \PY{l+s+s1}{\PYZsq{}}\PY{l+s+s1}{ (}\PY{l+s+se}{\PYZbs{}u0394}\PY{l+s+s1}{ = }\PY{l+s+s1}{\PYZsq{}} \PY{o}{+} \PY{n+nb}{str}\PY{p}{(}\PY{p}{(}\PY{n+nb}{round}\PY{p}{(}\PY{p}{(}\PY{n+nb}{float}\PY{p}{(}\PY{n}{Min\PYZus{}Max\PYZus{}PSRM}\PY{p}{[}\PY{l+m+mi}{0}\PY{p}{]}\PY{o}{/}\PY{n}{Min\PYZus{}Max\PYZus{}PSRM}\PY{p}{[}\PY{l+m+mi}{1}\PY{p}{]}\PY{p}{)}\PY{o}{\PYZhy{}}\PY{l+m+mi}{1}\PY{p}{)}\PY{o}{*}\PY{l+m+mi}{100}\PY{p}{,}\PY{l+m+mi}{2}\PY{p}{)}\PY{p}{)}\PY{p}{)} \PY{o}{+} \PY{l+s+s1}{\PYZsq{}}\PY{l+s+s1}{\PYZpc{}}\PY{l+s+s1}{).}\PY{l+s+s1}{\PYZsq{}}\PY{p}{)} \PY{o}{+} \PY{l+s+s1}{\PYZsq{}}\PY{l+s+s1}{| }\PY{l+s+s1}{\PYZsq{}}\PY{p}{)}
         \PY{n+nb}{print}\PY{p}{(}\PY{l+s+s1}{\PYZsq{}}\PY{l+s+s1}{\PYZsh{}}\PY{l+s+s1}{\PYZsq{}} \PY{o}{+} \PY{n}{SCREEN\PYZus{}WIDTH} \PY{o}{*} \PY{l+s+s1}{\PYZsq{}}\PY{l+s+s1}{\PYZhy{}}\PY{l+s+s1}{\PYZsq{}} \PY{o}{+} \PY{l+s+s1}{\PYZsq{}}\PY{l+s+s1}{\PYZsh{}}\PY{l+s+s1}{\PYZsq{}}\PY{p}{)}
         
         \PY{c+c1}{\PYZsh{}\PYZhy{}\PYZhy{}\PYZhy{}\PYZhy{}\PYZhy{}\PYZhy{}\PYZhy{}\PYZhy{}\PYZhy{}\PYZhy{}\PYZhy{}\PYZhy{}\PYZhy{}\PYZhy{}\PYZhy{}\PYZhy{}\PYZhy{}\PYZhy{}\PYZhy{}\PYZhy{}\PYZhy{}\PYZhy{}\PYZhy{}\PYZhy{}\PYZhy{}\PYZhy{}\PYZhy{}\PYZhy{}\PYZhy{}\PYZhy{}\PYZhy{}\PYZhy{}\PYZhy{}\PYZhy{}\PYZhy{}\PYZhy{}\PYZhy{}\PYZhy{}\PYZhy{}\PYZhy{}\PYZhy{}\PYZhy{}\PYZhy{}}
         \PY{c+c1}{\PYZsh{} Zerlege die mega\PYZus{}summe\PYZus{}list (beinhaltet alle Ergenisse) in Teillisten,}
         \PY{c+c1}{\PYZsh{} welche die Ergebnisse der einzelnen Simulationsläufe beinhalten}
         \PY{n}{counter\PYZus{}0}\PY{p}{,} \PY{n}{counter\PYZus{}1} \PY{o}{=} \PY{l+m+mi}{0}\PY{p}{,} \PY{n}{runs\PYZus{}sim}
         \PY{n}{array} \PY{o}{=} \PY{p}{[}\PY{p}{]}
         \PY{k}{for} \PY{n}{i} \PY{o+ow}{in} \PY{n+nb}{range}\PY{p}{(}\PY{l+m+mi}{0}\PY{p}{,} \PY{n}{runs\PYZus{}func}\PY{p}{)}\PY{p}{:}
             \PY{n}{x} \PY{o}{=} \PY{n}{mega\PYZus{}summe\PYZus{}list}\PY{p}{[}\PY{n}{counter\PYZus{}0}\PY{p}{:}\PY{n}{counter\PYZus{}1}\PY{p}{]}
             \PY{n}{array}\PY{o}{.}\PY{n}{append}\PY{p}{(}\PY{n}{x}\PY{p}{)}
             \PY{n}{counter\PYZus{}0} \PY{o}{+}\PY{o}{=} \PY{n}{runs\PYZus{}sim}
             \PY{n}{counter\PYZus{}1} \PY{o}{+}\PY{o}{=} \PY{n}{runs\PYZus{}sim}
         
         \PY{c+c1}{\PYZsh{}\PYZhy{}\PYZhy{}\PYZhy{}\PYZhy{}\PYZhy{}\PYZhy{}\PYZhy{}\PYZhy{}\PYZhy{}\PYZhy{}\PYZhy{}\PYZhy{}\PYZhy{}\PYZhy{}\PYZhy{}\PYZhy{}\PYZhy{}\PYZhy{}\PYZhy{}\PYZhy{}\PYZhy{}\PYZhy{}\PYZhy{}\PYZhy{}\PYZhy{}\PYZhy{}\PYZhy{}}
         \PY{c+c1}{\PYZsh{} Erstelle für jedes Dieser Teillisten ein Histogramm}
         \PY{c+c1}{\PYZsh{} und plotte anschlißend das Ergebnis}
         \PY{k}{for} \PY{n}{items} \PY{o+ow}{in} \PY{n}{array}\PY{p}{:}
             \PY{n}{values\PYZus{}PF} \PY{o}{=} \PY{n}{items}
             \PY{n}{bins} \PY{o}{=} \PY{n}{runs\PYZus{}sim}    
             \PY{n}{H}\PY{p}{,} \PY{n}{X1} \PY{o}{=} \PY{n}{np}\PY{o}{.}\PY{n}{histogram}\PY{p}{(}\PY{n}{values\PYZus{}PF}\PY{p}{,} \PY{n}{bins}\PY{p}{,} \PY{n}{density}\PY{o}{=}\PY{k+kc}{True}\PY{p}{)}
             \PY{n}{hist\PYZus{}func}\PY{p}{(}\PY{n}{H}\PY{p}{,} \PY{n}{X1}\PY{p}{)}
         \PY{n}{verteilung\PYZus{}func}\PY{p}{(}\PY{p}{)}
\end{Verbatim}

    \begin{Verbatim}[commandchars=\\\{\}]
\#-------------------------------------------------------------------------------------------------------------------\#
|   [INFO] Der DataFrame mit den auf den auf 10 mal 100 Durchläufen beruhenden Risikomaßen ergibt sich wie folgt:   | 
\#-------------------------------------------------------------------------------------------------------------------\#
         VaR  VaR-Change       CVaR  CVaR-Change      Power  Power-Change
0 -23.256615         NaN -21.636180          NaN  26.947987           NaN
1 -23.997290    0.031848 -22.211472     0.026589  28.046302      0.040757
2 -23.284165   -0.029717 -21.483862    -0.032758  26.760908     -0.045831
3 -22.889987   -0.016929 -20.777215    -0.032892  25.932862     -0.030942
4 -23.155260    0.011589 -21.574370     0.038367  27.073755      0.043994
5 -23.196542    0.001783 -22.062725     0.022636  27.388507      0.011626
6 -22.280270   -0.039500 -20.463326    -0.072493  27.269637     -0.004340
7 -23.833364    0.069707 -21.862745     0.068387  26.955147     -0.011533
8 -22.301710   -0.064265 -20.490398    -0.062771  26.400550     -0.020575
9 -24.801765    0.112102 -23.087812     0.126762  27.662976      0.047818
\#-------------------------------------------------------------------------------------------------------------------\#
|  Nach 10 Simulationsläufen mit je 100 Durchläufen beträgt der kleinste VaR -24.8, der größte -22.28 (Δ = 11.32\%). | 
| Nach 10 Simulationsläufen mit je 100 Durchläufen beträgt der kleinste CVaR -23.09, der größte -20.46 (Δ = 12.83\%).| 
| Nach 10 Simulationsläufen mit je 100 Durchläufen beträgt das kleinste P-SRM 25.93, das größte 28.05 (Δ = -7.54\%). | 
\#-------------------------------------------------------------------------------------------------------------------\#

    \end{Verbatim}

    \begin{center}
    \adjustimage{max size={0.9\linewidth}{0.9\paperheight}}{output_26_1.png}
    \end{center}
    { \hspace*{\fill} \\}
    

    % Add a bibliography block to the postdoc
    
    
    
    \end{document}
