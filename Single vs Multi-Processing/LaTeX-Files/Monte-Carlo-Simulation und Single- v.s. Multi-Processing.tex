\documentclass[paper=landscape]{scrartcl}
	\usepackage[paperwidth=40cm,paperheight=125cm,margin=1in]{geometry}

    \usepackage[breakable]{tcolorbox}
    \usepackage{parskip} % Stop auto-indenting (to mimic markdown behaviour)
    
    \usepackage{iftex}
    \ifPDFTeX
    	\usepackage[T1]{fontenc}
    	\usepackage{mathpazo}
    \else
    	\usepackage{fontspec}
    \fi

    % Basic figure setup, for now with no caption control since it's done
    % automatically by Pandoc (which extracts ![](path) syntax from Markdown).
    \usepackage{graphicx}
    % Maintain compatibility with old templates. Remove in nbconvert 6.0
    \let\Oldincludegraphics\includegraphics
    % Ensure that by default, figures have no caption (until we provide a
    % proper Figure object with a Caption API and a way to capture that
    % in the conversion process - todo).
    \usepackage{caption}
    \DeclareCaptionFormat{nocaption}{}
    \captionsetup{format=nocaption,aboveskip=0pt,belowskip=0pt}

    \usepackage[Export]{adjustbox} % Used to constrain images to a maximum size
    \adjustboxset{max size={0.9\linewidth}{0.9\paperheight}}
    \usepackage{float}
    \floatplacement{figure}{H} % forces figures to be placed at the correct location
    \usepackage{xcolor} % Allow colors to be defined
    \usepackage{enumerate} % Needed for markdown enumerations to work
    \usepackage{geometry} % Used to adjust the document margins
    \usepackage{amsmath} % Equations
    \usepackage{amssymb} % Equations
    \usepackage{textcomp} % defines textquotesingle
    % Hack from http://tex.stackexchange.com/a/47451/13684:
    \AtBeginDocument{%
        \def\PYZsq{\textquotesingle}% Upright quotes in Pygmentized code
    }
    \usepackage{upquote} % Upright quotes for verbatim code
    \usepackage{eurosym} % defines \euro
    \usepackage[mathletters]{ucs} % Extended unicode (utf-8) support
    \usepackage{fancyvrb} % verbatim replacement that allows latex
    \usepackage{grffile} % extends the file name processing of package graphics 
                         % to support a larger range
    \makeatletter % fix for grffile with XeLaTeX
    \def\Gread@@xetex#1{%
      \IfFileExists{"\Gin@base".bb}%
      {\Gread@eps{\Gin@base.bb}}%
      {\Gread@@xetex@aux#1}%
    }
    \makeatother

    % The hyperref package gives us a pdf with properly built
    % internal navigation ('pdf bookmarks' for the table of contents,
    % internal cross-reference links, web links for URLs, etc.)
    \usepackage{hyperref}
    % The default LaTeX title has an obnoxious amount of whitespace. By default,
    % titling removes some of it. It also provides customization options.
    \usepackage{titling}
    \usepackage{longtable} % longtable support required by pandoc >1.10
    \usepackage{booktabs}  % table support for pandoc > 1.12.2
    \usepackage[inline]{enumitem} % IRkernel/repr support (it uses the enumerate* environment)
    \usepackage[normalem]{ulem} % ulem is needed to support strikethroughs (\sout)
                                % normalem makes italics be italics, not underlines
    \usepackage{mathrsfs}
    

    
    % Colors for the hyperref package
    \definecolor{urlcolor}{rgb}{0,.145,.698}
    \definecolor{linkcolor}{rgb}{.71,0.21,0.01}
    \definecolor{citecolor}{rgb}{.12,.54,.11}

    % ANSI colors
    \definecolor{ansi-black}{HTML}{3E424D}
    \definecolor{ansi-black-intense}{HTML}{282C36}
    \definecolor{ansi-red}{HTML}{E75C58}
    \definecolor{ansi-red-intense}{HTML}{B22B31}
    \definecolor{ansi-green}{HTML}{00A250}
    \definecolor{ansi-green-intense}{HTML}{007427}
    \definecolor{ansi-yellow}{HTML}{DDB62B}
    \definecolor{ansi-yellow-intense}{HTML}{B27D12}
    \definecolor{ansi-blue}{HTML}{208FFB}
    \definecolor{ansi-blue-intense}{HTML}{0065CA}
    \definecolor{ansi-magenta}{HTML}{D160C4}
    \definecolor{ansi-magenta-intense}{HTML}{A03196}
    \definecolor{ansi-cyan}{HTML}{60C6C8}
    \definecolor{ansi-cyan-intense}{HTML}{258F8F}
    \definecolor{ansi-white}{HTML}{C5C1B4}
    \definecolor{ansi-white-intense}{HTML}{A1A6B2}
    \definecolor{ansi-default-inverse-fg}{HTML}{FFFFFF}
    \definecolor{ansi-default-inverse-bg}{HTML}{000000}

    % commands and environments needed by pandoc snippets
    % extracted from the output of `pandoc -s`
    \providecommand{\tightlist}{%
      \setlength{\itemsep}{0pt}\setlength{\parskip}{0pt}}
    \DefineVerbatimEnvironment{Highlighting}{Verbatim}{commandchars=\\\{\}}
    % Add ',fontsize=\small' for more characters per line
    \newenvironment{Shaded}{}{}
    \newcommand{\KeywordTok}[1]{\textcolor[rgb]{0.00,0.44,0.13}{\textbf{{#1}}}}
    \newcommand{\DataTypeTok}[1]{\textcolor[rgb]{0.56,0.13,0.00}{{#1}}}
    \newcommand{\DecValTok}[1]{\textcolor[rgb]{0.25,0.63,0.44}{{#1}}}
    \newcommand{\BaseNTok}[1]{\textcolor[rgb]{0.25,0.63,0.44}{{#1}}}
    \newcommand{\FloatTok}[1]{\textcolor[rgb]{0.25,0.63,0.44}{{#1}}}
    \newcommand{\CharTok}[1]{\textcolor[rgb]{0.25,0.44,0.63}{{#1}}}
    \newcommand{\StringTok}[1]{\textcolor[rgb]{0.25,0.44,0.63}{{#1}}}
    \newcommand{\CommentTok}[1]{\textcolor[rgb]{0.38,0.63,0.69}{\textit{{#1}}}}
    \newcommand{\OtherTok}[1]{\textcolor[rgb]{0.00,0.44,0.13}{{#1}}}
    \newcommand{\AlertTok}[1]{\textcolor[rgb]{1.00,0.00,0.00}{\textbf{{#1}}}}
    \newcommand{\FunctionTok}[1]{\textcolor[rgb]{0.02,0.16,0.49}{{#1}}}
    \newcommand{\RegionMarkerTok}[1]{{#1}}
    \newcommand{\ErrorTok}[1]{\textcolor[rgb]{1.00,0.00,0.00}{\textbf{{#1}}}}
    \newcommand{\NormalTok}[1]{{#1}}
    
    % Additional commands for more recent versions of Pandoc
    \newcommand{\ConstantTok}[1]{\textcolor[rgb]{0.53,0.00,0.00}{{#1}}}
    \newcommand{\SpecialCharTok}[1]{\textcolor[rgb]{0.25,0.44,0.63}{{#1}}}
    \newcommand{\VerbatimStringTok}[1]{\textcolor[rgb]{0.25,0.44,0.63}{{#1}}}
    \newcommand{\SpecialStringTok}[1]{\textcolor[rgb]{0.73,0.40,0.53}{{#1}}}
    \newcommand{\ImportTok}[1]{{#1}}
    \newcommand{\DocumentationTok}[1]{\textcolor[rgb]{0.73,0.13,0.13}{\textit{{#1}}}}
    \newcommand{\AnnotationTok}[1]{\textcolor[rgb]{0.38,0.63,0.69}{\textbf{\textit{{#1}}}}}
    \newcommand{\CommentVarTok}[1]{\textcolor[rgb]{0.38,0.63,0.69}{\textbf{\textit{{#1}}}}}
    \newcommand{\VariableTok}[1]{\textcolor[rgb]{0.10,0.09,0.49}{{#1}}}
    \newcommand{\ControlFlowTok}[1]{\textcolor[rgb]{0.00,0.44,0.13}{\textbf{{#1}}}}
    \newcommand{\OperatorTok}[1]{\textcolor[rgb]{0.40,0.40,0.40}{{#1}}}
    \newcommand{\BuiltInTok}[1]{{#1}}
    \newcommand{\ExtensionTok}[1]{{#1}}
    \newcommand{\PreprocessorTok}[1]{\textcolor[rgb]{0.74,0.48,0.00}{{#1}}}
    \newcommand{\AttributeTok}[1]{\textcolor[rgb]{0.49,0.56,0.16}{{#1}}}
    \newcommand{\InformationTok}[1]{\textcolor[rgb]{0.38,0.63,0.69}{\textbf{\textit{{#1}}}}}
    \newcommand{\WarningTok}[1]{\textcolor[rgb]{0.38,0.63,0.69}{\textbf{\textit{{#1}}}}}
    
    
    % Define a nice break command that doesn't care if a line doesn't already
    % exist.
    \def\br{\hspace*{\fill} \\* }
    % Math Jax compatibility definitions
    \def\gt{>}
    \def\lt{<}
    \let\Oldtex\TeX
    \let\Oldlatex\LaTeX
    \renewcommand{\TeX}{\textrm{\Oldtex}}
    \renewcommand{\LaTeX}{\textrm{\Oldlatex}}
    % Document parameters
    % Document title
    \title{Monte-Carlo-Simulation und Single- v.s. Multi-Processing}
    
    
    
    
    
% Pygments definitions
\makeatletter
\def\PY@reset{\let\PY@it=\relax \let\PY@bf=\relax%
    \let\PY@ul=\relax \let\PY@tc=\relax%
    \let\PY@bc=\relax \let\PY@ff=\relax}
\def\PY@tok#1{\csname PY@tok@#1\endcsname}
\def\PY@toks#1+{\ifx\relax#1\empty\else%
    \PY@tok{#1}\expandafter\PY@toks\fi}
\def\PY@do#1{\PY@bc{\PY@tc{\PY@ul{%
    \PY@it{\PY@bf{\PY@ff{#1}}}}}}}
\def\PY#1#2{\PY@reset\PY@toks#1+\relax+\PY@do{#2}}

\expandafter\def\csname PY@tok@w\endcsname{\def\PY@tc##1{\textcolor[rgb]{0.73,0.73,0.73}{##1}}}
\expandafter\def\csname PY@tok@c\endcsname{\let\PY@it=\textit\def\PY@tc##1{\textcolor[rgb]{0.25,0.50,0.50}{##1}}}
\expandafter\def\csname PY@tok@cp\endcsname{\def\PY@tc##1{\textcolor[rgb]{0.74,0.48,0.00}{##1}}}
\expandafter\def\csname PY@tok@k\endcsname{\let\PY@bf=\textbf\def\PY@tc##1{\textcolor[rgb]{0.00,0.50,0.00}{##1}}}
\expandafter\def\csname PY@tok@kp\endcsname{\def\PY@tc##1{\textcolor[rgb]{0.00,0.50,0.00}{##1}}}
\expandafter\def\csname PY@tok@kt\endcsname{\def\PY@tc##1{\textcolor[rgb]{0.69,0.00,0.25}{##1}}}
\expandafter\def\csname PY@tok@o\endcsname{\def\PY@tc##1{\textcolor[rgb]{0.40,0.40,0.40}{##1}}}
\expandafter\def\csname PY@tok@ow\endcsname{\let\PY@bf=\textbf\def\PY@tc##1{\textcolor[rgb]{0.67,0.13,1.00}{##1}}}
\expandafter\def\csname PY@tok@nb\endcsname{\def\PY@tc##1{\textcolor[rgb]{0.00,0.50,0.00}{##1}}}
\expandafter\def\csname PY@tok@nf\endcsname{\def\PY@tc##1{\textcolor[rgb]{0.00,0.00,1.00}{##1}}}
\expandafter\def\csname PY@tok@nc\endcsname{\let\PY@bf=\textbf\def\PY@tc##1{\textcolor[rgb]{0.00,0.00,1.00}{##1}}}
\expandafter\def\csname PY@tok@nn\endcsname{\let\PY@bf=\textbf\def\PY@tc##1{\textcolor[rgb]{0.00,0.00,1.00}{##1}}}
\expandafter\def\csname PY@tok@ne\endcsname{\let\PY@bf=\textbf\def\PY@tc##1{\textcolor[rgb]{0.82,0.25,0.23}{##1}}}
\expandafter\def\csname PY@tok@nv\endcsname{\def\PY@tc##1{\textcolor[rgb]{0.10,0.09,0.49}{##1}}}
\expandafter\def\csname PY@tok@no\endcsname{\def\PY@tc##1{\textcolor[rgb]{0.53,0.00,0.00}{##1}}}
\expandafter\def\csname PY@tok@nl\endcsname{\def\PY@tc##1{\textcolor[rgb]{0.63,0.63,0.00}{##1}}}
\expandafter\def\csname PY@tok@ni\endcsname{\let\PY@bf=\textbf\def\PY@tc##1{\textcolor[rgb]{0.60,0.60,0.60}{##1}}}
\expandafter\def\csname PY@tok@na\endcsname{\def\PY@tc##1{\textcolor[rgb]{0.49,0.56,0.16}{##1}}}
\expandafter\def\csname PY@tok@nt\endcsname{\let\PY@bf=\textbf\def\PY@tc##1{\textcolor[rgb]{0.00,0.50,0.00}{##1}}}
\expandafter\def\csname PY@tok@nd\endcsname{\def\PY@tc##1{\textcolor[rgb]{0.67,0.13,1.00}{##1}}}
\expandafter\def\csname PY@tok@s\endcsname{\def\PY@tc##1{\textcolor[rgb]{0.73,0.13,0.13}{##1}}}
\expandafter\def\csname PY@tok@sd\endcsname{\let\PY@it=\textit\def\PY@tc##1{\textcolor[rgb]{0.73,0.13,0.13}{##1}}}
\expandafter\def\csname PY@tok@si\endcsname{\let\PY@bf=\textbf\def\PY@tc##1{\textcolor[rgb]{0.73,0.40,0.53}{##1}}}
\expandafter\def\csname PY@tok@se\endcsname{\let\PY@bf=\textbf\def\PY@tc##1{\textcolor[rgb]{0.73,0.40,0.13}{##1}}}
\expandafter\def\csname PY@tok@sr\endcsname{\def\PY@tc##1{\textcolor[rgb]{0.73,0.40,0.53}{##1}}}
\expandafter\def\csname PY@tok@ss\endcsname{\def\PY@tc##1{\textcolor[rgb]{0.10,0.09,0.49}{##1}}}
\expandafter\def\csname PY@tok@sx\endcsname{\def\PY@tc##1{\textcolor[rgb]{0.00,0.50,0.00}{##1}}}
\expandafter\def\csname PY@tok@m\endcsname{\def\PY@tc##1{\textcolor[rgb]{0.40,0.40,0.40}{##1}}}
\expandafter\def\csname PY@tok@gh\endcsname{\let\PY@bf=\textbf\def\PY@tc##1{\textcolor[rgb]{0.00,0.00,0.50}{##1}}}
\expandafter\def\csname PY@tok@gu\endcsname{\let\PY@bf=\textbf\def\PY@tc##1{\textcolor[rgb]{0.50,0.00,0.50}{##1}}}
\expandafter\def\csname PY@tok@gd\endcsname{\def\PY@tc##1{\textcolor[rgb]{0.63,0.00,0.00}{##1}}}
\expandafter\def\csname PY@tok@gi\endcsname{\def\PY@tc##1{\textcolor[rgb]{0.00,0.63,0.00}{##1}}}
\expandafter\def\csname PY@tok@gr\endcsname{\def\PY@tc##1{\textcolor[rgb]{1.00,0.00,0.00}{##1}}}
\expandafter\def\csname PY@tok@ge\endcsname{\let\PY@it=\textit}
\expandafter\def\csname PY@tok@gs\endcsname{\let\PY@bf=\textbf}
\expandafter\def\csname PY@tok@gp\endcsname{\let\PY@bf=\textbf\def\PY@tc##1{\textcolor[rgb]{0.00,0.00,0.50}{##1}}}
\expandafter\def\csname PY@tok@go\endcsname{\def\PY@tc##1{\textcolor[rgb]{0.53,0.53,0.53}{##1}}}
\expandafter\def\csname PY@tok@gt\endcsname{\def\PY@tc##1{\textcolor[rgb]{0.00,0.27,0.87}{##1}}}
\expandafter\def\csname PY@tok@err\endcsname{\def\PY@bc##1{\setlength{\fboxsep}{0pt}\fcolorbox[rgb]{1.00,0.00,0.00}{1,1,1}{\strut ##1}}}
\expandafter\def\csname PY@tok@kc\endcsname{\let\PY@bf=\textbf\def\PY@tc##1{\textcolor[rgb]{0.00,0.50,0.00}{##1}}}
\expandafter\def\csname PY@tok@kd\endcsname{\let\PY@bf=\textbf\def\PY@tc##1{\textcolor[rgb]{0.00,0.50,0.00}{##1}}}
\expandafter\def\csname PY@tok@kn\endcsname{\let\PY@bf=\textbf\def\PY@tc##1{\textcolor[rgb]{0.00,0.50,0.00}{##1}}}
\expandafter\def\csname PY@tok@kr\endcsname{\let\PY@bf=\textbf\def\PY@tc##1{\textcolor[rgb]{0.00,0.50,0.00}{##1}}}
\expandafter\def\csname PY@tok@bp\endcsname{\def\PY@tc##1{\textcolor[rgb]{0.00,0.50,0.00}{##1}}}
\expandafter\def\csname PY@tok@fm\endcsname{\def\PY@tc##1{\textcolor[rgb]{0.00,0.00,1.00}{##1}}}
\expandafter\def\csname PY@tok@vc\endcsname{\def\PY@tc##1{\textcolor[rgb]{0.10,0.09,0.49}{##1}}}
\expandafter\def\csname PY@tok@vg\endcsname{\def\PY@tc##1{\textcolor[rgb]{0.10,0.09,0.49}{##1}}}
\expandafter\def\csname PY@tok@vi\endcsname{\def\PY@tc##1{\textcolor[rgb]{0.10,0.09,0.49}{##1}}}
\expandafter\def\csname PY@tok@vm\endcsname{\def\PY@tc##1{\textcolor[rgb]{0.10,0.09,0.49}{##1}}}
\expandafter\def\csname PY@tok@sa\endcsname{\def\PY@tc##1{\textcolor[rgb]{0.73,0.13,0.13}{##1}}}
\expandafter\def\csname PY@tok@sb\endcsname{\def\PY@tc##1{\textcolor[rgb]{0.73,0.13,0.13}{##1}}}
\expandafter\def\csname PY@tok@sc\endcsname{\def\PY@tc##1{\textcolor[rgb]{0.73,0.13,0.13}{##1}}}
\expandafter\def\csname PY@tok@dl\endcsname{\def\PY@tc##1{\textcolor[rgb]{0.73,0.13,0.13}{##1}}}
\expandafter\def\csname PY@tok@s2\endcsname{\def\PY@tc##1{\textcolor[rgb]{0.73,0.13,0.13}{##1}}}
\expandafter\def\csname PY@tok@sh\endcsname{\def\PY@tc##1{\textcolor[rgb]{0.73,0.13,0.13}{##1}}}
\expandafter\def\csname PY@tok@s1\endcsname{\def\PY@tc##1{\textcolor[rgb]{0.73,0.13,0.13}{##1}}}
\expandafter\def\csname PY@tok@mb\endcsname{\def\PY@tc##1{\textcolor[rgb]{0.40,0.40,0.40}{##1}}}
\expandafter\def\csname PY@tok@mf\endcsname{\def\PY@tc##1{\textcolor[rgb]{0.40,0.40,0.40}{##1}}}
\expandafter\def\csname PY@tok@mh\endcsname{\def\PY@tc##1{\textcolor[rgb]{0.40,0.40,0.40}{##1}}}
\expandafter\def\csname PY@tok@mi\endcsname{\def\PY@tc##1{\textcolor[rgb]{0.40,0.40,0.40}{##1}}}
\expandafter\def\csname PY@tok@il\endcsname{\def\PY@tc##1{\textcolor[rgb]{0.40,0.40,0.40}{##1}}}
\expandafter\def\csname PY@tok@mo\endcsname{\def\PY@tc##1{\textcolor[rgb]{0.40,0.40,0.40}{##1}}}
\expandafter\def\csname PY@tok@ch\endcsname{\let\PY@it=\textit\def\PY@tc##1{\textcolor[rgb]{0.25,0.50,0.50}{##1}}}
\expandafter\def\csname PY@tok@cm\endcsname{\let\PY@it=\textit\def\PY@tc##1{\textcolor[rgb]{0.25,0.50,0.50}{##1}}}
\expandafter\def\csname PY@tok@cpf\endcsname{\let\PY@it=\textit\def\PY@tc##1{\textcolor[rgb]{0.25,0.50,0.50}{##1}}}
\expandafter\def\csname PY@tok@c1\endcsname{\let\PY@it=\textit\def\PY@tc##1{\textcolor[rgb]{0.25,0.50,0.50}{##1}}}
\expandafter\def\csname PY@tok@cs\endcsname{\let\PY@it=\textit\def\PY@tc##1{\textcolor[rgb]{0.25,0.50,0.50}{##1}}}

\def\PYZbs{\char`\\}
\def\PYZus{\char`\_}
\def\PYZob{\char`\{}
\def\PYZcb{\char`\}}
\def\PYZca{\char`\^}
\def\PYZam{\char`\&}
\def\PYZlt{\char`\<}
\def\PYZgt{\char`\>}
\def\PYZsh{\char`\#}
\def\PYZpc{\char`\%}
\def\PYZdl{\char`\$}
\def\PYZhy{\char`\-}
\def\PYZsq{\char`\'}
\def\PYZdq{\char`\"}
\def\PYZti{\char`\~}
% for compatibility with earlier versions
\def\PYZat{@}
\def\PYZlb{[}
\def\PYZrb{]}
\makeatother


    % For linebreaks inside Verbatim environment from package fancyvrb. 
    \makeatletter
        \newbox\Wrappedcontinuationbox 
        \newbox\Wrappedvisiblespacebox 
        \newcommand*\Wrappedvisiblespace {\textcolor{red}{\textvisiblespace}} 
        \newcommand*\Wrappedcontinuationsymbol {\textcolor{red}{\llap{\tiny$\m@th\hookrightarrow$}}} 
        \newcommand*\Wrappedcontinuationindent {3ex } 
        \newcommand*\Wrappedafterbreak {\kern\Wrappedcontinuationindent\copy\Wrappedcontinuationbox} 
        % Take advantage of the already applied Pygments mark-up to insert 
        % potential linebreaks for TeX processing. 
        %        {, <, #, %, $, ' and ": go to next line. 
        %        _, }, ^, &, >, - and ~: stay at end of broken line. 
        % Use of \textquotesingle for straight quote. 
        \newcommand*\Wrappedbreaksatspecials {% 
            \def\PYGZus{\discretionary{\char`\_}{\Wrappedafterbreak}{\char`\_}}% 
            \def\PYGZob{\discretionary{}{\Wrappedafterbreak\char`\{}{\char`\{}}% 
            \def\PYGZcb{\discretionary{\char`\}}{\Wrappedafterbreak}{\char`\}}}% 
            \def\PYGZca{\discretionary{\char`\^}{\Wrappedafterbreak}{\char`\^}}% 
            \def\PYGZam{\discretionary{\char`\&}{\Wrappedafterbreak}{\char`\&}}% 
            \def\PYGZlt{\discretionary{}{\Wrappedafterbreak\char`\<}{\char`\<}}% 
            \def\PYGZgt{\discretionary{\char`\>}{\Wrappedafterbreak}{\char`\>}}% 
            \def\PYGZsh{\discretionary{}{\Wrappedafterbreak\char`\#}{\char`\#}}% 
            \def\PYGZpc{\discretionary{}{\Wrappedafterbreak\char`\%}{\char`\%}}% 
            \def\PYGZdl{\discretionary{}{\Wrappedafterbreak\char`\$}{\char`\$}}% 
            \def\PYGZhy{\discretionary{\char`\-}{\Wrappedafterbreak}{\char`\-}}% 
            \def\PYGZsq{\discretionary{}{\Wrappedafterbreak\textquotesingle}{\textquotesingle}}% 
            \def\PYGZdq{\discretionary{}{\Wrappedafterbreak\char`\"}{\char`\"}}% 
            \def\PYGZti{\discretionary{\char`\~}{\Wrappedafterbreak}{\char`\~}}% 
        } 
        % Some characters . , ; ? ! / are not pygmentized. 
        % This macro makes them "active" and they will insert potential linebreaks 
        \newcommand*\Wrappedbreaksatpunct {% 
            \lccode`\~`\.\lowercase{\def~}{\discretionary{\hbox{\char`\.}}{\Wrappedafterbreak}{\hbox{\char`\.}}}% 
            \lccode`\~`\,\lowercase{\def~}{\discretionary{\hbox{\char`\,}}{\Wrappedafterbreak}{\hbox{\char`\,}}}% 
            \lccode`\~`\;\lowercase{\def~}{\discretionary{\hbox{\char`\;}}{\Wrappedafterbreak}{\hbox{\char`\;}}}% 
            \lccode`\~`\:\lowercase{\def~}{\discretionary{\hbox{\char`\:}}{\Wrappedafterbreak}{\hbox{\char`\:}}}% 
            \lccode`\~`\?\lowercase{\def~}{\discretionary{\hbox{\char`\?}}{\Wrappedafterbreak}{\hbox{\char`\?}}}% 
            \lccode`\~`\!\lowercase{\def~}{\discretionary{\hbox{\char`\!}}{\Wrappedafterbreak}{\hbox{\char`\!}}}% 
            \lccode`\~`\/\lowercase{\def~}{\discretionary{\hbox{\char`\/}}{\Wrappedafterbreak}{\hbox{\char`\/}}}% 
            \catcode`\.\active
            \catcode`\,\active 
            \catcode`\;\active
            \catcode`\:\active
            \catcode`\?\active
            \catcode`\!\active
            \catcode`\/\active 
            \lccode`\~`\~ 	
        }
    \makeatother

    \let\OriginalVerbatim=\Verbatim
    \makeatletter
    \renewcommand{\Verbatim}[1][1]{%
        %\parskip\z@skip
        \sbox\Wrappedcontinuationbox {\Wrappedcontinuationsymbol}%
        \sbox\Wrappedvisiblespacebox {\FV@SetupFont\Wrappedvisiblespace}%
        \def\FancyVerbFormatLine ##1{\hsize\linewidth
            \vtop{\raggedright\hyphenpenalty\z@\exhyphenpenalty\z@
                \doublehyphendemerits\z@\finalhyphendemerits\z@
                \strut ##1\strut}%
        }%
        % If the linebreak is at a space, the latter will be displayed as visible
        % space at end of first line, and a continuation symbol starts next line.
        % Stretch/shrink are however usually zero for typewriter font.
        \def\FV@Space {%
            \nobreak\hskip\z@ plus\fontdimen3\font minus\fontdimen4\font
            \discretionary{\copy\Wrappedvisiblespacebox}{\Wrappedafterbreak}
            {\kern\fontdimen2\font}%
        }%
        
        % Allow breaks at special characters using \PYG... macros.
        \Wrappedbreaksatspecials
        % Breaks at punctuation characters . , ; ? ! and / need catcode=\active 	
        \OriginalVerbatim[#1,codes*=\Wrappedbreaksatpunct]%
    }
    \makeatother

    % Exact colors from NB
    \definecolor{incolor}{HTML}{303F9F}
    \definecolor{outcolor}{HTML}{D84315}
    \definecolor{cellborder}{HTML}{CFCFCF}
    \definecolor{cellbackground}{HTML}{F7F7F7}
    
    % prompt
    \makeatletter
    \newcommand{\boxspacing}{\kern\kvtcb@left@rule\kern\kvtcb@boxsep}
    \makeatother
    \newcommand{\prompt}[4]{
        \ttfamily\llap{{\color{#2}[#3]:\hspace{3pt}#4}}\vspace{-\baselineskip}
    }
    

    
    % Prevent overflowing lines due to hard-to-break entities
    \sloppy 
    % Setup hyperref package
    \hypersetup{
      breaklinks=true,  % so long urls are correctly broken across lines
      colorlinks=true,
      urlcolor=urlcolor,
      linkcolor=linkcolor,
      citecolor=citecolor,
      }
    % Slightly bigger margins than the latex defaults
    
    \geometry{verbose,tmargin=1in,bmargin=1in,lmargin=1in,rmargin=1in}
    
    

\begin{document}
    
    \maketitle
    
    

    
    \hypertarget{monte-carlo-simulation-und-single--v.s.-multi-processing}{%
\section{Monte-Carlo-Simulation und Single- v.s.
Multi-Processing}\label{monte-carlo-simulation-und-single--v.s.-multi-processing}}

© Thomas Robert Holy 2020 Version 1.0 Visit me on GitHub:
https://github.com/trh0ly \#\# Grundlegende Einstellungen: Zunächst
müssen die notwendigen Pakete (auch Module) importiert werden, damit auf
diese zugegriffen werden kann.

    \begin{tcolorbox}[breakable, size=fbox, boxrule=1pt, pad at break*=1mm,colback=cellbackground, colframe=cellborder]
\prompt{In}{incolor}{1}{\boxspacing}
\begin{Verbatim}[commandchars=\\\{\}]
\PY{k+kn}{import} \PY{n+nn}{pandas} \PY{k}{as} \PY{n+nn}{pd}
\PY{k+kn}{import} \PY{n+nn}{numpy} \PY{k}{as} \PY{n+nn}{np}
\PY{k+kn}{from} \PY{n+nn}{riskmeasure\PYZus{}module} \PY{k}{import} \PY{n}{risk\PYZus{}measure} \PY{k}{as} \PY{n}{rm}
\PY{k+kn}{import} \PY{n+nn}{math}
\PY{k+kn}{from} \PY{n+nn}{multiprocessing} \PY{k}{import} \PY{n}{Process}
\PY{k+kn}{from} \PY{n+nn}{multiprocessing} \PY{k}{import} \PY{n}{Manager}
\PY{k+kn}{from} \PY{n+nn}{multiprocessing}\PY{n+nn}{.}\PY{n+nn}{pool} \PY{k}{import} \PY{n}{Pool}
\PY{k+kn}{import} \PY{n+nn}{datetime} \PY{k}{as} \PY{n+nn}{dt}
\PY{k+kn}{import} \PY{n+nn}{operator}
\PY{k+kn}{from} \PY{n+nn}{IPython}\PY{n+nn}{.}\PY{n+nn}{core}\PY{n+nn}{.}\PY{n+nn}{display} \PY{k}{import} \PY{n}{display}\PY{p}{,} \PY{n}{HTML}
\PY{k+kn}{from} \PY{n+nn}{multi\PYZus{}1b} \PY{k}{import} \PY{n}{repeat\PYZus{}parallel}\PY{p}{,} \PY{n}{RM\PYZus{}frame\PYZus{}func}\PY{p}{,} \PY{n}{plotty\PYZus{}func}
\PY{k+kn}{from} \PY{n+nn}{Monte\PYZus{}Carlo\PYZus{}Simulation\PYZus{}lite} \PY{k}{import} \PY{n}{var\PYZus{}covar\PYZus{}matrix\PYZus{}func}\PY{p}{,} \PY{n}{var\PYZus{}func}\PY{p}{,} \PY{n}{cholesky\PYZus{}func}\PY{p}{,} \PY{n}{verteilung\PYZus{}func}\PY{p}{,} \PY{n}{copula\PYZus{}sim}\PY{p}{,} \PY{n}{hist\PYZus{}func}
\end{Verbatim}
\end{tcolorbox}

    Anschließend werden Einstellungen definiert, die die Formatierung der
Ausgaben betreffen. Hierfür wird das Modul \texttt{operator} genutzt.
Außerdem wird die Größe der Grafiken modifiziert, welche später
angezeigt werden sollen.

    \begin{tcolorbox}[breakable, size=fbox, boxrule=1pt, pad at break*=1mm,colback=cellbackground, colframe=cellborder]
\prompt{In}{incolor}{2}{\boxspacing}
\begin{Verbatim}[commandchars=\\\{\}]
\PY{o}{\PYZpc{}\PYZpc{}javascript}
\PY{n+nx}{IPython}\PY{p}{.}\PY{n+nx}{OutputArea}\PY{p}{.}\PY{n+nx}{auto\PYZus{}scroll\PYZus{}threshold} \PY{o}{=} \PY{l+m+mi}{9999}\PY{p}{;}
\end{Verbatim}
\end{tcolorbox}

    
    \begin{verbatim}
<IPython.core.display.Javascript object>
    \end{verbatim}

    
    \begin{tcolorbox}[breakable, size=fbox, boxrule=1pt, pad at break*=1mm,colback=cellbackground, colframe=cellborder]
\prompt{In}{incolor}{3}{\boxspacing}
\begin{Verbatim}[commandchars=\\\{\}]
\PY{n}{display}\PY{p}{(}\PY{n}{HTML}\PY{p}{(}\PY{l+s+s2}{\PYZdq{}}\PY{l+s+s2}{\PYZlt{}style\PYZgt{}.container }\PY{l+s+s2}{\PYZob{}}\PY{l+s+s2}{ width:100}\PY{l+s+s2}{\PYZpc{}}\PY{l+s+s2}{ !important; \PYZcb{}\PYZlt{}/style\PYZgt{}}\PY{l+s+s2}{\PYZdq{}}\PY{p}{)}\PY{p}{)}
\end{Verbatim}
\end{tcolorbox}

    
    \begin{verbatim}
<IPython.core.display.HTML object>
    \end{verbatim}

    
    \hypertarget{definition-einiger-funktionen-zum-vergleich-von-single--v.s.-multi-processing}{%
\subsection{Definition einiger Funktionen zum Vergleich von Single- v.s.
Multi-Processing}\label{definition-einiger-funktionen-zum-vergleich-von-single--v.s.-multi-processing}}

\hypertarget{definition-der-multi-processing-funktion}{%
\subsubsection{Definition der
Multi-Processing-Funktion}\label{definition-der-multi-processing-funktion}}

    \begin{tcolorbox}[breakable, size=fbox, boxrule=1pt, pad at break*=1mm,colback=cellbackground, colframe=cellborder]
\prompt{In}{incolor}{4}{\boxspacing}
\begin{Verbatim}[commandchars=\\\{\}]
\PY{c+c1}{\PYZsh{} Definiere die Funktion \PYZdq{}run\PYZdq{} in welcher die Parallelisierung der Monte\PYZhy{}Carlo\PYZhy{}Simulation auf \PYZdq{}Multi\PYZus{}1b\PYZdq{} gemanaget wird}
\PY{k}{def} \PY{n+nf}{run}\PY{p}{(}\PY{n}{runs\PYZus{}func}\PY{p}{,} \PY{n}{runs\PYZus{}sim}\PY{p}{,} \PY{n}{rand\PYZus{}x}\PY{p}{,} \PY{n}{rand\PYZus{}y}\PY{p}{,} \PY{n}{mu}\PY{p}{,} \PY{n}{std\PYZus{}list}\PY{p}{,} \PY{n}{corr\PYZus{}list}\PY{p}{,} \PY{n}{alpha}\PY{p}{,} \PY{n}{gamma}\PY{p}{,} \PY{n}{draw}\PY{o}{=}\PY{k+kc}{False}\PY{p}{,} \PY{n}{SCREEN\PYZus{}WIDTH}\PY{o}{=}\PY{l+m+mi}{115}\PY{p}{)}\PY{p}{:}
    \PY{n}{RM\PYZus{}VaR\PYZus{}list} \PY{o}{=} \PY{p}{[}\PY{p}{]}
    \PY{n}{RM\PYZus{}CVaR\PYZus{}list} \PY{o}{=} \PY{p}{[}\PY{p}{]}
    \PY{n}{RM\PYZus{}PSRM\PYZus{}list} \PY{o}{=} \PY{p}{[}\PY{p}{]}
    \PY{n}{mega\PYZus{}summe\PYZus{}list} \PY{o}{=} \PY{p}{[}\PY{p}{]}

    \PY{n}{centered}\PY{o}{=}\PY{n}{operator}\PY{o}{.}\PY{n}{methodcaller}\PY{p}{(}\PY{l+s+s1}{\PYZsq{}}\PY{l+s+s1}{center}\PY{l+s+s1}{\PYZsq{}}\PY{p}{,} \PY{n}{SCREEN\PYZus{}WIDTH}\PY{p}{)}

    \PY{n}{start} \PY{o}{=} \PY{n}{dt}\PY{o}{.}\PY{n}{datetime}\PY{o}{.}\PY{n}{now}\PY{p}{(}\PY{p}{)} \PY{c+c1}{\PYZsh{} Startpunkt Zeitmessung}
    \PY{k}{with} \PY{n}{Manager}\PY{p}{(}\PY{p}{)} \PY{k}{as} \PY{n}{manager}\PY{p}{:} \PY{c+c1}{\PYZsh{} Verwendung Funktion \PYZdq{}Manager()\PYZdq{} aus \PYZdq{}multiprocessing\PYZdq{} als Manager der geteilten Listen zwischen den parallel laufenden Simulationen}
        \PY{n}{shared\PYZus{}list} \PY{o}{=} \PY{n}{manager}\PY{o}{.}\PY{n}{list}\PY{p}{(}\PY{p}{)} \PY{c+c1}{\PYZsh{} Legt die leere Liste \PYZdq{}shared\PYZus{}list\PYZdq{} an, welche zwischen den parallel laufenden Simulationen geteilt wird und alle Realisationen der einzelnen Simulationsläufe enthält}
        \PY{n}{VaR\PYZus{}list} \PY{o}{=} \PY{n}{manager}\PY{o}{.}\PY{n}{list}\PY{p}{(}\PY{p}{)} \PY{c+c1}{\PYZsh{} Legt die leere Liste \PYZdq{}VaR\PYZus{}list\PYZdq{} an, welche zwischen den parallel laufenden Simulationen geteilt wird und alle VaR enthält}
        \PY{n}{CVaR\PYZus{}list} \PY{o}{=} \PY{n}{manager}\PY{o}{.}\PY{n}{list}\PY{p}{(}\PY{p}{)} \PY{c+c1}{\PYZsh{} Legt die leere Liste \PYZdq{}CVaR\PYZus{}list\PYZdq{} an, welche zwischen den parallel laufenden Simulationen geteilt wird und alle CVaR enthält}
        \PY{n}{PSRM\PYZus{}list} \PY{o}{=} \PY{n}{manager}\PY{o}{.}\PY{n}{list}\PY{p}{(}\PY{p}{)} \PY{c+c1}{\PYZsh{} Legt die leere Liste \PYZdq{}PSRM\PYZus{}list\PYZdq{} an, welche zwischen den parallel laufenden Simulationen geteilt wird und alle P\PYZhy{}SRM enthält}
        \PY{n}{processes} \PY{o}{=} \PY{p}{[}\PY{p}{]} \PY{c+c1}{\PYZsh{} Legt die leere Liste \PYZdq{}processes\PYZdq{} an, in welcher die auszuführenden Prozesse abgelegt werden}
        \PY{c+c1}{\PYZsh{} Für jedes i in der Range 0 bis runs\PYZus{}func (Simulationsläufe)...}
        \PY{k}{for} \PY{n}{i} \PY{o+ow}{in} \PY{n+nb}{range}\PY{p}{(}\PY{n}{runs\PYZus{}func}\PY{p}{)}\PY{p}{:}
            \PY{n}{p} \PY{o}{=} \PY{n}{Process}\PY{p}{(}\PY{n}{target}\PY{o}{=}\PY{n}{repeat\PYZus{}parallel}\PY{p}{,} \PY{n}{args}\PY{o}{=}\PY{p}{(}\PY{n}{runs\PYZus{}sim}\PY{p}{,} \PY{n}{rand\PYZus{}x}\PY{p}{,} \PY{n}{rand\PYZus{}y}\PY{p}{,} \PY{n}{mu}\PY{p}{,} \PY{n}{std\PYZus{}list}\PY{p}{,} \PY{n}{corr\PYZus{}list}\PY{p}{,} \PY{n}{alpha}\PY{p}{,} \PY{n}{gamma}\PY{p}{,} \PY{n}{shared\PYZus{}list}\PY{p}{,} \PY{n}{VaR\PYZus{}list}\PY{p}{,} \PY{n}{CVaR\PYZus{}list}\PY{p}{,} \PY{n}{PSRM\PYZus{}list}\PY{p}{,} \PY{n}{i}\PY{p}{)}\PY{p}{)} \PY{c+c1}{\PYZsh{} Erstelle den Prozess \PYZdq{}p\PYZdq{}, welcher die Funktion \PYZdq{}repeat\PYZus{}parallel\PYZdq{} ausführt mit den gebenen Paramtertn}
            \PY{n}{p}\PY{o}{.}\PY{n}{start}\PY{p}{(}\PY{p}{)} \PY{c+c1}{\PYZsh{} Starte den Prozess}
            \PY{n}{processes}\PY{o}{.}\PY{n}{append}\PY{p}{(}\PY{n}{p}\PY{p}{)} \PY{c+c1}{\PYZsh{} Füge den Prozess \PYZdq{}p\PYZdq{} der Liste \PYZdq{}processes\PYZdq{} an}
        \PY{c+c1}{\PYZsh{} Für jenden Prozess \PYZdq{}p\PYZdq{} in \PYZdq{}processes\PYZdq{} führe due Funktion join() aus (Clean Exit Process)}
        \PY{k}{for} \PY{n}{p} \PY{o+ow}{in} \PY{n}{processes}\PY{p}{:}
            \PY{n}{p}\PY{o}{.}\PY{n}{join}\PY{p}{(}\PY{p}{)}
        \PY{n}{RM\PYZus{}VaR\PYZus{}list} \PY{o}{+}\PY{o}{=} \PY{n}{VaR\PYZus{}list} \PY{c+c1}{\PYZsh{} Füge der \PYZdq{}RM\PYZus{}VaR\PYZus{}list\PYZdq{} den jeweiligen VaR an}
        \PY{n}{RM\PYZus{}CVaR\PYZus{}list} \PY{o}{+}\PY{o}{=} \PY{n}{CVaR\PYZus{}list} \PY{c+c1}{\PYZsh{} s.o.}
        \PY{n}{RM\PYZus{}PSRM\PYZus{}list} \PY{o}{+}\PY{o}{=} \PY{n}{PSRM\PYZus{}list} \PY{c+c1}{\PYZsh{} s.o.}
        \PY{c+c1}{\PYZsh{} Sofern draw == \PYZsq{}True\PYZsq{} füge die \PYZdq{}shared\PYZus{}list\PYZdq{} in der \PYZdq{}mega\PYZus{}summe\PYZus{}list\PYZdq{} an}
        \PY{k}{if} \PY{n}{draw} \PY{o}{==} \PY{k+kc}{True}\PY{p}{:}
            \PY{n}{mega\PYZus{}summe\PYZus{}list} \PY{o}{+}\PY{o}{=} \PY{n}{shared\PYZus{}list}
    \PY{n}{end} \PY{o}{=} \PY{n}{dt}\PY{o}{.}\PY{n}{datetime}\PY{o}{.}\PY{n}{now}\PY{p}{(}\PY{p}{)} \PY{c+c1}{\PYZsh{} nedpunkt für die Zeitmessung}
    \PY{n}{RM\PYZus{}frame\PYZus{}func}\PY{p}{(}\PY{n}{runs\PYZus{}sim}\PY{p}{,} \PY{n}{runs\PYZus{}func}\PY{p}{,} \PY{n}{RM\PYZus{}VaR\PYZus{}list}\PY{p}{,} \PY{n}{RM\PYZus{}CVaR\PYZus{}list}\PY{p}{,} \PY{n}{RM\PYZus{}PSRM\PYZus{}list}\PY{p}{,} \PY{n}{SCREEN\PYZus{}WIDTH}\PY{p}{,} \PY{n}{centered}\PY{p}{)} \PY{c+c1}{\PYZsh{} Ausführen der Funktion, welche den DataFrame mit den Risikomaßen ausgibt}
    \PY{c+c1}{\PYZsh{} Sofern draw == \PYZsq{}True\PYZsq{} führe die Funktion aus, welche die Verteilungfunktionen der Simulationsläufe plottet}
    \PY{k}{if} \PY{n}{draw} \PY{o}{==} \PY{k+kc}{True}\PY{p}{:}
        \PY{n}{plotty\PYZus{}func}\PY{p}{(}\PY{n}{runs\PYZus{}sim}\PY{p}{,} \PY{n}{runs\PYZus{}func}\PY{p}{,} \PY{n}{mega\PYZus{}summe\PYZus{}list}\PY{p}{)}
    \PY{n+nb}{print}\PY{p}{(}\PY{n}{end}\PY{o}{\PYZhy{}}\PY{n}{start}\PY{p}{)} \PY{c+c1}{\PYZsh{} Gibt das Zeit\PYZhy{}Delta zurück}
 
\PY{c+c1}{\PYZsh{}\PYZhy{}\PYZhy{}\PYZhy{}\PYZhy{}\PYZhy{}\PYZhy{}\PYZhy{}\PYZhy{}\PYZhy{}\PYZhy{}\PYZhy{}\PYZhy{}\PYZhy{}\PYZhy{}\PYZhy{}\PYZhy{}\PYZhy{}\PYZhy{}\PYZhy{}\PYZhy{}\PYZhy{}\PYZhy{}\PYZhy{}\PYZhy{}\PYZhy{}\PYZhy{}\PYZhy{}\PYZhy{}\PYZhy{}\PYZhy{}\PYZhy{}\PYZhy{}\PYZhy{}\PYZhy{}\PYZhy{}\PYZhy{}\PYZhy{}\PYZhy{}\PYZhy{}\PYZhy{}\PYZhy{}\PYZhy{}\PYZhy{}}
\PY{c+c1}{\PYZsh{} Vereinfachung}
\PY{k}{def} \PY{n+nf}{multithreading}\PY{p}{(}\PY{n}{n}\PY{p}{,} \PY{n}{rand\PYZus{}x}\PY{p}{,} \PY{n}{rand\PYZus{}y}\PY{p}{,} \PY{n}{var\PYZus{}x}\PY{p}{,} \PY{n}{var\PYZus{}y}\PY{p}{,} \PY{n}{corr\PYZus{}list}\PY{p}{,} \PY{n}{std\PYZus{}list}\PY{p}{,} \PY{n}{mu}\PY{p}{,} \PY{n}{alpha}\PY{p}{,} \PY{n}{gamma}\PY{p}{,} \PY{n}{runs\PYZus{}func}\PY{p}{,} \PY{n}{runs\PYZus{}sim}\PY{p}{)}\PY{p}{:}
    \PY{n}{run}\PY{p}{(}\PY{n}{runs\PYZus{}func}\PY{p}{,} \PY{n}{runs\PYZus{}sim}\PY{p}{,} \PY{n}{rand\PYZus{}x}\PY{p}{,} \PY{n}{rand\PYZus{}y}\PY{p}{,} \PY{n}{mu}\PY{p}{,} \PY{n}{std\PYZus{}list}\PY{p}{,} \PY{n}{corr\PYZus{}list}\PY{p}{,} \PY{n}{alpha}\PY{p}{,} \PY{n}{gamma}\PY{p}{,} \PY{n}{draw}\PY{o}{=}\PY{k+kc}{False}\PY{p}{)}    
\end{Verbatim}
\end{tcolorbox}

    \hypertarget{definition-der-sinlge-processing-funktion}{%
\subsubsection{Definition der
Sinlge-Processing-Funktion}\label{definition-der-sinlge-processing-funktion}}

    \begin{tcolorbox}[breakable, size=fbox, boxrule=1pt, pad at break*=1mm,colback=cellbackground, colframe=cellborder]
\prompt{In}{incolor}{5}{\boxspacing}
\begin{Verbatim}[commandchars=\\\{\}]
\PY{k}{def} \PY{n+nf}{single\PYZus{}thread}\PY{p}{(}\PY{n}{n}\PY{p}{,} \PY{n}{rand\PYZus{}x}\PY{p}{,} \PY{n}{rand\PYZus{}y}\PY{p}{,} \PY{n}{var\PYZus{}x}\PY{p}{,} \PY{n}{var\PYZus{}y}\PY{p}{,} \PY{n}{corr\PYZus{}list}\PY{p}{,} \PY{n}{std\PYZus{}list}\PY{p}{,} \PY{n}{mu}\PY{p}{,} \PY{n}{alpha}\PY{p}{,} \PY{n}{gamma}\PY{p}{,} \PY{n}{runs\PYZus{}func}\PY{p}{,} \PY{n}{runs\PYZus{}sim}\PY{p}{)}\PY{p}{:}
   
    \PY{n}{SCREEN\PYZus{}WIDTH} \PY{o}{=} \PY{l+m+mi}{115}
    \PY{n}{centered} \PY{o}{=} \PY{n}{operator}\PY{o}{.}\PY{n}{methodcaller}\PY{p}{(}\PY{l+s+s1}{\PYZsq{}}\PY{l+s+s1}{center}\PY{l+s+s1}{\PYZsq{}}\PY{p}{,} \PY{n}{SCREEN\PYZus{}WIDTH}\PY{p}{)}
    
    \PY{n}{RM\PYZus{}VaR\PYZus{}list}\PY{p}{,} \PY{n}{RM\PYZus{}CVaR\PYZus{}list} \PY{o}{=} \PY{p}{[}\PY{p}{]}\PY{p}{,} \PY{p}{[}\PY{p}{]}
    \PY{n}{RM\PYZus{}PSRM\PYZus{}list}\PY{p}{,} \PY{n}{mega\PYZus{}summe\PYZus{}list} \PY{o}{=} \PY{p}{[}\PY{p}{]}\PY{p}{,} \PY{p}{[}\PY{p}{]}

    \PY{c+c1}{\PYZsh{} Führe die Simulation \PYZdq{}runs\PYZus{}func mal\PYZdq{} durch und speichere die Ergebnisse in der jeweiligen Liste}
    \PY{n}{start} \PY{o}{=} \PY{n}{dt}\PY{o}{.}\PY{n}{datetime}\PY{o}{.}\PY{n}{now}\PY{p}{(}\PY{p}{)} 
    \PY{k}{for} \PY{n}{i} \PY{o+ow}{in} \PY{n+nb}{range}\PY{p}{(}\PY{l+m+mi}{0}\PY{p}{,} \PY{n}{runs\PYZus{}func}\PY{p}{)}\PY{p}{:}        
        \PY{n}{\PYZus{}}\PY{p}{,} \PY{n}{\PYZus{}}\PY{p}{,} \PY{n}{\PYZus{}}\PY{p}{,} \PY{n}{total\PYZus{}summe\PYZus{}liste} \PY{o}{=} \PY{n}{copula\PYZus{}sim}\PY{p}{(}\PY{n}{runs\PYZus{}sim}\PY{p}{,} \PY{n}{rand\PYZus{}x}\PY{p}{,} \PY{n}{rand\PYZus{}y}\PY{p}{,} \PY{n}{mu}\PY{p}{,} \PY{n}{std\PYZus{}list}\PY{p}{,} \PY{n}{corr\PYZus{}list}\PY{p}{,} \PY{n}{full\PYZus{}log}\PY{o}{=}\PY{k+kc}{False}\PY{p}{)}
        \PY{n}{mega\PYZus{}summe\PYZus{}list} \PY{o}{+}\PY{o}{=} \PY{n}{total\PYZus{}summe\PYZus{}liste}
        \PY{n}{x} \PY{o}{=} \PY{n}{rm}\PY{p}{(}\PY{n}{total\PYZus{}summe\PYZus{}liste}\PY{p}{,} \PY{n}{alpha}\PY{p}{,} \PY{n}{gamma}\PY{p}{)}    
        \PY{n}{RM\PYZus{}VaR\PYZus{}list}\PY{o}{.}\PY{n}{append}\PY{p}{(}\PY{n}{x}\PY{o}{.}\PY{n}{VaR}\PY{p}{(}\PY{p}{)}\PY{p}{)}    
        \PY{n}{RM\PYZus{}CVaR\PYZus{}list}\PY{o}{.}\PY{n}{append}\PY{p}{(}\PY{n}{x}\PY{o}{.}\PY{n}{CVaR}\PY{p}{(}\PY{p}{)}\PY{p}{)}
        \PY{n}{RM\PYZus{}PSRM\PYZus{}list}\PY{o}{.}\PY{n}{append}\PY{p}{(}\PY{n}{x}\PY{o}{.}\PY{n}{Power}\PY{p}{(}\PY{p}{)}\PY{p}{)}
    \PY{n}{end} \PY{o}{=} \PY{n}{dt}\PY{o}{.}\PY{n}{datetime}\PY{o}{.}\PY{n}{now}\PY{p}{(}\PY{p}{)}
    \PY{n+nb}{print}\PY{p}{(}\PY{n}{end}\PY{o}{\PYZhy{}}\PY{n}{start}\PY{p}{)}     
    
    \PY{c+c1}{\PYZsh{}\PYZhy{}\PYZhy{}\PYZhy{}\PYZhy{}\PYZhy{}\PYZhy{}\PYZhy{}\PYZhy{}\PYZhy{}\PYZhy{}\PYZhy{}\PYZhy{}\PYZhy{}\PYZhy{}\PYZhy{}\PYZhy{}\PYZhy{}\PYZhy{}\PYZhy{}\PYZhy{}\PYZhy{}\PYZhy{}\PYZhy{}\PYZhy{}\PYZhy{}\PYZhy{}\PYZhy{}}
    \PY{c+c1}{\PYZsh{} Erzeuge ein DataFrame mit den Simulationsvergebnissen}
    \PY{c+c1}{\PYZsh{} und deren prozentualen Änderung vom jeweils vorherigen Ergebnis}
    \PY{n}{RM\PYZus{}frame} \PY{o}{=} \PY{n}{pd}\PY{o}{.}\PY{n}{DataFrame}\PY{p}{(}\PY{p}{)}
    \PY{n}{RM\PYZus{}frame}\PY{p}{[}\PY{l+s+s1}{\PYZsq{}}\PY{l+s+s1}{VaR}\PY{l+s+s1}{\PYZsq{}}\PY{p}{]} \PY{o}{=} \PY{n}{RM\PYZus{}VaR\PYZus{}list}
    \PY{n}{RM\PYZus{}frame}\PY{p}{[}\PY{l+s+s1}{\PYZsq{}}\PY{l+s+s1}{VaR\PYZhy{}Change}\PY{l+s+s1}{\PYZsq{}}\PY{p}{]} \PY{o}{=} \PY{n}{RM\PYZus{}frame}\PY{p}{[}\PY{l+s+s1}{\PYZsq{}}\PY{l+s+s1}{VaR}\PY{l+s+s1}{\PYZsq{}}\PY{p}{]}\PY{o}{.}\PY{n}{pct\PYZus{}change}\PY{p}{(}\PY{p}{)}
    \PY{n}{RM\PYZus{}frame}\PY{p}{[}\PY{l+s+s1}{\PYZsq{}}\PY{l+s+s1}{CVaR}\PY{l+s+s1}{\PYZsq{}}\PY{p}{]} \PY{o}{=} \PY{n}{RM\PYZus{}CVaR\PYZus{}list}
    \PY{n}{RM\PYZus{}frame}\PY{p}{[}\PY{l+s+s1}{\PYZsq{}}\PY{l+s+s1}{CVaR\PYZhy{}Change}\PY{l+s+s1}{\PYZsq{}}\PY{p}{]} \PY{o}{=} \PY{n}{RM\PYZus{}frame}\PY{p}{[}\PY{l+s+s1}{\PYZsq{}}\PY{l+s+s1}{CVaR}\PY{l+s+s1}{\PYZsq{}}\PY{p}{]}\PY{o}{.}\PY{n}{pct\PYZus{}change}\PY{p}{(}\PY{p}{)}
    \PY{n}{RM\PYZus{}frame}\PY{p}{[}\PY{l+s+s1}{\PYZsq{}}\PY{l+s+s1}{Power}\PY{l+s+s1}{\PYZsq{}}\PY{p}{]} \PY{o}{=} \PY{n}{RM\PYZus{}PSRM\PYZus{}list}
    \PY{n}{RM\PYZus{}frame}\PY{p}{[}\PY{l+s+s1}{\PYZsq{}}\PY{l+s+s1}{Power\PYZhy{}Change}\PY{l+s+s1}{\PYZsq{}}\PY{p}{]} \PY{o}{=} \PY{n}{RM\PYZus{}frame}\PY{p}{[}\PY{l+s+s1}{\PYZsq{}}\PY{l+s+s1}{Power}\PY{l+s+s1}{\PYZsq{}}\PY{p}{]}\PY{o}{.}\PY{n}{pct\PYZus{}change}\PY{p}{(}\PY{p}{)}

    \PY{c+c1}{\PYZsh{}\PYZhy{}\PYZhy{}\PYZhy{}\PYZhy{}\PYZhy{}\PYZhy{}\PYZhy{}\PYZhy{}\PYZhy{}\PYZhy{}\PYZhy{}\PYZhy{}\PYZhy{}\PYZhy{}\PYZhy{}\PYZhy{}\PYZhy{}\PYZhy{}\PYZhy{}\PYZhy{}\PYZhy{}\PYZhy{}\PYZhy{}\PYZhy{}\PYZhy{}\PYZhy{}\PYZhy{}}
    \PY{c+c1}{\PYZsh{} Ermittle die kleinste und größte Relaisation des jweiligen Risikomaßes}
    \PY{n}{Min\PYZus{}Max\PYZus{}VaR} \PY{o}{=} \PY{p}{(}\PY{n+nb}{min}\PY{p}{(}\PY{n}{RM\PYZus{}VaR\PYZus{}list}\PY{p}{)}\PY{p}{,} \PY{n+nb}{max}\PY{p}{(}\PY{n}{RM\PYZus{}VaR\PYZus{}list}\PY{p}{)}\PY{p}{)}
    \PY{n}{Min\PYZus{}Max\PYZus{}CVaR} \PY{o}{=} \PY{p}{(}\PY{n+nb}{min}\PY{p}{(}\PY{n}{RM\PYZus{}CVaR\PYZus{}list}\PY{p}{)}\PY{p}{,} \PY{n+nb}{max}\PY{p}{(}\PY{n}{RM\PYZus{}CVaR\PYZus{}list}\PY{p}{)}\PY{p}{)}
    \PY{n}{Min\PYZus{}Max\PYZus{}PSRM} \PY{o}{=} \PY{p}{(}\PY{n+nb}{min}\PY{p}{(}\PY{n}{RM\PYZus{}PSRM\PYZus{}list}\PY{p}{)}\PY{p}{,} \PY{n+nb}{max}\PY{p}{(}\PY{n}{RM\PYZus{}PSRM\PYZus{}list}\PY{p}{)}\PY{p}{)}

    \PY{c+c1}{\PYZsh{}\PYZhy{}\PYZhy{}\PYZhy{}\PYZhy{}\PYZhy{}\PYZhy{}\PYZhy{}\PYZhy{}\PYZhy{}\PYZhy{}\PYZhy{}\PYZhy{}\PYZhy{}\PYZhy{}\PYZhy{}\PYZhy{}\PYZhy{}\PYZhy{}\PYZhy{}\PYZhy{}\PYZhy{}\PYZhy{}\PYZhy{}\PYZhy{}\PYZhy{}\PYZhy{}\PYZhy{}}
    \PY{c+c1}{\PYZsh{} Gib den DataFrame und die Infos zurück}
    \PY{n+nb}{print}\PY{p}{(}\PY{l+s+s1}{\PYZsq{}}\PY{l+s+s1}{\PYZsh{}}\PY{l+s+s1}{\PYZsq{}} \PY{o}{+} \PY{n}{SCREEN\PYZus{}WIDTH} \PY{o}{*} \PY{l+s+s1}{\PYZsq{}}\PY{l+s+s1}{\PYZhy{}}\PY{l+s+s1}{\PYZsq{}} \PY{o}{+} \PY{l+s+s1}{\PYZsq{}}\PY{l+s+s1}{\PYZsh{}}\PY{l+s+s1}{\PYZsq{}}\PY{p}{)}
    \PY{n+nb}{print}\PY{p}{(}\PY{l+s+s1}{\PYZsq{}}\PY{l+s+s1}{|}\PY{l+s+s1}{\PYZsq{}} \PY{o}{+} \PY{n}{centered}\PY{p}{(}\PY{l+s+s1}{\PYZsq{}}\PY{l+s+s1}{[INFO] Der DataFrame mit den auf den auf }\PY{l+s+s1}{\PYZsq{}} \PY{o}{+}\PY{n+nb}{str}\PY{p}{(}\PY{n}{runs\PYZus{}func}\PY{p}{)} \PY{o}{+} \PY{l+s+s1}{\PYZsq{}}\PY{l+s+s1}{ mal }\PY{l+s+s1}{\PYZsq{}} \PY{o}{+} \PY{n+nb}{str}\PY{p}{(}\PY{n}{runs\PYZus{}sim}\PY{p}{)} \PY{o}{+} \PY{l+s+s1}{\PYZsq{}}\PY{l+s+s1}{ Durchläufen beruhenden Risikomaßen ergibt sich wie folgt: }\PY{l+s+s1}{\PYZsq{}}\PY{p}{)} \PY{o}{+} \PY{l+s+s1}{\PYZsq{}}\PY{l+s+s1}{| }\PY{l+s+s1}{\PYZsq{}}\PY{p}{)}
    \PY{n+nb}{print}\PY{p}{(}\PY{l+s+s1}{\PYZsq{}}\PY{l+s+s1}{\PYZsh{}}\PY{l+s+s1}{\PYZsq{}} \PY{o}{+} \PY{n}{SCREEN\PYZus{}WIDTH} \PY{o}{*} \PY{l+s+s1}{\PYZsq{}}\PY{l+s+s1}{\PYZhy{}}\PY{l+s+s1}{\PYZsq{}} \PY{o}{+} \PY{l+s+s1}{\PYZsq{}}\PY{l+s+s1}{\PYZsh{}}\PY{l+s+s1}{\PYZsq{}}\PY{p}{)}
    \PY{n+nb}{print}\PY{p}{(}\PY{n}{RM\PYZus{}frame}\PY{p}{)}
    \PY{n+nb}{print}\PY{p}{(}\PY{l+s+s1}{\PYZsq{}}\PY{l+s+s1}{\PYZsh{}}\PY{l+s+s1}{\PYZsq{}} \PY{o}{+} \PY{n}{SCREEN\PYZus{}WIDTH} \PY{o}{*} \PY{l+s+s1}{\PYZsq{}}\PY{l+s+s1}{\PYZhy{}}\PY{l+s+s1}{\PYZsq{}} \PY{o}{+} \PY{l+s+s1}{\PYZsq{}}\PY{l+s+s1}{\PYZsh{}}\PY{l+s+s1}{\PYZsq{}}\PY{p}{)}
    \PY{n+nb}{print}\PY{p}{(}\PY{l+s+s1}{\PYZsq{}}\PY{l+s+s1}{|}\PY{l+s+s1}{\PYZsq{}} \PY{o}{+} \PY{n}{centered}\PY{p}{(}\PY{l+s+s1}{\PYZsq{}}\PY{l+s+s1}{Nach }\PY{l+s+s1}{\PYZsq{}} \PY{o}{+} \PY{n+nb}{str}\PY{p}{(}\PY{n}{runs\PYZus{}func}\PY{p}{)} \PY{o}{+} \PY{l+s+s1}{\PYZsq{}}\PY{l+s+s1}{ Simulationsläufen mit je }\PY{l+s+s1}{\PYZsq{}} \PY{o}{+} \PY{n+nb}{str}\PY{p}{(}\PY{n}{runs\PYZus{}sim}\PY{p}{)} \PY{o}{+} \PY{l+s+s1}{\PYZsq{}}\PY{l+s+s1}{ Durchläufen beträgt der kleinste VaR }\PY{l+s+s1}{\PYZsq{}} \PY{o}{+} \PY{n+nb}{str}\PY{p}{(}\PY{n+nb}{round}\PY{p}{(}\PY{n}{Min\PYZus{}Max\PYZus{}VaR}\PY{p}{[}\PY{l+m+mi}{0}\PY{p}{]}\PY{p}{,}\PY{l+m+mi}{2}\PY{p}{)}\PY{p}{)} \PY{o}{+}\PY{l+s+s1}{\PYZsq{}}\PY{l+s+s1}{, der größte }\PY{l+s+s1}{\PYZsq{}} \PY{o}{+} \PY{n+nb}{str}\PY{p}{(}\PY{n+nb}{round}\PY{p}{(}\PY{n}{Min\PYZus{}Max\PYZus{}VaR}\PY{p}{[}\PY{l+m+mi}{1}\PY{p}{]}\PY{p}{,}\PY{l+m+mi}{2}\PY{p}{)}\PY{p}{)} \PY{o}{+} \PY{l+s+s1}{\PYZsq{}}\PY{l+s+s1}{ (}\PY{l+s+se}{\PYZbs{}u0394}\PY{l+s+s1}{ = }\PY{l+s+s1}{\PYZsq{}} \PY{o}{+} \PY{n+nb}{str}\PY{p}{(}\PY{p}{(}\PY{n+nb}{round}\PY{p}{(}\PY{p}{(}\PY{n+nb}{float}\PY{p}{(}\PY{n}{Min\PYZus{}Max\PYZus{}VaR}\PY{p}{[}\PY{l+m+mi}{0}\PY{p}{]}\PY{o}{/}\PY{n}{Min\PYZus{}Max\PYZus{}VaR}\PY{p}{[}\PY{l+m+mi}{1}\PY{p}{]}\PY{p}{)}\PY{o}{\PYZhy{}}\PY{l+m+mi}{1}\PY{p}{)}\PY{o}{*}\PY{l+m+mi}{100}\PY{p}{,}\PY{l+m+mi}{2}\PY{p}{)}\PY{p}{)}\PY{p}{)} \PY{o}{+} \PY{l+s+s1}{\PYZsq{}}\PY{l+s+s1}{\PYZpc{}}\PY{l+s+s1}{).}\PY{l+s+s1}{\PYZsq{}}\PY{p}{)} \PY{o}{+} \PY{l+s+s1}{\PYZsq{}}\PY{l+s+s1}{| }\PY{l+s+s1}{\PYZsq{}}\PY{p}{)}
    \PY{n+nb}{print}\PY{p}{(}\PY{l+s+s1}{\PYZsq{}}\PY{l+s+s1}{|}\PY{l+s+s1}{\PYZsq{}} \PY{o}{+} \PY{n}{centered}\PY{p}{(}\PY{l+s+s1}{\PYZsq{}}\PY{l+s+s1}{Nach }\PY{l+s+s1}{\PYZsq{}} \PY{o}{+} \PY{n+nb}{str}\PY{p}{(}\PY{n}{runs\PYZus{}func}\PY{p}{)} \PY{o}{+} \PY{l+s+s1}{\PYZsq{}}\PY{l+s+s1}{ Simulationsläufen mit je }\PY{l+s+s1}{\PYZsq{}} \PY{o}{+} \PY{n+nb}{str}\PY{p}{(}\PY{n}{runs\PYZus{}sim}\PY{p}{)} \PY{o}{+} \PY{l+s+s1}{\PYZsq{}}\PY{l+s+s1}{ Durchläufen beträgt der kleinste CVaR }\PY{l+s+s1}{\PYZsq{}} \PY{o}{+} \PY{n+nb}{str}\PY{p}{(}\PY{n+nb}{round}\PY{p}{(}\PY{n}{Min\PYZus{}Max\PYZus{}CVaR}\PY{p}{[}\PY{l+m+mi}{0}\PY{p}{]}\PY{p}{,}\PY{l+m+mi}{2}\PY{p}{)}\PY{p}{)} \PY{o}{+}\PY{l+s+s1}{\PYZsq{}}\PY{l+s+s1}{, der größte }\PY{l+s+s1}{\PYZsq{}} \PY{o}{+} \PY{n+nb}{str}\PY{p}{(}\PY{n+nb}{round}\PY{p}{(}\PY{n}{Min\PYZus{}Max\PYZus{}CVaR}\PY{p}{[}\PY{l+m+mi}{1}\PY{p}{]}\PY{p}{,}\PY{l+m+mi}{2}\PY{p}{)}\PY{p}{)} \PY{o}{+} \PY{l+s+s1}{\PYZsq{}}\PY{l+s+s1}{ (}\PY{l+s+se}{\PYZbs{}u0394}\PY{l+s+s1}{ = }\PY{l+s+s1}{\PYZsq{}} \PY{o}{+} \PY{n+nb}{str}\PY{p}{(}\PY{p}{(}\PY{n+nb}{round}\PY{p}{(}\PY{p}{(}\PY{n+nb}{float}\PY{p}{(}\PY{n}{Min\PYZus{}Max\PYZus{}CVaR}\PY{p}{[}\PY{l+m+mi}{0}\PY{p}{]}\PY{o}{/}\PY{n}{Min\PYZus{}Max\PYZus{}CVaR}\PY{p}{[}\PY{l+m+mi}{1}\PY{p}{]}\PY{p}{)}\PY{o}{\PYZhy{}}\PY{l+m+mi}{1}\PY{p}{)}\PY{o}{*}\PY{l+m+mi}{100}\PY{p}{,}\PY{l+m+mi}{2}\PY{p}{)}\PY{p}{)}\PY{p}{)} \PY{o}{+} \PY{l+s+s1}{\PYZsq{}}\PY{l+s+s1}{\PYZpc{}}\PY{l+s+s1}{).}\PY{l+s+s1}{\PYZsq{}}\PY{p}{)} \PY{o}{+} \PY{l+s+s1}{\PYZsq{}}\PY{l+s+s1}{| }\PY{l+s+s1}{\PYZsq{}}\PY{p}{)}
    \PY{n+nb}{print}\PY{p}{(}\PY{l+s+s1}{\PYZsq{}}\PY{l+s+s1}{|}\PY{l+s+s1}{\PYZsq{}} \PY{o}{+} \PY{n}{centered}\PY{p}{(}\PY{l+s+s1}{\PYZsq{}}\PY{l+s+s1}{Nach }\PY{l+s+s1}{\PYZsq{}} \PY{o}{+} \PY{n+nb}{str}\PY{p}{(}\PY{n}{runs\PYZus{}func}\PY{p}{)} \PY{o}{+} \PY{l+s+s1}{\PYZsq{}}\PY{l+s+s1}{ Simulationsläufen mit je }\PY{l+s+s1}{\PYZsq{}} \PY{o}{+} \PY{n+nb}{str}\PY{p}{(}\PY{n}{runs\PYZus{}sim}\PY{p}{)} \PY{o}{+} \PY{l+s+s1}{\PYZsq{}}\PY{l+s+s1}{ Durchläufen beträgt das kleinste P\PYZhy{}SRM }\PY{l+s+s1}{\PYZsq{}} \PY{o}{+} \PY{n+nb}{str}\PY{p}{(}\PY{n+nb}{round}\PY{p}{(}\PY{n}{Min\PYZus{}Max\PYZus{}PSRM}\PY{p}{[}\PY{l+m+mi}{0}\PY{p}{]}\PY{p}{,}\PY{l+m+mi}{2}\PY{p}{)}\PY{p}{)} \PY{o}{+}\PY{l+s+s1}{\PYZsq{}}\PY{l+s+s1}{, das größte }\PY{l+s+s1}{\PYZsq{}} \PY{o}{+} \PY{n+nb}{str}\PY{p}{(}\PY{n+nb}{round}\PY{p}{(}\PY{n}{Min\PYZus{}Max\PYZus{}PSRM}\PY{p}{[}\PY{l+m+mi}{1}\PY{p}{]}\PY{p}{,}\PY{l+m+mi}{2}\PY{p}{)}\PY{p}{)} \PY{o}{+} \PY{l+s+s1}{\PYZsq{}}\PY{l+s+s1}{ (}\PY{l+s+se}{\PYZbs{}u0394}\PY{l+s+s1}{ = }\PY{l+s+s1}{\PYZsq{}} \PY{o}{+} \PY{n+nb}{str}\PY{p}{(}\PY{p}{(}\PY{n+nb}{round}\PY{p}{(}\PY{p}{(}\PY{n+nb}{float}\PY{p}{(}\PY{n}{Min\PYZus{}Max\PYZus{}PSRM}\PY{p}{[}\PY{l+m+mi}{0}\PY{p}{]}\PY{o}{/}\PY{n}{Min\PYZus{}Max\PYZus{}PSRM}\PY{p}{[}\PY{l+m+mi}{1}\PY{p}{]}\PY{p}{)}\PY{o}{\PYZhy{}}\PY{l+m+mi}{1}\PY{p}{)}\PY{o}{*}\PY{l+m+mi}{100}\PY{p}{,}\PY{l+m+mi}{2}\PY{p}{)}\PY{p}{)}\PY{p}{)} \PY{o}{+} \PY{l+s+s1}{\PYZsq{}}\PY{l+s+s1}{\PYZpc{}}\PY{l+s+s1}{).}\PY{l+s+s1}{\PYZsq{}}\PY{p}{)} \PY{o}{+} \PY{l+s+s1}{\PYZsq{}}\PY{l+s+s1}{| }\PY{l+s+s1}{\PYZsq{}}\PY{p}{)}
    \PY{n+nb}{print}\PY{p}{(}\PY{l+s+s1}{\PYZsq{}}\PY{l+s+s1}{\PYZsh{}}\PY{l+s+s1}{\PYZsq{}} \PY{o}{+} \PY{n}{SCREEN\PYZus{}WIDTH} \PY{o}{*} \PY{l+s+s1}{\PYZsq{}}\PY{l+s+s1}{\PYZhy{}}\PY{l+s+s1}{\PYZsq{}} \PY{o}{+} \PY{l+s+s1}{\PYZsq{}}\PY{l+s+s1}{\PYZsh{}}\PY{l+s+s1}{\PYZsq{}}\PY{p}{)}
\end{Verbatim}
\end{tcolorbox}

    \hypertarget{variablen-spezifizieren}{%
\subsection{Variablen spezifizieren}\label{variablen-spezifizieren}}

Die für die Simulation notwenigen Variablen werden definiert.

    \begin{tcolorbox}[breakable, size=fbox, boxrule=1pt, pad at break*=1mm,colback=cellbackground, colframe=cellborder]
\prompt{In}{incolor}{6}{\boxspacing}
\begin{Verbatim}[commandchars=\\\{\}]
\PY{c+c1}{\PYZsh{}\PYZhy{}\PYZhy{}\PYZhy{}\PYZhy{}\PYZhy{}\PYZhy{}\PYZhy{}\PYZhy{}\PYZhy{}\PYZhy{}\PYZhy{}\PYZhy{}\PYZhy{}\PYZhy{}\PYZhy{}\PYZhy{}\PYZhy{}\PYZhy{}\PYZhy{}\PYZhy{}\PYZhy{}\PYZhy{}\PYZhy{}\PYZhy{}\PYZhy{}\PYZhy{}\PYZhy{}\PYZhy{}\PYZhy{}\PYZhy{}\PYZhy{}\PYZhy{}\PYZhy{}\PYZhy{}\PYZhy{}\PYZhy{}\PYZhy{}\PYZhy{}\PYZhy{}\PYZhy{}\PYZhy{}\PYZhy{}\PYZhy{}}
\PY{c+c1}{\PYZsh{} Parameter für die Simulation}
\PY{c+c1}{\PYZsh{}\PYZhy{}\PYZhy{}\PYZhy{}\PYZhy{}\PYZhy{}\PYZhy{}\PYZhy{}\PYZhy{}\PYZhy{}\PYZhy{}\PYZhy{}\PYZhy{}\PYZhy{}\PYZhy{}\PYZhy{}\PYZhy{}\PYZhy{}\PYZhy{}\PYZhy{}\PYZhy{}\PYZhy{}\PYZhy{}\PYZhy{}\PYZhy{}\PYZhy{}\PYZhy{}\PYZhy{}\PYZhy{}\PYZhy{}\PYZhy{}}
\PY{c+c1}{\PYZsh{} Anzahl Simulationsdurchläufe}
\PY{n}{n} \PY{o}{=} \PY{l+m+mi}{10000}
\PY{c+c1}{\PYZsh{} Neue Randverteilungen (Gleichverteilung)}
\PY{n}{rand\PYZus{}x} \PY{o}{=} \PY{p}{[}\PY{l+m+mi}{10}\PY{p}{,}\PY{l+m+mi}{20}\PY{p}{]}
\PY{n}{rand\PYZus{}y} \PY{o}{=} \PY{p}{[}\PY{l+m+mi}{8}\PY{p}{,}\PY{l+m+mi}{22}\PY{p}{]}
\PY{c+c1}{\PYZsh{} Varianzen und Korrelation(en)}
\PY{n}{var\PYZus{}x} \PY{o}{=} \PY{l+m+mi}{4} 
\PY{n}{var\PYZus{}y} \PY{o}{=} \PY{l+m+mi}{9}
\PY{n}{corr\PYZus{}list} \PY{o}{=} \PY{p}{[}\PY{l+m+mi}{0}\PY{p}{]}
\PY{n}{std\PYZus{}list} \PY{o}{=} \PY{p}{[}\PY{n}{math}\PY{o}{.}\PY{n}{sqrt}\PY{p}{(}\PY{n}{var\PYZus{}x}\PY{p}{)}\PY{p}{,} \PY{n}{math}\PY{o}{.}\PY{n}{sqrt}\PY{p}{(}\PY{n}{var\PYZus{}y}\PY{p}{)}\PY{p}{]}
\PY{c+c1}{\PYZsh{} Erwartungswerte}
\PY{n}{mu} \PY{o}{=} \PY{p}{[}\PY{l+m+mi}{2}\PY{p}{,} \PY{l+m+mi}{3}\PY{p}{]}
\PY{c+c1}{\PYZsh{}\PYZhy{}\PYZhy{}\PYZhy{}\PYZhy{}\PYZhy{}\PYZhy{}\PYZhy{}\PYZhy{}\PYZhy{}\PYZhy{}\PYZhy{}\PYZhy{}\PYZhy{}\PYZhy{}\PYZhy{}\PYZhy{}\PYZhy{}\PYZhy{}\PYZhy{}\PYZhy{}\PYZhy{}\PYZhy{}\PYZhy{}\PYZhy{}\PYZhy{}\PYZhy{}\PYZhy{}\PYZhy{}\PYZhy{}\PYZhy{}}
\PY{c+c1}{\PYZsh{}\PYZhy{}\PYZhy{}\PYZhy{}\PYZhy{}\PYZhy{}\PYZhy{}\PYZhy{}\PYZhy{}\PYZhy{}\PYZhy{}\PYZhy{}\PYZhy{}\PYZhy{}\PYZhy{}\PYZhy{}\PYZhy{}\PYZhy{}\PYZhy{}\PYZhy{}\PYZhy{}\PYZhy{}\PYZhy{}\PYZhy{}\PYZhy{}\PYZhy{}\PYZhy{}\PYZhy{}\PYZhy{}\PYZhy{}\PYZhy{}\PYZhy{}\PYZhy{}\PYZhy{}\PYZhy{}\PYZhy{}\PYZhy{}\PYZhy{}\PYZhy{}\PYZhy{}\PYZhy{}\PYZhy{}\PYZhy{}\PYZhy{}}
\PY{c+c1}{\PYZsh{} Paramaeter Risikomaße}
\PY{n}{alpha} \PY{o}{=} \PY{l+m+mf}{0.05}
\PY{n}{gamma} \PY{o}{=} \PY{l+m+mf}{0.5}
\PY{c+c1}{\PYZsh{}\PYZhy{}\PYZhy{}\PYZhy{}\PYZhy{}\PYZhy{}\PYZhy{}\PYZhy{}\PYZhy{}\PYZhy{}\PYZhy{}\PYZhy{}\PYZhy{}\PYZhy{}\PYZhy{}\PYZhy{}\PYZhy{}\PYZhy{}\PYZhy{}\PYZhy{}\PYZhy{}\PYZhy{}\PYZhy{}\PYZhy{}\PYZhy{}\PYZhy{}\PYZhy{}\PYZhy{}\PYZhy{}\PYZhy{}\PYZhy{}\PYZhy{}\PYZhy{}\PYZhy{}\PYZhy{}\PYZhy{}\PYZhy{}\PYZhy{}\PYZhy{}\PYZhy{}\PYZhy{}\PYZhy{}\PYZhy{}\PYZhy{}}
\PY{c+c1}{\PYZsh{} Anzahl Simulationsläufe und Durchläufe pro Simulation}
\PY{n}{runs\PYZus{}func} \PY{o}{=} \PY{l+m+mi}{10}
\PY{n}{runs\PYZus{}sim} \PY{o}{=} \PY{l+m+mi}{100000}
\end{Verbatim}
\end{tcolorbox}

    \hypertarget{funktionsaufruf}{%
\subsection{Funktionsaufruf}\label{funktionsaufruf}}

    \begin{tcolorbox}[breakable, size=fbox, boxrule=1pt, pad at break*=1mm,colback=cellbackground, colframe=cellborder]
\prompt{In}{incolor}{7}{\boxspacing}
\begin{Verbatim}[commandchars=\\\{\}]
\PY{k}{if} \PY{n+nv+vm}{\PYZus{}\PYZus{}name\PYZus{}\PYZus{}}\PY{o}{==}\PY{l+s+s1}{\PYZsq{}}\PY{l+s+s1}{\PYZus{}\PYZus{}main\PYZus{}\PYZus{}}\PY{l+s+s1}{\PYZsq{}}\PY{p}{:}
    \PY{n+nb}{print}\PY{p}{(}\PY{l+s+s1}{\PYZsq{}}\PY{l+s+s1}{Ergebnis Multi\PYZhy{}threaded:}\PY{l+s+se}{\PYZbs{}n}\PY{l+s+s1}{\PYZsq{}}\PY{p}{)}
    \PY{n}{multithreading}\PY{p}{(}\PY{n}{n}\PY{p}{,} \PY{n}{rand\PYZus{}x}\PY{p}{,} \PY{n}{rand\PYZus{}y}\PY{p}{,} \PY{n}{var\PYZus{}x}\PY{p}{,} \PY{n}{var\PYZus{}y}\PY{p}{,} \PY{n}{corr\PYZus{}list}\PY{p}{,} \PY{n}{std\PYZus{}list}\PY{p}{,} \PY{n}{mu}\PY{p}{,} \PY{n}{alpha}\PY{p}{,} \PY{n}{gamma}\PY{p}{,} \PY{n}{runs\PYZus{}func}\PY{p}{,} \PY{n}{runs\PYZus{}sim}\PY{p}{)}
    \PY{n+nb}{print}\PY{p}{(}\PY{l+s+s1}{\PYZsq{}}\PY{l+s+se}{\PYZbs{}n}\PY{l+s+se}{\PYZbs{}n}\PY{l+s+s1}{Ergebnis Single\PYZhy{}threaded:}\PY{l+s+se}{\PYZbs{}n}\PY{l+s+s1}{\PYZsq{}}\PY{p}{)}
    \PY{n}{single\PYZus{}thread}\PY{p}{(}\PY{n}{n}\PY{p}{,} \PY{n}{rand\PYZus{}x}\PY{p}{,} \PY{n}{rand\PYZus{}y}\PY{p}{,} \PY{n}{var\PYZus{}x}\PY{p}{,} \PY{n}{var\PYZus{}y}\PY{p}{,} \PY{n}{corr\PYZus{}list}\PY{p}{,} \PY{n}{std\PYZus{}list}\PY{p}{,} \PY{n}{mu}\PY{p}{,} \PY{n}{alpha}\PY{p}{,} \PY{n}{gamma}\PY{p}{,} \PY{n}{runs\PYZus{}func}\PY{p}{,} \PY{n}{runs\PYZus{}sim}\PY{p}{)}
\end{Verbatim}
\end{tcolorbox}

    \begin{Verbatim}[commandchars=\\\{\}]
Ergebnis Multi-threaded:

\#-------------------------------------------------------------------------------
------------------------------------\#
| [INFO] Der DataFrame mit den auf den auf 10 mal 100000 Durchläufen beruhenden
Risikomaßen ergibt sich wie folgt:  |
\#-------------------------------------------------------------------------------
------------------------------------\#
         VaR  VaR-Change       CVaR  CVaR-Change      Power  Power-Change
0 -21.758507         NaN -20.504731          NaN  26.828263           NaN
1 -21.707793   -0.002331 -20.462340    -0.002067  26.817743     -0.000392
2 -21.717064    0.000427 -20.485782     0.001146  26.799571     -0.000678
3 -21.732111    0.000693 -20.483403    -0.000116  26.836461      0.001377
4 -21.740721    0.000396 -20.492935     0.000465  26.834742     -0.000064
5 -21.778684    0.001746 -20.540957     0.002343  26.870833      0.001345
6 -21.763236   -0.000709 -20.519158    -0.001061  26.850804     -0.000745
7 -21.730395   -0.001509 -20.482281    -0.001797  26.830540     -0.000755
8 -21.721129   -0.000426 -20.482142    -0.000007  26.826441     -0.000153
9 -21.789455    0.003146 -20.531516     0.002411  26.862185      0.001332
\#-------------------------------------------------------------------------------
------------------------------------\#
|Nach 10 Simulationsläufen mit je 100000 Durchläufen beträgt der kleinste VaR
-21.79, der größte -21.71 (0.38\%).|
|Nach 10 Simulationsläufen mit je 100000 Durchläufen beträgt der kleinste CVaR
-20.54, der größte -20.46 (0.38\%).|
|Nach 10 Simulationsläufen mit je 100000 Durchläufen beträgt das kleinste P-SRM
26.8, das größte 26.87 (-0.27\%).|
\#-------------------------------------------------------------------------------
------------------------------------\#
0:00:42.612356


Ergebnis Single-threaded:

0:03:51.417429
\#-------------------------------------------------------------------------------
------------------------------------\#
| [INFO] Der DataFrame mit den auf den auf 10 mal 100000 Durchläufen beruhenden
Risikomaßen ergibt sich wie folgt:  |
\#-------------------------------------------------------------------------------
------------------------------------\#
         VaR  VaR-Change       CVaR  CVaR-Change      Power  Power-Change
0 -21.721779         NaN -20.473054          NaN  26.811895           NaN
1 -21.731741    0.000459 -20.487463     0.000704  26.821491      0.000358
2 -21.749623    0.000823 -20.497513     0.000491  26.829076      0.000283
3 -21.759824    0.000469 -20.514984     0.000852  26.834125      0.000188
4 -21.759102   -0.000033 -20.514246    -0.000036  26.832232     -0.000071
5 -21.734240   -0.001143 -20.481392    -0.001602  26.826350     -0.000219
6 -21.769326    0.001614 -20.525132     0.002136  26.849975      0.000881
7 -21.744877   -0.001123 -20.499863    -0.001231  26.810697     -0.001463
8 -21.736873   -0.000368 -20.483276    -0.000809  26.816251      0.000207
9 -21.681204   -0.002561 -20.460268    -0.001123  26.813104     -0.000117
\#-------------------------------------------------------------------------------
------------------------------------\#
|Nach 10 Simulationsläufen mit je 100000 Durchläufen beträgt der kleinste VaR
-21.77, der größte -21.68 (0.41\%).|
|Nach 10 Simulationsläufen mit je 100000 Durchläufen beträgt der kleinste CVaR
-20.53, der größte -20.46 (0.32\%).|
|Nach 10 Simulationsläufen mit je 100000 Durchläufen beträgt das kleinste P-SRM
26.81, das größte 26.85 (-0.15\%).|
\#-------------------------------------------------------------------------------
------------------------------------\#
    \end{Verbatim}


    % Add a bibliography block to the postdoc
    
    
    
\end{document}
