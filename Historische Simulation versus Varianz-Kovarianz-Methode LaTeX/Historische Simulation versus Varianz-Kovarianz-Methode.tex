
% Default to the notebook output style

    


% Inherit from the specified cell style.




%\documentclass[11pt]{article}
\documentclass[paper=landscape]{scrartcl}
	\usepackage[paperwidth=40cm,paperheight=175cm,margin=1in]{geometry}

    
    
    \usepackage[T1]{fontenc}
    % Nicer default font (+ math font) than Computer Modern for most use cases
    \usepackage{mathpazo}

    % Basic figure setup, for now with no caption control since it's done
    % automatically by Pandoc (which extracts ![](path) syntax from Markdown).
    \usepackage{graphicx}
    % We will generate all images so they have a width \maxwidth. This means
    % that they will get their normal width if they fit onto the page, but
    % are scaled down if they would overflow the margins.
    \makeatletter
    \def\maxwidth{\ifdim\Gin@nat@width>\linewidth\linewidth
    \else\Gin@nat@width\fi}
    \makeatother
    \let\Oldincludegraphics\includegraphics
    % Set max figure width to be 80% of text width, for now hardcoded.
    \renewcommand{\includegraphics}[1]{\Oldincludegraphics[width=.8\maxwidth]{#1}}
    % Ensure that by default, figures have no caption (until we provide a
    % proper Figure object with a Caption API and a way to capture that
    % in the conversion process - todo).
    \usepackage{caption}
    \DeclareCaptionLabelFormat{nolabel}{}
    \captionsetup{labelformat=nolabel}

    \usepackage{adjustbox} % Used to constrain images to a maximum size 
    \usepackage{xcolor} % Allow colors to be defined
    \usepackage{enumerate} % Needed for markdown enumerations to work
    \usepackage{geometry} % Used to adjust the document margins
    \usepackage{amsmath} % Equations
    \usepackage{amssymb} % Equations
    \usepackage{textcomp} % defines textquotesingle
    % Hack from http://tex.stackexchange.com/a/47451/13684:
    \AtBeginDocument{%
        \def\PYZsq{\textquotesingle}% Upright quotes in Pygmentized code
    }
    \usepackage{upquote} % Upright quotes for verbatim code
    \usepackage{eurosym} % defines \euro
    \usepackage[mathletters]{ucs} % Extended unicode (utf-8) support
    \usepackage[utf8x]{inputenc} % Allow utf-8 characters in the tex document
    \usepackage{fancyvrb} % verbatim replacement that allows latex
    \usepackage{grffile} % extends the file name processing of package graphics 
                         % to support a larger range 
    % The hyperref package gives us a pdf with properly built
    % internal navigation ('pdf bookmarks' for the table of contents,
    % internal cross-reference links, web links for URLs, etc.)
    \usepackage{hyperref}
    \usepackage{longtable} % longtable support required by pandoc >1.10
    \usepackage{booktabs}  % table support for pandoc > 1.12.2
    \usepackage[inline]{enumitem} % IRkernel/repr support (it uses the enumerate* environment)
    \usepackage[normalem]{ulem} % ulem is needed to support strikethroughs (\sout)
                                % normalem makes italics be italics, not underlines
    \usepackage{mathrsfs}
    

    
    
    % Colors for the hyperref package
    \definecolor{urlcolor}{rgb}{0,.145,.698}
    \definecolor{linkcolor}{rgb}{.71,0.21,0.01}
    \definecolor{citecolor}{rgb}{.12,.54,.11}

    % ANSI colors
    \definecolor{ansi-black}{HTML}{3E424D}
    \definecolor{ansi-black-intense}{HTML}{282C36}
    \definecolor{ansi-red}{HTML}{E75C58}
    \definecolor{ansi-red-intense}{HTML}{B22B31}
    \definecolor{ansi-green}{HTML}{00A250}
    \definecolor{ansi-green-intense}{HTML}{007427}
    \definecolor{ansi-yellow}{HTML}{DDB62B}
    \definecolor{ansi-yellow-intense}{HTML}{B27D12}
    \definecolor{ansi-blue}{HTML}{208FFB}
    \definecolor{ansi-blue-intense}{HTML}{0065CA}
    \definecolor{ansi-magenta}{HTML}{D160C4}
    \definecolor{ansi-magenta-intense}{HTML}{A03196}
    \definecolor{ansi-cyan}{HTML}{60C6C8}
    \definecolor{ansi-cyan-intense}{HTML}{258F8F}
    \definecolor{ansi-white}{HTML}{C5C1B4}
    \definecolor{ansi-white-intense}{HTML}{A1A6B2}
    \definecolor{ansi-default-inverse-fg}{HTML}{FFFFFF}
    \definecolor{ansi-default-inverse-bg}{HTML}{000000}

    % commands and environments needed by pandoc snippets
    % extracted from the output of `pandoc -s`
    \providecommand{\tightlist}{%
      \setlength{\itemsep}{0pt}\setlength{\parskip}{0pt}}
    \DefineVerbatimEnvironment{Highlighting}{Verbatim}{commandchars=\\\{\}}
    % Add ',fontsize=\small' for more characters per line
    \newenvironment{Shaded}{}{}
    \newcommand{\KeywordTok}[1]{\textcolor[rgb]{0.00,0.44,0.13}{\textbf{{#1}}}}
    \newcommand{\DataTypeTok}[1]{\textcolor[rgb]{0.56,0.13,0.00}{{#1}}}
    \newcommand{\DecValTok}[1]{\textcolor[rgb]{0.25,0.63,0.44}{{#1}}}
    \newcommand{\BaseNTok}[1]{\textcolor[rgb]{0.25,0.63,0.44}{{#1}}}
    \newcommand{\FloatTok}[1]{\textcolor[rgb]{0.25,0.63,0.44}{{#1}}}
    \newcommand{\CharTok}[1]{\textcolor[rgb]{0.25,0.44,0.63}{{#1}}}
    \newcommand{\StringTok}[1]{\textcolor[rgb]{0.25,0.44,0.63}{{#1}}}
    \newcommand{\CommentTok}[1]{\textcolor[rgb]{0.38,0.63,0.69}{\textit{{#1}}}}
    \newcommand{\OtherTok}[1]{\textcolor[rgb]{0.00,0.44,0.13}{{#1}}}
    \newcommand{\AlertTok}[1]{\textcolor[rgb]{1.00,0.00,0.00}{\textbf{{#1}}}}
    \newcommand{\FunctionTok}[1]{\textcolor[rgb]{0.02,0.16,0.49}{{#1}}}
    \newcommand{\RegionMarkerTok}[1]{{#1}}
    \newcommand{\ErrorTok}[1]{\textcolor[rgb]{1.00,0.00,0.00}{\textbf{{#1}}}}
    \newcommand{\NormalTok}[1]{{#1}}
    
    % Additional commands for more recent versions of Pandoc
    \newcommand{\ConstantTok}[1]{\textcolor[rgb]{0.53,0.00,0.00}{{#1}}}
    \newcommand{\SpecialCharTok}[1]{\textcolor[rgb]{0.25,0.44,0.63}{{#1}}}
    \newcommand{\VerbatimStringTok}[1]{\textcolor[rgb]{0.25,0.44,0.63}{{#1}}}
    \newcommand{\SpecialStringTok}[1]{\textcolor[rgb]{0.73,0.40,0.53}{{#1}}}
    \newcommand{\ImportTok}[1]{{#1}}
    \newcommand{\DocumentationTok}[1]{\textcolor[rgb]{0.73,0.13,0.13}{\textit{{#1}}}}
    \newcommand{\AnnotationTok}[1]{\textcolor[rgb]{0.38,0.63,0.69}{\textbf{\textit{{#1}}}}}
    \newcommand{\CommentVarTok}[1]{\textcolor[rgb]{0.38,0.63,0.69}{\textbf{\textit{{#1}}}}}
    \newcommand{\VariableTok}[1]{\textcolor[rgb]{0.10,0.09,0.49}{{#1}}}
    \newcommand{\ControlFlowTok}[1]{\textcolor[rgb]{0.00,0.44,0.13}{\textbf{{#1}}}}
    \newcommand{\OperatorTok}[1]{\textcolor[rgb]{0.40,0.40,0.40}{{#1}}}
    \newcommand{\BuiltInTok}[1]{{#1}}
    \newcommand{\ExtensionTok}[1]{{#1}}
    \newcommand{\PreprocessorTok}[1]{\textcolor[rgb]{0.74,0.48,0.00}{{#1}}}
    \newcommand{\AttributeTok}[1]{\textcolor[rgb]{0.49,0.56,0.16}{{#1}}}
    \newcommand{\InformationTok}[1]{\textcolor[rgb]{0.38,0.63,0.69}{\textbf{\textit{{#1}}}}}
    \newcommand{\WarningTok}[1]{\textcolor[rgb]{0.38,0.63,0.69}{\textbf{\textit{{#1}}}}}
    
    
    % Define a nice break command that doesn't care if a line doesn't already
    % exist.
    \def\br{\hspace*{\fill} \\* }
    % Math Jax compatibility definitions
    \def\gt{>}
    \def\lt{<}
    \let\Oldtex\TeX
    \let\Oldlatex\LaTeX
    \renewcommand{\TeX}{\textrm{\Oldtex}}
    \renewcommand{\LaTeX}{\textrm{\Oldlatex}}
    % Document parameters
    % Document title
    \title{Historische Simulation versus Varianz-Kovarianz-Methode}
    
    
    
    
    

    % Pygments definitions
    
\makeatletter
\def\PY@reset{\let\PY@it=\relax \let\PY@bf=\relax%
    \let\PY@ul=\relax \let\PY@tc=\relax%
    \let\PY@bc=\relax \let\PY@ff=\relax}
\def\PY@tok#1{\csname PY@tok@#1\endcsname}
\def\PY@toks#1+{\ifx\relax#1\empty\else%
    \PY@tok{#1}\expandafter\PY@toks\fi}
\def\PY@do#1{\PY@bc{\PY@tc{\PY@ul{%
    \PY@it{\PY@bf{\PY@ff{#1}}}}}}}
\def\PY#1#2{\PY@reset\PY@toks#1+\relax+\PY@do{#2}}

\expandafter\def\csname PY@tok@w\endcsname{\def\PY@tc##1{\textcolor[rgb]{0.73,0.73,0.73}{##1}}}
\expandafter\def\csname PY@tok@c\endcsname{\let\PY@it=\textit\def\PY@tc##1{\textcolor[rgb]{0.25,0.50,0.50}{##1}}}
\expandafter\def\csname PY@tok@cp\endcsname{\def\PY@tc##1{\textcolor[rgb]{0.74,0.48,0.00}{##1}}}
\expandafter\def\csname PY@tok@k\endcsname{\let\PY@bf=\textbf\def\PY@tc##1{\textcolor[rgb]{0.00,0.50,0.00}{##1}}}
\expandafter\def\csname PY@tok@kp\endcsname{\def\PY@tc##1{\textcolor[rgb]{0.00,0.50,0.00}{##1}}}
\expandafter\def\csname PY@tok@kt\endcsname{\def\PY@tc##1{\textcolor[rgb]{0.69,0.00,0.25}{##1}}}
\expandafter\def\csname PY@tok@o\endcsname{\def\PY@tc##1{\textcolor[rgb]{0.40,0.40,0.40}{##1}}}
\expandafter\def\csname PY@tok@ow\endcsname{\let\PY@bf=\textbf\def\PY@tc##1{\textcolor[rgb]{0.67,0.13,1.00}{##1}}}
\expandafter\def\csname PY@tok@nb\endcsname{\def\PY@tc##1{\textcolor[rgb]{0.00,0.50,0.00}{##1}}}
\expandafter\def\csname PY@tok@nf\endcsname{\def\PY@tc##1{\textcolor[rgb]{0.00,0.00,1.00}{##1}}}
\expandafter\def\csname PY@tok@nc\endcsname{\let\PY@bf=\textbf\def\PY@tc##1{\textcolor[rgb]{0.00,0.00,1.00}{##1}}}
\expandafter\def\csname PY@tok@nn\endcsname{\let\PY@bf=\textbf\def\PY@tc##1{\textcolor[rgb]{0.00,0.00,1.00}{##1}}}
\expandafter\def\csname PY@tok@ne\endcsname{\let\PY@bf=\textbf\def\PY@tc##1{\textcolor[rgb]{0.82,0.25,0.23}{##1}}}
\expandafter\def\csname PY@tok@nv\endcsname{\def\PY@tc##1{\textcolor[rgb]{0.10,0.09,0.49}{##1}}}
\expandafter\def\csname PY@tok@no\endcsname{\def\PY@tc##1{\textcolor[rgb]{0.53,0.00,0.00}{##1}}}
\expandafter\def\csname PY@tok@nl\endcsname{\def\PY@tc##1{\textcolor[rgb]{0.63,0.63,0.00}{##1}}}
\expandafter\def\csname PY@tok@ni\endcsname{\let\PY@bf=\textbf\def\PY@tc##1{\textcolor[rgb]{0.60,0.60,0.60}{##1}}}
\expandafter\def\csname PY@tok@na\endcsname{\def\PY@tc##1{\textcolor[rgb]{0.49,0.56,0.16}{##1}}}
\expandafter\def\csname PY@tok@nt\endcsname{\let\PY@bf=\textbf\def\PY@tc##1{\textcolor[rgb]{0.00,0.50,0.00}{##1}}}
\expandafter\def\csname PY@tok@nd\endcsname{\def\PY@tc##1{\textcolor[rgb]{0.67,0.13,1.00}{##1}}}
\expandafter\def\csname PY@tok@s\endcsname{\def\PY@tc##1{\textcolor[rgb]{0.73,0.13,0.13}{##1}}}
\expandafter\def\csname PY@tok@sd\endcsname{\let\PY@it=\textit\def\PY@tc##1{\textcolor[rgb]{0.73,0.13,0.13}{##1}}}
\expandafter\def\csname PY@tok@si\endcsname{\let\PY@bf=\textbf\def\PY@tc##1{\textcolor[rgb]{0.73,0.40,0.53}{##1}}}
\expandafter\def\csname PY@tok@se\endcsname{\let\PY@bf=\textbf\def\PY@tc##1{\textcolor[rgb]{0.73,0.40,0.13}{##1}}}
\expandafter\def\csname PY@tok@sr\endcsname{\def\PY@tc##1{\textcolor[rgb]{0.73,0.40,0.53}{##1}}}
\expandafter\def\csname PY@tok@ss\endcsname{\def\PY@tc##1{\textcolor[rgb]{0.10,0.09,0.49}{##1}}}
\expandafter\def\csname PY@tok@sx\endcsname{\def\PY@tc##1{\textcolor[rgb]{0.00,0.50,0.00}{##1}}}
\expandafter\def\csname PY@tok@m\endcsname{\def\PY@tc##1{\textcolor[rgb]{0.40,0.40,0.40}{##1}}}
\expandafter\def\csname PY@tok@gh\endcsname{\let\PY@bf=\textbf\def\PY@tc##1{\textcolor[rgb]{0.00,0.00,0.50}{##1}}}
\expandafter\def\csname PY@tok@gu\endcsname{\let\PY@bf=\textbf\def\PY@tc##1{\textcolor[rgb]{0.50,0.00,0.50}{##1}}}
\expandafter\def\csname PY@tok@gd\endcsname{\def\PY@tc##1{\textcolor[rgb]{0.63,0.00,0.00}{##1}}}
\expandafter\def\csname PY@tok@gi\endcsname{\def\PY@tc##1{\textcolor[rgb]{0.00,0.63,0.00}{##1}}}
\expandafter\def\csname PY@tok@gr\endcsname{\def\PY@tc##1{\textcolor[rgb]{1.00,0.00,0.00}{##1}}}
\expandafter\def\csname PY@tok@ge\endcsname{\let\PY@it=\textit}
\expandafter\def\csname PY@tok@gs\endcsname{\let\PY@bf=\textbf}
\expandafter\def\csname PY@tok@gp\endcsname{\let\PY@bf=\textbf\def\PY@tc##1{\textcolor[rgb]{0.00,0.00,0.50}{##1}}}
\expandafter\def\csname PY@tok@go\endcsname{\def\PY@tc##1{\textcolor[rgb]{0.53,0.53,0.53}{##1}}}
\expandafter\def\csname PY@tok@gt\endcsname{\def\PY@tc##1{\textcolor[rgb]{0.00,0.27,0.87}{##1}}}
\expandafter\def\csname PY@tok@err\endcsname{\def\PY@bc##1{\setlength{\fboxsep}{0pt}\fcolorbox[rgb]{1.00,0.00,0.00}{1,1,1}{\strut ##1}}}
\expandafter\def\csname PY@tok@kc\endcsname{\let\PY@bf=\textbf\def\PY@tc##1{\textcolor[rgb]{0.00,0.50,0.00}{##1}}}
\expandafter\def\csname PY@tok@kd\endcsname{\let\PY@bf=\textbf\def\PY@tc##1{\textcolor[rgb]{0.00,0.50,0.00}{##1}}}
\expandafter\def\csname PY@tok@kn\endcsname{\let\PY@bf=\textbf\def\PY@tc##1{\textcolor[rgb]{0.00,0.50,0.00}{##1}}}
\expandafter\def\csname PY@tok@kr\endcsname{\let\PY@bf=\textbf\def\PY@tc##1{\textcolor[rgb]{0.00,0.50,0.00}{##1}}}
\expandafter\def\csname PY@tok@bp\endcsname{\def\PY@tc##1{\textcolor[rgb]{0.00,0.50,0.00}{##1}}}
\expandafter\def\csname PY@tok@fm\endcsname{\def\PY@tc##1{\textcolor[rgb]{0.00,0.00,1.00}{##1}}}
\expandafter\def\csname PY@tok@vc\endcsname{\def\PY@tc##1{\textcolor[rgb]{0.10,0.09,0.49}{##1}}}
\expandafter\def\csname PY@tok@vg\endcsname{\def\PY@tc##1{\textcolor[rgb]{0.10,0.09,0.49}{##1}}}
\expandafter\def\csname PY@tok@vi\endcsname{\def\PY@tc##1{\textcolor[rgb]{0.10,0.09,0.49}{##1}}}
\expandafter\def\csname PY@tok@vm\endcsname{\def\PY@tc##1{\textcolor[rgb]{0.10,0.09,0.49}{##1}}}
\expandafter\def\csname PY@tok@sa\endcsname{\def\PY@tc##1{\textcolor[rgb]{0.73,0.13,0.13}{##1}}}
\expandafter\def\csname PY@tok@sb\endcsname{\def\PY@tc##1{\textcolor[rgb]{0.73,0.13,0.13}{##1}}}
\expandafter\def\csname PY@tok@sc\endcsname{\def\PY@tc##1{\textcolor[rgb]{0.73,0.13,0.13}{##1}}}
\expandafter\def\csname PY@tok@dl\endcsname{\def\PY@tc##1{\textcolor[rgb]{0.73,0.13,0.13}{##1}}}
\expandafter\def\csname PY@tok@s2\endcsname{\def\PY@tc##1{\textcolor[rgb]{0.73,0.13,0.13}{##1}}}
\expandafter\def\csname PY@tok@sh\endcsname{\def\PY@tc##1{\textcolor[rgb]{0.73,0.13,0.13}{##1}}}
\expandafter\def\csname PY@tok@s1\endcsname{\def\PY@tc##1{\textcolor[rgb]{0.73,0.13,0.13}{##1}}}
\expandafter\def\csname PY@tok@mb\endcsname{\def\PY@tc##1{\textcolor[rgb]{0.40,0.40,0.40}{##1}}}
\expandafter\def\csname PY@tok@mf\endcsname{\def\PY@tc##1{\textcolor[rgb]{0.40,0.40,0.40}{##1}}}
\expandafter\def\csname PY@tok@mh\endcsname{\def\PY@tc##1{\textcolor[rgb]{0.40,0.40,0.40}{##1}}}
\expandafter\def\csname PY@tok@mi\endcsname{\def\PY@tc##1{\textcolor[rgb]{0.40,0.40,0.40}{##1}}}
\expandafter\def\csname PY@tok@il\endcsname{\def\PY@tc##1{\textcolor[rgb]{0.40,0.40,0.40}{##1}}}
\expandafter\def\csname PY@tok@mo\endcsname{\def\PY@tc##1{\textcolor[rgb]{0.40,0.40,0.40}{##1}}}
\expandafter\def\csname PY@tok@ch\endcsname{\let\PY@it=\textit\def\PY@tc##1{\textcolor[rgb]{0.25,0.50,0.50}{##1}}}
\expandafter\def\csname PY@tok@cm\endcsname{\let\PY@it=\textit\def\PY@tc##1{\textcolor[rgb]{0.25,0.50,0.50}{##1}}}
\expandafter\def\csname PY@tok@cpf\endcsname{\let\PY@it=\textit\def\PY@tc##1{\textcolor[rgb]{0.25,0.50,0.50}{##1}}}
\expandafter\def\csname PY@tok@c1\endcsname{\let\PY@it=\textit\def\PY@tc##1{\textcolor[rgb]{0.25,0.50,0.50}{##1}}}
\expandafter\def\csname PY@tok@cs\endcsname{\let\PY@it=\textit\def\PY@tc##1{\textcolor[rgb]{0.25,0.50,0.50}{##1}}}

\def\PYZbs{\char`\\}
\def\PYZus{\char`\_}
\def\PYZob{\char`\{}
\def\PYZcb{\char`\}}
\def\PYZca{\char`\^}
\def\PYZam{\char`\&}
\def\PYZlt{\char`\<}
\def\PYZgt{\char`\>}
\def\PYZsh{\char`\#}
\def\PYZpc{\char`\%}
\def\PYZdl{\char`\$}
\def\PYZhy{\char`\-}
\def\PYZsq{\char`\'}
\def\PYZdq{\char`\"}
\def\PYZti{\char`\~}
% for compatibility with earlier versions
\def\PYZat{@}
\def\PYZlb{[}
\def\PYZrb{]}
\makeatother


    % Exact colors from NB
    \definecolor{incolor}{rgb}{0.0, 0.0, 0.5}
    \definecolor{outcolor}{rgb}{0.545, 0.0, 0.0}



    
    % Prevent overflowing lines due to hard-to-break entities
    \sloppy 
    % Setup hyperref package
    \hypersetup{
      breaklinks=true,  % so long urls are correctly broken across lines
      colorlinks=true,
      urlcolor=urlcolor,
      linkcolor=linkcolor,
      citecolor=citecolor,
      }
    % Slightly bigger margins than the latex defaults
    
    \geometry{verbose,tmargin=1in,bmargin=1in,lmargin=1in,rmargin=1in}
    
    

    \begin{document}
    
    
    \maketitle
    
    

    
    \hypertarget{historische-simulation-versus-varianz-kovarianz-methode}{%
\section{Historische Simulation versus
Varianz-Kovarianz-Methode}\label{historische-simulation-versus-varianz-kovarianz-methode}}

© Thomas Robert Holy 2019 Version 1.2.1 Visit me on GitHub:
https://github.com/trh0ly

\hypertarget{voraussetzungen}{%
\subsection{Voraussetzungen:}\label{voraussetzungen}}

An dieser Stelle ist bereits alles vorbereitet. Das Notebook lässt sich
bereits vollständig ausführen

\hypertarget{optional}{%
\subsubsection{Optional:}\label{optional}}

Um ein Portfolio mit anderen Aktienkursen zusammenzustellen, können im
Home-Verzeichnis .csv-Datein hochgeladen werden. Diese müssen die
folgenden Kriterien erfüllen: - Sie umfassen alle denselben Zeitraum -
Es wird ein Punkt zur Dezimaltrennung verwendet - Es wird ein Komma zur
Spaltentrennung verwendet Abrufbar sind die Daten z.B. auf Yahoo
Finance: https://de.finance.yahoo.com/

\hypertarget{grundlegende-einstellungen}{%
\subsection{Grundlegende
Einstellungen:}\label{grundlegende-einstellungen}}

Zunächst müssen die notwendigen Pakete (auch Module) importiert werden,
damit auf diese zugegriffen werden kann.

    \begin{Verbatim}[commandchars=\\\{\}]
{\color{incolor}In [{\color{incolor}1}]:} \PY{k+kn}{import} \PY{n+nn}{numpy} \PY{k}{as} \PY{n+nn}{np} \PY{c+c1}{\PYZsh{} Programmbibliothek die eine einfache Handhabung von Vektoren, Matrizen oder generell großen mehrdimensionalen Arrays ermöglicht}
        \PY{k+kn}{import} \PY{n+nn}{pandas} \PY{k}{as} \PY{n+nn}{pd} \PY{c+c1}{\PYZsh{} Programmbibliothek die Hilfsmittel für die Verwaltung von Daten und deren Analyse anbietet}
        \PY{k+kn}{import} \PY{n+nn}{matplotlib}\PY{n+nn}{.}\PY{n+nn}{pyplot} \PY{k}{as} \PY{n+nn}{plt} \PY{c+c1}{\PYZsh{} Programmbibliothek die es erlaubt mathematische Darstellungen aller Art anzufertigen}
        \PY{k+kn}{import} \PY{n+nn}{matplotlib}\PY{n+nn}{.}\PY{n+nn}{patches} \PY{k}{as} \PY{n+nn}{mpatches}
        \PY{k+kn}{import} \PY{n+nn}{operator} \PY{c+c1}{\PYZsh{} Programmbibliothek, welche die Ausgaben übersichtlicher gestaltet}
        \PY{k+kn}{import} \PY{n+nn}{datetime} \PY{k}{as} \PY{n+nn}{dt} \PY{c+c1}{\PYZsh{} Das datetime\PYZhy{}Modul stellt Klassen bereit, mit denen Datums\PYZhy{} und Uhrzeitangaben auf einfache und komplexe Weise bearbeitet werden können}
        \PY{k+kn}{import} \PY{n+nn}{sys} \PY{c+c1}{\PYZsh{} Dieses Modul bietet Zugriff auf einige Variablen, die vom Interpreter verwendet oder verwaltet werden, sowie auf Funktionen, die stark mit dem Interpreter interagieren}
        \PY{k+kn}{from} \PY{n+nn}{scipy} \PY{k}{import} \PY{n}{stats} \PY{c+c1}{\PYZsh{} SciPy ist ein Python\PYZhy{}basiertes Ökosystem für Open\PYZhy{}Source\PYZhy{}Software für Mathematik, Naturwissenschaften und Ingenieurwissenschaften}
        \PY{k+kn}{import} \PY{n+nn}{prinzip}
        \PY{k+kn}{from} \PY{n+nn}{IPython}\PY{n+nn}{.}\PY{n+nn}{core}\PY{n+nn}{.}\PY{n+nn}{display} \PY{k}{import} \PY{n}{display}\PY{p}{,} \PY{n}{HTML}
\end{Verbatim}

    Anschließend werden Einstellungen definiert, die die Formatierung der
Ausgaben betreffen. Hierfür wird das Modul \texttt{operator} genutzt.
Außerdem wird die Breite des im Folgenden genutzten DataFrames erhöht
und die Größe der Grafiken modifiziert, welche später angezeigt werden
sollen.

    \begin{Verbatim}[commandchars=\\\{\}]
{\color{incolor}In [{\color{incolor}2}]:} \PY{n}{display}\PY{p}{(}\PY{n}{HTML}\PY{p}{(}\PY{l+s+s2}{\PYZdq{}}\PY{l+s+s2}{\PYZlt{}style\PYZgt{}.container }\PY{l+s+s2}{\PYZob{}}\PY{l+s+s2}{ width:100}\PY{l+s+s2}{\PYZpc{}}\PY{l+s+s2}{ !important; \PYZcb{}\PYZlt{}/style\PYZgt{}}\PY{l+s+s2}{\PYZdq{}}\PY{p}{)}\PY{p}{)}
        \PY{n}{SCREEN\PYZus{}WIDTH} \PY{o}{=} \PY{l+m+mi}{100}
        \PY{n}{centered} \PY{o}{=} \PY{n}{operator}\PY{o}{.}\PY{n}{methodcaller}\PY{p}{(}\PY{l+s+s1}{\PYZsq{}}\PY{l+s+s1}{center}\PY{l+s+s1}{\PYZsq{}}\PY{p}{,} \PY{n}{SCREEN\PYZus{}WIDTH}\PY{p}{)}
        \PY{n}{pd}\PY{o}{.}\PY{n}{set\PYZus{}option}\PY{p}{(}\PY{l+s+s1}{\PYZsq{}}\PY{l+s+s1}{display.width}\PY{l+s+s1}{\PYZsq{}}\PY{p}{,} \PY{l+m+mi}{125}\PY{p}{)}
        \PY{n}{plt}\PY{o}{.}\PY{n}{rcParams}\PY{p}{[}\PY{l+s+s2}{\PYZdq{}}\PY{l+s+s2}{figure.figsize}\PY{l+s+s2}{\PYZdq{}}\PY{p}{]} \PY{o}{=} \PY{l+m+mi}{15}\PY{p}{,}\PY{l+m+mf}{12.5}
\end{Verbatim}

    
    \begin{verbatim}
<IPython.core.display.HTML object>
    \end{verbatim}

    
    \hypertarget{datensuxe4tze-einlesen-und-manipulieren}{%
\subsection{Datensätze einlesen und
manipulieren:}\label{datensuxe4tze-einlesen-und-manipulieren}}

Nun werden Datensätze eingelesen. Die Datensätze werden manuell
definiert und anschließend zum Array ``dateinamen'' hinzufügt.
Standardmäßig werden zwei Datensätze (VOW3.DE, FME.DE) definiert und im
Array ``dateinamen'' gespeichert. Hinweis: An dieser Stelle können
andere/weitere Datensätze eingelesen werden, sofern die entsprechenden
Datensätze im Home-Verzeichnis hochgeladen wurden.

    \begin{Verbatim}[commandchars=\\\{\}]
{\color{incolor}In [{\color{incolor}3}]:} \PY{c+c1}{\PYZsh{}\PYZsh{}\PYZsh{}\PYZsh{}\PYZsh{}\PYZsh{}\PYZsh{}\PYZsh{}\PYZsh{}\PYZsh{}\PYZsh{}\PYZsh{}\PYZsh{}\PYZsh{}\PYZsh{}\PYZsh{}\PYZsh{}\PYZsh{}\PYZsh{}\PYZsh{}\PYZsh{}\PYZsh{}\PYZsh{}\PYZsh{}\PYZsh{}\PYZsh{}\PYZsh{}\PYZsh{}\PYZsh{}\PYZsh{}\PYZsh{}\PYZsh{}\PYZsh{}\PYZsh{}\PYZsh{}\PYZsh{}\PYZsh{}\PYZsh{}\PYZsh{}\PYZsh{}\PYZsh{}\PYZsh{}\PYZsh{}\PYZsh{}\PYZsh{}\PYZsh{}\PYZsh{}\PYZsh{}\PYZsh{}\PYZsh{}\PYZsh{}\PYZsh{}\PYZsh{}\PYZsh{}\PYZsh{}\PYZsh{}\PYZsh{}\PYZsh{}\PYZsh{}\PYZsh{}\PYZsh{}\PYZsh{}\PYZsh{}\PYZsh{}\PYZsh{}\PYZsh{}\PYZsh{}\PYZsh{}\PYZsh{}\PYZsh{}\PYZsh{}\PYZsh{}\PYZsh{}\PYZsh{}}
        \PY{c+c1}{\PYZsh{}\PYZhy{}\PYZhy{}\PYZhy{}\PYZhy{}\PYZhy{}\PYZhy{}\PYZhy{}\PYZhy{}\PYZhy{}\PYZhy{}\PYZhy{}\PYZhy{}\PYZhy{}\PYZhy{}\PYZhy{}\PYZhy{}\PYZhy{}\PYZhy{}\PYZhy{}\PYZhy{}\PYZhy{}\PYZhy{}\PYZhy{}\PYZhy{}\PYZhy{}\PYZhy{}\PYZhy{}\PYZhy{}\PYZhy{}\PYZhy{}\PYZhy{}\PYZhy{}\PYZhy{}\PYZhy{}\PYZhy{}\PYZhy{}\PYZhy{}\PYZhy{}\PYZhy{}\PYZhy{}\PYZhy{}\PYZhy{}\PYZhy{}\PYZhy{}\PYZhy{}\PYZhy{}\PYZhy{}\PYZhy{}\PYZhy{}\PYZhy{}\PYZhy{}\PYZhy{}\PYZhy{}\PYZhy{}\PYZhy{}\PYZhy{}\PYZhy{}\PYZhy{}\PYZhy{}\PYZhy{}\PYZhy{}\PYZhy{}\PYZhy{}\PYZhy{}\PYZhy{}\PYZhy{}\PYZhy{}\PYZhy{}\PYZhy{}\PYZhy{}\PYZhy{}\PYZhy{}\PYZhy{}}
        \PY{n}{datensatz1} \PY{o}{=} \PY{l+s+s1}{\PYZsq{}}\PY{l+s+s1}{VOW3.DE}\PY{l+s+s1}{\PYZsq{}}
        \PY{n}{datensatz2} \PY{o}{=} \PY{l+s+s1}{\PYZsq{}}\PY{l+s+s1}{FME.DE}\PY{l+s+s1}{\PYZsq{}}
        
        \PY{c+c1}{\PYZsh{}\PYZhy{}\PYZhy{}\PYZhy{}\PYZhy{}\PYZhy{}\PYZhy{}\PYZhy{}\PYZhy{}\PYZhy{}\PYZhy{}\PYZhy{}\PYZhy{}\PYZhy{}\PYZhy{}\PYZhy{}\PYZhy{}\PYZhy{}\PYZhy{}\PYZhy{}\PYZhy{}\PYZhy{}\PYZhy{}\PYZhy{}\PYZhy{}\PYZhy{}\PYZhy{}\PYZhy{}\PYZhy{}\PYZhy{}\PYZhy{}\PYZhy{}\PYZhy{}\PYZhy{}\PYZhy{}\PYZhy{}\PYZhy{}\PYZhy{}\PYZhy{}\PYZhy{}\PYZhy{}\PYZhy{}\PYZhy{}\PYZhy{}\PYZhy{}\PYZhy{}\PYZhy{}\PYZhy{}\PYZhy{}\PYZhy{}\PYZhy{}\PYZhy{}\PYZhy{}\PYZhy{}\PYZhy{}\PYZhy{}\PYZhy{}}
        \PY{c+c1}{\PYZsh{} Hier können weitere Datensätze definiert werden.}
        
        \PY{c+c1}{\PYZsh{}datensatz3 = \PYZsq{}\PYZsq{}}
        \PY{c+c1}{\PYZsh{}datensatz4 = \PYZsq{}\PYZsq{}}
        \PY{c+c1}{\PYZsh{}datensatz5 = \PYZsq{}\PYZsq{}}
        
        \PY{c+c1}{\PYZsh{}\PYZhy{}\PYZhy{}\PYZhy{}\PYZhy{}\PYZhy{}\PYZhy{}\PYZhy{}\PYZhy{}\PYZhy{}\PYZhy{}\PYZhy{}\PYZhy{}\PYZhy{}\PYZhy{}\PYZhy{}\PYZhy{}\PYZhy{}\PYZhy{}\PYZhy{}\PYZhy{}\PYZhy{}\PYZhy{}\PYZhy{}\PYZhy{}\PYZhy{}\PYZhy{}\PYZhy{}\PYZhy{}\PYZhy{}\PYZhy{}\PYZhy{}\PYZhy{}\PYZhy{}\PYZhy{}\PYZhy{}\PYZhy{}\PYZhy{}\PYZhy{}\PYZhy{}\PYZhy{}\PYZhy{}\PYZhy{}\PYZhy{}\PYZhy{}\PYZhy{}\PYZhy{}\PYZhy{}\PYZhy{}\PYZhy{}\PYZhy{}\PYZhy{}\PYZhy{}\PYZhy{}\PYZhy{}\PYZhy{}\PYZhy{}}
        \PY{c+c1}{\PYZsh{} Diese müssen ggf. noch in diesem Array ergänzt werden.}
        
        \PY{n}{dateinamen} \PY{o}{=} \PY{p}{[}\PY{n}{datensatz1}\PY{p}{,} \PY{n}{datensatz2}\PY{p}{]}
        \PY{c+c1}{\PYZsh{}\PYZhy{}\PYZhy{}\PYZhy{}\PYZhy{}\PYZhy{}\PYZhy{}\PYZhy{}\PYZhy{}\PYZhy{}\PYZhy{}\PYZhy{}\PYZhy{}\PYZhy{}\PYZhy{}\PYZhy{}\PYZhy{}\PYZhy{}\PYZhy{}\PYZhy{}\PYZhy{}\PYZhy{}\PYZhy{}\PYZhy{}\PYZhy{}\PYZhy{}\PYZhy{}\PYZhy{}\PYZhy{}\PYZhy{}\PYZhy{}\PYZhy{}\PYZhy{}\PYZhy{}\PYZhy{}\PYZhy{}\PYZhy{}\PYZhy{}\PYZhy{}\PYZhy{}\PYZhy{}\PYZhy{}\PYZhy{}\PYZhy{}\PYZhy{}\PYZhy{}\PYZhy{}\PYZhy{}\PYZhy{}\PYZhy{}\PYZhy{}\PYZhy{}\PYZhy{}\PYZhy{}\PYZhy{}\PYZhy{}\PYZhy{}\PYZhy{}\PYZhy{}\PYZhy{}\PYZhy{}\PYZhy{}\PYZhy{}\PYZhy{}\PYZhy{}\PYZhy{}\PYZhy{}\PYZhy{}\PYZhy{}\PYZhy{}\PYZhy{}\PYZhy{}\PYZhy{}\PYZhy{}}
        \PY{c+c1}{\PYZsh{}\PYZsh{}\PYZsh{}\PYZsh{}\PYZsh{}\PYZsh{}\PYZsh{}\PYZsh{}\PYZsh{}\PYZsh{}\PYZsh{}\PYZsh{}\PYZsh{}\PYZsh{}\PYZsh{}\PYZsh{}\PYZsh{}\PYZsh{}\PYZsh{}\PYZsh{}\PYZsh{}\PYZsh{}\PYZsh{}\PYZsh{}\PYZsh{}\PYZsh{}\PYZsh{}\PYZsh{}\PYZsh{}\PYZsh{}\PYZsh{}\PYZsh{}\PYZsh{}\PYZsh{}\PYZsh{}\PYZsh{}\PYZsh{}\PYZsh{}\PYZsh{}\PYZsh{}\PYZsh{}\PYZsh{}\PYZsh{}\PYZsh{}\PYZsh{}\PYZsh{}\PYZsh{}\PYZsh{}\PYZsh{}\PYZsh{}\PYZsh{}\PYZsh{}\PYZsh{}\PYZsh{}\PYZsh{}\PYZsh{}\PYZsh{}\PYZsh{}\PYZsh{}\PYZsh{}\PYZsh{}\PYZsh{}\PYZsh{}\PYZsh{}\PYZsh{}\PYZsh{}\PYZsh{}\PYZsh{}\PYZsh{}\PYZsh{}\PYZsh{}\PYZsh{}\PYZsh{}\PYZsh{}}
\end{Verbatim}

    Jetzt soll aus jedem eingelesen Datensatz der Aktienkurs zum jeweiligen
Tag extrahiert werden. Diesen Schritt kann man in Python automatisieren,
indem zunächst die leere Liste ``kurse'' anlegt wird und anschließend
von jedem sich in der Liste ``dateinamen'' befindenden Eintrag die
jeweiligen Spalten ``Date'' und ``Adj Close'' eingelesen werden. Dabei
werden die verschiedenen im Datensatz vorhanden Spalten mit jedem Komma
separiert und Punkte werden als Zeichen für die Dezimaltrennung
interpretiert. Anschließend werden die so extrahierten Daten zum Array
``kurse'' hinzugefügt.

    \begin{Verbatim}[commandchars=\\\{\}]
{\color{incolor}In [{\color{incolor}4}]:} \PY{n}{kurse} \PY{o}{=} \PY{p}{[}\PY{p}{]}     
        \PY{k}{for} \PY{n}{eintrag} \PY{o+ow}{in} \PY{n}{dateinamen}\PY{p}{:}
            \PY{n}{kurs} \PY{o}{=} \PY{n}{pd}\PY{o}{.}\PY{n}{read\PYZus{}csv}\PY{p}{(}\PY{n+nb}{str}\PY{p}{(}\PY{n}{eintrag}\PY{p}{)} \PY{o}{+} \PY{l+s+s1}{\PYZsq{}}\PY{l+s+s1}{.csv}\PY{l+s+s1}{\PYZsq{}}\PY{p}{,}
                        \PY{n}{sep}\PY{o}{=}\PY{l+s+s1}{\PYZsq{}}\PY{l+s+s1}{,}\PY{l+s+s1}{\PYZsq{}}\PY{p}{,}
                        \PY{n}{decimal}\PY{o}{=}\PY{l+s+s1}{\PYZsq{}}\PY{l+s+s1}{.}\PY{l+s+s1}{\PYZsq{}}\PY{p}{,}
                        \PY{n}{usecols}\PY{o}{=}\PY{p}{[}\PY{l+s+s1}{\PYZsq{}}\PY{l+s+s1}{Date}\PY{l+s+s1}{\PYZsq{}}\PY{p}{,}\PY{l+s+s1}{\PYZsq{}}\PY{l+s+s1}{Adj Close}\PY{l+s+s1}{\PYZsq{}}\PY{p}{]}\PY{p}{)}
            \PY{n}{kurse}\PY{o}{.}\PY{n}{append}\PY{p}{(}\PY{n}{kurs}\PY{p}{)}
\end{Verbatim}

    Nun wird das Modul \texttt{datetime} genutzt, um die Datumsspalte des
jeweiligen Datensatzes bearbeitbar zu machen. Zudem wird dem Programm
mitgeteilt, dass die Einträge der Spalte ``Adj Close'' numerisch sind
und mit ihnen gerechnet werden kann. Kommt es dabei zu Fehlern werden
die entsprechende Werte als NaN-Werte behandelt.

    \begin{Verbatim}[commandchars=\\\{\}]
{\color{incolor}In [{\color{incolor}5}]:} \PY{k}{for} \PY{n}{eintrag} \PY{o+ow}{in} \PY{n}{kurse}\PY{p}{:}
            \PY{n}{eintrag}\PY{p}{[}\PY{l+s+s1}{\PYZsq{}}\PY{l+s+s1}{Date}\PY{l+s+s1}{\PYZsq{}}\PY{p}{]} \PY{o}{=} \PY{n}{pd}\PY{o}{.}\PY{n}{to\PYZus{}datetime}\PY{p}{(}\PY{n}{eintrag}\PY{p}{[}\PY{l+s+s1}{\PYZsq{}}\PY{l+s+s1}{Date}\PY{l+s+s1}{\PYZsq{}}\PY{p}{]}\PY{p}{)}
            \PY{n}{eintrag}\PY{p}{[}\PY{l+s+s1}{\PYZsq{}}\PY{l+s+s1}{Adj Close}\PY{l+s+s1}{\PYZsq{}}\PY{p}{]} \PY{o}{=} \PY{n}{pd}\PY{o}{.}\PY{n}{to\PYZus{}numeric}\PY{p}{(}\PY{n}{eintrag}\PY{p}{[}\PY{l+s+s1}{\PYZsq{}}\PY{l+s+s1}{Adj Close}\PY{l+s+s1}{\PYZsq{}}\PY{p}{]}\PY{p}{,} \PY{n}{errors}\PY{o}{=}\PY{l+s+s1}{\PYZsq{}}\PY{l+s+s1}{coerce}\PY{l+s+s1}{\PYZsq{}}\PY{p}{)}
            
        \PY{n}{eintrag}
\end{Verbatim}

\begin{Verbatim}[commandchars=\\\{\}]
{\color{outcolor}Out[{\color{outcolor}5}]:}           Date  Adj Close
        0   2018-07-11  79.685799
        1   2018-07-12  81.349548
        2   2018-07-13  81.775162
        3   2018-07-16  82.316849
        4   2018-07-17  83.110023
        ..         {\ldots}        {\ldots}
        247 2019-07-05  71.059998
        248 2019-07-08  70.480003
        249 2019-07-09  66.639999
        250 2019-07-10  67.739998
        251 2019-07-11  69.559998
        
        [252 rows x 2 columns]
\end{Verbatim}
            
    \hypertarget{dataframe-erzeugen-und-daten-zusammentragen}{%
\subsection{Dataframe erzeugen und Daten
zusammentragen:}\label{dataframe-erzeugen-und-daten-zusammentragen}}

Anschließend werden die eingelesenen Daten in einem DataFrame
zusammengetragen, wofür das Modul \texttt{Pandas} verwendet wird. Ein
DataFrame kann ähnlich wie eine Excel-Tabelle verstanden werden. Dieser
Vorgang kann automatisiert werden, damit die einzelnen Spalten nicht
manuell hinzufügt werden müssen. Zunächst wird dafür der leere DataFrame
``kurschart'' angelegt. Als nächstes wird für jeden Eintrag in dem Array
``kurse'' zunächst der entsprechende Aktienkurs in den DataFrame
überführt und anschließend wird die dazugehörige Rendite berechnet. Um
Aktienkurse und dazugehörige Renditen auseinander halten zu können,
werden die Dateinamen aus dem Array ``dateinamen'' als Tabellenkopf
verwendet.

    \begin{Verbatim}[commandchars=\\\{\}]
{\color{incolor}In [{\color{incolor}6}]:} \PY{n}{kurschart} \PY{o}{=} \PY{n}{pd}\PY{o}{.}\PY{n}{DataFrame}\PY{p}{(}\PY{p}{)}
        \PY{n}{zaehler} \PY{o}{=} \PY{l+m+mi}{0}
        
        \PY{k}{for} \PY{n}{eintrag} \PY{o+ow}{in} \PY{n}{kurse}\PY{p}{:}
            \PY{n}{x} \PY{o}{=} \PY{n}{dateinamen}\PY{p}{[}\PY{n}{zaehler}\PY{p}{]}
            \PY{n}{kurschart}\PY{p}{[}\PY{l+s+s1}{\PYZsq{}}\PY{l+s+s1}{Aktienkurs }\PY{l+s+s1}{\PYZsq{}} \PY{o}{+} \PY{n+nb}{str}\PY{p}{(}\PY{n}{x}\PY{p}{)}\PY{p}{]} \PY{o}{=} \PY{n}{eintrag}\PY{p}{[}\PY{l+s+s1}{\PYZsq{}}\PY{l+s+s1}{Adj Close}\PY{l+s+s1}{\PYZsq{}}\PY{p}{]}
            \PY{n}{kurschart}\PY{p}{[}\PY{l+s+s1}{\PYZsq{}}\PY{l+s+s1}{Rendite }\PY{l+s+s1}{\PYZsq{}} \PY{o}{+} \PY{n+nb}{str}\PY{p}{(}\PY{n}{x}\PY{p}{)}\PY{p}{]} \PY{o}{=} \PY{p}{(}\PY{n}{eintrag}\PY{p}{[}\PY{l+s+s1}{\PYZsq{}}\PY{l+s+s1}{Adj Close}\PY{l+s+s1}{\PYZsq{}}\PY{p}{]} \PY{o}{\PYZhy{}} \PY{n}{eintrag}\PY{p}{[}\PY{l+s+s1}{\PYZsq{}}\PY{l+s+s1}{Adj Close}\PY{l+s+s1}{\PYZsq{}}\PY{p}{]}\PY{o}{.}\PY{n}{shift}\PY{p}{(}\PY{n}{periods} \PY{o}{=} \PY{l+m+mi}{1}\PY{p}{)}\PY{p}{)} \PY{o}{/} \PY{n}{eintrag}\PY{p}{[}\PY{l+s+s1}{\PYZsq{}}\PY{l+s+s1}{Adj Close}\PY{l+s+s1}{\PYZsq{}}\PY{p}{]}\PY{o}{.}\PY{n}{shift}\PY{p}{(}\PY{n}{periods} \PY{o}{=} \PY{l+m+mi}{1}\PY{p}{)} 
            \PY{n}{zaehler} \PY{o}{+}\PY{o}{=} \PY{l+m+mi}{1}
            
        \PY{n}{kurschart}
\end{Verbatim}

\begin{Verbatim}[commandchars=\\\{\}]
{\color{outcolor}Out[{\color{outcolor}6}]:}      Aktienkurs VOW3.DE  Rendite VOW3.DE  Aktienkurs FME.DE  Rendite FME.DE
        0            138.162201              NaN          79.685799             NaN
        1            138.239578         0.000560          81.349548        0.020879
        2            139.303711         0.007698          81.775162        0.005232
        3            138.065460        -0.008889          82.316849        0.006624
        4            139.381104         0.009529          83.110023        0.009636
        ..                  {\ldots}              {\ldots}                {\ldots}             {\ldots}
        247          154.600006        -0.000259          71.059998        0.005092
        248          154.800003         0.001294          70.480003       -0.008162
        249          153.960007        -0.005426          66.639999       -0.054484
        250          152.399994        -0.010133          67.739998        0.016507
        251          153.160004         0.004987          69.559998        0.026867
        
        [252 rows x 4 columns]
\end{Verbatim}
            
    Um die Portfolio-Rendite zu ermitteln, wird jede zweite Spalte des
DataFrames ``kurschart'' ausgewählt (dies sind die jeweiligen
Rendite-Spalten) und in einem zweiten DataFrame (``hilfs\_dataframe'')
abgespeichert. Anschließend werden die Spalten zeilenweise addiert und
durch die Anzahl der Datensätze geteilt, welche sich im Array
``dateinamen'' befinden. Dies entspricht einer naiven Diversifikation.

    \begin{Verbatim}[commandchars=\\\{\}]
{\color{incolor}In [{\color{incolor}7}]:} \PY{n}{hilfs\PYZus{}dataframe} \PY{o}{=} \PY{n}{kurschart}\PY{o}{.}\PY{n}{iloc}\PY{p}{[}\PY{p}{:}\PY{p}{,} \PY{l+m+mi}{1}\PY{p}{:}\PY{p}{:}\PY{l+m+mi}{2}\PY{p}{]}
        \PY{n}{hilfs\PYZus{}dataframe}\PY{p}{[}\PY{l+s+s1}{\PYZsq{}}\PY{l+s+s1}{PF\PYZhy{}Rendite}\PY{l+s+s1}{\PYZsq{}}\PY{p}{]} \PY{o}{=} \PY{n}{hilfs\PYZus{}dataframe}\PY{o}{.}\PY{n}{sum}\PY{p}{(}\PY{n}{axis} \PY{o}{=} \PY{l+m+mi}{1}\PY{p}{,} \PY{n}{skipna} \PY{o}{=} \PY{k+kc}{True}\PY{p}{)} \PY{o}{/} \PY{n+nb}{len}\PY{p}{(}\PY{n}{dateinamen}\PY{p}{)}
        
        \PY{n}{hilfs\PYZus{}dataframe}
\end{Verbatim}

\begin{Verbatim}[commandchars=\\\{\}]
{\color{outcolor}Out[{\color{outcolor}7}]:}      Rendite VOW3.DE  Rendite FME.DE  PF-Rendite
        0                NaN             NaN    0.000000
        1           0.000560        0.020879    0.010719
        2           0.007698        0.005232    0.006465
        3          -0.008889        0.006624   -0.001132
        4           0.009529        0.009636    0.009582
        ..               {\ldots}             {\ldots}         {\ldots}
        247        -0.000259        0.005092    0.002417
        248         0.001294       -0.008162   -0.003434
        249        -0.005426       -0.054484   -0.029955
        250        -0.010133        0.016507    0.003187
        251         0.004987        0.026867    0.015927
        
        [252 rows x 3 columns]
\end{Verbatim}
            
    Diese Portfolio-Rendite wird nun dem ursprünglichen DataFrame
``kurschart'' angefügt. Zudem erhält der DataFrame eine Datumsspalte,
welche gleichzeitig als Index verwendet wird.

    \begin{Verbatim}[commandchars=\\\{\}]
{\color{incolor}In [{\color{incolor}8}]:} \PY{n}{kurschart}\PY{p}{[}\PY{l+s+s1}{\PYZsq{}}\PY{l+s+s1}{Rendite\PYZhy{}PF}\PY{l+s+s1}{\PYZsq{}}\PY{p}{]} \PY{o}{=} \PY{n}{hilfs\PYZus{}dataframe}\PY{p}{[}\PY{l+s+s1}{\PYZsq{}}\PY{l+s+s1}{PF\PYZhy{}Rendite}\PY{l+s+s1}{\PYZsq{}}\PY{p}{]}
        \PY{n}{kurschart}\PY{p}{[}\PY{l+s+s1}{\PYZsq{}}\PY{l+s+s1}{Datum}\PY{l+s+s1}{\PYZsq{}}\PY{p}{]} \PY{o}{=} \PY{n}{eintrag}\PY{p}{[}\PY{l+s+s1}{\PYZsq{}}\PY{l+s+s1}{Date}\PY{l+s+s1}{\PYZsq{}}\PY{p}{]}
        \PY{n}{kurschart} \PY{o}{=} \PY{n}{kurschart}\PY{o}{.}\PY{n}{set\PYZus{}index}\PY{p}{(}\PY{l+s+s1}{\PYZsq{}}\PY{l+s+s1}{Datum}\PY{l+s+s1}{\PYZsq{}}\PY{p}{)}
\end{Verbatim}

    Nun kann das DataFrame ausgeben werden:

    \begin{Verbatim}[commandchars=\\\{\}]
{\color{incolor}In [{\color{incolor}9}]:} \PY{n+nb}{print}\PY{p}{(}\PY{l+s+s1}{\PYZsq{}}\PY{l+s+s1}{\PYZsh{}}\PY{l+s+s1}{\PYZsq{}} \PY{o}{+} \PY{n}{SCREEN\PYZus{}WIDTH} \PY{o}{*} \PY{l+s+s1}{\PYZsq{}}\PY{l+s+s1}{\PYZhy{}}\PY{l+s+s1}{\PYZsq{}} \PY{o}{+} \PY{l+s+s1}{\PYZsq{}}\PY{l+s+s1}{\PYZsh{}}\PY{l+s+s1}{\PYZsq{}}\PY{p}{)}
        \PY{n+nb}{print}\PY{p}{(}\PY{l+s+s1}{\PYZsq{}}\PY{l+s+s1}{|}\PY{l+s+s1}{\PYZsq{}} \PY{o}{+} \PY{n}{centered}\PY{p}{(}\PY{l+s+s1}{\PYZsq{}}\PY{l+s+s1}{[INFO] Der DataFrame mit den Aktienkursen und deren zugehörigen Renditen ergibt sich wie folgt: }\PY{l+s+s1}{\PYZsq{}}\PY{p}{)} \PY{o}{+} \PY{l+s+s1}{\PYZsq{}}\PY{l+s+s1}{|}\PY{l+s+s1}{\PYZsq{}}\PY{p}{)}
        \PY{n+nb}{print}\PY{p}{(}\PY{l+s+s1}{\PYZsq{}}\PY{l+s+s1}{\PYZsh{}}\PY{l+s+s1}{\PYZsq{}} \PY{o}{+} \PY{n}{SCREEN\PYZus{}WIDTH} \PY{o}{*} \PY{l+s+s1}{\PYZsq{}}\PY{l+s+s1}{\PYZhy{}}\PY{l+s+s1}{\PYZsq{}} \PY{o}{+} \PY{l+s+s1}{\PYZsq{}}\PY{l+s+s1}{\PYZsh{}}\PY{l+s+s1}{\PYZsq{}}\PY{p}{)}
        \PY{n+nb}{print}\PY{p}{(}\PY{n}{kurschart}\PY{p}{)}
        \PY{n+nb}{print}\PY{p}{(}\PY{l+s+s1}{\PYZsq{}}\PY{l+s+s1}{\PYZsh{}}\PY{l+s+s1}{\PYZsq{}} \PY{o}{+} \PY{n}{SCREEN\PYZus{}WIDTH} \PY{o}{*} \PY{l+s+s1}{\PYZsq{}}\PY{l+s+s1}{\PYZhy{}}\PY{l+s+s1}{\PYZsq{}} \PY{o}{+} \PY{l+s+s1}{\PYZsq{}}\PY{l+s+s1}{\PYZsh{}}\PY{l+s+s1}{\PYZsq{}}\PY{p}{)}
        \PY{n+nb}{print}\PY{p}{(}\PY{l+s+s1}{\PYZsq{}}\PY{l+s+s1}{|}\PY{l+s+s1}{\PYZsq{}} \PY{o}{+} \PY{n}{centered}\PY{p}{(}\PY{l+s+s1}{\PYZsq{}}\PY{l+s+s1}{[INFO] Überprüfung des obigen DataFrames auf NaN\PYZhy{}Werte: }\PY{l+s+s1}{\PYZsq{}}\PY{p}{)} \PY{o}{+} \PY{l+s+s1}{\PYZsq{}}\PY{l+s+s1}{|}\PY{l+s+s1}{\PYZsq{}}\PY{p}{)}
        \PY{n+nb}{print}\PY{p}{(}\PY{l+s+s1}{\PYZsq{}}\PY{l+s+s1}{\PYZsh{}}\PY{l+s+s1}{\PYZsq{}} \PY{o}{+} \PY{n}{SCREEN\PYZus{}WIDTH} \PY{o}{*} \PY{l+s+s1}{\PYZsq{}}\PY{l+s+s1}{\PYZhy{}}\PY{l+s+s1}{\PYZsq{}} \PY{o}{+} \PY{l+s+s1}{\PYZsq{}}\PY{l+s+s1}{\PYZsh{}}\PY{l+s+s1}{\PYZsq{}}\PY{p}{)}
        \PY{n+nb}{print}\PY{p}{(}\PY{n}{kurschart}\PY{o}{.}\PY{n}{isnull}\PY{p}{(}\PY{p}{)}\PY{o}{.}\PY{n}{sum}\PY{p}{(}\PY{p}{)}\PY{p}{)}
        \PY{n+nb}{print}\PY{p}{(}\PY{l+s+s1}{\PYZsq{}}\PY{l+s+s1}{\PYZsh{}}\PY{l+s+s1}{\PYZsq{}} \PY{o}{+} \PY{n}{SCREEN\PYZus{}WIDTH} \PY{o}{*} \PY{l+s+s1}{\PYZsq{}}\PY{l+s+s1}{\PYZhy{}}\PY{l+s+s1}{\PYZsq{}} \PY{o}{+} \PY{l+s+s1}{\PYZsq{}}\PY{l+s+s1}{\PYZsh{}}\PY{l+s+s1}{\PYZsq{}}\PY{p}{)}
\end{Verbatim}

    \begin{Verbatim}[commandchars=\\\{\}]
\#----------------------------------------------------------------------------------------------------\#
|  [INFO] Der DataFrame mit den Aktienkursen und deren zugehörigen Renditen ergibt sich wie folgt:   |
\#----------------------------------------------------------------------------------------------------\#
            Aktienkurs VOW3.DE  Rendite VOW3.DE  Aktienkurs FME.DE  Rendite FME.DE  Rendite-PF
Datum                                                                                         
2018-07-11          138.162201              NaN          79.685799             NaN    0.000000
2018-07-12          138.239578         0.000560          81.349548        0.020879    0.010719
2018-07-13          139.303711         0.007698          81.775162        0.005232    0.006465
2018-07-16          138.065460        -0.008889          82.316849        0.006624   -0.001132
2018-07-17          139.381104         0.009529          83.110023        0.009636    0.009582
{\ldots}                        {\ldots}              {\ldots}                {\ldots}             {\ldots}         {\ldots}
2019-07-05          154.600006        -0.000259          71.059998        0.005092    0.002417
2019-07-08          154.800003         0.001294          70.480003       -0.008162   -0.003434
2019-07-09          153.960007        -0.005426          66.639999       -0.054484   -0.029955
2019-07-10          152.399994        -0.010133          67.739998        0.016507    0.003187
2019-07-11          153.160004         0.004987          69.559998        0.026867    0.015927

[252 rows x 5 columns]
\#----------------------------------------------------------------------------------------------------\#
|                      [INFO] Überprüfung des obigen DataFrames auf NaN-Werte:                       |
\#----------------------------------------------------------------------------------------------------\#
Aktienkurs VOW3.DE    0
Rendite VOW3.DE       1
Aktienkurs FME.DE     0
Rendite FME.DE        1
Rendite-PF            0
dtype: int64
\#----------------------------------------------------------------------------------------------------\#

    \end{Verbatim}

    \hypertarget{streudiagramm-anzeigen}{%
\subsection{Streudiagramm anzeigen:}\label{streudiagramm-anzeigen}}

Mithilfe eines Streudiagramms kann die gemeinsame Verteilung von zwei
Datensätzen bzw. deren Abhängigkeitsstruktur betrachtet werden. Um ein
solches Streudiagramm zu plotten wird zunächst den Datensatz ``x'' und
den Datensatz ``y'' definiert. Anschließend werden die zugehörigen
Renditen in einer Liste gespeichert und um NaN-Werte bereinigt.

    \begin{Verbatim}[commandchars=\\\{\}]
{\color{incolor}In [{\color{incolor}10}]:} \PY{c+c1}{\PYZsh{}\PYZsh{}\PYZsh{}\PYZsh{}\PYZsh{}\PYZsh{}\PYZsh{}\PYZsh{}\PYZsh{}\PYZsh{}\PYZsh{}\PYZsh{}\PYZsh{}\PYZsh{}\PYZsh{}\PYZsh{}\PYZsh{}\PYZsh{}\PYZsh{}\PYZsh{}\PYZsh{}\PYZsh{}\PYZsh{}\PYZsh{}\PYZsh{}\PYZsh{}\PYZsh{}\PYZsh{}\PYZsh{}\PYZsh{}\PYZsh{}\PYZsh{}\PYZsh{}\PYZsh{}\PYZsh{}\PYZsh{}\PYZsh{}\PYZsh{}\PYZsh{}\PYZsh{}\PYZsh{}\PYZsh{}\PYZsh{}\PYZsh{}\PYZsh{}\PYZsh{}\PYZsh{}\PYZsh{}\PYZsh{}\PYZsh{}\PYZsh{}\PYZsh{}\PYZsh{}\PYZsh{}\PYZsh{}\PYZsh{}\PYZsh{}\PYZsh{}\PYZsh{}\PYZsh{}\PYZsh{}\PYZsh{}\PYZsh{}\PYZsh{}\PYZsh{}\PYZsh{}\PYZsh{}\PYZsh{}\PYZsh{}\PYZsh{}\PYZsh{}\PYZsh{}\PYZsh{}\PYZsh{}}
         \PY{c+c1}{\PYZsh{}\PYZhy{}\PYZhy{}\PYZhy{}\PYZhy{}\PYZhy{}\PYZhy{}\PYZhy{}\PYZhy{}\PYZhy{}\PYZhy{}\PYZhy{}\PYZhy{}\PYZhy{}\PYZhy{}\PYZhy{}\PYZhy{}\PYZhy{}\PYZhy{}\PYZhy{}\PYZhy{}\PYZhy{}\PYZhy{}\PYZhy{}\PYZhy{}\PYZhy{}\PYZhy{}\PYZhy{}\PYZhy{}\PYZhy{}\PYZhy{}\PYZhy{}\PYZhy{}\PYZhy{}\PYZhy{}\PYZhy{}\PYZhy{}\PYZhy{}\PYZhy{}\PYZhy{}\PYZhy{}\PYZhy{}\PYZhy{}\PYZhy{}\PYZhy{}\PYZhy{}\PYZhy{}\PYZhy{}\PYZhy{}\PYZhy{}\PYZhy{}\PYZhy{}\PYZhy{}\PYZhy{}\PYZhy{}\PYZhy{}\PYZhy{}\PYZhy{}\PYZhy{}\PYZhy{}\PYZhy{}\PYZhy{}\PYZhy{}\PYZhy{}\PYZhy{}\PYZhy{}\PYZhy{}\PYZhy{}\PYZhy{}\PYZhy{}\PYZhy{}\PYZhy{}\PYZhy{}\PYZhy{}}
         
         \PY{n}{x} \PY{o}{=} \PY{n}{dateinamen}\PY{p}{[}\PY{l+m+mi}{0}\PY{p}{]}
         \PY{n}{y} \PY{o}{=} \PY{n}{dateinamen}\PY{p}{[}\PY{l+m+mi}{1}\PY{p}{]}
         
         \PY{c+c1}{\PYZsh{}\PYZhy{}\PYZhy{}\PYZhy{}\PYZhy{}\PYZhy{}\PYZhy{}\PYZhy{}\PYZhy{}\PYZhy{}\PYZhy{}\PYZhy{}\PYZhy{}\PYZhy{}\PYZhy{}\PYZhy{}\PYZhy{}\PYZhy{}\PYZhy{}\PYZhy{}\PYZhy{}\PYZhy{}\PYZhy{}\PYZhy{}\PYZhy{}\PYZhy{}\PYZhy{}\PYZhy{}\PYZhy{}\PYZhy{}\PYZhy{}\PYZhy{}\PYZhy{}\PYZhy{}\PYZhy{}\PYZhy{}\PYZhy{}\PYZhy{}\PYZhy{}\PYZhy{}\PYZhy{}\PYZhy{}\PYZhy{}\PYZhy{}\PYZhy{}\PYZhy{}\PYZhy{}\PYZhy{}\PYZhy{}\PYZhy{}\PYZhy{}\PYZhy{}\PYZhy{}\PYZhy{}\PYZhy{}\PYZhy{}\PYZhy{}\PYZhy{}\PYZhy{}\PYZhy{}\PYZhy{}\PYZhy{}\PYZhy{}\PYZhy{}\PYZhy{}\PYZhy{}\PYZhy{}\PYZhy{}\PYZhy{}\PYZhy{}\PYZhy{}\PYZhy{}\PYZhy{}\PYZhy{}}
         \PY{c+c1}{\PYZsh{}\PYZsh{}\PYZsh{}\PYZsh{}\PYZsh{}\PYZsh{}\PYZsh{}\PYZsh{}\PYZsh{}\PYZsh{}\PYZsh{}\PYZsh{}\PYZsh{}\PYZsh{}\PYZsh{}\PYZsh{}\PYZsh{}\PYZsh{}\PYZsh{}\PYZsh{}\PYZsh{}\PYZsh{}\PYZsh{}\PYZsh{}\PYZsh{}\PYZsh{}\PYZsh{}\PYZsh{}\PYZsh{}\PYZsh{}\PYZsh{}\PYZsh{}\PYZsh{}\PYZsh{}\PYZsh{}\PYZsh{}\PYZsh{}\PYZsh{}\PYZsh{}\PYZsh{}\PYZsh{}\PYZsh{}\PYZsh{}\PYZsh{}\PYZsh{}\PYZsh{}\PYZsh{}\PYZsh{}\PYZsh{}\PYZsh{}\PYZsh{}\PYZsh{}\PYZsh{}\PYZsh{}\PYZsh{}\PYZsh{}\PYZsh{}\PYZsh{}\PYZsh{}\PYZsh{}\PYZsh{}\PYZsh{}\PYZsh{}\PYZsh{}\PYZsh{}\PYZsh{}\PYZsh{}\PYZsh{}\PYZsh{}\PYZsh{}\PYZsh{}\PYZsh{}\PYZsh{}\PYZsh{}}
         
         \PY{c+c1}{\PYZsh{} DataFrame\PYZhy{}Werte in eine Listen überführen}
         \PY{n}{values\PYZus{}datensatz1} \PY{o}{=} \PY{n}{kurschart}\PY{p}{[}\PY{l+s+s1}{\PYZsq{}}\PY{l+s+s1}{Rendite }\PY{l+s+s1}{\PYZsq{}} \PY{o}{+} \PY{n+nb}{str}\PY{p}{(}\PY{n}{x}\PY{p}{)}\PY{p}{]}\PY{o}{.}\PY{n}{values}\PY{o}{.}\PY{n}{tolist}\PY{p}{(}\PY{p}{)}
         \PY{n}{values\PYZus{}datensatz2} \PY{o}{=} \PY{n}{kurschart}\PY{p}{[}\PY{l+s+s1}{\PYZsq{}}\PY{l+s+s1}{Rendite }\PY{l+s+s1}{\PYZsq{}} \PY{o}{+} \PY{n+nb}{str}\PY{p}{(}\PY{n}{y}\PY{p}{)}\PY{p}{]}\PY{o}{.}\PY{n}{values}\PY{o}{.}\PY{n}{tolist}\PY{p}{(}\PY{p}{)}
         
         \PY{c+c1}{\PYZsh{}\PYZhy{}\PYZhy{}\PYZhy{}\PYZhy{}\PYZhy{}\PYZhy{}\PYZhy{}\PYZhy{}\PYZhy{}\PYZhy{}\PYZhy{}\PYZhy{}\PYZhy{}\PYZhy{}\PYZhy{}\PYZhy{}\PYZhy{}\PYZhy{}\PYZhy{}\PYZhy{}\PYZhy{}\PYZhy{}\PYZhy{}\PYZhy{}\PYZhy{}\PYZhy{}\PYZhy{}\PYZhy{}\PYZhy{}\PYZhy{}\PYZhy{}\PYZhy{}\PYZhy{}\PYZhy{}\PYZhy{}\PYZhy{}\PYZhy{}\PYZhy{}\PYZhy{}\PYZhy{}\PYZhy{}\PYZhy{}\PYZhy{}\PYZhy{}\PYZhy{}\PYZhy{}\PYZhy{}\PYZhy{}\PYZhy{}\PYZhy{}\PYZhy{}\PYZhy{}\PYZhy{}\PYZhy{}\PYZhy{}\PYZhy{}}
         \PY{c+c1}{\PYZsh{} Liste in ein Numpy\PYZhy{}Array transformieren}
         \PY{n}{values\PYZus{}datensatz1} \PY{o}{=} \PY{n}{np}\PY{o}{.}\PY{n}{array}\PY{p}{(}\PY{n}{values\PYZus{}datensatz1}\PY{p}{)}
         \PY{n}{values\PYZus{}datensatz2} \PY{o}{=} \PY{n}{np}\PY{o}{.}\PY{n}{array}\PY{p}{(}\PY{n}{values\PYZus{}datensatz2}\PY{p}{)}
         
         \PY{c+c1}{\PYZsh{}\PYZhy{}\PYZhy{}\PYZhy{}\PYZhy{}\PYZhy{}\PYZhy{}\PYZhy{}\PYZhy{}\PYZhy{}\PYZhy{}\PYZhy{}\PYZhy{}\PYZhy{}\PYZhy{}\PYZhy{}\PYZhy{}\PYZhy{}\PYZhy{}\PYZhy{}\PYZhy{}\PYZhy{}\PYZhy{}\PYZhy{}\PYZhy{}\PYZhy{}\PYZhy{}\PYZhy{}\PYZhy{}\PYZhy{}\PYZhy{}\PYZhy{}\PYZhy{}\PYZhy{}\PYZhy{}\PYZhy{}\PYZhy{}\PYZhy{}\PYZhy{}\PYZhy{}\PYZhy{}\PYZhy{}\PYZhy{}\PYZhy{}\PYZhy{}\PYZhy{}\PYZhy{}\PYZhy{}\PYZhy{}\PYZhy{}\PYZhy{}\PYZhy{}\PYZhy{}\PYZhy{}\PYZhy{}\PYZhy{}\PYZhy{}}
         \PY{c+c1}{\PYZsh{} Bereinigung um NaN\PYZhy{}Werte und Sicherstellung das beide Datensätze die gleiche Länge aufweisen}
         \PY{n}{values\PYZus{}datensatz1} \PY{o}{=} \PY{n}{values\PYZus{}datensatz1}\PY{p}{[}\PY{n}{np}\PY{o}{.}\PY{n}{logical\PYZus{}not}\PY{p}{(}\PY{n}{np}\PY{o}{.}\PY{n}{isnan}\PY{p}{(}\PY{n}{values\PYZus{}datensatz1}\PY{p}{)}\PY{p}{)}\PY{p}{]}
         \PY{n}{values\PYZus{}datensatz2} \PY{o}{=} \PY{n}{values\PYZus{}datensatz2}\PY{p}{[}\PY{n}{np}\PY{o}{.}\PY{n}{logical\PYZus{}not}\PY{p}{(}\PY{n}{np}\PY{o}{.}\PY{n}{isnan}\PY{p}{(}\PY{n}{values\PYZus{}datensatz2}\PY{p}{)}\PY{p}{)}\PY{p}{]}
         \PY{n}{values\PYZus{}datensatz1} \PY{o}{=} \PY{n}{values\PYZus{}datensatz1}\PY{p}{[}\PY{n}{values\PYZus{}datensatz1} \PY{o}{!=} \PY{l+m+mf}{0.0}\PY{p}{]}
         \PY{n}{values\PYZus{}datensatz2} \PY{o}{=} \PY{n}{values\PYZus{}datensatz2}\PY{p}{[}\PY{n}{values\PYZus{}datensatz2} \PY{o}{!=} \PY{l+m+mf}{0.0}\PY{p}{]}
         \PY{n}{min\PYZus{}len} \PY{o}{=} \PY{n+nb}{min}\PY{p}{(}\PY{n+nb}{len}\PY{p}{(}\PY{n}{values\PYZus{}datensatz1}\PY{p}{)}\PY{p}{,} \PY{n+nb}{len}\PY{p}{(}\PY{n}{values\PYZus{}datensatz2}\PY{p}{)}\PY{p}{)}
         \PY{n}{values\PYZus{}datensatz1} \PY{o}{=} \PY{n}{values\PYZus{}datensatz1}\PY{p}{[}\PY{p}{:}\PY{n}{min\PYZus{}len}\PY{p}{]}
         \PY{n}{values\PYZus{}datensatz2} \PY{o}{=} \PY{n}{values\PYZus{}datensatz2}\PY{p}{[}\PY{p}{:}\PY{n}{min\PYZus{}len}\PY{p}{]}
         
         \PY{c+c1}{\PYZsh{}\PYZhy{}\PYZhy{}\PYZhy{}\PYZhy{}\PYZhy{}\PYZhy{}\PYZhy{}\PYZhy{}\PYZhy{}\PYZhy{}\PYZhy{}\PYZhy{}\PYZhy{}\PYZhy{}\PYZhy{}\PYZhy{}\PYZhy{}\PYZhy{}\PYZhy{}\PYZhy{}\PYZhy{}\PYZhy{}\PYZhy{}\PYZhy{}\PYZhy{}\PYZhy{}\PYZhy{}\PYZhy{}\PYZhy{}\PYZhy{}\PYZhy{}\PYZhy{}\PYZhy{}\PYZhy{}\PYZhy{}\PYZhy{}\PYZhy{}\PYZhy{}\PYZhy{}\PYZhy{}\PYZhy{}\PYZhy{}\PYZhy{}\PYZhy{}\PYZhy{}\PYZhy{}\PYZhy{}\PYZhy{}\PYZhy{}\PYZhy{}\PYZhy{}\PYZhy{}\PYZhy{}\PYZhy{}\PYZhy{}\PYZhy{}}
         \PY{c+c1}{\PYZsh{} Berechnung der Korrelation}
         \PY{n}{corr} \PY{o}{=} \PY{n}{np}\PY{o}{.}\PY{n}{ma}\PY{o}{.}\PY{n}{corrcoef}\PY{p}{(}\PY{n}{values\PYZus{}datensatz1}\PY{p}{,} \PY{n}{values\PYZus{}datensatz2}\PY{p}{)}
\end{Verbatim}

    Nun kann das Streudiagramm für die jeweiligen Aktienkurse geplottet
werden:

    \begin{Verbatim}[commandchars=\\\{\}]
{\color{incolor}In [{\color{incolor}11}]:} \PY{c+c1}{\PYZsh{}\PYZhy{}\PYZhy{}\PYZhy{}\PYZhy{}\PYZhy{}\PYZhy{}\PYZhy{}\PYZhy{}\PYZhy{}\PYZhy{}\PYZhy{}\PYZhy{}\PYZhy{}\PYZhy{}\PYZhy{}\PYZhy{}\PYZhy{}\PYZhy{}\PYZhy{}\PYZhy{}\PYZhy{}\PYZhy{}\PYZhy{}\PYZhy{}\PYZhy{}\PYZhy{}\PYZhy{}\PYZhy{}\PYZhy{}\PYZhy{}\PYZhy{}\PYZhy{}\PYZhy{}\PYZhy{}\PYZhy{}\PYZhy{}\PYZhy{}\PYZhy{}\PYZhy{}\PYZhy{}\PYZhy{}\PYZhy{}\PYZhy{}\PYZhy{}\PYZhy{}\PYZhy{}\PYZhy{}\PYZhy{}\PYZhy{}\PYZhy{}\PYZhy{}\PYZhy{}\PYZhy{}\PYZhy{}\PYZhy{}\PYZhy{}}
         \PY{c+c1}{\PYZsh{} Grafik plotten }
         \PY{n}{plt}\PY{o}{.}\PY{n}{scatter}\PY{p}{(}\PY{n}{values\PYZus{}datensatz1}\PY{p}{,} \PY{n}{values\PYZus{}datensatz2}\PY{p}{)}
         \PY{n}{plt}\PY{o}{.}\PY{n}{xlabel}\PY{p}{(}\PY{l+s+s1}{\PYZsq{}}\PY{l+s+s1}{Rendite des Datensatzes }\PY{l+s+s1}{\PYZsq{}} \PY{o}{+} \PY{n+nb}{str}\PY{p}{(}\PY{n}{datensatz1}\PY{p}{)}\PY{p}{)}
         \PY{n}{plt}\PY{o}{.}\PY{n}{ylabel}\PY{p}{(}\PY{l+s+s1}{\PYZsq{}}\PY{l+s+s1}{Rendite des Datensatzes }\PY{l+s+s1}{\PYZsq{}} \PY{o}{+} \PY{n+nb}{str}\PY{p}{(}\PY{n}{datensatz2}\PY{p}{)}\PY{p}{)} 
         \PY{n}{plt}\PY{o}{.}\PY{n}{title}\PY{p}{(}\PY{l+s+s1}{\PYZsq{}}\PY{l+s+s1}{Gemeinsame Verteilung der Datensätze }\PY{l+s+s1}{\PYZsq{}} \PY{o}{+} \PY{n+nb}{str}\PY{p}{(}\PY{n}{datensatz1}\PY{p}{)} \PY{o}{+} \PY{l+s+s1}{\PYZsq{}}\PY{l+s+s1}{ und }\PY{l+s+s1}{\PYZsq{}} \PY{o}{+} \PY{n+nb}{str}\PY{p}{(}\PY{n}{datensatz2}\PY{p}{)}\PY{p}{)}
         \PY{k}{def} \PY{n+nf}{easy\PYZus{}plot}\PY{p}{(}\PY{p}{)}\PY{p}{:}
             \PY{n}{plt}\PY{o}{.}\PY{n}{grid}\PY{p}{(}\PY{p}{)}
             \PY{n}{plt}\PY{o}{.}\PY{n}{axhline}\PY{p}{(}\PY{l+m+mi}{0}\PY{p}{,} \PY{n}{color}\PY{o}{=}\PY{l+s+s1}{\PYZsq{}}\PY{l+s+s1}{black}\PY{l+s+s1}{\PYZsq{}}\PY{p}{)}
             \PY{n}{plt}\PY{o}{.}\PY{n}{axvline}\PY{p}{(}\PY{l+m+mi}{0}\PY{p}{,} \PY{n}{color}\PY{o}{=}\PY{l+s+s1}{\PYZsq{}}\PY{l+s+s1}{black}\PY{l+s+s1}{\PYZsq{}}\PY{p}{)}
             \PY{n}{plt}\PY{o}{.}\PY{n}{show}\PY{p}{(}\PY{p}{)}
         \PY{n}{easy\PYZus{}plot}\PY{p}{(}\PY{p}{)} 
         
         \PY{c+c1}{\PYZsh{}\PYZhy{}\PYZhy{}\PYZhy{}\PYZhy{}\PYZhy{}\PYZhy{}\PYZhy{}\PYZhy{}\PYZhy{}\PYZhy{}\PYZhy{}\PYZhy{}\PYZhy{}\PYZhy{}\PYZhy{}\PYZhy{}\PYZhy{}\PYZhy{}\PYZhy{}\PYZhy{}\PYZhy{}\PYZhy{}\PYZhy{}\PYZhy{}\PYZhy{}\PYZhy{}\PYZhy{}\PYZhy{}\PYZhy{}\PYZhy{}\PYZhy{}\PYZhy{}\PYZhy{}\PYZhy{}\PYZhy{}\PYZhy{}\PYZhy{}\PYZhy{}\PYZhy{}\PYZhy{}\PYZhy{}\PYZhy{}\PYZhy{}\PYZhy{}\PYZhy{}\PYZhy{}\PYZhy{}\PYZhy{}\PYZhy{}\PYZhy{}\PYZhy{}\PYZhy{}\PYZhy{}\PYZhy{}\PYZhy{}\PYZhy{}}
         \PY{c+c1}{\PYZsh{} Korrelation ausgeben lassen}
         \PY{n+nb}{print}\PY{p}{(}\PY{l+s+s1}{\PYZsq{}}\PY{l+s+s1}{\PYZsh{}}\PY{l+s+s1}{\PYZsq{}} \PY{o}{+} \PY{n}{SCREEN\PYZus{}WIDTH} \PY{o}{*} \PY{l+s+s1}{\PYZsq{}}\PY{l+s+s1}{\PYZhy{}}\PY{l+s+s1}{\PYZsq{}} \PY{o}{+} \PY{l+s+s1}{\PYZsq{}}\PY{l+s+s1}{\PYZsh{}}\PY{l+s+s1}{\PYZsq{}}\PY{p}{)}
         \PY{n+nb}{print}\PY{p}{(}\PY{l+s+s1}{\PYZsq{}}\PY{l+s+s1}{|}\PY{l+s+s1}{\PYZsq{}} \PY{o}{+} \PY{n}{centered}\PY{p}{(}\PY{l+s+s1}{\PYZsq{}}\PY{l+s+s1}{[INFO] Der Korrelationskoeffizient beträgt: }\PY{l+s+si}{\PYZob{}\PYZcb{}}\PY{l+s+s1}{\PYZsq{}}\PY{o}{.}\PY{n}{format}\PY{p}{(}\PY{n}{corr}\PY{p}{[}\PY{l+m+mi}{0}\PY{p}{]}\PY{p}{[}\PY{l+m+mi}{1}\PY{p}{]}\PY{p}{)}\PY{p}{)} \PY{o}{+} \PY{l+s+s1}{\PYZsq{}}\PY{l+s+s1}{|}\PY{l+s+s1}{\PYZsq{}}\PY{p}{)}
         \PY{n+nb}{print}\PY{p}{(}\PY{l+s+s1}{\PYZsq{}}\PY{l+s+s1}{\PYZsh{}}\PY{l+s+s1}{\PYZsq{}} \PY{o}{+} \PY{n}{SCREEN\PYZus{}WIDTH} \PY{o}{*} \PY{l+s+s1}{\PYZsq{}}\PY{l+s+s1}{\PYZhy{}}\PY{l+s+s1}{\PYZsq{}} \PY{o}{+} \PY{l+s+s1}{\PYZsq{}}\PY{l+s+s1}{\PYZsh{}}\PY{l+s+s1}{\PYZsq{}}\PY{p}{)}
\end{Verbatim}

    \begin{center}
    \adjustimage{max size={0.9\linewidth}{0.9\paperheight}}{output_21_0.png}
    \end{center}
    { \hspace*{\fill} \\}
    
    \begin{Verbatim}[commandchars=\\\{\}]
\#----------------------------------------------------------------------------------------------------\#
|                   [INFO] Der Korrelationskoeffizient beträgt: 0.2529532401634806                   |
\#----------------------------------------------------------------------------------------------------\#

    \end{Verbatim}

    Die zugehörigen Erwartungswerte und Standardabweichungen können auch
ausgegeben werden. Anschließend wird das Modul \texttt{Numpy} genutzt,
welches den Erwartungswert und die Standardabweichung berechnet.

    \begin{Verbatim}[commandchars=\\\{\}]
{\color{incolor}In [{\color{incolor}12}]:} \PY{c+c1}{\PYZsh{}\PYZhy{}\PYZhy{}\PYZhy{}\PYZhy{}\PYZhy{}\PYZhy{}\PYZhy{}\PYZhy{}\PYZhy{}\PYZhy{}\PYZhy{}\PYZhy{}\PYZhy{}\PYZhy{}\PYZhy{}\PYZhy{}\PYZhy{}\PYZhy{}\PYZhy{}\PYZhy{}\PYZhy{}\PYZhy{}\PYZhy{}\PYZhy{}\PYZhy{}\PYZhy{}\PYZhy{}\PYZhy{}\PYZhy{}\PYZhy{}\PYZhy{}\PYZhy{}\PYZhy{}\PYZhy{}\PYZhy{}\PYZhy{}\PYZhy{}\PYZhy{}\PYZhy{}\PYZhy{}\PYZhy{}\PYZhy{}\PYZhy{}\PYZhy{}\PYZhy{}\PYZhy{}\PYZhy{}\PYZhy{}\PYZhy{}\PYZhy{}\PYZhy{}\PYZhy{}\PYZhy{}\PYZhy{}\PYZhy{}\PYZhy{}}
         \PY{c+c1}{\PYZsh{} Berechnung Mittelwerte}
         \PY{n}{mu\PYZus{}datensatz1} \PY{o}{=} \PY{n}{np}\PY{o}{.}\PY{n}{mean}\PY{p}{(}\PY{n}{values\PYZus{}datensatz1}\PY{p}{)}
         \PY{n}{mu\PYZus{}datensatz2} \PY{o}{=}  \PY{n}{np}\PY{o}{.}\PY{n}{mean}\PY{p}{(}\PY{n}{values\PYZus{}datensatz2}\PY{p}{)}
         
         \PY{c+c1}{\PYZsh{}\PYZhy{}\PYZhy{}\PYZhy{}\PYZhy{}\PYZhy{}\PYZhy{}\PYZhy{}\PYZhy{}\PYZhy{}\PYZhy{}\PYZhy{}\PYZhy{}\PYZhy{}\PYZhy{}\PYZhy{}\PYZhy{}\PYZhy{}\PYZhy{}\PYZhy{}\PYZhy{}\PYZhy{}\PYZhy{}\PYZhy{}\PYZhy{}\PYZhy{}\PYZhy{}\PYZhy{}\PYZhy{}\PYZhy{}\PYZhy{}\PYZhy{}\PYZhy{}\PYZhy{}\PYZhy{}\PYZhy{}\PYZhy{}\PYZhy{}\PYZhy{}\PYZhy{}\PYZhy{}\PYZhy{}\PYZhy{}\PYZhy{}\PYZhy{}\PYZhy{}\PYZhy{}\PYZhy{}\PYZhy{}\PYZhy{}\PYZhy{}\PYZhy{}\PYZhy{}\PYZhy{}\PYZhy{}\PYZhy{}\PYZhy{}}
         \PY{c+c1}{\PYZsh{} Ausgabe der Mittelwerte}
         \PY{n+nb}{print}\PY{p}{(}\PY{l+s+s1}{\PYZsq{}}\PY{l+s+s1}{\PYZsh{}}\PY{l+s+s1}{\PYZsq{}} \PY{o}{+} \PY{n}{SCREEN\PYZus{}WIDTH} \PY{o}{*} \PY{l+s+s1}{\PYZsq{}}\PY{l+s+s1}{\PYZhy{}}\PY{l+s+s1}{\PYZsq{}} \PY{o}{+} \PY{l+s+s1}{\PYZsq{}}\PY{l+s+s1}{\PYZsh{}}\PY{l+s+s1}{\PYZsq{}}\PY{p}{)}
         \PY{n+nb}{print}\PY{p}{(}\PY{l+s+s1}{\PYZsq{}}\PY{l+s+s1}{|}\PY{l+s+s1}{\PYZsq{}} \PY{o}{+} \PY{n}{centered}\PY{p}{(}\PY{l+s+s1}{\PYZsq{}}\PY{l+s+s1}{[INFO] Die erwartete Rendite des Datensatzes }\PY{l+s+s1}{\PYZsq{}} \PY{o}{+} \PY{n+nb}{str}\PY{p}{(}\PY{n}{datensatz1}\PY{p}{)} \PY{o}{+} \PY{l+s+s1}{\PYZsq{}}\PY{l+s+s1}{ beträgt }\PY{l+s+s1}{\PYZsq{}} \PY{o}{+} \PY{n+nb}{str}\PY{p}{(}\PY{n}{mu\PYZus{}datensatz1}\PY{p}{)} \PY{o}{+} \PY{l+s+s1}{\PYZsq{}}\PY{l+s+s1}{.}\PY{l+s+s1}{\PYZsq{}}\PY{p}{)} \PY{o}{+} \PY{l+s+s1}{\PYZsq{}}\PY{l+s+s1}{| }\PY{l+s+s1}{\PYZsq{}}\PY{p}{)}
         \PY{n+nb}{print}\PY{p}{(}\PY{l+s+s1}{\PYZsq{}}\PY{l+s+s1}{|}\PY{l+s+s1}{\PYZsq{}} \PY{o}{+} \PY{n}{centered}\PY{p}{(}\PY{l+s+s1}{\PYZsq{}}\PY{l+s+s1}{[INFO] Die erwartete Rendite des Datensatzes }\PY{l+s+s1}{\PYZsq{}} \PY{o}{+} \PY{n+nb}{str}\PY{p}{(}\PY{n}{datensatz2}\PY{p}{)} \PY{o}{+} \PY{l+s+s1}{\PYZsq{}}\PY{l+s+s1}{ beträgt }\PY{l+s+s1}{\PYZsq{}} \PY{o}{+} \PY{n+nb}{str}\PY{p}{(}\PY{n}{mu\PYZus{}datensatz2}\PY{p}{)} \PY{o}{+} \PY{l+s+s1}{\PYZsq{}}\PY{l+s+s1}{.}\PY{l+s+s1}{\PYZsq{}}\PY{p}{)} \PY{o}{+} \PY{l+s+s1}{\PYZsq{}}\PY{l+s+s1}{| }\PY{l+s+s1}{\PYZsq{}}\PY{p}{)}
         \PY{n+nb}{print}\PY{p}{(}\PY{l+s+s1}{\PYZsq{}}\PY{l+s+s1}{\PYZsh{}}\PY{l+s+s1}{\PYZsq{}} \PY{o}{+} \PY{n}{SCREEN\PYZus{}WIDTH} \PY{o}{*} \PY{l+s+s1}{\PYZsq{}}\PY{l+s+s1}{\PYZhy{}}\PY{l+s+s1}{\PYZsq{}} \PY{o}{+} \PY{l+s+s1}{\PYZsq{}}\PY{l+s+s1}{\PYZsh{}}\PY{l+s+s1}{\PYZsq{}}\PY{p}{)}
         
         \PY{c+c1}{\PYZsh{}\PYZhy{}\PYZhy{}\PYZhy{}\PYZhy{}\PYZhy{}\PYZhy{}\PYZhy{}\PYZhy{}\PYZhy{}\PYZhy{}\PYZhy{}\PYZhy{}\PYZhy{}\PYZhy{}\PYZhy{}\PYZhy{}\PYZhy{}\PYZhy{}\PYZhy{}\PYZhy{}\PYZhy{}\PYZhy{}\PYZhy{}\PYZhy{}\PYZhy{}\PYZhy{}\PYZhy{}\PYZhy{}\PYZhy{}\PYZhy{}\PYZhy{}\PYZhy{}\PYZhy{}\PYZhy{}\PYZhy{}\PYZhy{}\PYZhy{}\PYZhy{}\PYZhy{}\PYZhy{}\PYZhy{}\PYZhy{}\PYZhy{}\PYZhy{}\PYZhy{}\PYZhy{}\PYZhy{}\PYZhy{}\PYZhy{}\PYZhy{}\PYZhy{}\PYZhy{}\PYZhy{}\PYZhy{}\PYZhy{}\PYZhy{}}
         \PY{c+c1}{\PYZsh{} Berechnung Standardabweichung}
         \PY{n}{std\PYZus{}dattensatz1} \PY{o}{=} \PY{n}{np}\PY{o}{.}\PY{n}{std}\PY{p}{(}\PY{n}{values\PYZus{}datensatz1}\PY{p}{)}
         \PY{n}{std\PYZus{}dattensatz2} \PY{o}{=} \PY{n}{np}\PY{o}{.}\PY{n}{std}\PY{p}{(}\PY{n}{values\PYZus{}datensatz2}\PY{p}{)}
         
         \PY{c+c1}{\PYZsh{}\PYZhy{}\PYZhy{}\PYZhy{}\PYZhy{}\PYZhy{}\PYZhy{}\PYZhy{}\PYZhy{}\PYZhy{}\PYZhy{}\PYZhy{}\PYZhy{}\PYZhy{}\PYZhy{}\PYZhy{}\PYZhy{}\PYZhy{}\PYZhy{}\PYZhy{}\PYZhy{}\PYZhy{}\PYZhy{}\PYZhy{}\PYZhy{}\PYZhy{}\PYZhy{}\PYZhy{}\PYZhy{}\PYZhy{}\PYZhy{}\PYZhy{}\PYZhy{}\PYZhy{}\PYZhy{}\PYZhy{}\PYZhy{}\PYZhy{}\PYZhy{}\PYZhy{}\PYZhy{}\PYZhy{}\PYZhy{}\PYZhy{}\PYZhy{}\PYZhy{}\PYZhy{}\PYZhy{}\PYZhy{}\PYZhy{}\PYZhy{}\PYZhy{}\PYZhy{}\PYZhy{}\PYZhy{}\PYZhy{}\PYZhy{}}
         \PY{c+c1}{\PYZsh{} Ausgabe Standardabweichung}
         \PY{n+nb}{print}\PY{p}{(}\PY{l+s+s1}{\PYZsq{}}\PY{l+s+s1}{|}\PY{l+s+s1}{\PYZsq{}} \PY{o}{+} \PY{n}{centered}\PY{p}{(}\PY{l+s+s1}{\PYZsq{}}\PY{l+s+s1}{[INFO] Der Datensatz }\PY{l+s+s1}{\PYZsq{}} \PY{o}{+} \PY{n+nb}{str}\PY{p}{(}\PY{n}{datensatz1}\PY{p}{)} \PY{o}{+} \PY{l+s+s1}{\PYZsq{}}\PY{l+s+s1}{ hat eine Standardabweichung i.H.v. }\PY{l+s+s1}{\PYZsq{}} \PY{o}{+} \PY{n+nb}{str}\PY{p}{(}\PY{n}{std\PYZus{}dattensatz1}\PY{p}{)} \PY{o}{+} \PY{l+s+s1}{\PYZsq{}}\PY{l+s+s1}{.}\PY{l+s+s1}{\PYZsq{}}\PY{p}{)} \PY{o}{+} \PY{l+s+s1}{\PYZsq{}}\PY{l+s+s1}{| }\PY{l+s+s1}{\PYZsq{}}\PY{p}{)}
         \PY{n+nb}{print}\PY{p}{(}\PY{l+s+s1}{\PYZsq{}}\PY{l+s+s1}{|}\PY{l+s+s1}{\PYZsq{}} \PY{o}{+} \PY{n}{centered}\PY{p}{(}\PY{l+s+s1}{\PYZsq{}}\PY{l+s+s1}{[INFO] Der Datensatz }\PY{l+s+s1}{\PYZsq{}} \PY{o}{+} \PY{n+nb}{str}\PY{p}{(}\PY{n}{datensatz2}\PY{p}{)} \PY{o}{+} \PY{l+s+s1}{\PYZsq{}}\PY{l+s+s1}{ hat eine Standardabweichung i.H.v. }\PY{l+s+s1}{\PYZsq{}} \PY{o}{+} \PY{n+nb}{str}\PY{p}{(}\PY{n}{std\PYZus{}dattensatz2}\PY{p}{)} \PY{o}{+} \PY{l+s+s1}{\PYZsq{}}\PY{l+s+s1}{.}\PY{l+s+s1}{\PYZsq{}}\PY{p}{)} \PY{o}{+} \PY{l+s+s1}{\PYZsq{}}\PY{l+s+s1}{| }\PY{l+s+s1}{\PYZsq{}}\PY{p}{)}
         \PY{n+nb}{print}\PY{p}{(}\PY{l+s+s1}{\PYZsq{}}\PY{l+s+s1}{\PYZsh{}}\PY{l+s+s1}{\PYZsq{}} \PY{o}{+} \PY{n}{SCREEN\PYZus{}WIDTH} \PY{o}{*} \PY{l+s+s1}{\PYZsq{}}\PY{l+s+s1}{\PYZhy{}}\PY{l+s+s1}{\PYZsq{}} \PY{o}{+} \PY{l+s+s1}{\PYZsq{}}\PY{l+s+s1}{\PYZsh{}}\PY{l+s+s1}{\PYZsq{}}\PY{p}{)}
\end{Verbatim}

    \begin{Verbatim}[commandchars=\\\{\}]
\#----------------------------------------------------------------------------------------------------\#
|        [INFO] Die erwartete Rendite des Datensatzes VOW3.DE beträgt 0.0005292171712249241.         | 
|        [INFO] Die erwartete Rendite des Datensatzes FME.DE beträgt -0.00033376251501250566.        | 
\#----------------------------------------------------------------------------------------------------\#
|       [INFO] Der Datensatz VOW3.DE hat eine Standardabweichung i.H.v. 0.016551642943328664.        | 
|        [INFO] Der Datensatz FME.DE hat eine Standardabweichung i.H.v. 0.020156241318251833.        | 
\#----------------------------------------------------------------------------------------------------\#

    \end{Verbatim}

    \hypertarget{portfolio-renditedaten-bereinigen-und-analysieren}{%
\subsection{Portfolio-Renditedaten bereinigen und
analysieren:}\label{portfolio-renditedaten-bereinigen-und-analysieren}}

Da für die Simulationsverfahren fortan die Portfolio-Rendite benötigt
wird, wird diese ebenfalls in einer Liste abgespeichert und um NaN-Werte
bereinigt. Anschließend wird wiederum der Erwartungswert und die
Standardabweichung berechnet.

    \begin{Verbatim}[commandchars=\\\{\}]
{\color{incolor}In [{\color{incolor}13}]:} \PY{c+c1}{\PYZsh{}\PYZhy{}\PYZhy{}\PYZhy{}\PYZhy{}\PYZhy{}\PYZhy{}\PYZhy{}\PYZhy{}\PYZhy{}\PYZhy{}\PYZhy{}\PYZhy{}\PYZhy{}\PYZhy{}\PYZhy{}\PYZhy{}\PYZhy{}\PYZhy{}\PYZhy{}\PYZhy{}\PYZhy{}\PYZhy{}\PYZhy{}\PYZhy{}\PYZhy{}\PYZhy{}\PYZhy{}\PYZhy{}\PYZhy{}\PYZhy{}\PYZhy{}\PYZhy{}\PYZhy{}\PYZhy{}\PYZhy{}\PYZhy{}\PYZhy{}\PYZhy{}\PYZhy{}\PYZhy{}\PYZhy{}\PYZhy{}\PYZhy{}\PYZhy{}\PYZhy{}\PYZhy{}\PYZhy{}\PYZhy{}\PYZhy{}\PYZhy{}\PYZhy{}\PYZhy{}\PYZhy{}\PYZhy{}\PYZhy{}\PYZhy{}}
         \PY{c+c1}{\PYZsh{} DataFrame\PYZhy{}Werte in Liste überführen und Bereinigen}
         \PY{n}{values\PYZus{}PF} \PY{o}{=} \PY{n}{kurschart}\PY{p}{[}\PY{l+s+s1}{\PYZsq{}}\PY{l+s+s1}{Rendite\PYZhy{}PF}\PY{l+s+s1}{\PYZsq{}}\PY{p}{]}\PY{o}{.}\PY{n}{values}\PY{o}{.}\PY{n}{tolist}\PY{p}{(}\PY{p}{)}
         \PY{n}{values\PYZus{}PF} \PY{o}{=} \PY{n}{np}\PY{o}{.}\PY{n}{array}\PY{p}{(}\PY{n}{values\PYZus{}PF}\PY{p}{)}
         \PY{n}{values\PYZus{}PF} \PY{o}{=} \PY{n}{values\PYZus{}PF}\PY{p}{[}\PY{n}{np}\PY{o}{.}\PY{n}{logical\PYZus{}not}\PY{p}{(}\PY{n}{np}\PY{o}{.}\PY{n}{isnan}\PY{p}{(}\PY{n}{values\PYZus{}PF}\PY{p}{)}\PY{p}{)}\PY{p}{]}
         \PY{n}{values\PYZus{}PF} \PY{o}{=} \PY{n}{values\PYZus{}PF}\PY{p}{[}\PY{n}{values\PYZus{}PF} \PY{o}{!=} \PY{l+m+mf}{0.0}\PY{p}{]}
         
         \PY{c+c1}{\PYZsh{}\PYZhy{}\PYZhy{}\PYZhy{}\PYZhy{}\PYZhy{}\PYZhy{}\PYZhy{}\PYZhy{}\PYZhy{}\PYZhy{}\PYZhy{}\PYZhy{}\PYZhy{}\PYZhy{}\PYZhy{}\PYZhy{}\PYZhy{}\PYZhy{}\PYZhy{}\PYZhy{}\PYZhy{}\PYZhy{}\PYZhy{}\PYZhy{}\PYZhy{}\PYZhy{}\PYZhy{}\PYZhy{}\PYZhy{}\PYZhy{}\PYZhy{}\PYZhy{}\PYZhy{}\PYZhy{}\PYZhy{}\PYZhy{}\PYZhy{}\PYZhy{}\PYZhy{}\PYZhy{}\PYZhy{}\PYZhy{}\PYZhy{}\PYZhy{}\PYZhy{}\PYZhy{}\PYZhy{}\PYZhy{}\PYZhy{}\PYZhy{}\PYZhy{}\PYZhy{}\PYZhy{}\PYZhy{}\PYZhy{}\PYZhy{}}
         \PY{c+c1}{\PYZsh{} Berechnung Mittelwert und Standardabweichung}
         \PY{n}{mu\PYZus{}PF} \PY{o}{=} \PY{n}{np}\PY{o}{.}\PY{n}{mean}\PY{p}{(}\PY{n}{values\PYZus{}PF}\PY{p}{)}
         \PY{n}{std\PYZus{}PF} \PY{o}{=} \PY{n}{np}\PY{o}{.}\PY{n}{std}\PY{p}{(}\PY{n}{values\PYZus{}PF}\PY{p}{)}
         
         \PY{c+c1}{\PYZsh{}\PYZhy{}\PYZhy{}\PYZhy{}\PYZhy{}\PYZhy{}\PYZhy{}\PYZhy{}\PYZhy{}\PYZhy{}\PYZhy{}\PYZhy{}\PYZhy{}\PYZhy{}\PYZhy{}\PYZhy{}\PYZhy{}\PYZhy{}\PYZhy{}\PYZhy{}\PYZhy{}\PYZhy{}\PYZhy{}\PYZhy{}\PYZhy{}\PYZhy{}\PYZhy{}\PYZhy{}\PYZhy{}\PYZhy{}\PYZhy{}\PYZhy{}\PYZhy{}\PYZhy{}\PYZhy{}\PYZhy{}\PYZhy{}\PYZhy{}\PYZhy{}\PYZhy{}\PYZhy{}\PYZhy{}\PYZhy{}\PYZhy{}\PYZhy{}\PYZhy{}\PYZhy{}\PYZhy{}\PYZhy{}\PYZhy{}\PYZhy{}\PYZhy{}\PYZhy{}\PYZhy{}\PYZhy{}\PYZhy{}\PYZhy{}}
         \PY{c+c1}{\PYZsh{} Ausgabe der Informationen}
         \PY{n+nb}{print}\PY{p}{(}\PY{l+s+s1}{\PYZsq{}}\PY{l+s+s1}{\PYZsh{}}\PY{l+s+s1}{\PYZsq{}} \PY{o}{+} \PY{n}{SCREEN\PYZus{}WIDTH} \PY{o}{*} \PY{l+s+s1}{\PYZsq{}}\PY{l+s+s1}{\PYZhy{}}\PY{l+s+s1}{\PYZsq{}} \PY{o}{+} \PY{l+s+s1}{\PYZsq{}}\PY{l+s+s1}{\PYZsh{}}\PY{l+s+s1}{\PYZsq{}}\PY{p}{)}
         \PY{n+nb}{print}\PY{p}{(}\PY{l+s+s1}{\PYZsq{}}\PY{l+s+s1}{|}\PY{l+s+s1}{\PYZsq{}} \PY{o}{+} \PY{n}{centered}\PY{p}{(}\PY{l+s+s1}{\PYZsq{}}\PY{l+s+s1}{[INFO] Die Porfolio\PYZhy{}Rendite hat einen Erwartunswert i.H.v. }\PY{l+s+s1}{\PYZsq{}} \PY{o}{+} \PY{n+nb}{str}\PY{p}{(}\PY{n}{mu\PYZus{}PF}\PY{p}{)} \PY{o}{+} \PY{l+s+s1}{\PYZsq{}}\PY{l+s+s1}{.}\PY{l+s+s1}{\PYZsq{}}\PY{p}{)} \PY{o}{+} \PY{l+s+s1}{\PYZsq{}}\PY{l+s+s1}{| }\PY{l+s+s1}{\PYZsq{}}\PY{p}{)}
         \PY{n+nb}{print}\PY{p}{(}\PY{l+s+s1}{\PYZsq{}}\PY{l+s+s1}{\PYZsh{}}\PY{l+s+s1}{\PYZsq{}} \PY{o}{+} \PY{n}{SCREEN\PYZus{}WIDTH} \PY{o}{*} \PY{l+s+s1}{\PYZsq{}}\PY{l+s+s1}{\PYZhy{}}\PY{l+s+s1}{\PYZsq{}} \PY{o}{+} \PY{l+s+s1}{\PYZsq{}}\PY{l+s+s1}{\PYZsh{}}\PY{l+s+s1}{\PYZsq{}}\PY{p}{)}
         \PY{n+nb}{print}\PY{p}{(}\PY{l+s+s1}{\PYZsq{}}\PY{l+s+s1}{|}\PY{l+s+s1}{\PYZsq{}} \PY{o}{+} \PY{n}{centered}\PY{p}{(}\PY{l+s+s1}{\PYZsq{}}\PY{l+s+s1}{[INFO] Das Porfolio hat eine Standardabweichung i.H.v. }\PY{l+s+s1}{\PYZsq{}} \PY{o}{+} \PY{n+nb}{str}\PY{p}{(}\PY{n}{std\PYZus{}PF}\PY{p}{)} \PY{o}{+} \PY{l+s+s1}{\PYZsq{}}\PY{l+s+s1}{.}\PY{l+s+s1}{\PYZsq{}}\PY{p}{)} \PY{o}{+} \PY{l+s+s1}{\PYZsq{}}\PY{l+s+s1}{| }\PY{l+s+s1}{\PYZsq{}}\PY{p}{)}
         \PY{n+nb}{print}\PY{p}{(}\PY{l+s+s1}{\PYZsq{}}\PY{l+s+s1}{\PYZsh{}}\PY{l+s+s1}{\PYZsq{}} \PY{o}{+} \PY{n}{SCREEN\PYZus{}WIDTH} \PY{o}{*} \PY{l+s+s1}{\PYZsq{}}\PY{l+s+s1}{\PYZhy{}}\PY{l+s+s1}{\PYZsq{}} \PY{o}{+} \PY{l+s+s1}{\PYZsq{}}\PY{l+s+s1}{\PYZsh{}}\PY{l+s+s1}{\PYZsq{}}\PY{p}{)}
\end{Verbatim}

    \begin{Verbatim}[commandchars=\\\{\}]
\#----------------------------------------------------------------------------------------------------\#
|         [INFO] Die Porfolio-Rendite hat einen Erwartunswert i.H.v. 0.0001072721245152569.          | 
\#----------------------------------------------------------------------------------------------------\#
|            [INFO] Das Porfolio hat eine Standardabweichung i.H.v. 0.014593523569872993.            | 
\#----------------------------------------------------------------------------------------------------\#

    \end{Verbatim}

    \hypertarget{warum-ist-der-mittelwert-der-zu-erwartenden-renditen-standardabweichungen-nicht-gleich-dem-erwartungswert-der-pf-rendite-standardabweichung}{%
\subsubsection{Warum ist der Mittelwert der zu erwartenden Renditen /
Standardabweichungen nicht gleich dem Erwartungswert der PF-Rendite /
Standardabweichung?}\label{warum-ist-der-mittelwert-der-zu-erwartenden-renditen-standardabweichungen-nicht-gleich-dem-erwartungswert-der-pf-rendite-standardabweichung}}

Grund: Rechnung erfolgt mit Float-Werten, welche eine Annäherung an
reelle Zahlen darstellen. Diese sind nicht exakt.

    \begin{Verbatim}[commandchars=\\\{\}]
{\color{incolor}In [{\color{incolor}14}]:} \PY{p}{(}\PY{n}{std\PYZus{}dattensatz1} \PY{o}{+} \PY{n}{std\PYZus{}dattensatz2}\PY{p}{)} \PY{o}{/} \PY{l+m+mi}{2} \PY{o}{==} \PY{n}{std\PYZus{}PF}
\end{Verbatim}

\begin{Verbatim}[commandchars=\\\{\}]
{\color{outcolor}Out[{\color{outcolor}14}]:} False
\end{Verbatim}
            
    \begin{Verbatim}[commandchars=\\\{\}]
{\color{incolor}In [{\color{incolor}15}]:} \PY{p}{(}\PY{n}{mu\PYZus{}datensatz1} \PY{o}{+} \PY{n}{mu\PYZus{}datensatz2}\PY{p}{)} \PY{o}{/} \PY{l+m+mi}{2} \PY{o}{==} \PY{n}{mu\PYZus{}PF}
\end{Verbatim}

\begin{Verbatim}[commandchars=\\\{\}]
{\color{outcolor}Out[{\color{outcolor}15}]:} False
\end{Verbatim}
            
    \begin{Verbatim}[commandchars=\\\{\}]
{\color{incolor}In [{\color{incolor}16}]:} \PY{l+m+mf}{0.1} \PY{o}{+} \PY{l+m+mf}{0.1} \PY{o}{+} \PY{l+m+mf}{0.1} \PY{o}{==} \PY{l+m+mf}{0.3}
\end{Verbatim}

\begin{Verbatim}[commandchars=\\\{\}]
{\color{outcolor}Out[{\color{outcolor}16}]:} False
\end{Verbatim}
            
    Um das Intervall beider Simulationsverfahren zu begrenzen, wird der
kleinste und größte Rendite-Wert der in der Liste ``values\_PF'' zu
finden ist ermittelt.

    \begin{Verbatim}[commandchars=\\\{\}]
{\color{incolor}In [{\color{incolor}17}]:} \PY{n}{mini\PYZus{}values\PYZus{}PF} \PY{o}{=} \PY{n+nb}{min}\PY{p}{(}\PY{n}{values\PYZus{}PF}\PY{p}{)}
         \PY{n}{maxi\PYZus{}values\PYZus{}PF} \PY{o}{=} \PY{n+nb}{max}\PY{p}{(}\PY{n}{values\PYZus{}PF}\PY{p}{)}
\end{Verbatim}

    \hypertarget{abfragefunktion-definieren}{%
\subsection{Abfragefunktion
definieren:}\label{abfragefunktion-definieren}}

Nun wird eine Funktion definiert mit deren Hilfe die Feinheit der
jeweiligen Verteilungsfunktion bestimmt werden kann. Standardmäßig wird
die höchste Feinheit gewählt, d.h. jede einzelne Realisation wird
einzeln erfasst, sodass die Verteilungsfunktion eine sehr hohe
Genauigkeit aufweist. Dies muss jedoch nicht immer sinnvoll sein,
weshalb mithilfe einer Abfrage auch andere Werte akzeptiert werden
sollen. Im Grunde die Funktion ob eine Anpassung vorgenommen werden soll
oder nicht und wenn ja, welche.

    \begin{Verbatim}[commandchars=\\\{\}]
{\color{incolor}In [{\color{incolor}18}]:} \PY{k}{def} \PY{n+nf}{abfrage}\PY{p}{(}\PY{p}{)}\PY{p}{:}
             \PY{n}{abfrage} \PY{o}{=} \PY{k+kc}{None}
             \PY{c+c1}{\PYZsh{} Solange die Nutzereingabe nicht in folgenden Liste, frage erneut..}
             \PY{k}{while} \PY{n}{abfrage} \PY{o+ow}{not} \PY{o+ow}{in} \PY{p}{(}\PY{l+s+s1}{\PYZsq{}}\PY{l+s+s1}{Ja}\PY{l+s+s1}{\PYZsq{}}\PY{p}{,} \PY{l+s+s1}{\PYZsq{}}\PY{l+s+s1}{Nein}\PY{l+s+s1}{\PYZsq{}}\PY{p}{,} \PY{l+s+s1}{\PYZsq{}}\PY{l+s+s1}{ja}\PY{l+s+s1}{\PYZsq{}}\PY{p}{,} \PY{l+s+s1}{\PYZsq{}}\PY{l+s+s1}{nein}\PY{l+s+s1}{\PYZsq{}}\PY{p}{,} \PY{l+s+s1}{\PYZsq{}}\PY{l+s+s1}{j}\PY{l+s+s1}{\PYZsq{}}\PY{p}{,} \PY{l+s+s1}{\PYZsq{}}\PY{l+s+s1}{n}\PY{l+s+s1}{\PYZsq{}}\PY{p}{)}\PY{p}{:}
                 \PY{n}{abfrage} \PY{o}{=} \PY{n+nb}{input}\PY{p}{(}\PY{l+s+s1}{\PYZsq{}}\PY{l+s+s1}{|}\PY{l+s+s1}{\PYZsq{}} \PY{o}{+} \PY{n}{centered}\PY{p}{(}\PY{l+s+s1}{\PYZsq{}}\PY{l+s+s1}{[EINGABE] Möchten Sie die Genauigkeit anpassen? Geben Sie }\PY{l+s+s1}{\PYZdq{}}\PY{l+s+s1}{Ja}\PY{l+s+s1}{\PYZdq{}}\PY{l+s+s1}{ oder }\PY{l+s+s1}{\PYZdq{}}\PY{l+s+s1}{Nein}\PY{l+s+s1}{\PYZdq{}}\PY{l+s+s1}{ ein: }\PY{l+s+s1}{\PYZsq{}}\PY{p}{)} \PY{o}{+} \PY{l+s+s1}{\PYZsq{}}\PY{l+s+s1}{| }\PY{l+s+s1}{\PYZsq{}}\PY{p}{)}        
                 \PY{c+c1}{\PYZsh{} Wenn Anwort Nein, dann fahre mit Standardwert (Höchste Genauigkeit) fort}
                 \PY{k}{if} \PY{n}{abfrage} \PY{o}{==} \PY{l+s+s1}{\PYZsq{}}\PY{l+s+s1}{Nein}\PY{l+s+s1}{\PYZsq{}} \PY{o+ow}{or} \PY{n}{abfrage} \PY{o}{==} \PY{l+s+s1}{\PYZsq{}}\PY{l+s+s1}{nein}\PY{l+s+s1}{\PYZsq{}} \PY{o+ow}{or} \PY{n}{abfrage} \PY{o}{==} \PY{l+s+s1}{\PYZsq{}}\PY{l+s+s1}{n}\PY{l+s+s1}{\PYZsq{}}\PY{p}{:}
                     \PY{k}{return} \PY{n+nb}{len}\PY{p}{(}\PY{n}{values\PYZus{}PF}\PY{p}{)}
                 \PY{c+c1}{\PYZsh{} Wenn Antwort Ja, dann warte auf Input und fahre mit diesem fort}
                 \PY{k}{elif} \PY{n}{abfrage} \PY{o}{==} \PY{l+s+s1}{\PYZsq{}}\PY{l+s+s1}{Ja}\PY{l+s+s1}{\PYZsq{}} \PY{o+ow}{or} \PY{n}{abfrage} \PY{o}{==} \PY{l+s+s1}{\PYZsq{}}\PY{l+s+s1}{ja}\PY{l+s+s1}{\PYZsq{}} \PY{o+ow}{or} \PY{n}{abfrage} \PY{o}{==} \PY{l+s+s1}{\PYZsq{}}\PY{l+s+s1}{j}\PY{l+s+s1}{\PYZsq{}}\PY{p}{:}
                     \PY{k}{return} \PY{n+nb}{int}\PY{p}{(}\PY{n+nb}{input}\PY{p}{(}\PY{l+s+s1}{\PYZsq{}}\PY{l+s+s1}{|}\PY{l+s+s1}{\PYZsq{}} \PY{o}{+} \PY{n}{centered}\PY{p}{(}\PY{l+s+s1}{\PYZsq{}}\PY{l+s+s1}{[EINGABE] Geben sie eine Zahl zwischen }\PY{l+s+s1}{\PYZsq{}} \PY{o}{+} \PY{n+nb}{str}\PY{p}{(}\PY{n+nb}{round}\PY{p}{(}\PY{n+nb}{len}\PY{p}{(}\PY{n}{values\PYZus{}PF}\PY{p}{)}\PY{o}{/}\PY{l+m+mi}{10}\PY{p}{)}\PY{p}{)} \PY{o}{+} \PY{l+s+s1}{\PYZsq{}}\PY{l+s+s1}{ und }\PY{l+s+s1}{\PYZsq{}} \PY{o}{+} \PY{n+nb}{str}\PY{p}{(}\PY{n+nb}{len}\PY{p}{(}\PY{n}{values\PYZus{}PF}\PY{p}{)}\PY{p}{)} \PY{o}{+} \PY{l+s+s1}{\PYZsq{}}\PY{l+s+s1}{ ein}\PY{l+s+s1}{\PYZsq{}}\PY{p}{)} \PY{o}{+} \PY{l+s+s1}{\PYZsq{}}\PY{l+s+s1}{| }\PY{l+s+s1}{\PYZsq{}}\PY{p}{)}\PY{p}{)}    
                 \PY{c+c1}{\PYZsh{} Sonst gib eine Fehlermeldung aus}
                 \PY{k}{else}\PY{p}{:}
                     \PY{n+nb}{print}\PY{p}{(}\PY{l+s+s1}{\PYZsq{}}\PY{l+s+s1}{|}\PY{l+s+s1}{\PYZsq{}} \PY{o}{+} \PY{n}{centered}\PY{p}{(}\PY{l+s+s1}{\PYZsq{}}\PY{l+s+s1}{[WARNUNG] Geben Sie }\PY{l+s+s1}{\PYZdq{}}\PY{l+s+s1}{Ja}\PY{l+s+s1}{\PYZdq{}}\PY{l+s+s1}{ oder }\PY{l+s+s1}{\PYZdq{}}\PY{l+s+s1}{Nein}\PY{l+s+s1}{\PYZdq{}}\PY{l+s+s1}{ ein! }\PY{l+s+s1}{\PYZsq{}}\PY{p}{)} \PY{o}{+} \PY{l+s+s1}{\PYZsq{}}\PY{l+s+s1}{| }\PY{l+s+s1}{\PYZsq{}}\PY{p}{)}
\end{Verbatim}

    \hypertarget{verteilungsfunktionen-definieren-und-plotten}{%
\subsection{Verteilungsfunktionen definieren und
plotten:}\label{verteilungsfunktionen-definieren-und-plotten}}

Im letzten Schritt gilt es nun alle Informationen zusammenzutragen und
die beiden Verteilungsfunktionen zu zeichnen.

    \begin{Verbatim}[commandchars=\\\{\}]
{\color{incolor}In [{\color{incolor}19}]:} \PY{c+c1}{\PYZsh{}\PYZhy{}\PYZhy{}\PYZhy{}\PYZhy{}\PYZhy{}\PYZhy{}\PYZhy{}\PYZhy{}\PYZhy{}\PYZhy{}\PYZhy{}\PYZhy{}\PYZhy{}\PYZhy{}\PYZhy{}\PYZhy{}\PYZhy{}\PYZhy{}\PYZhy{}\PYZhy{}\PYZhy{}\PYZhy{}\PYZhy{}\PYZhy{}\PYZhy{}\PYZhy{}\PYZhy{}\PYZhy{}\PYZhy{}\PYZhy{}\PYZhy{}\PYZhy{}\PYZhy{}\PYZhy{}\PYZhy{}\PYZhy{}\PYZhy{}\PYZhy{}\PYZhy{}\PYZhy{}\PYZhy{}\PYZhy{}\PYZhy{}\PYZhy{}\PYZhy{}\PYZhy{}\PYZhy{}\PYZhy{}\PYZhy{}\PYZhy{}\PYZhy{}\PYZhy{}\PYZhy{}\PYZhy{}\PYZhy{}\PYZhy{}}
         \PY{c+c1}{\PYZsh{} Aufruf der Abfrage}
         \PY{n+nb}{print}\PY{p}{(}\PY{l+s+s1}{\PYZsq{}}\PY{l+s+s1}{\PYZsh{}}\PY{l+s+s1}{\PYZsq{}} \PY{o}{+} \PY{n}{SCREEN\PYZus{}WIDTH} \PY{o}{*} \PY{l+s+s1}{\PYZsq{}}\PY{l+s+s1}{\PYZhy{}}\PY{l+s+s1}{\PYZsq{}} \PY{o}{+} \PY{l+s+s1}{\PYZsq{}}\PY{l+s+s1}{\PYZsh{}}\PY{l+s+s1}{\PYZsq{}}\PY{p}{)}
         \PY{n}{bins} \PY{o}{=} \PY{n}{abfrage}\PY{p}{(}\PY{p}{)}
         \PY{n+nb}{print}\PY{p}{(}\PY{l+s+s1}{\PYZsq{}}\PY{l+s+s1}{\PYZsh{}}\PY{l+s+s1}{\PYZsq{}} \PY{o}{+} \PY{n}{SCREEN\PYZus{}WIDTH} \PY{o}{*} \PY{l+s+s1}{\PYZsq{}}\PY{l+s+s1}{\PYZhy{}}\PY{l+s+s1}{\PYZsq{}} \PY{o}{+} \PY{l+s+s1}{\PYZsq{}}\PY{l+s+s1}{\PYZsh{}}\PY{l+s+s1}{\PYZsq{}}\PY{p}{)}
         
         \PY{c+c1}{\PYZsh{}\PYZhy{}\PYZhy{}\PYZhy{}\PYZhy{}\PYZhy{}\PYZhy{}\PYZhy{}\PYZhy{}\PYZhy{}\PYZhy{}\PYZhy{}\PYZhy{}\PYZhy{}\PYZhy{}\PYZhy{}\PYZhy{}\PYZhy{}\PYZhy{}\PYZhy{}\PYZhy{}\PYZhy{}\PYZhy{}\PYZhy{}\PYZhy{}\PYZhy{}\PYZhy{}\PYZhy{}\PYZhy{}\PYZhy{}\PYZhy{}\PYZhy{}\PYZhy{}\PYZhy{}\PYZhy{}\PYZhy{}\PYZhy{}\PYZhy{}\PYZhy{}\PYZhy{}\PYZhy{}\PYZhy{}\PYZhy{}\PYZhy{}\PYZhy{}\PYZhy{}\PYZhy{}\PYZhy{}\PYZhy{}\PYZhy{}\PYZhy{}\PYZhy{}\PYZhy{}\PYZhy{}\PYZhy{}\PYZhy{}\PYZhy{}\PYZhy{}\PYZhy{}}
         \PY{c+c1}{\PYZsh{} Plot für die Historische Simulation}
         \PY{n}{H}\PY{p}{,} \PY{n}{X1} \PY{o}{=} \PY{n}{np}\PY{o}{.}\PY{n}{histogram}\PY{p}{(}\PY{n}{values\PYZus{}PF}\PY{p}{,} \PY{n}{bins}\PY{p}{,} \PY{n}{density}\PY{o}{=}\PY{k+kc}{True}\PY{p}{)}
         \PY{n}{dx} \PY{o}{=} \PY{n}{X1}\PY{p}{[}\PY{l+m+mi}{1}\PY{p}{]} \PY{o}{\PYZhy{}} \PY{n}{X1}\PY{p}{[}\PY{l+m+mi}{0}\PY{p}{]}
         \PY{n}{F1} \PY{o}{=} \PY{n}{np}\PY{o}{.}\PY{n}{cumsum}\PY{p}{(}\PY{n}{H}\PY{p}{)} \PY{o}{*} \PY{n}{dx}
         \PY{n}{plt}\PY{o}{.}\PY{n}{plot}\PY{p}{(}\PY{n}{X1}\PY{p}{[}\PY{l+m+mi}{1}\PY{p}{:}\PY{p}{]}\PY{p}{,} \PY{n}{F1}\PY{p}{)}
         
         \PY{c+c1}{\PYZsh{}\PYZhy{}\PYZhy{}\PYZhy{}\PYZhy{}\PYZhy{}\PYZhy{}\PYZhy{}\PYZhy{}\PYZhy{}\PYZhy{}\PYZhy{}\PYZhy{}\PYZhy{}\PYZhy{}\PYZhy{}\PYZhy{}\PYZhy{}\PYZhy{}\PYZhy{}\PYZhy{}\PYZhy{}\PYZhy{}\PYZhy{}\PYZhy{}\PYZhy{}\PYZhy{}\PYZhy{}\PYZhy{}\PYZhy{}\PYZhy{}\PYZhy{}\PYZhy{}\PYZhy{}\PYZhy{}\PYZhy{}\PYZhy{}\PYZhy{}\PYZhy{}\PYZhy{}\PYZhy{}\PYZhy{}\PYZhy{}\PYZhy{}\PYZhy{}\PYZhy{}\PYZhy{}\PYZhy{}\PYZhy{}\PYZhy{}\PYZhy{}\PYZhy{}\PYZhy{}\PYZhy{}\PYZhy{}\PYZhy{}\PYZhy{}\PYZhy{}\PYZhy{}}
         \PY{c+c1}{\PYZsh{} Kumulierte Verteilungfunktion für die Varianz\PYZhy{}Kovarianz\PYZhy{}Methode}
         \PY{n}{array} \PY{o}{=} \PY{n}{np}\PY{o}{.}\PY{n}{array}\PY{p}{(}\PY{n}{np}\PY{o}{.}\PY{n}{arange}\PY{p}{(}\PY{l+m+mf}{0.0001}\PY{p}{,} \PY{l+m+mi}{1}\PY{p}{,} \PY{l+m+mf}{0.0001}\PY{p}{)}\PY{p}{)} \PY{c+c1}{\PYZsh{} Array von 0.0001 bis 1 in 0.0001\PYZhy{}er Schritten}
         \PY{n}{var\PYZus{}covar\PYZus{}results} \PY{o}{=} \PY{n}{stats}\PY{o}{.}\PY{n}{norm}\PY{o}{.}\PY{n}{ppf}\PY{p}{(}\PY{n}{array}\PY{p}{,} \PY{n}{mu\PYZus{}PF}\PY{p}{,} \PY{n}{std\PYZus{}PF}\PY{p}{)} \PY{c+c1}{\PYZsh{} Erzeugt eine parametrisierte percent point function }
         \PY{n}{var\PYZus{}covar\PYZus{}range} \PY{o}{=} \PY{n}{np}\PY{o}{.}\PY{n}{linspace}\PY{p}{(}\PY{n}{mini\PYZus{}values\PYZus{}PF}\PY{p}{,} \PY{n}{maxi\PYZus{}values\PYZus{}PF}\PY{p}{,} \PY{n}{bins}\PY{p}{)} \PY{c+c1}{\PYZsh{} Erzeugt ein Intervall für die X\PYZhy{}Achse}
         \PY{n}{plt}\PY{o}{.}\PY{n}{plot}\PY{p}{(}\PY{n}{var\PYZus{}covar\PYZus{}range}\PY{p}{,} \PY{n}{stats}\PY{o}{.}\PY{n}{norm}\PY{o}{.}\PY{n}{cdf}\PY{p}{(}\PY{n}{var\PYZus{}covar\PYZus{}range}\PY{p}{,} \PY{n}{mu\PYZus{}PF}\PY{p}{,} \PY{n}{std\PYZus{}PF}\PY{p}{)}\PY{p}{)} \PY{c+c1}{\PYZsh{} Plottet die Verteilungsfunktion}
         
         \PY{c+c1}{\PYZsh{}\PYZhy{}\PYZhy{}\PYZhy{}\PYZhy{}\PYZhy{}\PYZhy{}\PYZhy{}\PYZhy{}\PYZhy{}\PYZhy{}\PYZhy{}\PYZhy{}\PYZhy{}\PYZhy{}\PYZhy{}\PYZhy{}\PYZhy{}\PYZhy{}\PYZhy{}\PYZhy{}\PYZhy{}\PYZhy{}\PYZhy{}\PYZhy{}\PYZhy{}\PYZhy{}\PYZhy{}\PYZhy{}\PYZhy{}\PYZhy{}\PYZhy{}\PYZhy{}\PYZhy{}\PYZhy{}\PYZhy{}\PYZhy{}\PYZhy{}\PYZhy{}\PYZhy{}\PYZhy{}\PYZhy{}\PYZhy{}\PYZhy{}\PYZhy{}\PYZhy{}\PYZhy{}\PYZhy{}\PYZhy{}\PYZhy{}\PYZhy{}\PYZhy{}\PYZhy{}\PYZhy{}\PYZhy{}\PYZhy{}\PYZhy{}\PYZhy{}\PYZhy{}}
         \PY{c+c1}{\PYZsh{} Restliche Einstellungen für die Grafik}
         \PY{n}{plt}\PY{o}{.}\PY{n}{xlabel}\PY{p}{(}\PY{l+s+s1}{\PYZsq{}}\PY{l+s+s1}{Rendite}\PY{l+s+s1}{\PYZsq{}}\PY{p}{)}
         \PY{n}{plt}\PY{o}{.}\PY{n}{ylabel}\PY{p}{(}\PY{l+s+s1}{\PYZsq{}}\PY{l+s+s1}{Wahrscheinlichkeit}\PY{l+s+s1}{\PYZsq{}}\PY{p}{)}
         \PY{n}{blue\PYZus{}patch} \PY{o}{=} \PY{n}{mpatches}\PY{o}{.}\PY{n}{Patch}\PY{p}{(}\PY{n}{color}\PY{o}{=}\PY{l+s+s1}{\PYZsq{}}\PY{l+s+s1}{blue}\PY{l+s+s1}{\PYZsq{}}\PY{p}{,} \PY{n}{label}\PY{o}{=}\PY{l+s+s1}{\PYZsq{}}\PY{l+s+s1}{Historische Simulation}\PY{l+s+s1}{\PYZsq{}}\PY{p}{)}
         \PY{n}{orange\PYZus{}patch} \PY{o}{=} \PY{n}{mpatches}\PY{o}{.}\PY{n}{Patch}\PY{p}{(}\PY{n}{color}\PY{o}{=}\PY{l+s+s1}{\PYZsq{}}\PY{l+s+s1}{orange}\PY{l+s+s1}{\PYZsq{}}\PY{p}{,} \PY{n}{label}\PY{o}{=}\PY{l+s+s1}{\PYZsq{}}\PY{l+s+s1}{Varianz\PYZhy{}Kovarianzmethode}\PY{l+s+s1}{\PYZsq{}}\PY{p}{)}
         \PY{n}{plt}\PY{o}{.}\PY{n}{legend}\PY{p}{(}\PY{n}{handles}\PY{o}{=}\PY{p}{[}\PY{n}{orange\PYZus{}patch}\PY{p}{,} \PY{n}{blue\PYZus{}patch}\PY{p}{]}\PY{p}{)}
         \PY{n}{plt}\PY{o}{.}\PY{n}{title}\PY{p}{(}\PY{l+s+s1}{\PYZsq{}}\PY{l+s+s1}{Verteilungsfunktion: Historische Simulation versus Varianz\PYZhy{}Kovarianz\PYZhy{}Methode}\PY{l+s+s1}{\PYZsq{}}\PY{p}{)}
         \PY{n}{easy\PYZus{}plot}\PY{p}{(}\PY{p}{)}
\end{Verbatim}

    \begin{Verbatim}[commandchars=\\\{\}]
\#----------------------------------------------------------------------------------------------------\#
|          [EINGABE] Möchten Sie die Genauigkeit anpassen? Geben Sie "Ja" oder "Nein" ein:           | n
\#----------------------------------------------------------------------------------------------------\#

    \end{Verbatim}

    \begin{center}
    \adjustimage{max size={0.9\linewidth}{0.9\paperheight}}{output_35_1.png}
    \end{center}
    { \hspace*{\fill} \\}
    
    \hypertarget{risikomauxdfe-schuxe4tzen---historische-simulation}{%
\subsection{Risikomaße schätzen - Historische
Simulation:}\label{risikomauxdfe-schuxe4tzen---historische-simulation}}

\hypertarget{value-at-risk}{%
\subsubsection{Value at Risk}\label{value-at-risk}}

Um den Value at Risk für die historische Simulation zu bestimmen, werden
die Portfolio-Realisationen zunächst der Größe nach sortiert, wobei
dieser Schritt gleichzeitig der Ausgangspunkt für die Berechnung des
Conditional Value at Risk ist. Anschließend wird das Alpha-Quantil der
Verlustfunktion bestimmt, indem der Parameter ``alpha'' mit der Länge
der Liste ``RM\_list'' multipliziert wird. Der so ermittelte Wert gibt
die Position in der Liste ``RM\_list'' an, an welcher sich der Value at
Risk zum Konfidenzniveau ``alpha'' befindet.

    \begin{Verbatim}[commandchars=\\\{\}]
{\color{incolor}In [{\color{incolor}20}]:} \PY{k}{def} \PY{n+nf}{VaR}\PY{p}{(}\PY{n}{liste}\PY{p}{,} \PY{n}{alpha}\PY{p}{)}\PY{p}{:} 
             \PY{n}{liste} \PY{o}{=} \PY{n+nb}{sorted}\PY{p}{(}\PY{n}{liste}\PY{p}{)} \PY{c+c1}{\PYZsh{} Liste sortieren}
             \PY{n}{item} \PY{o}{=} \PY{p}{(}\PY{n+nb}{int}\PY{p}{(}\PY{p}{(}\PY{n}{alpha} \PY{o}{*} \PY{n+nb}{len}\PY{p}{(}\PY{n}{liste}\PY{p}{)}\PY{p}{)}\PY{p}{)} \PY{o}{\PYZhy{}} \PY{l+m+mi}{1}\PY{p}{)} \PY{c+c1}{\PYZsh{} Index\PYZhy{}Wert bestimmen}
             \PY{n}{VaR} \PY{o}{=} \PY{o}{\PYZhy{}}\PY{p}{(}\PY{n}{liste}\PY{p}{[}\PY{n}{item}\PY{p}{]}\PY{p}{)} \PY{c+c1}{\PYZsh{} Value at Risik ist der Wert an der Stelle des Index\PYZhy{}Wertes}
             \PY{n+nb}{print}\PY{p}{(}\PY{l+s+s1}{\PYZsq{}}\PY{l+s+s1}{\PYZsh{}}\PY{l+s+s1}{\PYZsq{}} \PY{o}{+} \PY{n}{SCREEN\PYZus{}WIDTH} \PY{o}{*} \PY{l+s+s1}{\PYZsq{}}\PY{l+s+s1}{\PYZhy{}}\PY{l+s+s1}{\PYZsq{}} \PY{o}{+} \PY{l+s+s1}{\PYZsq{}}\PY{l+s+s1}{\PYZsh{}}\PY{l+s+s1}{\PYZsq{}}\PY{p}{)}
             \PY{n+nb}{print}\PY{p}{(}\PY{l+s+s1}{\PYZsq{}}\PY{l+s+s1}{|}\PY{l+s+s1}{\PYZsq{}} \PY{o}{+} \PY{n}{centered}\PY{p}{(}\PY{l+s+s1}{\PYZsq{}}\PY{l+s+s1}{Der VaR beträgt: }\PY{l+s+s1}{\PYZsq{}} \PY{o}{+} \PY{n+nb}{str}\PY{p}{(}\PY{n}{VaR}\PY{p}{)} \PY{o}{+} \PY{l+s+s1}{\PYZsq{}}\PY{l+s+s1}{.}\PY{l+s+s1}{\PYZsq{}}\PY{p}{)} \PY{o}{+} \PY{l+s+s1}{\PYZsq{}}\PY{l+s+s1}{| }\PY{l+s+s1}{\PYZsq{}}\PY{p}{)}
\end{Verbatim}

    \hypertarget{conditional-value-at-risk}{%
\subsubsection{Conditional Value at
Risk}\label{conditional-value-at-risk}}

Der Conditional Value at Risk wird grundsätzlich wie der Value at Risk
bestimmt, wobei hier jedoch der Mittelwert über alle Realisationen bis
zum Alpha-Quantil gebildet wird. Daher wird hier nach der
Positionsbestimmung die Liste ``CVaR\_list'' mit denjenigen
Realisationen aus der Liste ``RM\_list'' gefüllt, welche den Bereich von
der kleinsten Realisation bis zum Alpha-Quantil abdecken. Die Summe
dieser Liste wird anschließend durch die Anzahl ihrer Elemente geteilt.

    \begin{Verbatim}[commandchars=\\\{\}]
{\color{incolor}In [{\color{incolor}21}]:} \PY{k}{def} \PY{n+nf}{CVaR}\PY{p}{(}\PY{n}{liste}\PY{p}{,} \PY{n}{alpha}\PY{p}{)}\PY{p}{:}
             \PY{n}{liste} \PY{o}{=} \PY{n+nb}{sorted}\PY{p}{(}\PY{n}{liste}\PY{p}{)} \PY{c+c1}{\PYZsh{} Liste sortieren}
             \PY{n}{item} \PY{o}{=} \PY{n+nb}{int}\PY{p}{(}\PY{p}{(}\PY{n}{alpha} \PY{o}{*} \PY{n+nb}{len}\PY{p}{(}\PY{n}{liste}\PY{p}{)}\PY{p}{)}\PY{p}{)} \PY{c+c1}{\PYZsh{} Index\PYZhy{}Wert bestimmen}
             \PY{n}{CVaR\PYZus{}list} \PY{o}{=} \PY{n}{liste}\PY{p}{[}\PY{l+m+mi}{0}\PY{p}{:}\PY{n}{item}\PY{p}{]} \PY{c+c1}{\PYZsh{} Teilliste bilden}
             \PY{n}{CVaR} \PY{o}{=} \PY{o}{\PYZhy{}}\PY{p}{(}\PY{n}{np}\PY{o}{.}\PY{n}{sum}\PY{p}{(}\PY{n}{CVaR\PYZus{}list}\PY{p}{)} \PY{o}{/} \PY{n+nb}{len}\PY{p}{(}\PY{n}{CVaR\PYZus{}list}\PY{p}{)}\PY{p}{)} \PY{c+c1}{\PYZsh{} Erwartungswert der Teilliste bilden}
             \PY{n+nb}{print}\PY{p}{(}\PY{l+s+s1}{\PYZsq{}}\PY{l+s+s1}{\PYZsh{}}\PY{l+s+s1}{\PYZsq{}} \PY{o}{+} \PY{n}{SCREEN\PYZus{}WIDTH} \PY{o}{*} \PY{l+s+s1}{\PYZsq{}}\PY{l+s+s1}{\PYZhy{}}\PY{l+s+s1}{\PYZsq{}} \PY{o}{+} \PY{l+s+s1}{\PYZsq{}}\PY{l+s+s1}{\PYZsh{}}\PY{l+s+s1}{\PYZsq{}}\PY{p}{)}
             \PY{n+nb}{print}\PY{p}{(}\PY{l+s+s1}{\PYZsq{}}\PY{l+s+s1}{|}\PY{l+s+s1}{\PYZsq{}} \PY{o}{+} \PY{n}{centered}\PY{p}{(}\PY{l+s+s1}{\PYZsq{}}\PY{l+s+s1}{Der CVaR beträgt: }\PY{l+s+s1}{\PYZsq{}} \PY{o}{+} \PY{n+nb}{str}\PY{p}{(}\PY{n}{CVaR}\PY{p}{)} \PY{o}{+} \PY{l+s+s1}{\PYZsq{}}\PY{l+s+s1}{.}\PY{l+s+s1}{\PYZsq{}}\PY{p}{)} \PY{o}{+} \PY{l+s+s1}{\PYZsq{}}\PY{l+s+s1}{| }\PY{l+s+s1}{\PYZsq{}}\PY{p}{)}
\end{Verbatim}

    \hypertarget{power-spektrales-risikomauxdf-phi_bpb-cdot-pb-1-text-mit-p-in-left01right-b-in-left01right}{%
\subsubsection{\texorpdfstring{Power-Spektrales Risikomaß
(\(\Phi_b(p)=b \cdot p^{b-1} \text{ mit } p \in \left[0,1\right], ~b \in \left(0,1\right)\))}{Power-Spektrales Risikomaß (\textbackslash Phi\_b(p)=b \textbackslash cdot p\^{}\{b-1\} \textbackslash text\{ mit \} p \textbackslash in \textbackslash left{[}0,1\textbackslash right{]}, \textasciitilde b \textbackslash in \textbackslash left(0,1\textbackslash right))}}\label{power-spektrales-risikomauxdf-phi_bpb-cdot-pb-1-text-mit-p-in-left01right-b-in-left01right}}

Für das Power-Spektrale Risikomaß ergibt sich der Erwartungswert aus dem
Mittelwert der ``RM\_list'' (der Mittelwert der Portfolio-Realisationen)
und das Risiko ergibt sich aus dem Matrixprodukt der transponierten
``RM\_list'' mit der ``subj\_ws\_list'', welche subjektive
Wahrscheinlichkeiten beinhaltet. Die Elemente letzterer Liste werden
berechnet, indem die Laufvariable jeder Realisation in der geordneten
Statisitk ``RM\_list'' (bei ``example1'' 1 bis 364) durch die
Gesamtanzahl der Realisationen (bei ``BAS.DE'' 253) geteilt und dann mit
``gamma'' potenziert wird (daher heißt es Power-Spektrales Risikomaß).
Dabei ist bei jeder Berechnung der jeweils vorher errechnete Wert zu
subtrahieren.

    \begin{Verbatim}[commandchars=\\\{\}]
{\color{incolor}In [{\color{incolor}22}]:} \PY{k}{def} \PY{n+nf}{power}\PY{p}{(}\PY{n}{liste}\PY{p}{,} \PY{n}{gamma}\PY{p}{,} \PY{n}{VarKoVar}\PY{o}{=}\PY{k+kc}{False}\PY{p}{)}\PY{p}{:}  
             \PY{n}{liste} \PY{o}{=} \PY{n+nb}{sorted}\PY{p}{(}\PY{n}{liste}\PY{p}{)} \PY{c+c1}{\PYZsh{} Liste sortieren}
             
             \PY{n}{EW} \PY{o}{=} \PY{n}{np}\PY{o}{.}\PY{n}{mean}\PY{p}{(}\PY{n}{liste}\PY{p}{)} \PY{c+c1}{\PYZsh{} Erwartungswert bestimmen}
             \PY{n+nb}{print}\PY{p}{(}\PY{l+s+s1}{\PYZsq{}}\PY{l+s+s1}{\PYZsh{}}\PY{l+s+s1}{\PYZsq{}} \PY{o}{+} \PY{n}{SCREEN\PYZus{}WIDTH} \PY{o}{*} \PY{l+s+s1}{\PYZsq{}}\PY{l+s+s1}{\PYZhy{}}\PY{l+s+s1}{\PYZsq{}} \PY{o}{+} \PY{l+s+s1}{\PYZsq{}}\PY{l+s+s1}{\PYZsh{}}\PY{l+s+s1}{\PYZsq{}}\PY{p}{)}
             \PY{k}{if} \PY{n}{VarKoVar} \PY{o}{==} \PY{k+kc}{False}\PY{p}{:}
                 \PY{n+nb}{print}\PY{p}{(}\PY{l+s+s1}{\PYZsq{}}\PY{l+s+s1}{|}\PY{l+s+s1}{\PYZsq{}} \PY{o}{+} \PY{n}{centered}\PY{p}{(}\PY{l+s+s1}{\PYZsq{}}\PY{l+s+s1}{Power\PYZhy{}Spektrales Risikomaß bei der historischen Simulation:}\PY{l+s+s1}{\PYZsq{}}\PY{p}{)} \PY{o}{+} \PY{l+s+s1}{\PYZsq{}}\PY{l+s+s1}{| }\PY{l+s+s1}{\PYZsq{}}\PY{p}{)}
             \PY{k}{else}\PY{p}{:}
                 \PY{n+nb}{print}\PY{p}{(}\PY{l+s+s1}{\PYZsq{}}\PY{l+s+s1}{|}\PY{l+s+s1}{\PYZsq{}} \PY{o}{+} \PY{n}{centered}\PY{p}{(}\PY{l+s+s1}{\PYZsq{}}\PY{l+s+s1}{Power\PYZhy{}Spektrales Risikomaß bei der Varianz\PYZhy{}Kovarianz\PYZhy{}Methode:}\PY{l+s+s1}{\PYZsq{}}\PY{p}{)} \PY{o}{+} \PY{l+s+s1}{\PYZsq{}}\PY{l+s+s1}{| }\PY{l+s+s1}{\PYZsq{}}\PY{p}{)}
             \PY{n+nb}{print}\PY{p}{(}\PY{l+s+s1}{\PYZsq{}}\PY{l+s+s1}{\PYZsh{}}\PY{l+s+s1}{\PYZsq{}} \PY{o}{+} \PY{n}{SCREEN\PYZus{}WIDTH} \PY{o}{*} \PY{l+s+s1}{\PYZsq{}}\PY{l+s+s1}{\PYZhy{}}\PY{l+s+s1}{\PYZsq{}} \PY{o}{+} \PY{l+s+s1}{\PYZsq{}}\PY{l+s+s1}{\PYZsh{}}\PY{l+s+s1}{\PYZsq{}}\PY{p}{)}
             \PY{n+nb}{print}\PY{p}{(}\PY{l+s+s1}{\PYZsq{}}\PY{l+s+s1}{|}\PY{l+s+s1}{\PYZsq{}} \PY{o}{+} \PY{n}{centered}\PY{p}{(}\PY{l+s+s1}{\PYZsq{}}\PY{l+s+s1}{Der Erwartungswert beträgt: }\PY{l+s+s1}{\PYZsq{}} \PY{o}{+} \PY{n+nb}{str}\PY{p}{(}\PY{n}{EW}\PY{p}{)} \PY{o}{+} \PY{l+s+s1}{\PYZsq{}}\PY{l+s+s1}{.}\PY{l+s+s1}{\PYZsq{}}\PY{p}{)} \PY{o}{+} \PY{l+s+s1}{\PYZsq{}}\PY{l+s+s1}{| }\PY{l+s+s1}{\PYZsq{}}\PY{p}{)}
         
             \PY{n}{subj\PYZus{}ws\PYZus{}list} \PY{o}{=} \PY{p}{[}\PY{p}{]}
             \PY{n}{counter1}\PY{p}{,} \PY{n}{counter2} \PY{o}{=} \PY{n+nb}{len}\PY{p}{(}\PY{n}{liste}\PY{p}{)}\PY{p}{,} \PY{n+nb}{len}\PY{p}{(}\PY{n}{liste}\PY{p}{)} \PY{o}{\PYZhy{}} \PY{l+m+mi}{1}
             \PY{c+c1}{\PYZsh{} Subjektive Wahrscheinlichkieten bestimmen und der Liste anfügen}
             \PY{k}{for} \PY{n}{i} \PY{o+ow}{in} \PY{n}{liste}\PY{p}{:}
                 \PY{n}{subj\PYZus{}ws} \PY{o}{=} \PY{p}{(}\PY{n}{np}\PY{o}{.}\PY{n}{power}\PY{p}{(}\PY{p}{(}\PY{n}{counter1} \PY{o}{/} \PY{n+nb}{len}\PY{p}{(}\PY{n}{liste}\PY{p}{)}\PY{p}{)}\PY{p}{,} \PY{n}{gamma}\PY{p}{)}\PY{p}{)} \PY{o}{\PYZhy{}} \PY{p}{(}\PY{n}{np}\PY{o}{.}\PY{n}{power}\PY{p}{(}\PY{p}{(}\PY{n}{counter2} \PY{o}{/} \PY{n+nb}{len}\PY{p}{(}\PY{n}{liste}\PY{p}{)}\PY{p}{)}\PY{p}{,} \PY{n}{gamma}\PY{p}{)}\PY{p}{)} 
                 \PY{n}{counter1} \PY{o}{\PYZhy{}}\PY{o}{=} \PY{l+m+mi}{1}
                 \PY{n}{counter2} \PY{o}{\PYZhy{}}\PY{o}{=} \PY{l+m+mi}{1}
                 \PY{n}{subj\PYZus{}ws\PYZus{}list}\PY{o}{.}\PY{n}{append}\PY{p}{(}\PY{n}{subj\PYZus{}ws}\PY{p}{)}
             \PY{n}{subj\PYZus{}ws\PYZus{}list} \PY{o}{=} \PY{n}{subj\PYZus{}ws\PYZus{}list}\PY{p}{[}\PY{p}{:}\PY{p}{:}\PY{o}{\PYZhy{}}\PY{l+m+mi}{1}\PY{p}{]} \PY{c+c1}{\PYZsh{} Liste inverieren}
             \PY{c+c1}{\PYZsh{} Risiko der Liste berechnen}
             \PY{n}{risk} \PY{o}{=} \PY{p}{(}\PY{o}{\PYZhy{}} \PY{n}{np}\PY{o}{.}\PY{n}{matmul}\PY{p}{(}\PY{n}{np}\PY{o}{.}\PY{n}{transpose}\PY{p}{(}\PY{n}{liste}\PY{p}{)}\PY{p}{,} \PY{n}{subj\PYZus{}ws\PYZus{}list}\PY{p}{)}\PY{p}{)}
             \PY{n+nb}{print}\PY{p}{(}\PY{l+s+s1}{\PYZsq{}}\PY{l+s+s1}{|}\PY{l+s+s1}{\PYZsq{}} \PY{o}{+} \PY{n}{centered}\PY{p}{(}\PY{l+s+s1}{\PYZsq{}}\PY{l+s+s1}{Das Risiko beträgt: }\PY{l+s+s1}{\PYZsq{}} \PY{o}{+} \PY{n+nb}{str}\PY{p}{(}\PY{n}{risk}\PY{p}{)} \PY{o}{+} \PY{l+s+s1}{\PYZsq{}}\PY{l+s+s1}{.}\PY{l+s+s1}{\PYZsq{}}\PY{p}{)} \PY{o}{+} \PY{l+s+s1}{\PYZsq{}}\PY{l+s+s1}{| }\PY{l+s+s1}{\PYZsq{}}\PY{p}{)}
             \PY{n+nb}{print}\PY{p}{(}\PY{l+s+s1}{\PYZsq{}}\PY{l+s+s1}{\PYZsh{}}\PY{l+s+s1}{\PYZsq{}} \PY{o}{+} \PY{n}{SCREEN\PYZus{}WIDTH} \PY{o}{*} \PY{l+s+s1}{\PYZsq{}}\PY{l+s+s1}{\PYZhy{}}\PY{l+s+s1}{\PYZsq{}} \PY{o}{+} \PY{l+s+s1}{\PYZsq{}}\PY{l+s+s1}{\PYZsh{}}\PY{l+s+s1}{\PYZsq{}}\PY{p}{)}
\end{Verbatim}

    \hypertarget{prinzip}{%
\paragraph{Prinzip:}\label{prinzip}}

Das erste Bild zeigt die Stammfunktion und das zweite Bild zeigt die
subjektiven Gewichte mit denen die PF-Realisationen multipliziert
werden. Je größer die Realisation, desto geringer die Gewichtung.

    \begin{Verbatim}[commandchars=\\\{\}]
{\color{incolor}In [{\color{incolor}23}]:} \PY{c+c1}{\PYZsh{}\PYZsh{}\PYZsh{}\PYZsh{}\PYZsh{}\PYZsh{}\PYZsh{}\PYZsh{}\PYZsh{}\PYZsh{}\PYZsh{}\PYZsh{}\PYZsh{}\PYZsh{}\PYZsh{}\PYZsh{}\PYZsh{}\PYZsh{}\PYZsh{}\PYZsh{}\PYZsh{}\PYZsh{}\PYZsh{}\PYZsh{}\PYZsh{}\PYZsh{}\PYZsh{}\PYZsh{}\PYZsh{}\PYZsh{}\PYZsh{}\PYZsh{}\PYZsh{}\PYZsh{}\PYZsh{}\PYZsh{}\PYZsh{}\PYZsh{}\PYZsh{}\PYZsh{}\PYZsh{}\PYZsh{}\PYZsh{}\PYZsh{}\PYZsh{}\PYZsh{}\PYZsh{}\PYZsh{}\PYZsh{}\PYZsh{}\PYZsh{}\PYZsh{}\PYZsh{}\PYZsh{}\PYZsh{}\PYZsh{}\PYZsh{}\PYZsh{}\PYZsh{}\PYZsh{}\PYZsh{}\PYZsh{}\PYZsh{}\PYZsh{}\PYZsh{}\PYZsh{}\PYZsh{}\PYZsh{}\PYZsh{}\PYZsh{}\PYZsh{}\PYZsh{}\PYZsh{}\PYZsh{}}
         \PY{c+c1}{\PYZsh{}\PYZhy{}\PYZhy{}\PYZhy{}\PYZhy{}\PYZhy{}\PYZhy{}\PYZhy{}\PYZhy{}\PYZhy{}\PYZhy{}\PYZhy{}\PYZhy{}\PYZhy{}\PYZhy{}\PYZhy{}\PYZhy{}\PYZhy{}\PYZhy{}\PYZhy{}\PYZhy{}\PYZhy{}\PYZhy{}\PYZhy{}\PYZhy{}\PYZhy{}\PYZhy{}\PYZhy{}\PYZhy{}\PYZhy{}\PYZhy{}\PYZhy{}\PYZhy{}\PYZhy{}\PYZhy{}\PYZhy{}\PYZhy{}\PYZhy{}\PYZhy{}\PYZhy{}\PYZhy{}\PYZhy{}\PYZhy{}\PYZhy{}\PYZhy{}\PYZhy{}\PYZhy{}\PYZhy{}\PYZhy{}\PYZhy{}\PYZhy{}\PYZhy{}\PYZhy{}\PYZhy{}\PYZhy{}\PYZhy{}\PYZhy{}\PYZhy{}\PYZhy{}\PYZhy{}\PYZhy{}\PYZhy{}\PYZhy{}\PYZhy{}\PYZhy{}\PYZhy{}\PYZhy{}\PYZhy{}\PYZhy{}\PYZhy{}\PYZhy{}\PYZhy{}\PYZhy{}\PYZhy{}}
         
         \PY{n}{gamma} \PY{o}{=} \PY{l+m+mf}{0.5}
         
         \PY{c+c1}{\PYZsh{}\PYZhy{}\PYZhy{}\PYZhy{}\PYZhy{}\PYZhy{}\PYZhy{}\PYZhy{}\PYZhy{}\PYZhy{}\PYZhy{}\PYZhy{}\PYZhy{}\PYZhy{}\PYZhy{}\PYZhy{}\PYZhy{}\PYZhy{}\PYZhy{}\PYZhy{}\PYZhy{}\PYZhy{}\PYZhy{}\PYZhy{}\PYZhy{}\PYZhy{}\PYZhy{}\PYZhy{}\PYZhy{}\PYZhy{}\PYZhy{}\PYZhy{}\PYZhy{}\PYZhy{}\PYZhy{}\PYZhy{}\PYZhy{}\PYZhy{}\PYZhy{}\PYZhy{}\PYZhy{}\PYZhy{}\PYZhy{}\PYZhy{}\PYZhy{}\PYZhy{}\PYZhy{}\PYZhy{}\PYZhy{}\PYZhy{}\PYZhy{}\PYZhy{}\PYZhy{}\PYZhy{}\PYZhy{}\PYZhy{}\PYZhy{}\PYZhy{}\PYZhy{}\PYZhy{}\PYZhy{}\PYZhy{}\PYZhy{}\PYZhy{}\PYZhy{}\PYZhy{}\PYZhy{}\PYZhy{}\PYZhy{}\PYZhy{}\PYZhy{}\PYZhy{}\PYZhy{}\PYZhy{}}
         \PY{c+c1}{\PYZsh{}\PYZsh{}\PYZsh{}\PYZsh{}\PYZsh{}\PYZsh{}\PYZsh{}\PYZsh{}\PYZsh{}\PYZsh{}\PYZsh{}\PYZsh{}\PYZsh{}\PYZsh{}\PYZsh{}\PYZsh{}\PYZsh{}\PYZsh{}\PYZsh{}\PYZsh{}\PYZsh{}\PYZsh{}\PYZsh{}\PYZsh{}\PYZsh{}\PYZsh{}\PYZsh{}\PYZsh{}\PYZsh{}\PYZsh{}\PYZsh{}\PYZsh{}\PYZsh{}\PYZsh{}\PYZsh{}\PYZsh{}\PYZsh{}\PYZsh{}\PYZsh{}\PYZsh{}\PYZsh{}\PYZsh{}\PYZsh{}\PYZsh{}\PYZsh{}\PYZsh{}\PYZsh{}\PYZsh{}\PYZsh{}\PYZsh{}\PYZsh{}\PYZsh{}\PYZsh{}\PYZsh{}\PYZsh{}\PYZsh{}\PYZsh{}\PYZsh{}\PYZsh{}\PYZsh{}\PYZsh{}\PYZsh{}\PYZsh{}\PYZsh{}\PYZsh{}\PYZsh{}\PYZsh{}\PYZsh{}\PYZsh{}\PYZsh{}\PYZsh{}\PYZsh{}\PYZsh{}\PYZsh{}}
         
         \PY{n}{prinzip}\PY{o}{.}\PY{n}{prinzip}\PY{p}{(}\PY{n}{gamma}\PY{p}{,} \PY{n}{values\PYZus{}PF}\PY{p}{)}
\end{Verbatim}

    \begin{center}
    \adjustimage{max size={0.9\linewidth}{0.9\paperheight}}{output_43_0.png}
    \end{center}
    { \hspace*{\fill} \\}
    
    \hypertarget{parameterfestlegung-und-aufruf-der-funktionen}{%
\subsubsection{Parameterfestlegung und Aufruf der
Funktionen}\label{parameterfestlegung-und-aufruf-der-funktionen}}

    \begin{Verbatim}[commandchars=\\\{\}]
{\color{incolor}In [{\color{incolor}24}]:} \PY{c+c1}{\PYZsh{}\PYZsh{}\PYZsh{}\PYZsh{}\PYZsh{}\PYZsh{}\PYZsh{}\PYZsh{}\PYZsh{}\PYZsh{}\PYZsh{}\PYZsh{}\PYZsh{}\PYZsh{}\PYZsh{}\PYZsh{}\PYZsh{}\PYZsh{}\PYZsh{}\PYZsh{}\PYZsh{}\PYZsh{}\PYZsh{}\PYZsh{}\PYZsh{}\PYZsh{}\PYZsh{}\PYZsh{}\PYZsh{}\PYZsh{}\PYZsh{}\PYZsh{}\PYZsh{}\PYZsh{}\PYZsh{}\PYZsh{}\PYZsh{}\PYZsh{}\PYZsh{}\PYZsh{}\PYZsh{}\PYZsh{}\PYZsh{}\PYZsh{}\PYZsh{}\PYZsh{}\PYZsh{}\PYZsh{}\PYZsh{}\PYZsh{}\PYZsh{}\PYZsh{}\PYZsh{}\PYZsh{}\PYZsh{}\PYZsh{}\PYZsh{}\PYZsh{}\PYZsh{}\PYZsh{}\PYZsh{}\PYZsh{}\PYZsh{}\PYZsh{}\PYZsh{}\PYZsh{}\PYZsh{}\PYZsh{}\PYZsh{}\PYZsh{}\PYZsh{}\PYZsh{}\PYZsh{}\PYZsh{}}
         \PY{c+c1}{\PYZsh{}\PYZhy{}\PYZhy{}\PYZhy{}\PYZhy{}\PYZhy{}\PYZhy{}\PYZhy{}\PYZhy{}\PYZhy{}\PYZhy{}\PYZhy{}\PYZhy{}\PYZhy{}\PYZhy{}\PYZhy{}\PYZhy{}\PYZhy{}\PYZhy{}\PYZhy{}\PYZhy{}\PYZhy{}\PYZhy{}\PYZhy{}\PYZhy{}\PYZhy{}\PYZhy{}\PYZhy{}\PYZhy{}\PYZhy{}\PYZhy{}\PYZhy{}\PYZhy{}\PYZhy{}\PYZhy{}\PYZhy{}\PYZhy{}\PYZhy{}\PYZhy{}\PYZhy{}\PYZhy{}\PYZhy{}\PYZhy{}\PYZhy{}\PYZhy{}\PYZhy{}\PYZhy{}\PYZhy{}\PYZhy{}\PYZhy{}\PYZhy{}\PYZhy{}\PYZhy{}\PYZhy{}\PYZhy{}\PYZhy{}\PYZhy{}\PYZhy{}\PYZhy{}\PYZhy{}\PYZhy{}\PYZhy{}\PYZhy{}\PYZhy{}\PYZhy{}\PYZhy{}\PYZhy{}\PYZhy{}\PYZhy{}\PYZhy{}\PYZhy{}\PYZhy{}\PYZhy{}\PYZhy{}}
         \PY{n}{alpha} \PY{o}{=} \PY{l+m+mf}{0.1}
         \PY{n}{VaR}\PY{p}{(}\PY{n}{values\PYZus{}PF}\PY{p}{,} \PY{n}{alpha}\PY{p}{)}
         
         \PY{n}{alpha} \PY{o}{=} \PY{l+m+mf}{0.1}
         \PY{n}{CVaR}\PY{p}{(}\PY{n}{values\PYZus{}PF}\PY{p}{,} \PY{n}{alpha}\PY{p}{)}
         
         \PY{n}{gamma} \PY{o}{=} \PY{l+m+mf}{0.5}
         \PY{n}{power}\PY{p}{(}\PY{n}{values\PYZus{}PF}\PY{p}{,} \PY{n}{gamma}\PY{p}{)}
         \PY{c+c1}{\PYZsh{}\PYZhy{}\PYZhy{}\PYZhy{}\PYZhy{}\PYZhy{}\PYZhy{}\PYZhy{}\PYZhy{}\PYZhy{}\PYZhy{}\PYZhy{}\PYZhy{}\PYZhy{}\PYZhy{}\PYZhy{}\PYZhy{}\PYZhy{}\PYZhy{}\PYZhy{}\PYZhy{}\PYZhy{}\PYZhy{}\PYZhy{}\PYZhy{}\PYZhy{}\PYZhy{}\PYZhy{}\PYZhy{}\PYZhy{}\PYZhy{}\PYZhy{}\PYZhy{}\PYZhy{}\PYZhy{}\PYZhy{}\PYZhy{}\PYZhy{}\PYZhy{}\PYZhy{}\PYZhy{}\PYZhy{}\PYZhy{}\PYZhy{}\PYZhy{}\PYZhy{}\PYZhy{}\PYZhy{}\PYZhy{}\PYZhy{}\PYZhy{}\PYZhy{}\PYZhy{}\PYZhy{}\PYZhy{}\PYZhy{}\PYZhy{}\PYZhy{}\PYZhy{}\PYZhy{}\PYZhy{}\PYZhy{}\PYZhy{}\PYZhy{}\PYZhy{}\PYZhy{}\PYZhy{}\PYZhy{}\PYZhy{}\PYZhy{}\PYZhy{}\PYZhy{}\PYZhy{}\PYZhy{}}
         \PY{c+c1}{\PYZsh{}\PYZsh{}\PYZsh{}\PYZsh{}\PYZsh{}\PYZsh{}\PYZsh{}\PYZsh{}\PYZsh{}\PYZsh{}\PYZsh{}\PYZsh{}\PYZsh{}\PYZsh{}\PYZsh{}\PYZsh{}\PYZsh{}\PYZsh{}\PYZsh{}\PYZsh{}\PYZsh{}\PYZsh{}\PYZsh{}\PYZsh{}\PYZsh{}\PYZsh{}\PYZsh{}\PYZsh{}\PYZsh{}\PYZsh{}\PYZsh{}\PYZsh{}\PYZsh{}\PYZsh{}\PYZsh{}\PYZsh{}\PYZsh{}\PYZsh{}\PYZsh{}\PYZsh{}\PYZsh{}\PYZsh{}\PYZsh{}\PYZsh{}\PYZsh{}\PYZsh{}\PYZsh{}\PYZsh{}\PYZsh{}\PYZsh{}\PYZsh{}\PYZsh{}\PYZsh{}\PYZsh{}\PYZsh{}\PYZsh{}\PYZsh{}\PYZsh{}\PYZsh{}\PYZsh{}\PYZsh{}\PYZsh{}\PYZsh{}\PYZsh{}\PYZsh{}\PYZsh{}\PYZsh{}\PYZsh{}\PYZsh{}\PYZsh{}\PYZsh{}\PYZsh{}\PYZsh{}\PYZsh{}}
\end{Verbatim}

    \begin{Verbatim}[commandchars=\\\{\}]
\#----------------------------------------------------------------------------------------------------\#
|                               Der VaR beträgt: 0.017789645491228443.                               | 
\#----------------------------------------------------------------------------------------------------\#
|                              Der CVaR beträgt: 0.026073806207793263.                               | 
\#----------------------------------------------------------------------------------------------------\#
|                    Power-Spektrales Risikomaß bei der historischen Simulation:                     | 
\#----------------------------------------------------------------------------------------------------\#
|                        Der Erwartungswert beträgt: 0.00010727212451525667.                         | 
|                             Das Risiko beträgt: 0.012121607385303547.                              | 
\#----------------------------------------------------------------------------------------------------\#

    \end{Verbatim}

    \hypertarget{risikomauxdfe-schuxe4tzen---varianz-kovarianz-methode}{%
\subsection{Risikomaße schätzen -
Varianz-Kovarianz-Methode:}\label{risikomauxdfe-schuxe4tzen---varianz-kovarianz-methode}}

\hypertarget{value-at-risk-conditional-value-at-risk-und-power-spektrales-risikomauxdf}{%
\subsubsection{Value at Risk, Conditional Value at Risk und
Power-Spektrales
Risikomaß}\label{value-at-risk-conditional-value-at-risk-und-power-spektrales-risikomauxdf}}

Die Berechnung des Value at Risk für die Varianz-Kovarianz-Methode ist
analog zu der bei der historischen Simulation. Der einzige Unterschied
ist, dass hierbei auf die auf (mu, sigma) parametrisierten Elemente
zurückgegriffen wird, welche zuvor als ``var\_covar\_results'' bei der
Generierung der analytischen Verteilungsfunktion gespeichert wurden. Für
den Conditional Value at Risk und das Power-Spektrale Risikomaß gilt
selbiges. \#\#\# Parameterfestlegung und Aufruf der Funktionen

    \begin{Verbatim}[commandchars=\\\{\}]
{\color{incolor}In [{\color{incolor}25}]:} \PY{c+c1}{\PYZsh{}\PYZsh{}\PYZsh{}\PYZsh{}\PYZsh{}\PYZsh{}\PYZsh{}\PYZsh{}\PYZsh{}\PYZsh{}\PYZsh{}\PYZsh{}\PYZsh{}\PYZsh{}\PYZsh{}\PYZsh{}\PYZsh{}\PYZsh{}\PYZsh{}\PYZsh{}\PYZsh{}\PYZsh{}\PYZsh{}\PYZsh{}\PYZsh{}\PYZsh{}\PYZsh{}\PYZsh{}\PYZsh{}\PYZsh{}\PYZsh{}\PYZsh{}\PYZsh{}\PYZsh{}\PYZsh{}\PYZsh{}\PYZsh{}\PYZsh{}\PYZsh{}\PYZsh{}\PYZsh{}\PYZsh{}\PYZsh{}\PYZsh{}\PYZsh{}\PYZsh{}\PYZsh{}\PYZsh{}\PYZsh{}\PYZsh{}\PYZsh{}\PYZsh{}\PYZsh{}\PYZsh{}\PYZsh{}\PYZsh{}\PYZsh{}\PYZsh{}\PYZsh{}\PYZsh{}\PYZsh{}\PYZsh{}\PYZsh{}\PYZsh{}\PYZsh{}\PYZsh{}\PYZsh{}\PYZsh{}\PYZsh{}\PYZsh{}\PYZsh{}\PYZsh{}\PYZsh{}\PYZsh{}}
         \PY{c+c1}{\PYZsh{}\PYZhy{}\PYZhy{}\PYZhy{}\PYZhy{}\PYZhy{}\PYZhy{}\PYZhy{}\PYZhy{}\PYZhy{}\PYZhy{}\PYZhy{}\PYZhy{}\PYZhy{}\PYZhy{}\PYZhy{}\PYZhy{}\PYZhy{}\PYZhy{}\PYZhy{}\PYZhy{}\PYZhy{}\PYZhy{}\PYZhy{}\PYZhy{}\PYZhy{}\PYZhy{}\PYZhy{}\PYZhy{}\PYZhy{}\PYZhy{}\PYZhy{}\PYZhy{}\PYZhy{}\PYZhy{}\PYZhy{}\PYZhy{}\PYZhy{}\PYZhy{}\PYZhy{}\PYZhy{}\PYZhy{}\PYZhy{}\PYZhy{}\PYZhy{}\PYZhy{}\PYZhy{}\PYZhy{}\PYZhy{}\PYZhy{}\PYZhy{}\PYZhy{}\PYZhy{}\PYZhy{}\PYZhy{}\PYZhy{}\PYZhy{}\PYZhy{}\PYZhy{}\PYZhy{}\PYZhy{}\PYZhy{}\PYZhy{}\PYZhy{}\PYZhy{}\PYZhy{}\PYZhy{}\PYZhy{}\PYZhy{}\PYZhy{}\PYZhy{}\PYZhy{}\PYZhy{}\PYZhy{}}
         \PY{n}{alpha} \PY{o}{=} \PY{l+m+mf}{0.1}
         \PY{n}{VaR}\PY{p}{(}\PY{n}{var\PYZus{}covar\PYZus{}results}\PY{p}{,} \PY{n}{alpha}\PY{p}{)}
         
         \PY{n}{alpha} \PY{o}{=} \PY{l+m+mf}{0.1}
         \PY{n}{CVaR}\PY{p}{(}\PY{n}{var\PYZus{}covar\PYZus{}results}\PY{p}{,} \PY{n}{alpha}\PY{p}{)}
         
         \PY{n}{gamma} \PY{o}{=} \PY{l+m+mf}{0.5}
         \PY{n}{power}\PY{p}{(}\PY{n}{var\PYZus{}covar\PYZus{}results}\PY{p}{,} \PY{n}{gamma}\PY{p}{,} \PY{n}{VarKoVar}\PY{o}{=}\PY{k+kc}{True}\PY{p}{)}
         \PY{c+c1}{\PYZsh{}\PYZhy{}\PYZhy{}\PYZhy{}\PYZhy{}\PYZhy{}\PYZhy{}\PYZhy{}\PYZhy{}\PYZhy{}\PYZhy{}\PYZhy{}\PYZhy{}\PYZhy{}\PYZhy{}\PYZhy{}\PYZhy{}\PYZhy{}\PYZhy{}\PYZhy{}\PYZhy{}\PYZhy{}\PYZhy{}\PYZhy{}\PYZhy{}\PYZhy{}\PYZhy{}\PYZhy{}\PYZhy{}\PYZhy{}\PYZhy{}\PYZhy{}\PYZhy{}\PYZhy{}\PYZhy{}\PYZhy{}\PYZhy{}\PYZhy{}\PYZhy{}\PYZhy{}\PYZhy{}\PYZhy{}\PYZhy{}\PYZhy{}\PYZhy{}\PYZhy{}\PYZhy{}\PYZhy{}\PYZhy{}\PYZhy{}\PYZhy{}\PYZhy{}\PYZhy{}\PYZhy{}\PYZhy{}\PYZhy{}\PYZhy{}\PYZhy{}\PYZhy{}\PYZhy{}\PYZhy{}\PYZhy{}\PYZhy{}\PYZhy{}\PYZhy{}\PYZhy{}\PYZhy{}\PYZhy{}\PYZhy{}\PYZhy{}\PYZhy{}\PYZhy{}\PYZhy{}\PYZhy{}}
         \PY{c+c1}{\PYZsh{}\PYZsh{}\PYZsh{}\PYZsh{}\PYZsh{}\PYZsh{}\PYZsh{}\PYZsh{}\PYZsh{}\PYZsh{}\PYZsh{}\PYZsh{}\PYZsh{}\PYZsh{}\PYZsh{}\PYZsh{}\PYZsh{}\PYZsh{}\PYZsh{}\PYZsh{}\PYZsh{}\PYZsh{}\PYZsh{}\PYZsh{}\PYZsh{}\PYZsh{}\PYZsh{}\PYZsh{}\PYZsh{}\PYZsh{}\PYZsh{}\PYZsh{}\PYZsh{}\PYZsh{}\PYZsh{}\PYZsh{}\PYZsh{}\PYZsh{}\PYZsh{}\PYZsh{}\PYZsh{}\PYZsh{}\PYZsh{}\PYZsh{}\PYZsh{}\PYZsh{}\PYZsh{}\PYZsh{}\PYZsh{}\PYZsh{}\PYZsh{}\PYZsh{}\PYZsh{}\PYZsh{}\PYZsh{}\PYZsh{}\PYZsh{}\PYZsh{}\PYZsh{}\PYZsh{}\PYZsh{}\PYZsh{}\PYZsh{}\PYZsh{}\PYZsh{}\PYZsh{}\PYZsh{}\PYZsh{}\PYZsh{}\PYZsh{}\PYZsh{}\PYZsh{}\PYZsh{}\PYZsh{}}
\end{Verbatim}

    \begin{Verbatim}[commandchars=\\\{\}]
\#----------------------------------------------------------------------------------------------------\#
|                               Der VaR beträgt: 0.018603399368407544.                               | 
\#----------------------------------------------------------------------------------------------------\#
|                              Der CVaR beträgt: 0.025490031878150177.                               | 
\#----------------------------------------------------------------------------------------------------\#
|                   Power-Spektrales Risikomaß bei der Varianz-Kovarianz-Methode:                    | 
\#----------------------------------------------------------------------------------------------------\#
|                        Der Erwartungswert beträgt: 0.00010727212451525916.                         | 
|                              Das Risiko beträgt: 0.01008497086603982.                              | 
\#----------------------------------------------------------------------------------------------------\#

    \end{Verbatim}


    % Add a bibliography block to the postdoc
    
    
    
    \end{document}
